%!TEX root = ../Thesis.tex
\chapter{Conclusions\label{ch:Conclusions}}
In this dissertation I have developed the first Monte Carlo implementation of Arnoldi's method for neutron transport.  This implementation uses explicitly restarted Arnoldi's method to estimate multiple eigenvalues of the transport-fission operator of the Boltzmann transport equation \eqref{eq:BoltzmannEquation}.  

Using Arnoldi's method for estimating eigenvalues is a new technique in the Monte Carlo particle transport field; traditionally, the power method has been used.  Arnoldi's method has been used in the numerical analysis community for many years to estimate multiple eigenvalues and eigenvectors of a linear operator, but this is the first time it has been used with a Monte Carlo application of the linear operator.

I have demonstrated the ability to use Arnoldi's method to estimate up to three eigenvalues of the transport-fission operator for a variety of homogeneous and heterogeneous one-dimensional problems.  The eigenvalue estimates have been compared to and agree with published results and independent deterministic calculations within statistical uncertainty.  The eigenvectors have also been estimated and compared with deterministic calculations; again the results from Arnoldi's method are in harmony with the deterministic calculations.  In some situations, Arnoldi's estimated eigenvectors are improvements over the power method estimated eigenvectors.

Arnoldi's method can be used to calculate more than three eigenmodes.  Calculating additional higher-order eigenmodes will require additional iterations in a restart, and it may be necessary to track more particles in an iteration or use a finer spatial discretization. 

Arnoldi's method requires the fission source to be discretized.  The simplest way to discretize the fission source is to use a constant in space or first-order accurate spatial approximation of the fission source which can, unfortunately, cause an error in the eigenvalue estimate if too few spatial bins are used to discretize the source.  I have implemented a second order accurate approximation to the fission source and have used it to reduce the error in the eigenvalue calculation by an order of magnitude for the same number of particles tracked.  The eigenvector estimates from the second order accurate approximation are a great improvement over the first order accurate approximation.  Rather than a jagged, step-wise approximation to the eigenvector, the second-order accurate approximation is a smooth and nearly continuous function across bins.  In addition, the figure of merit for the second-order accurate approximation is 2--3 times larger than the first order accurate approximation.

I have investigated relaxing the precision to which the transport-fission operator is applied at every Arnoldi iteration.  Studies have shown that relaxing Arnoldi's method has no effect on the convergence to the correct eigenvalues.  Relaxing Arnoldi's method in a Monte Carlo particle transport application simply involves tracking fewer particles in an iteration than was initially specified; tracking fewer particles is less computationally expensive and reduces the overall time for the simulation.  

Relaxing Arnoldi's method for Monte Carlo criticality applications can save on computation time.  Relaxing too much, however,  can cause the eigenvalue estimates to be incorrect.  In addition, relaxing Arnoldi's method only a little can cause the figure of merit to be smaller than for a non-relaxed Arnoldi's method.  Relaxing Arnoldi's method turns out to be not a sufficient improvement to make it worthwhile to use in practice.  

Two important topics currently being investigated in the Monte Carlo particle transport community are the underestimation of the variance of the mean eigenvalue estimate and the convergence of the fission source.  The power method \emph{under}estimates the variance because it ignores the correlation between power method iterations.  My implementation of Arnoldi's method also ignores the inter-iteration correlation, but the reported variance appears to be more conservative.  In one of the problems discussed in this dissertation, I have shown that Arnoldi's method \emph{over}estimates the uncertainty in the eigenvalue estimate by approximately 10\%.

The power method can take a long time to converge the fission source---especially for problems with a large dominance ratio.  In Monte Carlo criticality calculations, both the eigenvalue estimate and the fission source must be converged before tallying begins.  If convergence is slow, more iterations must be discarded and computation wasted.  I have shown that Arnoldi's method is superior to the power method in converging both the eigenvalue estimate and the fission source.  Arnoldi's method appears to converge both the eigenvalue estimate and the fission source immediately, while the power method can require several hundreds of iterations.  

\section{Future Work \label{sec:FutureWork}}
Work on Monte Carlo Arnoldi's method for criticality calculations is far from complete.  This dissertation represents the first work performed in this field.  Some of the many topics that still need to be explored are described next.

\subsection{Implicit Restarts} \label{sec:IRAM}
One of the most intriguing ways that restarted Arnoldi's method could be improved is by implementing \emph{implicit} restarts.  Implicit restarts were developed by \citet{Sorensen:1992Impli-0} as a way to restart Arnoldi's method with an improved starting vector, and at the same time reduce the computational expense of the algorithm and increase the stability of maintaining orthogonality between Arnoldi vectors.  

Implicitly restarted Arnoldi's method (IRAM), as it is called, performs iterations of the shifted \QR algorithm on the upper Hessenberg matrix, $H_m$, in the Arnoldi factorization in exchange of Arnoldi iterations.  It can be shown that performing these shifted \QR iterations allows Arnoldi's method to jump into the middle of the next restart, skipping several iterations.  IRAM is mathematically equivalent to picking a vector for the beginning of a new restart as a linear combination of the eigenvectors associated with the desired region of the spectrum of \A{} as we did in explicitly restarted Arnoldi's method.  The full derivation and proof of implicitly restarted Arnoldi's method is given in Appendix \ref{ch:IRAM}.

In Arnoldi's method, applying the linear operator \A{} is the most computationally expensive part, using greater than 80\% of the computer cycles in a given iteration.  Since the number of iterations in an Arnoldi restart is small, performing \QR iterations on $H_m$ will be inexpensive.  Trading computationally expensive applications of \A{} for inexpensive \QR iterations on $H_m$ should significantly reduce the computational expense of Arnoldi's method.

\subsection{Calculating Eigenvalue Estimates at Every Iteration}
In \Fref{ch:ArnoldiMethod} it was suggested that an Arnoldi restart could be treated similarly to the power method; that is, at the end of an Arnoldi restart an estimate for the eigenvalues are calculated and stored.  In the power method, an eigenvalue estimate is calculated at every iteration; since multiple iterations make up one Arnoldi restart, the power method has many more eigenvalue estimates than does Arnoldi's method for the same number of iterations and number of particles tracked.

The variance of the mean of the eigenvalue estimates goes as one over the square-root of the number of eigenvalue estimates.  Thus, the power method has an advantage over Arnoldi's method in that it has more eigenvalue estimates---the variance for the mean eigenvalue from the power method will almost certainly be smaller.  

There is no reason why an eigenvalue estimate could not be calculated at every Arnoldi iteration instead of just at the end of a restart.  Of course, after the first iteration we could only estimate the fundamental eigenvalue; we would have to wait for additional iterations to estimate higher order eigenvalues.  It would be slightly more computationally expensive, but the decrease in the variance may be worth the extra expense.  In all of the calculations in which Arnoldi's method was directly compared to the power method, the figures of merit from the power method calculations were larger than the figures of merit for Arnoldi's method even though both methods took approximately the same amount of computational time.  The figure of merit is larger for the power method because the variance is smaller.  Calculating an eigenvalue estimate at every iteration would increase the number of estimates in Arnoldi's method, but is more computationally expensive.

A preliminary test has been performed to see how estimating the eigenvalue at each iteration might work.  I have repeated the 20 mfp simulation in \Fref{ch:ArnoldiMethod} from Arnoldi's method, but instead of estimating three eigenvalues only the fundamental eigenvalue is estimated and only two iterations are done per restart.  In each iteration 1E5 particles are tracked.  The number of restarts are 125 inactive and 500 active.  The total number of particles tracked and the total number of iterations is the same for this simulation and the power method and Arnoldi's method from \Fref{ch:ArnoldiMethod}.

The results of this simulation are given in \Fref{tab:N1Arnoldi}.  The results for the new simulation are given first and denoted with a star.  The eigenvalue estimate is within statistical uncertainty of the reference value of $\lambda_0 = 4.82780$.  We can see that the standard deviation of this simulation is identical to the power method simulation from \Fref{ch:ArnoldiMethod}, but that the figure of merit is larger.
\begin{table}[h] \centering
    \begin{tabular}{rccc}
        \toprule
        & $\lambda_0$ & $\sigma$ & FOM \\
        \midrule
        Arnoldi* & 4.82806 & 6.3\e{-4} & 6.6\e{3} \\
        Power    & 4.82734 & 6.3\e{-4} & 5.4\e{3} \\
        Arnoldi  &  4.8290 & 1.5\e{-3} & 1.1\e{3} \\
        \bottomrule
    \end{tabular}
    \caption{Eigenvalue estimate and figure of merit for Arnoldi's method (Arnoldi*) with just 2 iterations per restart and only saving the fundamental eigenmode.  Also included are results from \Fref{tab:BasicResults} for comparison.  (Reference $\lambda_0 = 4.82780$.)}
    \label{tab:N1Arnoldi}
\end{table}

\Fref{fig:N1ArnoldiValues} shows the eigenvalue estimate convergence as well as the Shannon Entropy.  In this particular simulation, Arnoldi's method does not converge the eigenvalue estimate or the Shannon entropy immediately as we have seen previously, but it still only requires a few restarts to converge.  \Fref{fig:N1ArnoldiVectors} shows the estimated fundamental eigenvector along with the reference solution.  Again we see that Arnoldi's method can accurately estimate the fundamental eigenvector.  
\begin{figure} \centering
    \subfloat[Eigenvalue estimate and Shannon Entropy]{\label{fig:N1ArnoldiValues}% GNUPLOT: LaTeX picture with Postscript
\begingroup%
\makeatletter%
\newcommand{\GNUPLOTspecial}{%
  \@sanitize\catcode`\%=14\relax\special}%
\setlength{\unitlength}{0.0500bp}%
\begin{picture}(8640,5760)(0,0)%
  {\GNUPLOTspecial{"
%!PS-Adobe-2.0 EPSF-2.0
%%Title: N1ArnoldiValues.tex
%%Creator: gnuplot 4.3 patchlevel 0
%%CreationDate: Sat Aug  1 02:05:56 2009
%%DocumentFonts: 
%%BoundingBox: 0 0 432 288
%%EndComments
%%BeginProlog
/gnudict 256 dict def
gnudict begin
%
% The following true/false flags may be edited by hand if desired.
% The unit line width and grayscale image gamma correction may also be changed.
%
/Color true def
/Blacktext true def
/Solid false def
/Dashlength 1 def
/Landscape false def
/Level1 false def
/Rounded false def
/ClipToBoundingBox false def
/TransparentPatterns false def
/gnulinewidth 5.000 def
/userlinewidth gnulinewidth def
/Gamma 1.0 def
%
/vshift -66 def
/dl1 {
  10.0 Dashlength mul mul
  Rounded { currentlinewidth 0.75 mul sub dup 0 le { pop 0.01 } if } if
} def
/dl2 {
  10.0 Dashlength mul mul
  Rounded { currentlinewidth 0.75 mul add } if
} def
/hpt_ 31.5 def
/vpt_ 31.5 def
/hpt hpt_ def
/vpt vpt_ def
Level1 {} {
/SDict 10 dict def
systemdict /pdfmark known not {
  userdict /pdfmark systemdict /cleartomark get put
} if
SDict begin [
  /Title (N1ArnoldiValues.tex)
  /Subject (gnuplot plot)
  /Creator (gnuplot 4.3 patchlevel 0)
  /Author (Jeremy Conlin)
%  /Producer (gnuplot)
%  /Keywords ()
  /CreationDate (Sat Aug  1 02:05:56 2009)
  /DOCINFO pdfmark
end
} ifelse
/doclip {
  ClipToBoundingBox {
    newpath 0 0 moveto 432 0 lineto 432 288 lineto 0 288 lineto closepath
    clip
  } if
} def
%
% Gnuplot Prolog Version 4.2 (November 2007)
%
/M {moveto} bind def
/L {lineto} bind def
/R {rmoveto} bind def
/V {rlineto} bind def
/N {newpath moveto} bind def
/Z {closepath} bind def
/C {setrgbcolor} bind def
/f {rlineto fill} bind def
/Gshow {show} def   % May be redefined later in the file to support UTF-8
/vpt2 vpt 2 mul def
/hpt2 hpt 2 mul def
/Lshow {currentpoint stroke M 0 vshift R 
	Blacktext {gsave 0 setgray show grestore} {show} ifelse} def
/Rshow {currentpoint stroke M dup stringwidth pop neg vshift R
	Blacktext {gsave 0 setgray show grestore} {show} ifelse} def
/Cshow {currentpoint stroke M dup stringwidth pop -2 div vshift R 
	Blacktext {gsave 0 setgray show grestore} {show} ifelse} def
/UP {dup vpt_ mul /vpt exch def hpt_ mul /hpt exch def
  /hpt2 hpt 2 mul def /vpt2 vpt 2 mul def} def
/DL {Color {setrgbcolor Solid {pop []} if 0 setdash}
 {pop pop pop 0 setgray Solid {pop []} if 0 setdash} ifelse} def
/BL {stroke userlinewidth 2 mul setlinewidth
	Rounded {1 setlinejoin 1 setlinecap} if} def
/AL {stroke userlinewidth 2 div setlinewidth
	Rounded {1 setlinejoin 1 setlinecap} if} def
/UL {dup gnulinewidth mul /userlinewidth exch def
	dup 1 lt {pop 1} if 10 mul /udl exch def} def
/PL {stroke userlinewidth setlinewidth
	Rounded {1 setlinejoin 1 setlinecap} if} def
% Default Line colors
/LCw {1 1 1} def
/LCb {0 0 0} def
/LCa {0 0 0} def
/LC0 {1 0 0} def
/LC1 {0 1 0} def
/LC2 {0 0 1} def
/LC3 {1 0 1} def
/LC4 {0 1 1} def
/LC5 {1 1 0} def
/LC6 {0 0 0} def
/LC7 {1 0.3 0} def
/LC8 {0.5 0.5 0.5} def
% Default Line Types
/LTw {PL [] 1 setgray} def
/LTb {BL [] LCb DL} def
/LTa {AL [1 udl mul 2 udl mul] 0 setdash LCa setrgbcolor} def
/LT0 {PL [] LC0 DL} def
/LT1 {PL [4 dl1 2 dl2] LC1 DL} def
/LT2 {PL [2 dl1 3 dl2] LC2 DL} def
/LT3 {PL [1 dl1 1.5 dl2] LC3 DL} def
/LT4 {PL [6 dl1 2 dl2 1 dl1 2 dl2] LC4 DL} def
/LT5 {PL [3 dl1 3 dl2 1 dl1 3 dl2] LC5 DL} def
/LT6 {PL [2 dl1 2 dl2 2 dl1 6 dl2] LC6 DL} def
/LT7 {PL [1 dl1 2 dl2 6 dl1 2 dl2 1 dl1 2 dl2] LC7 DL} def
/LT8 {PL [2 dl1 2 dl2 2 dl1 2 dl2 2 dl1 2 dl2 2 dl1 4 dl2] LC8 DL} def
/Pnt {stroke [] 0 setdash gsave 1 setlinecap M 0 0 V stroke grestore} def
/Dia {stroke [] 0 setdash 2 copy vpt add M
  hpt neg vpt neg V hpt vpt neg V
  hpt vpt V hpt neg vpt V closepath stroke
  Pnt} def
/Pls {stroke [] 0 setdash vpt sub M 0 vpt2 V
  currentpoint stroke M
  hpt neg vpt neg R hpt2 0 V stroke
 } def
/Box {stroke [] 0 setdash 2 copy exch hpt sub exch vpt add M
  0 vpt2 neg V hpt2 0 V 0 vpt2 V
  hpt2 neg 0 V closepath stroke
  Pnt} def
/Crs {stroke [] 0 setdash exch hpt sub exch vpt add M
  hpt2 vpt2 neg V currentpoint stroke M
  hpt2 neg 0 R hpt2 vpt2 V stroke} def
/TriU {stroke [] 0 setdash 2 copy vpt 1.12 mul add M
  hpt neg vpt -1.62 mul V
  hpt 2 mul 0 V
  hpt neg vpt 1.62 mul V closepath stroke
  Pnt} def
/Star {2 copy Pls Crs} def
/BoxF {stroke [] 0 setdash exch hpt sub exch vpt add M
  0 vpt2 neg V hpt2 0 V 0 vpt2 V
  hpt2 neg 0 V closepath fill} def
/TriUF {stroke [] 0 setdash vpt 1.12 mul add M
  hpt neg vpt -1.62 mul V
  hpt 2 mul 0 V
  hpt neg vpt 1.62 mul V closepath fill} def
/TriD {stroke [] 0 setdash 2 copy vpt 1.12 mul sub M
  hpt neg vpt 1.62 mul V
  hpt 2 mul 0 V
  hpt neg vpt -1.62 mul V closepath stroke
  Pnt} def
/TriDF {stroke [] 0 setdash vpt 1.12 mul sub M
  hpt neg vpt 1.62 mul V
  hpt 2 mul 0 V
  hpt neg vpt -1.62 mul V closepath fill} def
/DiaF {stroke [] 0 setdash vpt add M
  hpt neg vpt neg V hpt vpt neg V
  hpt vpt V hpt neg vpt V closepath fill} def
/Pent {stroke [] 0 setdash 2 copy gsave
  translate 0 hpt M 4 {72 rotate 0 hpt L} repeat
  closepath stroke grestore Pnt} def
/PentF {stroke [] 0 setdash gsave
  translate 0 hpt M 4 {72 rotate 0 hpt L} repeat
  closepath fill grestore} def
/Circle {stroke [] 0 setdash 2 copy
  hpt 0 360 arc stroke Pnt} def
/CircleF {stroke [] 0 setdash hpt 0 360 arc fill} def
/C0 {BL [] 0 setdash 2 copy moveto vpt 90 450 arc} bind def
/C1 {BL [] 0 setdash 2 copy moveto
	2 copy vpt 0 90 arc closepath fill
	vpt 0 360 arc closepath} bind def
/C2 {BL [] 0 setdash 2 copy moveto
	2 copy vpt 90 180 arc closepath fill
	vpt 0 360 arc closepath} bind def
/C3 {BL [] 0 setdash 2 copy moveto
	2 copy vpt 0 180 arc closepath fill
	vpt 0 360 arc closepath} bind def
/C4 {BL [] 0 setdash 2 copy moveto
	2 copy vpt 180 270 arc closepath fill
	vpt 0 360 arc closepath} bind def
/C5 {BL [] 0 setdash 2 copy moveto
	2 copy vpt 0 90 arc
	2 copy moveto
	2 copy vpt 180 270 arc closepath fill
	vpt 0 360 arc} bind def
/C6 {BL [] 0 setdash 2 copy moveto
	2 copy vpt 90 270 arc closepath fill
	vpt 0 360 arc closepath} bind def
/C7 {BL [] 0 setdash 2 copy moveto
	2 copy vpt 0 270 arc closepath fill
	vpt 0 360 arc closepath} bind def
/C8 {BL [] 0 setdash 2 copy moveto
	2 copy vpt 270 360 arc closepath fill
	vpt 0 360 arc closepath} bind def
/C9 {BL [] 0 setdash 2 copy moveto
	2 copy vpt 270 450 arc closepath fill
	vpt 0 360 arc closepath} bind def
/C10 {BL [] 0 setdash 2 copy 2 copy moveto vpt 270 360 arc closepath fill
	2 copy moveto
	2 copy vpt 90 180 arc closepath fill
	vpt 0 360 arc closepath} bind def
/C11 {BL [] 0 setdash 2 copy moveto
	2 copy vpt 0 180 arc closepath fill
	2 copy moveto
	2 copy vpt 270 360 arc closepath fill
	vpt 0 360 arc closepath} bind def
/C12 {BL [] 0 setdash 2 copy moveto
	2 copy vpt 180 360 arc closepath fill
	vpt 0 360 arc closepath} bind def
/C13 {BL [] 0 setdash 2 copy moveto
	2 copy vpt 0 90 arc closepath fill
	2 copy moveto
	2 copy vpt 180 360 arc closepath fill
	vpt 0 360 arc closepath} bind def
/C14 {BL [] 0 setdash 2 copy moveto
	2 copy vpt 90 360 arc closepath fill
	vpt 0 360 arc} bind def
/C15 {BL [] 0 setdash 2 copy vpt 0 360 arc closepath fill
	vpt 0 360 arc closepath} bind def
/Rec {newpath 4 2 roll moveto 1 index 0 rlineto 0 exch rlineto
	neg 0 rlineto closepath} bind def
/Square {dup Rec} bind def
/Bsquare {vpt sub exch vpt sub exch vpt2 Square} bind def
/S0 {BL [] 0 setdash 2 copy moveto 0 vpt rlineto BL Bsquare} bind def
/S1 {BL [] 0 setdash 2 copy vpt Square fill Bsquare} bind def
/S2 {BL [] 0 setdash 2 copy exch vpt sub exch vpt Square fill Bsquare} bind def
/S3 {BL [] 0 setdash 2 copy exch vpt sub exch vpt2 vpt Rec fill Bsquare} bind def
/S4 {BL [] 0 setdash 2 copy exch vpt sub exch vpt sub vpt Square fill Bsquare} bind def
/S5 {BL [] 0 setdash 2 copy 2 copy vpt Square fill
	exch vpt sub exch vpt sub vpt Square fill Bsquare} bind def
/S6 {BL [] 0 setdash 2 copy exch vpt sub exch vpt sub vpt vpt2 Rec fill Bsquare} bind def
/S7 {BL [] 0 setdash 2 copy exch vpt sub exch vpt sub vpt vpt2 Rec fill
	2 copy vpt Square fill Bsquare} bind def
/S8 {BL [] 0 setdash 2 copy vpt sub vpt Square fill Bsquare} bind def
/S9 {BL [] 0 setdash 2 copy vpt sub vpt vpt2 Rec fill Bsquare} bind def
/S10 {BL [] 0 setdash 2 copy vpt sub vpt Square fill 2 copy exch vpt sub exch vpt Square fill
	Bsquare} bind def
/S11 {BL [] 0 setdash 2 copy vpt sub vpt Square fill 2 copy exch vpt sub exch vpt2 vpt Rec fill
	Bsquare} bind def
/S12 {BL [] 0 setdash 2 copy exch vpt sub exch vpt sub vpt2 vpt Rec fill Bsquare} bind def
/S13 {BL [] 0 setdash 2 copy exch vpt sub exch vpt sub vpt2 vpt Rec fill
	2 copy vpt Square fill Bsquare} bind def
/S14 {BL [] 0 setdash 2 copy exch vpt sub exch vpt sub vpt2 vpt Rec fill
	2 copy exch vpt sub exch vpt Square fill Bsquare} bind def
/S15 {BL [] 0 setdash 2 copy Bsquare fill Bsquare} bind def
/D0 {gsave translate 45 rotate 0 0 S0 stroke grestore} bind def
/D1 {gsave translate 45 rotate 0 0 S1 stroke grestore} bind def
/D2 {gsave translate 45 rotate 0 0 S2 stroke grestore} bind def
/D3 {gsave translate 45 rotate 0 0 S3 stroke grestore} bind def
/D4 {gsave translate 45 rotate 0 0 S4 stroke grestore} bind def
/D5 {gsave translate 45 rotate 0 0 S5 stroke grestore} bind def
/D6 {gsave translate 45 rotate 0 0 S6 stroke grestore} bind def
/D7 {gsave translate 45 rotate 0 0 S7 stroke grestore} bind def
/D8 {gsave translate 45 rotate 0 0 S8 stroke grestore} bind def
/D9 {gsave translate 45 rotate 0 0 S9 stroke grestore} bind def
/D10 {gsave translate 45 rotate 0 0 S10 stroke grestore} bind def
/D11 {gsave translate 45 rotate 0 0 S11 stroke grestore} bind def
/D12 {gsave translate 45 rotate 0 0 S12 stroke grestore} bind def
/D13 {gsave translate 45 rotate 0 0 S13 stroke grestore} bind def
/D14 {gsave translate 45 rotate 0 0 S14 stroke grestore} bind def
/D15 {gsave translate 45 rotate 0 0 S15 stroke grestore} bind def
/DiaE {stroke [] 0 setdash vpt add M
  hpt neg vpt neg V hpt vpt neg V
  hpt vpt V hpt neg vpt V closepath stroke} def
/BoxE {stroke [] 0 setdash exch hpt sub exch vpt add M
  0 vpt2 neg V hpt2 0 V 0 vpt2 V
  hpt2 neg 0 V closepath stroke} def
/TriUE {stroke [] 0 setdash vpt 1.12 mul add M
  hpt neg vpt -1.62 mul V
  hpt 2 mul 0 V
  hpt neg vpt 1.62 mul V closepath stroke} def
/TriDE {stroke [] 0 setdash vpt 1.12 mul sub M
  hpt neg vpt 1.62 mul V
  hpt 2 mul 0 V
  hpt neg vpt -1.62 mul V closepath stroke} def
/PentE {stroke [] 0 setdash gsave
  translate 0 hpt M 4 {72 rotate 0 hpt L} repeat
  closepath stroke grestore} def
/CircE {stroke [] 0 setdash 
  hpt 0 360 arc stroke} def
/Opaque {gsave closepath 1 setgray fill grestore 0 setgray closepath} def
/DiaW {stroke [] 0 setdash vpt add M
  hpt neg vpt neg V hpt vpt neg V
  hpt vpt V hpt neg vpt V Opaque stroke} def
/BoxW {stroke [] 0 setdash exch hpt sub exch vpt add M
  0 vpt2 neg V hpt2 0 V 0 vpt2 V
  hpt2 neg 0 V Opaque stroke} def
/TriUW {stroke [] 0 setdash vpt 1.12 mul add M
  hpt neg vpt -1.62 mul V
  hpt 2 mul 0 V
  hpt neg vpt 1.62 mul V Opaque stroke} def
/TriDW {stroke [] 0 setdash vpt 1.12 mul sub M
  hpt neg vpt 1.62 mul V
  hpt 2 mul 0 V
  hpt neg vpt -1.62 mul V Opaque stroke} def
/PentW {stroke [] 0 setdash gsave
  translate 0 hpt M 4 {72 rotate 0 hpt L} repeat
  Opaque stroke grestore} def
/CircW {stroke [] 0 setdash 
  hpt 0 360 arc Opaque stroke} def
/BoxFill {gsave Rec 1 setgray fill grestore} def
/Density {
  /Fillden exch def
  currentrgbcolor
  /ColB exch def /ColG exch def /ColR exch def
  /ColR ColR Fillden mul Fillden sub 1 add def
  /ColG ColG Fillden mul Fillden sub 1 add def
  /ColB ColB Fillden mul Fillden sub 1 add def
  ColR ColG ColB setrgbcolor} def
/BoxColFill {gsave Rec PolyFill} def
/PolyFill {gsave Density fill grestore grestore} def
/h {rlineto rlineto rlineto gsave closepath fill grestore} bind def
%
% PostScript Level 1 Pattern Fill routine for rectangles
% Usage: x y w h s a XX PatternFill
%	x,y = lower left corner of box to be filled
%	w,h = width and height of box
%	  a = angle in degrees between lines and x-axis
%	 XX = 0/1 for no/yes cross-hatch
%
/PatternFill {gsave /PFa [ 9 2 roll ] def
  PFa 0 get PFa 2 get 2 div add PFa 1 get PFa 3 get 2 div add translate
  PFa 2 get -2 div PFa 3 get -2 div PFa 2 get PFa 3 get Rec
  gsave 1 setgray fill grestore clip
  currentlinewidth 0.5 mul setlinewidth
  /PFs PFa 2 get dup mul PFa 3 get dup mul add sqrt def
  0 0 M PFa 5 get rotate PFs -2 div dup translate
  0 1 PFs PFa 4 get div 1 add floor cvi
	{PFa 4 get mul 0 M 0 PFs V} for
  0 PFa 6 get ne {
	0 1 PFs PFa 4 get div 1 add floor cvi
	{PFa 4 get mul 0 2 1 roll M PFs 0 V} for
 } if
  stroke grestore} def
%
/languagelevel where
 {pop languagelevel} {1} ifelse
 2 lt
	{/InterpretLevel1 true def}
	{/InterpretLevel1 Level1 def}
 ifelse
%
% PostScript level 2 pattern fill definitions
%
/Level2PatternFill {
/Tile8x8 {/PaintType 2 /PatternType 1 /TilingType 1 /BBox [0 0 8 8] /XStep 8 /YStep 8}
	bind def
/KeepColor {currentrgbcolor [/Pattern /DeviceRGB] setcolorspace} bind def
<< Tile8x8
 /PaintProc {0.5 setlinewidth pop 0 0 M 8 8 L 0 8 M 8 0 L stroke} 
>> matrix makepattern
/Pat1 exch def
<< Tile8x8
 /PaintProc {0.5 setlinewidth pop 0 0 M 8 8 L 0 8 M 8 0 L stroke
	0 4 M 4 8 L 8 4 L 4 0 L 0 4 L stroke}
>> matrix makepattern
/Pat2 exch def
<< Tile8x8
 /PaintProc {0.5 setlinewidth pop 0 0 M 0 8 L
	8 8 L 8 0 L 0 0 L fill}
>> matrix makepattern
/Pat3 exch def
<< Tile8x8
 /PaintProc {0.5 setlinewidth pop -4 8 M 8 -4 L
	0 12 M 12 0 L stroke}
>> matrix makepattern
/Pat4 exch def
<< Tile8x8
 /PaintProc {0.5 setlinewidth pop -4 0 M 8 12 L
	0 -4 M 12 8 L stroke}
>> matrix makepattern
/Pat5 exch def
<< Tile8x8
 /PaintProc {0.5 setlinewidth pop -2 8 M 4 -4 L
	0 12 M 8 -4 L 4 12 M 10 0 L stroke}
>> matrix makepattern
/Pat6 exch def
<< Tile8x8
 /PaintProc {0.5 setlinewidth pop -2 0 M 4 12 L
	0 -4 M 8 12 L 4 -4 M 10 8 L stroke}
>> matrix makepattern
/Pat7 exch def
<< Tile8x8
 /PaintProc {0.5 setlinewidth pop 8 -2 M -4 4 L
	12 0 M -4 8 L 12 4 M 0 10 L stroke}
>> matrix makepattern
/Pat8 exch def
<< Tile8x8
 /PaintProc {0.5 setlinewidth pop 0 -2 M 12 4 L
	-4 0 M 12 8 L -4 4 M 8 10 L stroke}
>> matrix makepattern
/Pat9 exch def
/Pattern1 {PatternBgnd KeepColor Pat1 setpattern} bind def
/Pattern2 {PatternBgnd KeepColor Pat2 setpattern} bind def
/Pattern3 {PatternBgnd KeepColor Pat3 setpattern} bind def
/Pattern4 {PatternBgnd KeepColor Landscape {Pat5} {Pat4} ifelse setpattern} bind def
/Pattern5 {PatternBgnd KeepColor Landscape {Pat4} {Pat5} ifelse setpattern} bind def
/Pattern6 {PatternBgnd KeepColor Landscape {Pat9} {Pat6} ifelse setpattern} bind def
/Pattern7 {PatternBgnd KeepColor Landscape {Pat8} {Pat7} ifelse setpattern} bind def
} def
%
%
%End of PostScript Level 2 code
%
/PatternBgnd {
  TransparentPatterns {} {gsave 1 setgray fill grestore} ifelse
} def
%
% Substitute for Level 2 pattern fill codes with
% grayscale if Level 2 support is not selected.
%
/Level1PatternFill {
/Pattern1 {0.250 Density} bind def
/Pattern2 {0.500 Density} bind def
/Pattern3 {0.750 Density} bind def
/Pattern4 {0.125 Density} bind def
/Pattern5 {0.375 Density} bind def
/Pattern6 {0.625 Density} bind def
/Pattern7 {0.875 Density} bind def
} def
%
% Now test for support of Level 2 code
%
Level1 {Level1PatternFill} {Level2PatternFill} ifelse
%
/Symbol-Oblique /Symbol findfont [1 0 .167 1 0 0] makefont
dup length dict begin {1 index /FID eq {pop pop} {def} ifelse} forall
currentdict end definefont pop
end
%%EndProlog
gnudict begin
gsave
doclip
0 0 translate
0.050 0.050 scale
0 setgray
newpath
1.000 UL
LTb
1100 640 M
63 0 V
-63 488 R
63 0 V
-63 488 R
63 0 V
-63 488 R
63 0 V
-63 488 R
63 0 V
-63 488 R
63 0 V
-63 487 R
63 0 V
-63 488 R
63 0 V
-63 488 R
63 0 V
-63 488 R
63 0 V
-63 488 R
63 0 V
1100 640 M
0 63 V
0 4816 R
0 -63 V
2103 640 M
0 63 V
0 4816 R
0 -63 V
3106 640 M
0 63 V
0 4816 R
0 -63 V
4109 640 M
0 63 V
0 4816 R
0 -63 V
5112 640 M
0 63 V
0 4816 R
0 -63 V
6115 640 M
0 63 V
0 4816 R
0 -63 V
7118 640 M
0 63 V
0 4816 R
0 -63 V
7419 640 M
-63 0 V
63 813 R
-63 0 V
63 813 R
-63 0 V
63 814 R
-63 0 V
63 813 R
-63 0 V
63 813 R
-63 0 V
63 813 R
-63 0 V
stroke
1100 5519 N
0 -4879 V
6319 0 V
0 4879 V
-6319 0 V
Z stroke
LCb setrgbcolor
LTb
LCb setrgbcolor
LTb
LCb setrgbcolor
LTb
LCb setrgbcolor
LTb
1.000 UP
1.000 UL
LTb
1.000 UL
LTb
5196 703 N
0 800 V
2103 0 V
0 -800 V
-2103 0 V
Z stroke
5196 1503 M
2103 0 V
1.000 UP
stroke
LT0
LCb setrgbcolor
LT0
6636 1303 M
543 0 V
-543 31 R
0 -62 V
543 62 R
0 -62 V
1115 640 M
5 1524 V
10 1207 V
10 569 V
10 432 V
10 245 V
10 116 V
10 61 V
10 145 V
10 -3 V
10 87 V
10 29 V
10 8 V
10 -7 V
10 61 V
10 -12 V
11 -132 V
10 114 V
10 -51 V
10 48 V
10 -17 V
10 53 V
10 -56 V
10 47 V
10 -35 V
10 31 V
10 -62 V
10 50 V
10 -9 V
10 44 V
10 -48 V
10 50 V
10 11 V
10 -93 V
10 99 V
10 -87 V
10 26 V
10 26 V
10 45 V
10 -75 V
10 18 V
10 -19 V
10 70 V
10 -76 V
10 -20 V
10 99 V
10 -102 V
10 30 V
10 41 V
11 -4 V
10 -29 V
10 3 V
10 -8 V
10 70 V
10 -36 V
10 24 V
10 -47 V
10 4 V
10 17 V
10 4 V
10 -31 V
10 -13 V
10 -47 V
10 120 V
10 -80 V
10 58 V
10 -23 V
10 20 V
10 3 V
10 -29 V
10 36 V
10 -89 V
10 7 V
10 122 V
10 -80 V
10 58 V
10 -18 V
10 -7 V
10 9 V
10 -59 V
10 88 V
10 -106 V
11 -2 V
10 51 V
10 -31 V
10 91 V
10 -18 V
10 -31 V
10 9 V
10 -57 V
10 39 V
10 9 V
10 75 V
10 -61 V
10 -25 V
10 -11 V
10 2 V
10 68 V
10 -91 V
stroke 2093 5076 M
10 -3 V
10 -1 V
10 63 V
10 -40 V
10 23 V
10 -16 V
10 -29 V
10 -20 V
10 30 V
10 25 V
10 36 V
10 -39 V
10 8 V
10 20 V
10 -41 V
10 -6 V
10 19 V
11 -35 V
10 40 V
10 -39 V
10 -66 V
10 49 V
10 105 V
10 -81 V
10 19 V
10 0 V
10 31 V
10 -30 V
10 -4 V
10 15 V
10 -5 V
10 -4 V
10 -5 V
10 -3 V
10 4 V
10 0 V
10 1 V
10 -2 V
10 1 V
10 3 V
10 -1 V
10 -1 V
10 0 V
10 1 V
10 2 V
10 -3 V
10 -2 V
10 -1 V
10 -2 V
10 0 V
11 0 V
10 1 V
10 0 V
10 1 V
10 0 V
10 -1 V
10 2 V
10 -2 V
10 -1 V
10 -1 V
10 1 V
10 1 V
10 -1 V
10 0 V
10 -1 V
10 -1 V
10 1 V
10 1 V
10 0 V
10 1 V
10 0 V
10 0 V
10 0 V
10 0 V
10 0 V
10 1 V
10 0 V
10 0 V
10 0 V
10 0 V
10 -1 V
10 1 V
10 -1 V
11 1 V
10 0 V
10 0 V
10 0 V
10 0 V
10 0 V
10 0 V
10 -1 V
10 1 V
10 0 V
10 -1 V
10 -1 V
10 1 V
10 0 V
10 1 V
10 0 V
10 0 V
10 1 V
10 -1 V
10 0 V
10 0 V
stroke 3136 5094 M
10 0 V
10 0 V
10 0 V
10 0 V
10 -1 V
10 0 V
10 0 V
10 0 V
10 -1 V
10 0 V
10 0 V
10 -1 V
11 0 V
10 0 V
10 0 V
10 1 V
10 0 V
10 0 V
10 0 V
10 0 V
10 -1 V
10 0 V
10 0 V
10 1 V
10 0 V
10 0 V
10 0 V
10 0 V
10 -1 V
10 0 V
10 1 V
10 0 V
10 0 V
10 0 V
10 0 V
10 0 V
10 0 V
10 1 V
10 0 V
10 0 V
10 0 V
10 0 V
10 1 V
10 0 V
10 0 V
11 0 V
10 -1 V
10 0 V
10 0 V
10 0 V
10 1 V
10 0 V
10 0 V
10 0 V
10 0 V
10 0 V
10 0 V
10 0 V
10 0 V
10 0 V
10 0 V
10 0 V
10 0 V
10 0 V
10 0 V
10 0 V
10 0 V
10 0 V
10 0 V
10 0 V
10 1 V
10 0 V
10 0 V
10 -1 V
10 0 V
10 1 V
10 0 V
10 0 V
11 0 V
10 0 V
10 0 V
10 0 V
10 0 V
10 0 V
10 0 V
10 0 V
10 0 V
10 1 V
10 0 V
10 0 V
10 0 V
10 0 V
10 0 V
10 0 V
10 0 V
10 0 V
10 1 V
10 -1 V
10 0 V
10 0 V
10 0 V
10 0 V
10 0 V
10 0 V
stroke 4179 5096 M
10 0 V
10 0 V
10 0 V
10 -1 V
10 0 V
10 0 V
10 0 V
11 0 V
10 0 V
10 0 V
10 0 V
10 0 V
10 0 V
10 0 V
10 0 V
10 0 V
10 0 V
10 0 V
10 0 V
10 0 V
10 0 V
10 0 V
10 0 V
10 0 V
10 0 V
10 0 V
10 -1 V
10 1 V
10 -1 V
10 0 V
10 0 V
10 1 V
10 -1 V
10 1 V
10 -1 V
10 0 V
10 1 V
10 0 V
10 0 V
10 0 V
10 0 V
11 0 V
10 0 V
10 0 V
10 0 V
10 0 V
10 0 V
10 0 V
10 0 V
10 0 V
10 1 V
10 0 V
10 0 V
10 0 V
10 0 V
10 1 V
10 0 V
10 0 V
10 0 V
10 0 V
10 0 V
10 0 V
10 0 V
10 0 V
10 0 V
10 0 V
10 -1 V
10 0 V
10 1 V
10 0 V
10 0 V
10 0 V
10 0 V
10 0 V
11 0 V
10 0 V
10 0 V
10 0 V
10 0 V
10 0 V
10 0 V
10 0 V
10 0 V
10 0 V
10 0 V
10 0 V
10 0 V
10 0 V
10 0 V
10 0 V
10 0 V
10 0 V
10 0 V
10 1 V
10 0 V
10 -1 V
10 1 V
10 0 V
10 0 V
10 0 V
10 0 V
10 -1 V
10 0 V
10 0 V
stroke 5222 5097 M
10 0 V
10 0 V
10 1 V
11 0 V
10 0 V
10 -1 V
10 0 V
10 0 V
10 0 V
10 0 V
10 0 V
10 0 V
10 0 V
10 0 V
10 0 V
10 0 V
10 0 V
10 0 V
10 0 V
10 0 V
10 0 V
10 0 V
10 0 V
10 0 V
10 0 V
10 0 V
10 0 V
10 0 V
10 0 V
10 0 V
10 0 V
10 0 V
10 0 V
10 0 V
10 0 V
10 0 V
11 0 V
10 0 V
10 0 V
10 0 V
10 1 V
10 0 V
10 0 V
10 0 V
10 -1 V
10 0 V
10 0 V
10 0 V
10 0 V
10 0 V
10 0 V
10 0 V
10 0 V
10 0 V
10 0 V
10 0 V
10 0 V
10 0 V
10 0 V
10 0 V
10 0 V
10 0 V
10 1 V
10 0 V
10 0 V
10 -1 V
10 0 V
10 0 V
10 0 V
11 0 V
10 0 V
10 0 V
10 0 V
10 0 V
10 0 V
10 0 V
10 1 V
10 0 V
10 0 V
10 0 V
10 0 V
10 0 V
10 0 V
10 0 V
10 0 V
10 0 V
10 0 V
10 0 V
10 0 V
10 0 V
10 0 V
10 0 V
10 0 V
10 0 V
10 1 V
10 0 V
10 0 V
10 0 V
10 0 V
10 0 V
10 -1 V
10 0 V
11 1 V
10 -1 V
stroke 6266 5098 M
10 1 V
10 -1 V
10 1 V
10 0 V
10 0 V
10 0 V
10 0 V
10 -1 V
10 0 V
10 0 V
10 0 V
10 0 V
10 0 V
10 0 V
10 0 V
10 0 V
10 0 V
10 0 V
10 0 V
10 0 V
10 0 V
10 0 V
10 0 V
10 0 V
10 0 V
10 0 V
10 0 V
10 0 V
10 0 V
10 0 V
10 0 V
10 0 V
11 0 V
10 0 V
10 0 V
10 0 V
10 0 V
10 0 V
10 0 V
10 0 V
10 0 V
10 0 V
10 0 V
10 0 V
10 0 V
10 0 V
10 0 V
10 1 V
10 0 V
10 0 V
10 -1 V
10 0 V
10 1 V
10 -1 V
10 1 V
10 0 V
10 0 V
10 0 V
10 0 V
10 0 V
10 0 V
10 0 V
10 0 V
10 0 V
10 0 V
11 0 V
10 0 V
10 0 V
10 0 V
10 0 V
10 0 V
10 0 V
10 0 V
10 0 V
10 0 V
10 0 V
10 0 V
10 0 V
10 0 V
10 0 V
10 0 V
10 0 V
10 0 V
10 0 V
10 0 V
10 0 V
10 0 V
10 0 V
10 0 V
10 0 V
10 0 V
10 0 V
10 0 V
10 0 V
10 0 V
10 0 V
10 0 V
10 0 V
11 0 V
10 0 V
10 0 V
10 0 V
10 0 V
10 1 V
stroke 7309 5100 M
10 0 V
10 0 V
10 0 V
10 0 V
10 0 V
10 0 V
1120 2164 M
-31 0 R
62 0 V
-62 0 R
62 0 V
-21 1207 R
-31 0 R
62 0 V
-62 0 R
62 0 V
-21 569 R
-31 0 R
62 0 V
-62 0 R
62 0 V
-21 432 R
-31 0 R
62 0 V
-62 0 R
62 0 V
-21 245 R
-31 0 R
62 0 V
-62 0 R
62 0 V
-21 116 R
-31 0 R
62 0 V
-62 0 R
62 0 V
-21 61 R
-31 0 R
62 0 V
-62 0 R
62 0 V
-21 145 R
-31 0 R
62 0 V
-62 0 R
62 0 V
-21 -3 R
-31 0 R
62 0 V
-62 0 R
62 0 V
-21 87 R
-31 0 R
62 0 V
-62 0 R
62 0 V
-21 29 R
-31 0 R
62 0 V
-62 0 R
62 0 V
-21 8 R
-31 0 R
62 0 V
-62 0 R
62 0 V
-21 -7 R
-31 0 R
62 0 V
-62 0 R
62 0 V
-21 61 R
-31 0 R
62 0 V
-62 0 R
62 0 V
-21 -12 R
-31 0 R
62 0 V
-62 0 R
62 0 V
-20 -132 R
-31 0 R
62 0 V
-62 0 R
62 0 V
-21 114 R
-31 0 R
62 0 V
-62 0 R
62 0 V
-21 -51 R
-31 0 R
62 0 V
-62 0 R
62 0 V
-21 48 R
-31 0 R
62 0 V
-62 0 R
62 0 V
-21 -17 R
-31 0 R
62 0 V
stroke 1342 5064 M
-62 0 R
62 0 V
-21 53 R
-31 0 R
62 0 V
-62 0 R
62 0 V
-21 -56 R
-31 0 R
62 0 V
-62 0 R
62 0 V
-21 47 R
-31 0 R
62 0 V
-62 0 R
62 0 V
-21 -35 R
-31 0 R
62 0 V
-62 0 R
62 0 V
-21 31 R
-31 0 R
62 0 V
-62 0 R
62 0 V
-21 -62 R
-31 0 R
62 0 V
-62 0 R
62 0 V
-21 50 R
-31 0 R
62 0 V
-62 0 R
62 0 V
-21 -9 R
-31 0 R
62 0 V
-62 0 R
62 0 V
-21 44 R
-31 0 R
62 0 V
-62 0 R
62 0 V
-21 -48 R
-31 0 R
62 0 V
-62 0 R
62 0 V
-21 50 R
-31 0 R
62 0 V
-62 0 R
62 0 V
-21 11 R
-31 0 R
62 0 V
-62 0 R
62 0 V
-21 -93 R
-31 0 R
62 0 V
-62 0 R
62 0 V
-21 99 R
-31 0 R
62 0 V
-62 0 R
62 0 V
-21 -87 R
-31 0 R
62 0 V
-62 0 R
62 0 V
-21 26 R
-31 0 R
62 0 V
-62 0 R
62 0 V
-21 26 R
-31 0 R
62 0 V
-62 0 R
62 0 V
-21 45 R
-31 0 R
62 0 V
-62 0 R
62 0 V
-21 -75 R
-31 0 R
62 0 V
-62 0 R
62 0 V
-21 18 R
-31 0 R
62 0 V
-62 0 R
62 0 V
-21 -19 R
-31 0 R
62 0 V
stroke 1552 5080 M
-62 0 R
62 0 V
-21 70 R
-31 0 R
62 0 V
-62 0 R
62 0 V
-21 -76 R
-31 0 R
62 0 V
-62 0 R
62 0 V
-21 -20 R
-31 0 R
62 0 V
-62 0 R
62 0 V
-21 99 R
-31 0 R
62 0 V
-62 0 R
62 0 V
-21 -102 R
-31 0 R
62 0 V
-62 0 R
62 0 V
-21 30 R
-31 0 R
62 0 V
-62 0 R
62 0 V
-21 41 R
-31 0 R
62 0 V
-62 0 R
62 0 V
-20 -4 R
-31 0 R
62 0 V
-62 0 R
62 0 V
-21 -29 R
-31 0 R
62 0 V
-62 0 R
62 0 V
-21 3 R
-31 0 R
62 0 V
-62 0 R
62 0 V
-21 -8 R
-31 0 R
62 0 V
-62 0 R
62 0 V
-21 70 R
-31 0 R
62 0 V
-62 0 R
62 0 V
-21 -36 R
-31 0 R
62 0 V
-62 0 R
62 0 V
-21 24 R
-31 0 R
62 0 V
-62 0 R
62 0 V
-21 -47 R
-31 0 R
62 0 V
-62 0 R
62 0 V
-21 4 R
-31 0 R
62 0 V
-62 0 R
62 0 V
-21 17 R
-31 0 R
62 0 V
-62 0 R
62 0 V
-21 4 R
-31 0 R
62 0 V
-62 0 R
62 0 V
-21 -31 R
-31 0 R
62 0 V
-62 0 R
62 0 V
-21 -13 R
-31 0 R
62 0 V
-62 0 R
62 0 V
-21 -47 R
-31 0 R
62 0 V
stroke 1763 5029 M
-62 0 R
62 0 V
-21 120 R
-31 0 R
62 0 V
-62 0 R
62 0 V
-21 -80 R
-31 0 R
62 0 V
-62 0 R
62 0 V
-21 58 R
-31 0 R
62 0 V
-62 0 R
62 0 V
-21 -23 R
-31 0 R
62 0 V
-62 0 R
62 0 V
-21 20 R
-31 0 R
62 0 V
-62 0 R
62 0 V
-21 3 R
-31 0 R
62 0 V
-62 0 R
62 0 V
-21 -29 R
-31 0 R
62 0 V
-62 0 R
62 0 V
-21 36 R
-31 0 R
62 0 V
-62 0 R
62 0 V
-21 -89 R
-31 0 R
62 0 V
-62 0 R
62 0 V
-21 7 R
-31 0 R
62 0 V
-62 0 R
62 0 V
-21 122 R
-31 0 R
62 0 V
-62 0 R
62 0 V
-21 -80 R
-31 0 R
62 0 V
-62 0 R
62 0 V
-21 58 R
-31 0 R
62 0 V
-62 0 R
62 0 V
-21 -18 R
-31 0 R
62 0 V
-62 0 R
62 0 V
-21 -7 R
-31 0 R
62 0 V
-62 0 R
62 0 V
-21 9 R
-31 0 R
62 0 V
-62 0 R
62 0 V
-21 -59 R
-31 0 R
62 0 V
-62 0 R
62 0 V
-21 88 R
-31 0 R
62 0 V
-62 0 R
62 0 V
-21 -106 R
-31 0 R
62 0 V
-62 0 R
62 0 V
-20 -2 R
-31 0 R
62 0 V
-62 0 R
62 0 V
-21 51 R
-31 0 R
62 0 V
stroke 1974 5108 M
-62 0 R
62 0 V
-21 -31 R
-31 0 R
62 0 V
-62 0 R
62 0 V
-21 91 R
-31 0 R
62 0 V
-62 0 R
62 0 V
-21 -18 R
-31 0 R
62 0 V
-62 0 R
62 0 V
-21 -31 R
-31 0 R
62 0 V
-62 0 R
62 0 V
-21 9 R
-31 0 R
62 0 V
-62 0 R
62 0 V
-21 -57 R
-31 0 R
62 0 V
-62 0 R
62 0 V
-21 39 R
-31 0 R
62 0 V
-62 0 R
62 0 V
-21 9 R
-31 0 R
62 0 V
-62 0 R
62 0 V
-21 75 R
-31 0 R
62 0 V
-62 0 R
62 0 V
-21 -61 R
-31 0 R
62 0 V
-62 0 R
62 0 V
-21 -25 R
-31 0 R
62 0 V
-62 0 R
62 0 V
-21 -11 R
-31 0 R
62 0 V
-62 0 R
62 0 V
-21 2 R
-31 0 R
62 0 V
-62 0 R
62 0 V
-21 68 R
-31 0 R
62 0 V
-62 0 R
62 0 V
-21 -91 R
-31 0 R
62 0 V
-62 0 R
62 0 V
-21 -3 R
-31 0 R
62 0 V
-62 0 R
62 0 V
-21 -1 R
-31 0 R
62 0 V
-62 0 R
62 0 V
-21 63 R
-31 0 R
62 0 V
-62 0 R
62 0 V
-21 -40 R
-31 0 R
62 0 V
-62 0 R
62 0 V
-21 23 R
-31 0 R
62 0 V
-62 0 R
62 0 V
-21 -16 R
-31 0 R
62 0 V
stroke 2184 5102 M
-62 0 R
62 0 V
-21 -29 R
-31 0 R
62 0 V
-62 0 R
62 0 V
-21 -20 R
-31 0 R
62 0 V
-62 0 R
62 0 V
-21 30 R
-31 0 R
62 0 V
-62 0 R
62 0 V
-21 25 R
-31 0 R
62 0 V
-62 0 R
62 0 V
-21 36 R
-31 0 R
62 0 V
-62 0 R
62 0 V
-21 -39 R
-31 0 R
62 0 V
-62 0 R
62 0 V
-21 8 R
-31 0 R
62 0 V
-62 0 R
62 0 V
-21 20 R
-31 0 R
62 0 V
-62 0 R
62 0 V
-21 -41 R
-31 0 R
62 0 V
-62 0 R
62 0 V
-21 -6 R
-31 0 R
62 0 V
-62 0 R
62 0 V
-21 19 R
-31 0 R
62 0 V
-62 0 R
62 0 V
-20 -35 R
-31 0 R
62 0 V
-62 0 R
62 0 V
-21 40 R
-31 0 R
62 0 V
-62 0 R
62 0 V
-21 -39 R
-31 0 R
62 0 V
-62 0 R
62 0 V
-21 -66 R
-31 0 R
62 0 V
-62 0 R
62 0 V
-21 49 R
-31 0 R
62 0 V
-62 0 R
62 0 V
-21 105 R
-31 0 R
62 0 V
-62 0 R
62 0 V
-21 -81 R
-31 0 R
62 0 V
-62 0 R
62 0 V
-21 19 R
-31 0 R
62 0 V
-62 0 R
62 0 V
-21 0 R
-31 0 R
62 0 V
-62 0 R
62 0 V
-21 31 R
-31 0 R
62 0 V
stroke 2395 5128 M
-62 0 R
62 0 V
-21 -52 R
0 43 V
-31 -43 R
62 0 V
-62 43 R
62 0 V
-21 -40 R
0 30 V
-31 -30 R
62 0 V
-62 30 R
62 0 V
-21 -17 R
0 33 V
-31 -33 R
62 0 V
-62 33 R
62 0 V
-21 -35 R
0 28 V
-31 -28 R
62 0 V
-62 28 R
62 0 V
-21 -30 R
0 24 V
-31 -24 R
62 0 V
-62 24 R
62 0 V
-21 -29 R
0 23 V
-31 -23 R
62 0 V
-62 23 R
62 0 V
-21 -24 R
0 21 V
-31 -21 R
62 0 V
-62 21 R
62 0 V
-21 -17 R
0 20 V
-31 -20 R
62 0 V
-62 20 R
62 0 V
-21 -19 R
0 18 V
-31 -18 R
62 0 V
-62 18 R
62 0 V
-21 -17 R
0 17 V
-31 -17 R
62 0 V
-62 17 R
62 0 V
-21 -17 R
0 15 V
-31 -15 R
62 0 V
-62 15 R
62 0 V
-21 -14 R
0 14 V
-31 -14 R
62 0 V
-62 14 R
62 0 V
-21 -11 R
0 14 V
-31 -14 R
62 0 V
-62 14 R
62 0 V
-21 -14 R
0 13 V
-31 -13 R
62 0 V
-62 13 R
62 0 V
-21 -15 R
0 13 V
-31 -13 R
62 0 V
-62 13 R
62 0 V
-21 -12 R
0 12 V
-31 -12 R
62 0 V
-62 12 R
62 0 V
-21 -11 R
0 11 V
-31 -11 R
62 0 V
-62 11 R
62 0 V
stroke 2565 5103 M
-21 -9 R
0 12 V
-31 -12 R
62 0 V
-62 12 R
62 0 V
-21 -15 R
0 12 V
-31 -12 R
62 0 V
-62 12 R
62 0 V
-21 -14 R
0 12 V
-31 -12 R
62 0 V
-62 12 R
62 0 V
-21 -13 R
0 12 V
-31 -12 R
62 0 V
-62 12 R
62 0 V
-21 -14 R
0 12 V
-31 -12 R
62 0 V
-62 12 R
62 0 V
-21 -12 R
0 12 V
-31 -12 R
62 0 V
-62 12 R
62 0 V
-20 -11 R
0 11 V
-31 -11 R
62 0 V
-62 11 R
62 0 V
-21 -11 R
0 11 V
-31 -11 R
62 0 V
-62 11 R
62 0 V
-21 -10 R
0 11 V
-31 -11 R
62 0 V
-62 11 R
62 0 V
-21 -10 R
0 11 V
-31 -11 R
62 0 V
-62 11 R
62 0 V
-21 -12 R
0 11 V
-31 -11 R
62 0 V
-62 11 R
62 0 V
-21 -11 R
0 10 V
-31 -10 R
62 0 V
-62 10 R
62 0 V
-21 -8 R
0 10 V
-31 -10 R
62 0 V
-62 10 R
62 0 V
-21 -13 R
0 11 V
-31 -11 R
62 0 V
-62 11 R
62 0 V
-21 -11 R
0 10 V
-31 -10 R
62 0 V
-62 10 R
62 0 V
-21 -11 R
0 10 V
-31 -10 R
62 0 V
-62 10 R
62 0 V
-21 -9 R
0 10 V
-31 -10 R
62 0 V
-62 10 R
62 0 V
-21 -9 R
0 10 V
stroke 2715 5098 M
-31 -10 R
62 0 V
-62 10 R
62 0 V
-21 -11 R
0 10 V
-31 -10 R
62 0 V
-62 10 R
62 0 V
-21 -10 R
0 9 V
-31 -9 R
62 0 V
-62 9 R
62 0 V
-21 -10 R
0 9 V
-31 -9 R
62 0 V
-62 9 R
62 0 V
-21 -10 R
0 9 V
-31 -9 R
62 0 V
-62 9 R
62 0 V
-21 -8 R
0 9 V
-31 -9 R
62 0 V
-62 9 R
62 0 V
-21 -8 R
0 9 V
-31 -9 R
62 0 V
-62 9 R
62 0 V
-21 -9 R
0 9 V
-31 -9 R
62 0 V
-62 9 R
62 0 V
-21 -8 R
0 9 V
-31 -9 R
62 0 V
-62 9 R
62 0 V
-21 -9 R
0 9 V
-31 -9 R
62 0 V
-62 9 R
62 0 V
-21 -9 R
0 9 V
-31 -9 R
62 0 V
-62 9 R
62 0 V
-21 -8 R
0 9 V
-31 -9 R
62 0 V
-62 9 R
62 0 V
-21 -9 R
0 8 V
-31 -8 R
62 0 V
-62 8 R
62 0 V
-21 -8 R
0 8 V
-31 -8 R
62 0 V
-62 8 R
62 0 V
-21 -7 R
0 8 V
-31 -8 R
62 0 V
-62 8 R
62 0 V
-21 -8 R
0 8 V
-31 -8 R
62 0 V
-62 8 R
62 0 V
-21 -8 R
0 8 V
-31 -8 R
62 0 V
-62 8 R
62 0 V
-21 -8 R
0 8 V
-31 -8 R
62 0 V
stroke 2916 5090 M
-62 8 R
62 0 V
-21 -8 R
0 8 V
-31 -8 R
62 0 V
-62 8 R
62 0 V
-21 -9 R
0 8 V
-31 -8 R
62 0 V
-62 8 R
62 0 V
-21 -7 R
0 7 V
-31 -7 R
62 0 V
-62 7 R
62 0 V
-21 -7 R
0 7 V
-31 -7 R
62 0 V
-62 7 R
62 0 V
-20 -7 R
0 7 V
-31 -7 R
62 0 V
-62 7 R
62 0 V
-21 -7 R
0 7 V
-31 -7 R
62 0 V
-62 7 R
62 0 V
-21 -7 R
0 8 V
-31 -8 R
62 0 V
-62 8 R
62 0 V
-21 -8 R
0 8 V
-31 -8 R
62 0 V
-62 8 R
62 0 V
-21 -8 R
0 7 V
-31 -7 R
62 0 V
-62 7 R
62 0 V
-21 -6 R
0 7 V
-31 -7 R
62 0 V
-62 7 R
62 0 V
-21 -8 R
0 7 V
-31 -7 R
62 0 V
-62 7 R
62 0 V
-21 -7 R
0 7 V
-31 -7 R
62 0 V
-62 7 R
62 0 V
-21 -7 R
0 7 V
-31 -7 R
62 0 V
-62 7 R
62 0 V
-21 -7 R
0 7 V
-31 -7 R
62 0 V
-62 7 R
62 0 V
-21 -7 R
0 6 V
-31 -6 R
62 0 V
-62 6 R
62 0 V
-21 -7 R
0 7 V
-31 -7 R
62 0 V
-62 7 R
62 0 V
-21 -7 R
0 7 V
-31 -7 R
62 0 V
-62 7 R
62 0 V
stroke 3087 5096 M
-21 -7 R
0 7 V
-31 -7 R
62 0 V
-62 7 R
62 0 V
-21 -6 R
0 7 V
-31 -7 R
62 0 V
-62 7 R
62 0 V
-21 -6 R
0 7 V
-31 -7 R
62 0 V
-62 7 R
62 0 V
-21 -7 R
0 7 V
-31 -7 R
62 0 V
-62 7 R
62 0 V
-21 -7 R
0 7 V
-31 -7 R
62 0 V
-62 7 R
62 0 V
-21 -7 R
0 6 V
-31 -6 R
62 0 V
-62 6 R
62 0 V
-21 -6 R
0 6 V
-31 -6 R
62 0 V
-62 6 R
62 0 V
-21 -6 R
0 6 V
-31 -6 R
62 0 V
-62 6 R
62 0 V
-21 -7 R
0 7 V
-31 -7 R
62 0 V
-62 7 R
62 0 V
-21 -6 R
0 7 V
-31 -7 R
62 0 V
-62 7 R
62 0 V
-21 -7 R
0 6 V
-31 -6 R
62 0 V
-62 6 R
62 0 V
-21 -6 R
0 6 V
-31 -6 R
62 0 V
-62 6 R
62 0 V
-21 -7 R
0 7 V
-31 -7 R
62 0 V
-62 7 R
62 0 V
-21 -7 R
0 6 V
-31 -6 R
62 0 V
-62 6 R
62 0 V
-21 -6 R
0 6 V
-31 -6 R
62 0 V
-62 6 R
62 0 V
-21 -6 R
0 6 V
-31 -6 R
62 0 V
-62 6 R
62 0 V
-21 -7 R
0 6 V
-31 -6 R
62 0 V
-62 6 R
62 0 V
-21 -6 R
0 6 V
stroke 3236 5095 M
-31 -6 R
62 0 V
-62 6 R
62 0 V
-21 -7 R
0 7 V
-31 -7 R
62 0 V
-62 7 R
62 0 V
-21 -7 R
0 7 V
-31 -7 R
62 0 V
-62 7 R
62 0 V
-20 -7 R
0 6 V
-31 -6 R
62 0 V
-62 6 R
62 0 V
-21 -6 R
0 7 V
-31 -7 R
62 0 V
-62 7 R
62 0 V
-21 -7 R
0 6 V
-31 -6 R
62 0 V
-62 6 R
62 0 V
-21 -5 R
0 6 V
-31 -6 R
62 0 V
-62 6 R
62 0 V
-21 -6 R
0 6 V
-31 -6 R
62 0 V
-62 6 R
62 0 V
-21 -6 R
0 6 V
-31 -6 R
62 0 V
-62 6 R
62 0 V
-21 -6 R
0 6 V
-31 -6 R
62 0 V
-62 6 R
62 0 V
-21 -6 R
0 6 V
-31 -6 R
62 0 V
-62 6 R
62 0 V
-21 -6 R
0 5 V
-31 -5 R
62 0 V
-62 5 R
62 0 V
-21 -6 R
0 6 V
-31 -6 R
62 0 V
-62 6 R
62 0 V
-21 -6 R
0 6 V
-31 -6 R
62 0 V
-62 6 R
62 0 V
-21 -5 R
0 6 V
-31 -6 R
62 0 V
-62 6 R
62 0 V
-21 -6 R
0 6 V
-31 -6 R
62 0 V
-62 6 R
62 0 V
-21 -6 R
0 6 V
-31 -6 R
62 0 V
-62 6 R
62 0 V
-21 -6 R
0 5 V
-31 -5 R
62 0 V
stroke 3438 5089 M
-62 5 R
62 0 V
-21 -5 R
0 6 V
-31 -6 R
62 0 V
-62 6 R
62 0 V
-21 -6 R
0 5 V
-31 -5 R
62 0 V
-62 5 R
62 0 V
-21 -6 R
0 6 V
-31 -6 R
62 0 V
-62 6 R
62 0 V
-21 -5 R
0 6 V
-31 -6 R
62 0 V
-62 6 R
62 0 V
-21 -6 R
0 6 V
-31 -6 R
62 0 V
-62 6 R
62 0 V
-21 -6 R
0 6 V
-31 -6 R
62 0 V
-62 6 R
62 0 V
-21 -6 R
0 6 V
-31 -6 R
62 0 V
-62 6 R
62 0 V
-21 -6 R
0 6 V
-31 -6 R
62 0 V
-62 6 R
62 0 V
-21 -5 R
0 5 V
-31 -5 R
62 0 V
-62 5 R
62 0 V
-21 -5 R
0 5 V
-31 -5 R
62 0 V
-62 5 R
62 0 V
-21 -5 R
0 6 V
-31 -6 R
62 0 V
-62 6 R
62 0 V
-21 -6 R
0 6 V
-31 -6 R
62 0 V
-62 6 R
62 0 V
-21 -5 R
0 5 V
-31 -5 R
62 0 V
-62 5 R
62 0 V
-21 -6 R
0 5 V
-31 -5 R
62 0 V
-62 5 R
62 0 V
-21 -5 R
0 6 V
-31 -6 R
62 0 V
-62 6 R
62 0 V
-21 -5 R
0 5 V
-31 -5 R
62 0 V
-62 5 R
62 0 V
-21 -5 R
0 6 V
-31 -6 R
62 0 V
-62 6 R
62 0 V
stroke 3608 5097 M
-21 -6 R
0 6 V
-31 -6 R
62 0 V
-62 6 R
62 0 V
-20 -6 R
0 5 V
-31 -5 R
62 0 V
-62 5 R
62 0 V
-21 -5 R
0 5 V
-31 -5 R
62 0 V
-62 5 R
62 0 V
-21 -6 R
0 6 V
-31 -6 R
62 0 V
-62 6 R
62 0 V
-21 -5 R
0 5 V
-31 -5 R
62 0 V
-62 5 R
62 0 V
-21 -6 R
0 6 V
-31 -6 R
62 0 V
-62 6 R
62 0 V
-21 -5 R
0 5 V
-31 -5 R
62 0 V
-62 5 R
62 0 V
-21 -5 R
0 5 V
-31 -5 R
62 0 V
-62 5 R
62 0 V
-21 -5 R
0 5 V
-31 -5 R
62 0 V
-62 5 R
62 0 V
-21 -5 R
0 5 V
-31 -5 R
62 0 V
-62 5 R
62 0 V
-21 -5 R
0 5 V
-31 -5 R
62 0 V
-62 5 R
62 0 V
-21 -5 R
0 6 V
-31 -6 R
62 0 V
-62 6 R
62 0 V
-21 -5 R
0 5 V
-31 -5 R
62 0 V
-62 5 R
62 0 V
-21 -5 R
0 5 V
-31 -5 R
62 0 V
-62 5 R
62 0 V
-21 -5 R
0 5 V
-31 -5 R
62 0 V
-62 5 R
62 0 V
-21 -5 R
0 5 V
-31 -5 R
62 0 V
-62 5 R
62 0 V
-21 -5 R
0 5 V
-31 -5 R
62 0 V
-62 5 R
62 0 V
-21 -5 R
0 5 V
stroke 3758 5097 M
-31 -5 R
62 0 V
-62 5 R
62 0 V
-21 -6 R
0 6 V
-31 -6 R
62 0 V
-62 6 R
62 0 V
-21 -5 R
0 5 V
-31 -5 R
62 0 V
-62 5 R
62 0 V
-21 -6 R
0 6 V
-31 -6 R
62 0 V
-62 6 R
62 0 V
-21 -6 R
0 6 V
-31 -6 R
62 0 V
-62 6 R
62 0 V
-21 -6 R
0 6 V
-31 -6 R
62 0 V
-62 6 R
62 0 V
-21 -6 R
0 6 V
-31 -6 R
62 0 V
-62 6 R
62 0 V
-21 -5 R
0 5 V
-31 -5 R
62 0 V
-62 5 R
62 0 V
-21 -5 R
0 5 V
-31 -5 R
62 0 V
-62 5 R
62 0 V
-21 -5 R
0 5 V
-31 -5 R
62 0 V
-62 5 R
62 0 V
-21 -5 R
0 5 V
-31 -5 R
62 0 V
-62 5 R
62 0 V
-21 -5 R
0 5 V
-31 -5 R
62 0 V
-62 5 R
62 0 V
-21 -5 R
0 5 V
-31 -5 R
62 0 V
-62 5 R
62 0 V
-21 -5 R
0 5 V
-31 -5 R
62 0 V
-62 5 R
62 0 V
-21 -4 R
0 5 V
-31 -5 R
62 0 V
-62 5 R
62 0 V
-21 -5 R
0 5 V
-31 -5 R
62 0 V
-62 5 R
62 0 V
-21 -6 R
0 5 V
-31 -5 R
62 0 V
-62 5 R
62 0 V
-20 -5 R
0 5 V
-31 -5 R
62 0 V
stroke 3960 5092 M
-62 5 R
62 0 V
-21 -5 R
0 5 V
-31 -5 R
62 0 V
-62 5 R
62 0 V
-21 -5 R
0 5 V
-31 -5 R
62 0 V
-62 5 R
62 0 V
-21 -4 R
0 5 V
-31 -5 R
62 0 V
-62 5 R
62 0 V
-21 -6 R
0 5 V
-31 -5 R
62 0 V
-62 5 R
62 0 V
-21 -5 R
0 5 V
-31 -5 R
62 0 V
-62 5 R
62 0 V
-21 -4 R
0 5 V
-31 -5 R
62 0 V
-62 5 R
62 0 V
-21 -6 R
0 6 V
-31 -6 R
62 0 V
-62 6 R
62 0 V
-21 -6 R
0 6 V
-31 -6 R
62 0 V
-62 6 R
62 0 V
-21 -5 R
0 5 V
-31 -5 R
62 0 V
-62 5 R
62 0 V
-21 -5 R
0 5 V
-31 -5 R
62 0 V
-62 5 R
62 0 V
-21 -5 R
0 5 V
-31 -5 R
62 0 V
-62 5 R
62 0 V
-21 -5 R
0 5 V
-31 -5 R
62 0 V
-62 5 R
62 0 V
-21 -5 R
0 6 V
-31 -6 R
62 0 V
-62 6 R
62 0 V
-21 -5 R
0 5 V
-31 -5 R
62 0 V
-62 5 R
62 0 V
-21 -6 R
0 6 V
-31 -6 R
62 0 V
-62 6 R
62 0 V
-21 -6 R
0 6 V
-31 -6 R
62 0 V
-62 6 R
62 0 V
-21 -5 R
0 5 V
-31 -5 R
62 0 V
-62 5 R
62 0 V
stroke 4130 5099 M
-21 -5 R
0 5 V
-31 -5 R
62 0 V
-62 5 R
62 0 V
-21 -5 R
0 5 V
-31 -5 R
62 0 V
-62 5 R
62 0 V
-21 -5 R
0 5 V
-31 -5 R
62 0 V
-62 5 R
62 0 V
-21 -6 R
0 6 V
-31 -6 R
62 0 V
-62 6 R
62 0 V
-21 -6 R
0 6 V
-31 -6 R
62 0 V
-62 6 R
62 0 V
-21 -6 R
0 6 V
-31 -6 R
62 0 V
-62 6 R
62 0 V
-21 -6 R
0 5 V
-31 -5 R
62 0 V
-62 5 R
62 0 V
-21 -5 R
0 5 V
-31 -5 R
62 0 V
-62 5 R
62 0 V
-21 -5 R
0 5 V
-31 -5 R
62 0 V
-62 5 R
62 0 V
-21 -5 R
0 5 V
-31 -5 R
62 0 V
-62 5 R
62 0 V
-21 -5 R
0 5 V
-31 -5 R
62 0 V
-62 5 R
62 0 V
-21 -5 R
0 5 V
-31 -5 R
62 0 V
-62 5 R
62 0 V
-21 -5 R
0 5 V
-31 -5 R
62 0 V
-62 5 R
62 0 V
-21 -5 R
0 5 V
-31 -5 R
62 0 V
-62 5 R
62 0 V
-21 -5 R
0 5 V
-31 -5 R
62 0 V
-62 5 R
62 0 V
-20 -5 R
0 5 V
-31 -5 R
62 0 V
-62 5 R
62 0 V
-21 -5 R
0 5 V
-31 -5 R
62 0 V
-62 5 R
62 0 V
-21 -5 R
0 5 V
stroke 4280 5098 M
-31 -5 R
62 0 V
-62 5 R
62 0 V
-21 -5 R
0 5 V
-31 -5 R
62 0 V
-62 5 R
62 0 V
-21 -5 R
0 5 V
-31 -5 R
62 0 V
-62 5 R
62 0 V
-21 -5 R
0 5 V
-31 -5 R
62 0 V
-62 5 R
62 0 V
-21 -5 R
0 5 V
-31 -5 R
62 0 V
-62 5 R
62 0 V
-21 -5 R
0 5 V
-31 -5 R
62 0 V
-62 5 R
62 0 V
-21 -6 R
0 5 V
-31 -5 R
62 0 V
-62 5 R
62 0 V
-21 -4 R
0 5 V
-31 -5 R
62 0 V
-62 5 R
62 0 V
-21 -5 R
0 5 V
-31 -5 R
62 0 V
-62 5 R
62 0 V
-21 -5 R
0 4 V
-31 -4 R
62 0 V
-62 4 R
62 0 V
-21 -5 R
0 5 V
-31 -5 R
62 0 V
-62 5 R
62 0 V
-21 -5 R
0 5 V
-31 -5 R
62 0 V
-62 5 R
62 0 V
-21 -5 R
0 5 V
-31 -5 R
62 0 V
-62 5 R
62 0 V
-21 -5 R
0 5 V
-31 -5 R
62 0 V
-62 5 R
62 0 V
-21 -4 R
0 5 V
-31 -5 R
62 0 V
-62 5 R
62 0 V
-21 -6 R
0 5 V
-31 -5 R
62 0 V
-62 5 R
62 0 V
-21 -5 R
0 5 V
-31 -5 R
62 0 V
-62 5 R
62 0 V
-21 -5 R
0 5 V
-31 -5 R
62 0 V
stroke 4481 5092 M
-62 5 R
62 0 V
-21 -5 R
0 5 V
-31 -5 R
62 0 V
-62 5 R
62 0 V
-21 -5 R
0 5 V
-31 -5 R
62 0 V
-62 5 R
62 0 V
-21 -5 R
0 5 V
-31 -5 R
62 0 V
-62 5 R
62 0 V
-21 -5 R
0 5 V
-31 -5 R
62 0 V
-62 5 R
62 0 V
-21 -5 R
0 5 V
-31 -5 R
62 0 V
-62 5 R
62 0 V
-21 -5 R
0 5 V
-31 -5 R
62 0 V
-62 5 R
62 0 V
-21 -5 R
0 5 V
-31 -5 R
62 0 V
-62 5 R
62 0 V
-21 -5 R
0 5 V
-31 -5 R
62 0 V
-62 5 R
62 0 V
-21 -5 R
0 5 V
-31 -5 R
62 0 V
-62 5 R
62 0 V
-21 -5 R
0 5 V
-31 -5 R
62 0 V
-62 5 R
62 0 V
-21 -5 R
0 5 V
-31 -5 R
62 0 V
-62 5 R
62 0 V
-21 -5 R
0 5 V
-31 -5 R
62 0 V
-62 5 R
62 0 V
-21 -5 R
0 5 V
-31 -5 R
62 0 V
-62 5 R
62 0 V
-21 -5 R
0 5 V
-31 -5 R
62 0 V
-62 5 R
62 0 V
-20 -5 R
0 5 V
-31 -5 R
62 0 V
-62 5 R
62 0 V
-21 -5 R
0 5 V
-31 -5 R
62 0 V
-62 5 R
62 0 V
-21 -5 R
0 5 V
-31 -5 R
62 0 V
-62 5 R
62 0 V
stroke 4652 5097 M
-21 -4 R
0 4 V
-31 -4 R
62 0 V
-62 4 R
62 0 V
-21 -4 R
0 4 V
-31 -4 R
62 0 V
-62 4 R
62 0 V
-21 -4 R
0 4 V
-31 -4 R
62 0 V
-62 4 R
62 0 V
-21 -4 R
0 4 V
-31 -4 R
62 0 V
-62 4 R
62 0 V
-21 -4 R
0 4 V
-31 -4 R
62 0 V
-62 4 R
62 0 V
-21 -4 R
0 5 V
-31 -5 R
62 0 V
-62 5 R
62 0 V
-21 -5 R
0 5 V
-31 -5 R
62 0 V
-62 5 R
62 0 V
-21 -4 R
0 4 V
-31 -4 R
62 0 V
-62 4 R
62 0 V
-21 -4 R
0 5 V
-31 -5 R
62 0 V
-62 5 R
62 0 V
-21 -5 R
0 5 V
-31 -5 R
62 0 V
-62 5 R
62 0 V
-21 -5 R
0 5 V
-31 -5 R
62 0 V
-62 5 R
62 0 V
-21 -5 R
0 5 V
-31 -5 R
62 0 V
-62 5 R
62 0 V
-21 -5 R
0 5 V
-31 -5 R
62 0 V
-62 5 R
62 0 V
-21 -4 R
0 4 V
-31 -4 R
62 0 V
-62 4 R
62 0 V
-21 -4 R
0 4 V
-31 -4 R
62 0 V
-62 4 R
62 0 V
-21 -4 R
0 4 V
-31 -4 R
62 0 V
-62 4 R
62 0 V
-21 -4 R
0 4 V
-31 -4 R
62 0 V
-62 4 R
62 0 V
-21 -5 R
0 5 V
stroke 4801 5099 M
-31 -5 R
62 0 V
-62 5 R
62 0 V
-21 -4 R
0 4 V
-31 -4 R
62 0 V
-62 4 R
62 0 V
-21 -4 R
0 4 V
-31 -4 R
62 0 V
-62 4 R
62 0 V
-21 -5 R
0 5 V
-31 -5 R
62 0 V
-62 5 R
62 0 V
-21 -5 R
0 5 V
-31 -5 R
62 0 V
-62 5 R
62 0 V
-21 -5 R
0 5 V
-31 -5 R
62 0 V
-62 5 R
62 0 V
-21 -5 R
0 5 V
-31 -5 R
62 0 V
-62 5 R
62 0 V
-21 -5 R
0 5 V
-31 -5 R
62 0 V
-62 5 R
62 0 V
-21 -4 R
0 4 V
-31 -4 R
62 0 V
-62 4 R
62 0 V
-21 -4 R
0 4 V
-31 -4 R
62 0 V
-62 4 R
62 0 V
-21 -4 R
0 4 V
-31 -4 R
62 0 V
-62 4 R
62 0 V
-21 -4 R
0 4 V
-31 -4 R
62 0 V
-62 4 R
62 0 V
-21 -4 R
0 4 V
-31 -4 R
62 0 V
-62 4 R
62 0 V
-20 -4 R
0 4 V
-31 -4 R
62 0 V
-62 4 R
62 0 V
-21 -4 R
0 4 V
-31 -4 R
62 0 V
-62 4 R
62 0 V
-21 -4 R
0 4 V
-31 -4 R
62 0 V
-62 4 R
62 0 V
-21 -5 R
0 5 V
-31 -5 R
62 0 V
-62 5 R
62 0 V
-21 -5 R
0 5 V
-31 -5 R
62 0 V
stroke 5003 5094 M
-62 5 R
62 0 V
-21 -4 R
0 4 V
-31 -4 R
62 0 V
-62 4 R
62 0 V
-21 -4 R
0 4 V
-31 -4 R
62 0 V
-62 4 R
62 0 V
-21 -4 R
0 4 V
-31 -4 R
62 0 V
-62 4 R
62 0 V
-21 -4 R
0 4 V
-31 -4 R
62 0 V
-62 4 R
62 0 V
-21 -4 R
0 4 V
-31 -4 R
62 0 V
-62 4 R
62 0 V
-21 -5 R
0 5 V
-31 -5 R
62 0 V
-62 5 R
62 0 V
-21 -5 R
0 5 V
-31 -5 R
62 0 V
-62 5 R
62 0 V
-21 -5 R
0 5 V
-31 -5 R
62 0 V
-62 5 R
62 0 V
-21 -5 R
0 5 V
-31 -5 R
62 0 V
-62 5 R
62 0 V
-21 -5 R
0 5 V
-31 -5 R
62 0 V
-62 5 R
62 0 V
-21 -5 R
0 5 V
-31 -5 R
62 0 V
-62 5 R
62 0 V
-21 -4 R
0 4 V
-31 -4 R
62 0 V
-62 4 R
62 0 V
-21 -4 R
0 4 V
-31 -4 R
62 0 V
-62 4 R
62 0 V
-21 -4 R
0 4 V
-31 -4 R
62 0 V
-62 4 R
62 0 V
-21 -4 R
0 5 V
-31 -5 R
62 0 V
-62 5 R
62 0 V
-21 -5 R
0 5 V
-31 -5 R
62 0 V
-62 5 R
62 0 V
-21 -5 R
0 5 V
-31 -5 R
62 0 V
-62 5 R
62 0 V
stroke 5173 5100 M
-21 -5 R
0 5 V
-31 -5 R
62 0 V
-62 5 R
62 0 V
-21 -5 R
0 5 V
-31 -5 R
62 0 V
-62 5 R
62 0 V
-21 -5 R
0 5 V
-31 -5 R
62 0 V
-62 5 R
62 0 V
-21 -5 R
0 5 V
-31 -5 R
62 0 V
-62 5 R
62 0 V
-21 -5 R
0 5 V
-31 -5 R
62 0 V
-62 5 R
62 0 V
-21 -5 R
0 5 V
-31 -5 R
62 0 V
-62 5 R
62 0 V
-21 -5 R
0 5 V
-31 -5 R
62 0 V
-62 5 R
62 0 V
-21 -5 R
0 5 V
-31 -5 R
62 0 V
-62 5 R
62 0 V
-21 -5 R
0 4 V
-31 -4 R
62 0 V
-62 4 R
62 0 V
-21 -4 R
0 5 V
-31 -5 R
62 0 V
-62 5 R
62 0 V
-21 -5 R
0 5 V
-31 -5 R
62 0 V
-62 5 R
62 0 V
-20 -5 R
0 5 V
-31 -5 R
62 0 V
-62 5 R
62 0 V
-21 -5 R
0 5 V
-31 -5 R
62 0 V
-62 5 R
62 0 V
-21 -5 R
0 5 V
-31 -5 R
62 0 V
-62 5 R
62 0 V
-21 -5 R
0 5 V
-31 -5 R
62 0 V
-62 5 R
62 0 V
-21 -5 R
0 4 V
-31 -4 R
62 0 V
-62 4 R
62 0 V
-21 -4 R
0 4 V
-31 -4 R
62 0 V
-62 4 R
62 0 V
-21 -4 R
0 4 V
stroke 5323 5099 M
-31 -4 R
62 0 V
-62 4 R
62 0 V
-21 -4 R
0 4 V
-31 -4 R
62 0 V
-62 4 R
62 0 V
-21 -4 R
0 5 V
-31 -5 R
62 0 V
-62 5 R
62 0 V
-21 -5 R
0 4 V
-31 -4 R
62 0 V
-62 4 R
62 0 V
-21 -4 R
0 4 V
-31 -4 R
62 0 V
-62 4 R
62 0 V
-21 -4 R
0 4 V
-31 -4 R
62 0 V
-62 4 R
62 0 V
-21 -4 R
0 4 V
-31 -4 R
62 0 V
-62 4 R
62 0 V
-21 -4 R
0 4 V
-31 -4 R
62 0 V
-62 4 R
62 0 V
-21 -4 R
0 5 V
-31 -5 R
62 0 V
-62 5 R
62 0 V
-21 -5 R
0 4 V
-31 -4 R
62 0 V
-62 4 R
62 0 V
-21 -4 R
0 4 V
-31 -4 R
62 0 V
-62 4 R
62 0 V
-21 -4 R
0 4 V
-31 -4 R
62 0 V
-62 4 R
62 0 V
-21 -4 R
0 4 V
-31 -4 R
62 0 V
-62 4 R
62 0 V
-21 -4 R
0 4 V
-31 -4 R
62 0 V
-62 4 R
62 0 V
-21 -4 R
0 4 V
-31 -4 R
62 0 V
-62 4 R
62 0 V
-21 -4 R
0 4 V
-31 -4 R
62 0 V
-62 4 R
62 0 V
-21 -4 R
0 4 V
-31 -4 R
62 0 V
-62 4 R
62 0 V
-21 -4 R
0 4 V
-31 -4 R
62 0 V
stroke 5524 5095 M
-62 4 R
62 0 V
-21 -4 R
0 4 V
-31 -4 R
62 0 V
-62 4 R
62 0 V
-21 -4 R
0 4 V
-31 -4 R
62 0 V
-62 4 R
62 0 V
-21 -4 R
0 4 V
-31 -4 R
62 0 V
-62 4 R
62 0 V
-21 -4 R
0 4 V
-31 -4 R
62 0 V
-62 4 R
62 0 V
-21 -4 R
0 4 V
-31 -4 R
62 0 V
-62 4 R
62 0 V
-21 -4 R
0 4 V
-31 -4 R
62 0 V
-62 4 R
62 0 V
-21 -4 R
0 4 V
-31 -4 R
62 0 V
-62 4 R
62 0 V
-21 -4 R
0 4 V
-31 -4 R
62 0 V
-62 4 R
62 0 V
-21 -4 R
0 4 V
-31 -4 R
62 0 V
-62 4 R
62 0 V
-20 -4 R
0 4 V
-31 -4 R
62 0 V
-62 4 R
62 0 V
-21 -4 R
0 4 V
-31 -4 R
62 0 V
-62 4 R
62 0 V
-21 -4 R
0 4 V
-31 -4 R
62 0 V
-62 4 R
62 0 V
-21 -4 R
0 4 V
-31 -4 R
62 0 V
-62 4 R
62 0 V
-21 -3 R
0 4 V
-31 -4 R
62 0 V
-62 4 R
62 0 V
-21 -4 R
0 4 V
-31 -4 R
62 0 V
-62 4 R
62 0 V
-21 -4 R
0 4 V
-31 -4 R
62 0 V
-62 4 R
62 0 V
-21 -4 R
0 4 V
-31 -4 R
62 0 V
-62 4 R
62 0 V
stroke 5695 5100 M
-21 -5 R
0 4 V
-31 -4 R
62 0 V
-62 4 R
62 0 V
-21 -4 R
0 4 V
-31 -4 R
62 0 V
-62 4 R
62 0 V
-21 -4 R
0 4 V
-31 -4 R
62 0 V
-62 4 R
62 0 V
-21 -4 R
0 4 V
-31 -4 R
62 0 V
-62 4 R
62 0 V
-21 -4 R
0 4 V
-31 -4 R
62 0 V
-62 4 R
62 0 V
-21 -4 R
0 4 V
-31 -4 R
62 0 V
-62 4 R
62 0 V
-21 -4 R
0 4 V
-31 -4 R
62 0 V
-62 4 R
62 0 V
-21 -4 R
0 4 V
-31 -4 R
62 0 V
-62 4 R
62 0 V
-21 -4 R
0 4 V
-31 -4 R
62 0 V
-62 4 R
62 0 V
-21 -4 R
0 4 V
-31 -4 R
62 0 V
-62 4 R
62 0 V
-21 -4 R
0 4 V
-31 -4 R
62 0 V
-62 4 R
62 0 V
-21 -4 R
0 4 V
-31 -4 R
62 0 V
-62 4 R
62 0 V
-21 -4 R
0 4 V
-31 -4 R
62 0 V
-62 4 R
62 0 V
-21 -4 R
0 4 V
-31 -4 R
62 0 V
-62 4 R
62 0 V
-21 -4 R
0 4 V
-31 -4 R
62 0 V
-62 4 R
62 0 V
-21 -4 R
0 4 V
-31 -4 R
62 0 V
-62 4 R
62 0 V
-21 -4 R
0 4 V
-31 -4 R
62 0 V
-62 4 R
62 0 V
-21 -4 R
0 4 V
stroke 5844 5099 M
-31 -4 R
62 0 V
-62 4 R
62 0 V
-21 -3 R
0 3 V
-31 -3 R
62 0 V
-62 3 R
62 0 V
-21 -3 R
0 3 V
-31 -3 R
62 0 V
-62 3 R
62 0 V
-21 -3 R
0 3 V
-31 -3 R
62 0 V
-62 3 R
62 0 V
-21 -3 R
0 3 V
-31 -3 R
62 0 V
-62 3 R
62 0 V
-21 -3 R
0 3 V
-31 -3 R
62 0 V
-62 3 R
62 0 V
-21 -4 R
0 4 V
-31 -4 R
62 0 V
-62 4 R
62 0 V
-21 -4 R
0 4 V
-31 -4 R
62 0 V
-62 4 R
62 0 V
-20 -4 R
0 4 V
-31 -4 R
62 0 V
-62 4 R
62 0 V
-21 -3 R
0 3 V
-31 -3 R
62 0 V
-62 3 R
62 0 V
-21 -3 R
0 3 V
-31 -3 R
62 0 V
-62 3 R
62 0 V
-21 -3 R
0 3 V
-31 -3 R
62 0 V
-62 3 R
62 0 V
-21 -3 R
0 3 V
-31 -3 R
62 0 V
-62 3 R
62 0 V
-21 -3 R
0 3 V
-31 -3 R
62 0 V
-62 3 R
62 0 V
-21 -3 R
0 3 V
-31 -3 R
62 0 V
-62 3 R
62 0 V
-21 -3 R
0 3 V
-31 -3 R
62 0 V
-62 3 R
62 0 V
-21 -3 R
0 3 V
-31 -3 R
62 0 V
-62 3 R
62 0 V
-21 -3 R
0 4 V
-31 -4 R
62 0 V
stroke 6046 5096 M
-62 4 R
62 0 V
-21 -4 R
0 4 V
-31 -4 R
62 0 V
-62 4 R
62 0 V
-21 -4 R
0 4 V
-31 -4 R
62 0 V
-62 4 R
62 0 V
-21 -4 R
0 4 V
-31 -4 R
62 0 V
-62 4 R
62 0 V
-21 -4 R
0 4 V
-31 -4 R
62 0 V
-62 4 R
62 0 V
-21 -4 R
0 4 V
-31 -4 R
62 0 V
-62 4 R
62 0 V
-21 -4 R
0 4 V
-31 -4 R
62 0 V
-62 4 R
62 0 V
-21 -4 R
0 4 V
-31 -4 R
62 0 V
-62 4 R
62 0 V
-21 -4 R
0 4 V
-31 -4 R
62 0 V
-62 4 R
62 0 V
-21 -4 R
0 4 V
-31 -4 R
62 0 V
-62 4 R
62 0 V
-21 -4 R
0 4 V
-31 -4 R
62 0 V
-62 4 R
62 0 V
-21 -4 R
0 4 V
-31 -4 R
62 0 V
-62 4 R
62 0 V
-21 -4 R
0 4 V
-31 -4 R
62 0 V
-62 4 R
62 0 V
-21 -4 R
0 4 V
-31 -4 R
62 0 V
-62 4 R
62 0 V
-21 -3 R
0 3 V
-31 -3 R
62 0 V
-62 3 R
62 0 V
-21 -3 R
0 3 V
-31 -3 R
62 0 V
-62 3 R
62 0 V
-21 -3 R
0 3 V
-31 -3 R
62 0 V
-62 3 R
62 0 V
-21 -3 R
0 3 V
-31 -3 R
62 0 V
-62 3 R
62 0 V
stroke 6216 5100 M
-21 -3 R
0 3 V
-31 -3 R
62 0 V
-62 3 R
62 0 V
-21 -3 R
0 3 V
-31 -3 R
62 0 V
-62 3 R
62 0 V
-21 -3 R
0 3 V
-31 -3 R
62 0 V
-62 3 R
62 0 V
-21 -3 R
0 3 V
-31 -3 R
62 0 V
-62 3 R
62 0 V
-21 -3 R
0 3 V
-31 -3 R
62 0 V
-62 3 R
62 0 V
-21 -3 R
0 3 V
-31 -3 R
62 0 V
-62 3 R
62 0 V
-20 -3 R
0 3 V
-31 -3 R
62 0 V
-62 3 R
62 0 V
-21 -3 R
0 3 V
-31 -3 R
62 0 V
-62 3 R
62 0 V
-21 -3 R
0 3 V
-31 -3 R
62 0 V
-62 3 R
62 0 V
-21 -3 R
0 3 V
-31 -3 R
62 0 V
-62 3 R
62 0 V
-21 -3 R
0 3 V
-31 -3 R
62 0 V
-62 3 R
62 0 V
-21 -3 R
0 3 V
-31 -3 R
62 0 V
-62 3 R
62 0 V
-21 -3 R
0 3 V
-31 -3 R
62 0 V
-62 3 R
62 0 V
-21 -3 R
0 3 V
-31 -3 R
62 0 V
-62 3 R
62 0 V
-21 -3 R
0 3 V
-31 -3 R
62 0 V
-62 3 R
62 0 V
-21 -3 R
0 3 V
-31 -3 R
62 0 V
-62 3 R
62 0 V
-21 -3 R
0 3 V
-31 -3 R
62 0 V
-62 3 R
62 0 V
-21 -4 R
0 4 V
stroke 6366 5100 M
-31 -4 R
62 0 V
-62 4 R
62 0 V
-21 -4 R
0 4 V
-31 -4 R
62 0 V
-62 4 R
62 0 V
-21 -4 R
0 4 V
-31 -4 R
62 0 V
-62 4 R
62 0 V
-21 -3 R
0 3 V
-31 -3 R
62 0 V
-62 3 R
62 0 V
-21 -3 R
0 3 V
-31 -3 R
62 0 V
-62 3 R
62 0 V
-21 -4 R
0 4 V
-31 -4 R
62 0 V
-62 4 R
62 0 V
-21 -4 R
0 4 V
-31 -4 R
62 0 V
-62 4 R
62 0 V
-21 -4 R
0 4 V
-31 -4 R
62 0 V
-62 4 R
62 0 V
-21 -4 R
0 4 V
-31 -4 R
62 0 V
-62 4 R
62 0 V
-21 -4 R
0 4 V
-31 -4 R
62 0 V
-62 4 R
62 0 V
-21 -4 R
0 4 V
-31 -4 R
62 0 V
-62 4 R
62 0 V
-21 -4 R
0 4 V
-31 -4 R
62 0 V
-62 4 R
62 0 V
-21 -4 R
0 4 V
-31 -4 R
62 0 V
-62 4 R
62 0 V
-21 -4 R
0 4 V
-31 -4 R
62 0 V
-62 4 R
62 0 V
-21 -4 R
0 4 V
-31 -4 R
62 0 V
-62 4 R
62 0 V
-21 -4 R
0 4 V
-31 -4 R
62 0 V
-62 4 R
62 0 V
-21 -3 R
0 3 V
-31 -3 R
62 0 V
-62 3 R
62 0 V
-21 -3 R
0 3 V
-31 -3 R
62 0 V
stroke 6567 5097 M
-62 3 R
62 0 V
-21 -3 R
0 3 V
-31 -3 R
62 0 V
-62 3 R
62 0 V
-21 -3 R
0 3 V
-31 -3 R
62 0 V
-62 3 R
62 0 V
-21 -3 R
0 3 V
-31 -3 R
62 0 V
-62 3 R
62 0 V
-21 -3 R
0 3 V
-31 -3 R
62 0 V
-62 3 R
62 0 V
-21 -3 R
0 3 V
-31 -3 R
62 0 V
-62 3 R
62 0 V
-20 -3 R
0 3 V
-31 -3 R
62 0 V
-62 3 R
62 0 V
-21 -3 R
0 3 V
-31 -3 R
62 0 V
-62 3 R
62 0 V
-21 -4 R
0 4 V
-31 -4 R
62 0 V
-62 4 R
62 0 V
-21 -3 R
0 3 V
-31 -3 R
62 0 V
-62 3 R
62 0 V
-21 -4 R
0 4 V
-31 -4 R
62 0 V
-62 4 R
62 0 V
-21 -4 R
0 4 V
-31 -4 R
62 0 V
-62 4 R
62 0 V
-21 -4 R
0 4 V
-31 -4 R
62 0 V
-62 4 R
62 0 V
-21 -3 R
0 3 V
-31 -3 R
62 0 V
-62 3 R
62 0 V
-21 -4 R
0 4 V
-31 -4 R
62 0 V
-62 4 R
62 0 V
-21 -3 R
0 3 V
-31 -3 R
62 0 V
-62 3 R
62 0 V
-21 -3 R
0 3 V
-31 -3 R
62 0 V
-62 3 R
62 0 V
-21 -3 R
0 3 V
-31 -3 R
62 0 V
-62 3 R
62 0 V
stroke 6738 5100 M
-21 -3 R
0 3 V
-31 -3 R
62 0 V
-62 3 R
62 0 V
-21 -3 R
0 3 V
-31 -3 R
62 0 V
-62 3 R
62 0 V
-21 -3 R
0 3 V
-31 -3 R
62 0 V
-62 3 R
62 0 V
-21 -3 R
0 3 V
-31 -3 R
62 0 V
-62 3 R
62 0 V
-21 -3 R
0 3 V
-31 -3 R
62 0 V
-62 3 R
62 0 V
-21 -3 R
0 3 V
-31 -3 R
62 0 V
-62 3 R
62 0 V
-21 -3 R
0 3 V
-31 -3 R
62 0 V
-62 3 R
62 0 V
-21 -3 R
0 3 V
-31 -3 R
62 0 V
-62 3 R
62 0 V
-21 -3 R
0 3 V
-31 -3 R
62 0 V
-62 3 R
62 0 V
-21 -3 R
0 3 V
-31 -3 R
62 0 V
-62 3 R
62 0 V
-21 -3 R
0 3 V
-31 -3 R
62 0 V
-62 3 R
62 0 V
-21 -3 R
0 3 V
-31 -3 R
62 0 V
-62 3 R
62 0 V
-21 -3 R
0 3 V
-31 -3 R
62 0 V
-62 3 R
62 0 V
-21 -3 R
0 3 V
-31 -3 R
62 0 V
-62 3 R
62 0 V
-21 -3 R
0 3 V
-31 -3 R
62 0 V
-62 3 R
62 0 V
-21 -3 R
0 4 V
-31 -4 R
62 0 V
-62 4 R
62 0 V
-21 -4 R
0 4 V
-31 -4 R
62 0 V
-62 4 R
62 0 V
-21 -4 R
0 4 V
stroke 6887 5101 M
-31 -4 R
62 0 V
-62 4 R
62 0 V
-21 -4 R
0 4 V
-31 -4 R
62 0 V
-62 4 R
62 0 V
-21 -3 R
0 3 V
-31 -3 R
62 0 V
-62 3 R
62 0 V
-21 -3 R
0 3 V
-31 -3 R
62 0 V
-62 3 R
62 0 V
-20 -3 R
0 3 V
-31 -3 R
62 0 V
-62 3 R
62 0 V
-21 -3 R
0 3 V
-31 -3 R
62 0 V
-62 3 R
62 0 V
-21 -3 R
0 3 V
-31 -3 R
62 0 V
-62 3 R
62 0 V
-21 -3 R
0 3 V
-31 -3 R
62 0 V
-62 3 R
62 0 V
-21 -3 R
0 3 V
-31 -3 R
62 0 V
-62 3 R
62 0 V
-21 -3 R
0 3 V
-31 -3 R
62 0 V
-62 3 R
62 0 V
-21 -3 R
0 3 V
-31 -3 R
62 0 V
-62 3 R
62 0 V
-21 -3 R
0 3 V
-31 -3 R
62 0 V
-62 3 R
62 0 V
-21 -3 R
0 3 V
-31 -3 R
62 0 V
-62 3 R
62 0 V
-21 -3 R
0 3 V
-31 -3 R
62 0 V
-62 3 R
62 0 V
-21 -3 R
0 3 V
-31 -3 R
62 0 V
-62 3 R
62 0 V
-21 -3 R
0 3 V
-31 -3 R
62 0 V
-62 3 R
62 0 V
-21 -3 R
0 3 V
-31 -3 R
62 0 V
-62 3 R
62 0 V
-21 -3 R
0 3 V
-31 -3 R
62 0 V
stroke 7089 5098 M
-62 3 R
62 0 V
-21 -3 R
0 3 V
-31 -3 R
62 0 V
-62 3 R
62 0 V
-21 -3 R
0 3 V
-31 -3 R
62 0 V
-62 3 R
62 0 V
-21 -3 R
0 3 V
-31 -3 R
62 0 V
-62 3 R
62 0 V
-21 -3 R
0 3 V
-31 -3 R
62 0 V
-62 3 R
62 0 V
-21 -3 R
0 3 V
-31 -3 R
62 0 V
-62 3 R
62 0 V
-21 -3 R
0 3 V
-31 -3 R
62 0 V
-62 3 R
62 0 V
-21 -3 R
0 3 V
-31 -3 R
62 0 V
-62 3 R
62 0 V
-21 -3 R
0 3 V
-31 -3 R
62 0 V
-62 3 R
62 0 V
-21 -3 R
0 3 V
-31 -3 R
62 0 V
-62 3 R
62 0 V
-21 -3 R
0 3 V
-31 -3 R
62 0 V
-62 3 R
62 0 V
-21 -3 R
0 3 V
-31 -3 R
62 0 V
-62 3 R
62 0 V
-21 -3 R
0 3 V
-31 -3 R
62 0 V
-62 3 R
62 0 V
-21 -3 R
0 3 V
-31 -3 R
62 0 V
-62 3 R
62 0 V
-21 -3 R
0 3 V
-31 -3 R
62 0 V
-62 3 R
62 0 V
-21 -3 R
0 3 V
-31 -3 R
62 0 V
-62 3 R
62 0 V
-21 -3 R
0 3 V
-31 -3 R
62 0 V
-62 3 R
62 0 V
-21 -3 R
0 3 V
-31 -3 R
62 0 V
-62 3 R
62 0 V
stroke 7259 5101 M
-21 -3 R
0 3 V
-31 -3 R
62 0 V
-62 3 R
62 0 V
-21 -3 R
0 3 V
-31 -3 R
62 0 V
-62 3 R
62 0 V
-20 -3 R
0 3 V
-31 -3 R
62 0 V
-62 3 R
62 0 V
-21 -3 R
0 3 V
-31 -3 R
62 0 V
-62 3 R
62 0 V
-21 -3 R
0 3 V
-31 -3 R
62 0 V
-62 3 R
62 0 V
-21 -3 R
0 3 V
-31 -3 R
62 0 V
-62 3 R
62 0 V
-21 -3 R
0 3 V
-31 -3 R
62 0 V
-62 3 R
62 0 V
-21 -3 R
0 3 V
-31 -3 R
62 0 V
-62 3 R
62 0 V
-21 -3 R
0 3 V
-31 -3 R
62 0 V
-62 3 R
62 0 V
-21 -3 R
0 3 V
-31 -3 R
62 0 V
-62 3 R
62 0 V
-21 -3 R
0 3 V
-31 -3 R
62 0 V
-62 3 R
62 0 V
-21 -3 R
0 3 V
-31 -3 R
62 0 V
-62 3 R
62 0 V
-21 -3 R
0 3 V
-31 -3 R
62 0 V
-62 3 R
62 0 V
-21 -3 R
0 3 V
-31 -3 R
62 0 V
-62 3 R
62 0 V
1120 2164 Pls
1130 3371 Pls
1140 3940 Pls
1150 4372 Pls
1160 4617 Pls
1170 4733 Pls
1180 4794 Pls
1190 4939 Pls
1200 4936 Pls
1210 5023 Pls
1220 5052 Pls
1230 5060 Pls
1240 5053 Pls
1250 5114 Pls
1260 5102 Pls
1271 4970 Pls
1281 5084 Pls
1291 5033 Pls
1301 5081 Pls
1311 5064 Pls
1321 5117 Pls
1331 5061 Pls
1341 5108 Pls
1351 5073 Pls
1361 5104 Pls
1371 5042 Pls
1381 5092 Pls
1391 5083 Pls
1401 5127 Pls
1411 5079 Pls
1421 5129 Pls
1431 5140 Pls
1441 5047 Pls
1451 5146 Pls
1461 5059 Pls
1471 5085 Pls
1481 5111 Pls
1491 5156 Pls
1501 5081 Pls
1511 5099 Pls
1521 5080 Pls
1531 5150 Pls
1541 5074 Pls
1551 5054 Pls
1561 5153 Pls
1571 5051 Pls
1581 5081 Pls
1591 5122 Pls
1602 5118 Pls
1612 5089 Pls
1622 5092 Pls
1632 5084 Pls
1642 5154 Pls
1652 5118 Pls
1662 5142 Pls
1672 5095 Pls
1682 5099 Pls
1692 5116 Pls
1702 5120 Pls
1712 5089 Pls
1722 5076 Pls
1732 5029 Pls
1742 5149 Pls
1752 5069 Pls
1762 5127 Pls
1772 5104 Pls
1782 5124 Pls
1792 5127 Pls
1802 5098 Pls
1812 5134 Pls
1822 5045 Pls
1832 5052 Pls
1842 5174 Pls
1852 5094 Pls
1862 5152 Pls
1872 5134 Pls
1882 5127 Pls
1892 5136 Pls
1902 5077 Pls
1912 5165 Pls
1922 5059 Pls
1933 5057 Pls
1943 5108 Pls
1953 5077 Pls
1963 5168 Pls
1973 5150 Pls
1983 5119 Pls
1993 5128 Pls
2003 5071 Pls
2013 5110 Pls
2023 5119 Pls
2033 5194 Pls
2043 5133 Pls
2053 5108 Pls
2063 5097 Pls
2073 5099 Pls
2083 5167 Pls
2093 5076 Pls
2103 5073 Pls
2113 5072 Pls
2123 5135 Pls
2133 5095 Pls
2143 5118 Pls
2153 5102 Pls
2163 5073 Pls
2173 5053 Pls
2183 5083 Pls
2193 5108 Pls
2203 5144 Pls
2213 5105 Pls
2223 5113 Pls
2233 5133 Pls
2243 5092 Pls
2253 5086 Pls
2263 5105 Pls
2274 5070 Pls
2284 5110 Pls
2294 5071 Pls
2304 5005 Pls
2314 5054 Pls
2324 5159 Pls
2334 5078 Pls
2344 5097 Pls
2354 5097 Pls
2364 5128 Pls
2374 5098 Pls
2384 5094 Pls
2394 5109 Pls
2404 5104 Pls
2414 5100 Pls
2424 5095 Pls
2434 5092 Pls
2444 5096 Pls
2454 5096 Pls
2464 5097 Pls
2474 5095 Pls
2484 5096 Pls
2494 5099 Pls
2504 5098 Pls
2514 5097 Pls
2524 5097 Pls
2534 5098 Pls
2544 5100 Pls
2554 5097 Pls
2564 5095 Pls
2574 5094 Pls
2584 5092 Pls
2594 5092 Pls
2605 5092 Pls
2615 5093 Pls
2625 5093 Pls
2635 5094 Pls
2645 5094 Pls
2655 5093 Pls
2665 5095 Pls
2675 5093 Pls
2685 5092 Pls
2695 5091 Pls
2705 5092 Pls
2715 5093 Pls
2725 5092 Pls
2735 5092 Pls
2745 5091 Pls
2755 5090 Pls
2765 5091 Pls
2775 5092 Pls
2785 5092 Pls
2795 5093 Pls
2805 5093 Pls
2815 5093 Pls
2825 5093 Pls
2835 5093 Pls
2845 5093 Pls
2855 5094 Pls
2865 5094 Pls
2875 5094 Pls
2885 5094 Pls
2895 5094 Pls
2905 5093 Pls
2915 5094 Pls
2925 5093 Pls
2936 5094 Pls
2946 5094 Pls
2956 5094 Pls
2966 5094 Pls
2976 5094 Pls
2986 5094 Pls
2996 5094 Pls
3006 5093 Pls
3016 5094 Pls
3026 5094 Pls
3036 5093 Pls
3046 5092 Pls
3056 5093 Pls
3066 5093 Pls
3076 5094 Pls
3086 5094 Pls
3096 5094 Pls
3106 5095 Pls
3116 5094 Pls
3126 5094 Pls
3136 5094 Pls
3146 5094 Pls
3156 5094 Pls
3166 5094 Pls
3176 5094 Pls
3186 5093 Pls
3196 5093 Pls
3206 5093 Pls
3216 5093 Pls
3226 5092 Pls
3236 5092 Pls
3246 5092 Pls
3256 5091 Pls
3267 5091 Pls
3277 5091 Pls
3287 5091 Pls
3297 5092 Pls
3307 5092 Pls
3317 5092 Pls
3327 5092 Pls
3337 5092 Pls
3347 5091 Pls
3357 5091 Pls
3367 5091 Pls
3377 5092 Pls
3387 5092 Pls
3397 5092 Pls
3407 5092 Pls
3417 5092 Pls
3427 5091 Pls
3437 5091 Pls
3447 5092 Pls
3457 5092 Pls
3467 5092 Pls
3477 5092 Pls
3487 5092 Pls
3497 5092 Pls
3507 5092 Pls
3517 5093 Pls
3527 5093 Pls
3537 5093 Pls
3547 5093 Pls
3557 5093 Pls
3567 5094 Pls
3577 5094 Pls
3587 5094 Pls
3598 5094 Pls
3608 5093 Pls
3618 5093 Pls
3628 5093 Pls
3638 5093 Pls
3648 5094 Pls
3658 5094 Pls
3668 5094 Pls
3678 5094 Pls
3688 5094 Pls
3698 5094 Pls
3708 5094 Pls
3718 5094 Pls
3728 5094 Pls
3738 5094 Pls
3748 5094 Pls
3758 5094 Pls
3768 5094 Pls
3778 5094 Pls
3788 5094 Pls
3798 5094 Pls
3808 5094 Pls
3818 5094 Pls
3828 5094 Pls
3838 5094 Pls
3848 5095 Pls
3858 5095 Pls
3868 5095 Pls
3878 5094 Pls
3888 5094 Pls
3898 5095 Pls
3908 5095 Pls
3918 5095 Pls
3929 5095 Pls
3939 5095 Pls
3949 5095 Pls
3959 5095 Pls
3969 5095 Pls
3979 5095 Pls
3989 5095 Pls
3999 5095 Pls
4009 5095 Pls
4019 5096 Pls
4029 5096 Pls
4039 5096 Pls
4049 5096 Pls
4059 5096 Pls
4069 5096 Pls
4079 5096 Pls
4089 5096 Pls
4099 5096 Pls
4109 5097 Pls
4119 5096 Pls
4129 5096 Pls
4139 5096 Pls
4149 5096 Pls
4159 5096 Pls
4169 5096 Pls
4179 5096 Pls
4189 5096 Pls
4199 5096 Pls
4209 5096 Pls
4219 5095 Pls
4229 5095 Pls
4239 5095 Pls
4249 5095 Pls
4260 5095 Pls
4270 5095 Pls
4280 5095 Pls
4290 5095 Pls
4300 5095 Pls
4310 5095 Pls
4320 5095 Pls
4330 5095 Pls
4340 5095 Pls
4350 5095 Pls
4360 5095 Pls
4370 5095 Pls
4380 5095 Pls
4390 5095 Pls
4400 5095 Pls
4410 5095 Pls
4420 5095 Pls
4430 5095 Pls
4440 5095 Pls
4450 5094 Pls
4460 5095 Pls
4470 5094 Pls
4480 5094 Pls
4490 5094 Pls
4500 5095 Pls
4510 5094 Pls
4520 5095 Pls
4530 5094 Pls
4540 5094 Pls
4550 5095 Pls
4560 5095 Pls
4570 5095 Pls
4580 5095 Pls
4590 5095 Pls
4601 5095 Pls
4611 5095 Pls
4621 5095 Pls
4631 5095 Pls
4641 5095 Pls
4651 5095 Pls
4661 5095 Pls
4671 5095 Pls
4681 5095 Pls
4691 5096 Pls
4701 5096 Pls
4711 5096 Pls
4721 5096 Pls
4731 5096 Pls
4741 5097 Pls
4751 5097 Pls
4761 5097 Pls
4771 5097 Pls
4781 5097 Pls
4791 5097 Pls
4801 5097 Pls
4811 5097 Pls
4821 5097 Pls
4831 5097 Pls
4841 5097 Pls
4851 5096 Pls
4861 5096 Pls
4871 5097 Pls
4881 5097 Pls
4891 5097 Pls
4901 5097 Pls
4911 5097 Pls
4921 5097 Pls
4932 5097 Pls
4942 5097 Pls
4952 5097 Pls
4962 5097 Pls
4972 5097 Pls
4982 5097 Pls
4992 5097 Pls
5002 5097 Pls
5012 5097 Pls
5022 5097 Pls
5032 5097 Pls
5042 5097 Pls
5052 5097 Pls
5062 5097 Pls
5072 5097 Pls
5082 5097 Pls
5092 5097 Pls
5102 5097 Pls
5112 5097 Pls
5122 5098 Pls
5132 5098 Pls
5142 5097 Pls
5152 5098 Pls
5162 5098 Pls
5172 5098 Pls
5182 5098 Pls
5192 5098 Pls
5202 5097 Pls
5212 5097 Pls
5222 5097 Pls
5232 5097 Pls
5242 5097 Pls
5252 5098 Pls
5263 5098 Pls
5273 5098 Pls
5283 5097 Pls
5293 5097 Pls
5303 5097 Pls
5313 5097 Pls
5323 5097 Pls
5333 5097 Pls
5343 5097 Pls
5353 5097 Pls
5363 5097 Pls
5373 5097 Pls
5383 5097 Pls
5393 5097 Pls
5403 5097 Pls
5413 5097 Pls
5423 5097 Pls
5433 5097 Pls
5443 5097 Pls
5453 5097 Pls
5463 5097 Pls
5473 5097 Pls
5483 5097 Pls
5493 5097 Pls
5503 5097 Pls
5513 5097 Pls
5523 5097 Pls
5533 5097 Pls
5543 5097 Pls
5553 5097 Pls
5563 5097 Pls
5573 5097 Pls
5583 5097 Pls
5594 5097 Pls
5604 5097 Pls
5614 5097 Pls
5624 5097 Pls
5634 5098 Pls
5644 5098 Pls
5654 5098 Pls
5664 5098 Pls
5674 5097 Pls
5684 5097 Pls
5694 5097 Pls
5704 5097 Pls
5714 5097 Pls
5724 5097 Pls
5734 5097 Pls
5744 5097 Pls
5754 5097 Pls
5764 5097 Pls
5774 5097 Pls
5784 5097 Pls
5794 5097 Pls
5804 5097 Pls
5814 5097 Pls
5824 5097 Pls
5834 5097 Pls
5844 5097 Pls
5854 5098 Pls
5864 5098 Pls
5874 5098 Pls
5884 5097 Pls
5894 5097 Pls
5904 5097 Pls
5914 5097 Pls
5925 5097 Pls
5935 5097 Pls
5945 5097 Pls
5955 5097 Pls
5965 5097 Pls
5975 5097 Pls
5985 5097 Pls
5995 5098 Pls
6005 5098 Pls
6015 5098 Pls
6025 5098 Pls
6035 5098 Pls
6045 5098 Pls
6055 5098 Pls
6065 5098 Pls
6075 5098 Pls
6085 5098 Pls
6095 5098 Pls
6105 5098 Pls
6115 5098 Pls
6125 5098 Pls
6135 5098 Pls
6145 5098 Pls
6155 5098 Pls
6165 5098 Pls
6175 5099 Pls
6185 5099 Pls
6195 5099 Pls
6205 5099 Pls
6215 5099 Pls
6225 5099 Pls
6235 5098 Pls
6245 5098 Pls
6256 5099 Pls
6266 5098 Pls
6276 5099 Pls
6286 5098 Pls
6296 5099 Pls
6306 5099 Pls
6316 5099 Pls
6326 5099 Pls
6336 5099 Pls
6346 5098 Pls
6356 5098 Pls
6366 5098 Pls
6376 5098 Pls
6386 5098 Pls
6396 5098 Pls
6406 5098 Pls
6416 5098 Pls
6426 5098 Pls
6436 5098 Pls
6446 5098 Pls
6456 5098 Pls
6466 5098 Pls
6476 5098 Pls
6486 5098 Pls
6496 5098 Pls
6506 5098 Pls
6516 5098 Pls
6526 5098 Pls
6536 5098 Pls
6546 5098 Pls
6556 5098 Pls
6566 5098 Pls
6576 5098 Pls
6586 5098 Pls
6597 5098 Pls
6607 5098 Pls
6617 5098 Pls
6627 5098 Pls
6637 5098 Pls
6647 5098 Pls
6657 5098 Pls
6667 5098 Pls
6677 5098 Pls
6687 5098 Pls
6697 5098 Pls
6707 5098 Pls
6717 5098 Pls
6727 5098 Pls
6737 5098 Pls
6747 5099 Pls
6757 5099 Pls
6767 5099 Pls
6777 5098 Pls
6787 5098 Pls
6797 5099 Pls
6807 5098 Pls
6817 5099 Pls
6827 5099 Pls
6837 5099 Pls
6847 5099 Pls
6857 5099 Pls
6867 5099 Pls
6877 5099 Pls
6887 5099 Pls
6897 5099 Pls
6907 5099 Pls
6917 5099 Pls
6928 5099 Pls
6938 5099 Pls
6948 5099 Pls
6958 5099 Pls
6968 5099 Pls
6978 5099 Pls
6988 5099 Pls
6998 5099 Pls
7008 5099 Pls
7018 5099 Pls
7028 5099 Pls
7038 5099 Pls
7048 5099 Pls
7058 5099 Pls
7068 5099 Pls
7078 5099 Pls
7088 5099 Pls
7098 5099 Pls
7108 5099 Pls
7118 5099 Pls
7128 5099 Pls
7138 5099 Pls
7148 5099 Pls
7158 5099 Pls
7168 5099 Pls
7178 5099 Pls
7188 5099 Pls
7198 5099 Pls
7208 5099 Pls
7218 5099 Pls
7228 5099 Pls
7238 5099 Pls
7248 5099 Pls
7259 5099 Pls
7269 5099 Pls
7279 5099 Pls
7289 5099 Pls
7299 5099 Pls
7309 5100 Pls
7319 5100 Pls
7329 5100 Pls
7339 5100 Pls
7349 5100 Pls
7359 5100 Pls
7369 5100 Pls
6907 1303 Pls
1.000 UL
LT0
LC2 setrgbcolor
LCb setrgbcolor
LT0
LC2 setrgbcolor
6636 1103 M
543 0 V
1110 1207 M
10 1718 V
10 940 V
10 302 V
10 388 V
10 -40 V
10 220 V
10 -16 V
10 44 V
10 100 V
10 -96 V
10 -11 V
10 39 V
10 -115 V
10 86 V
10 -84 V
11 119 V
10 -97 V
10 13 V
10 -12 V
10 -6 V
10 73 V
10 38 V
10 -103 V
10 80 V
10 27 V
10 -126 V
10 15 V
10 113 V
10 -87 V
10 0 V
10 -31 V
10 -35 V
10 97 V
10 35 V
10 -87 V
10 -141 V
10 84 V
10 83 V
10 -71 V
10 160 V
10 7 V
10 -142 V
10 -41 V
10 77 V
10 -93 V
10 139 V
10 -121 V
10 82 V
11 -45 V
10 115 V
10 -110 V
10 92 V
10 10 V
10 -129 V
10 42 V
10 54 V
10 -103 V
10 -65 V
10 289 V
10 0 V
10 -95 V
10 31 V
10 -97 V
10 48 V
10 43 V
10 -166 V
10 70 V
10 27 V
10 1 V
10 -42 V
10 15 V
10 -24 V
10 47 V
10 -40 V
10 2 V
10 -13 V
10 -103 V
10 -12 V
10 245 V
10 -168 V
10 32 V
11 -144 V
10 12 V
10 117 V
10 17 V
10 80 V
10 -21 V
10 102 V
10 -85 V
10 54 V
10 -153 V
10 -55 V
10 195 V
10 -172 V
10 57 V
10 5 V
10 -39 V
10 5 V
10 11 V
10 143 V
10 -230 V
10 180 V
stroke 2133 4761 M
10 -13 V
10 -83 V
10 16 V
10 125 V
10 -79 V
10 21 V
10 -128 V
10 -170 V
10 447 V
10 -233 V
10 99 V
10 12 V
10 -91 V
11 64 V
10 -71 V
10 28 V
10 82 V
10 -77 V
10 42 V
10 38 V
10 39 V
10 -186 V
10 -32 V
10 19 V
10 148 V
10 -36 V
10 -34 V
10 51 V
10 58 V
10 -28 V
10 -33 V
10 -10 V
10 88 V
10 -61 V
10 -82 V
10 8 V
10 64 V
10 -68 V
10 4 V
10 31 V
10 -76 V
10 88 V
10 -126 V
10 120 V
10 -13 V
10 -78 V
11 100 V
10 -143 V
10 21 V
10 53 V
10 54 V
10 -63 V
10 109 V
10 9 V
10 -95 V
10 -37 V
10 27 V
10 42 V
10 108 V
10 -139 V
10 -93 V
10 101 V
10 0 V
10 46 V
10 -10 V
10 -96 V
10 89 V
10 52 V
10 -64 V
10 128 V
10 -141 V
10 -53 V
10 63 V
10 68 V
10 -117 V
10 149 V
10 47 V
10 -42 V
10 -61 V
11 -58 V
10 -5 V
10 30 V
10 33 V
10 -8 V
10 -69 V
10 60 V
10 -36 V
10 -42 V
10 1 V
10 134 V
10 -65 V
10 -29 V
10 -8 V
10 58 V
10 -97 V
10 96 V
10 -2 V
10 11 V
10 3 V
10 89 V
10 -45 V
10 -135 V
10 131 V
10 -57 V
stroke 3176 4747 M
10 -36 V
10 -124 V
10 141 V
10 72 V
10 -117 V
10 150 V
10 -49 V
10 -41 V
11 -23 V
10 -23 V
10 60 V
10 -60 V
10 37 V
10 -34 V
10 65 V
10 -5 V
10 3 V
10 -10 V
10 -8 V
10 -50 V
10 46 V
10 -228 V
10 136 V
10 96 V
10 108 V
10 -176 V
10 -25 V
10 169 V
10 -127 V
10 -25 V
10 -16 V
10 27 V
10 46 V
10 64 V
10 -35 V
10 -5 V
10 43 V
10 -106 V
10 -3 V
10 -65 V
10 177 V
11 0 V
10 -98 V
10 32 V
10 -45 V
10 97 V
10 -22 V
10 -68 V
10 103 V
10 -127 V
10 -45 V
10 153 V
10 -28 V
10 16 V
10 -40 V
10 -48 V
10 56 V
10 -19 V
10 22 V
10 -16 V
10 -78 V
10 61 V
10 24 V
10 -30 V
10 51 V
10 -89 V
10 -12 V
10 154 V
10 -78 V
10 20 V
10 -10 V
10 92 V
10 -102 V
10 -14 V
11 14 V
10 39 V
10 45 V
10 -58 V
10 -62 V
10 -1 V
10 39 V
10 27 V
10 -172 V
10 247 V
10 -50 V
10 84 V
10 -205 V
10 155 V
10 -39 V
10 -132 V
10 227 V
10 -184 V
10 62 V
10 18 V
10 -34 V
10 -21 V
10 -90 V
10 126 V
10 -80 V
10 38 V
10 -16 V
10 -23 V
10 61 V
10 -83 V
stroke 4219 4640 M
10 107 V
10 -36 V
10 -22 V
11 151 V
10 -115 V
10 -69 V
10 94 V
10 -17 V
10 -76 V
10 10 V
10 9 V
10 72 V
10 -116 V
10 63 V
10 15 V
10 39 V
10 -65 V
10 -100 V
10 227 V
10 -109 V
10 47 V
10 -120 V
10 16 V
10 -6 V
10 40 V
10 159 V
10 -61 V
10 -60 V
10 -80 V
10 27 V
10 108 V
10 -110 V
10 164 V
10 -79 V
10 51 V
10 -80 V
10 -53 V
11 144 V
10 -67 V
10 49 V
10 53 V
10 -88 V
10 12 V
10 -169 V
10 174 V
10 61 V
10 -207 V
10 100 V
10 -16 V
10 23 V
10 -107 V
10 214 V
10 -215 V
10 72 V
10 161 V
10 -82 V
10 -21 V
10 19 V
10 -10 V
10 -78 V
10 110 V
10 -71 V
10 -40 V
10 150 V
10 -139 V
10 31 V
10 128 V
10 6 V
10 -124 V
10 64 V
11 14 V
10 -147 V
10 -82 V
10 97 V
10 69 V
10 -135 V
10 146 V
10 9 V
10 -68 V
10 -64 V
10 63 V
10 -84 V
10 208 V
10 -30 V
10 -47 V
10 -84 V
10 6 V
10 35 V
10 55 V
10 20 V
10 11 V
10 -51 V
10 -73 V
10 89 V
10 103 V
10 -160 V
10 61 V
10 -62 V
10 -59 V
10 -9 V
10 43 V
10 4 V
10 -113 V
11 225 V
stroke 5263 4792 M
10 -184 V
10 196 V
10 -78 V
10 -36 V
10 0 V
10 -64 V
10 211 V
10 -29 V
10 -60 V
10 -143 V
10 146 V
10 -14 V
10 48 V
10 -62 V
10 -105 V
10 138 V
10 -125 V
10 -1 V
10 145 V
10 -53 V
10 57 V
10 -69 V
10 -39 V
10 166 V
10 -85 V
10 57 V
10 -87 V
10 -148 V
10 179 V
10 11 V
10 -64 V
10 49 V
11 -11 V
10 -101 V
10 150 V
10 -19 V
10 -148 V
10 88 V
10 49 V
10 -125 V
10 84 V
10 59 V
10 -61 V
10 87 V
10 -105 V
10 22 V
10 -44 V
10 62 V
10 -55 V
10 1 V
10 31 V
10 16 V
10 85 V
10 -70 V
10 2 V
10 -140 V
10 94 V
10 126 V
10 -142 V
10 128 V
10 -104 V
10 -56 V
10 -87 V
10 181 V
10 -21 V
11 41 V
10 14 V
10 -225 V
10 246 V
10 36 V
10 -69 V
10 1 V
10 -76 V
10 66 V
10 99 V
10 -89 V
10 -102 V
10 96 V
10 -39 V
10 59 V
10 -147 V
10 34 V
10 -75 V
10 88 V
10 -226 V
10 308 V
10 -34 V
10 -6 V
10 -26 V
10 19 V
10 -13 V
10 184 V
10 -266 V
10 115 V
10 -161 V
10 132 V
10 47 V
10 50 V
11 -120 V
10 9 V
10 82 V
10 21 V
10 4 V
10 -143 V
stroke 6306 4659 M
10 83 V
10 9 V
10 -22 V
10 28 V
10 -55 V
10 56 V
10 -154 V
10 39 V
10 47 V
10 110 V
10 -75 V
10 42 V
10 -57 V
10 -54 V
10 59 V
10 -1 V
10 91 V
10 -146 V
10 126 V
10 -186 V
10 125 V
10 -2 V
10 -63 V
10 61 V
10 126 V
10 -117 V
10 -33 V
10 86 V
11 -13 V
10 -115 V
10 147 V
10 -119 V
10 31 V
10 -10 V
10 128 V
10 -97 V
10 -135 V
10 73 V
10 131 V
10 -213 V
10 222 V
10 -211 V
10 194 V
10 -58 V
10 -229 V
10 202 V
10 121 V
10 -13 V
10 -161 V
10 139 V
10 -90 V
10 96 V
10 -90 V
10 154 V
10 -173 V
10 -13 V
10 32 V
10 -40 V
10 22 V
10 97 V
10 -65 V
11 39 V
10 24 V
10 0 V
10 -272 V
10 192 V
10 -75 V
10 40 V
10 101 V
10 -1 V
10 23 V
10 -82 V
10 -69 V
10 106 V
10 -135 V
10 95 V
10 -67 V
10 9 V
10 122 V
10 19 V
10 35 V
10 -257 V
10 32 V
10 99 V
10 13 V
10 0 V
10 3 V
10 -200 V
10 192 V
10 54 V
10 18 V
10 -216 V
10 58 V
10 148 V
11 -98 V
10 -58 V
10 -24 V
10 199 V
10 -108 V
10 -46 V
10 81 V
10 83 V
10 -32 V
10 -87 V
stroke 7349 4684 M
10 23 V
10 19 V
stroke
LTb
LCb setrgbcolor
LTb
6636 903 M
543 0 V
1110 5099 M
64 0 V
63 0 V
64 0 V
64 0 V
64 0 V
63 0 V
64 0 V
64 0 V
64 0 V
63 0 V
64 0 V
64 0 V
63 0 V
64 0 V
64 0 V
64 0 V
63 0 V
64 0 V
64 0 V
64 0 V
63 0 V
64 0 V
64 0 V
63 0 V
64 0 V
64 0 V
64 0 V
63 0 V
64 0 V
64 0 V
64 0 V
63 0 V
64 0 V
64 0 V
63 0 V
64 0 V
64 0 V
64 0 V
63 0 V
64 0 V
64 0 V
64 0 V
63 0 V
64 0 V
64 0 V
63 0 V
64 0 V
64 0 V
64 0 V
63 0 V
64 0 V
64 0 V
64 0 V
63 0 V
64 0 V
64 0 V
63 0 V
64 0 V
64 0 V
64 0 V
63 0 V
64 0 V
64 0 V
64 0 V
63 0 V
64 0 V
64 0 V
63 0 V
64 0 V
64 0 V
64 0 V
63 0 V
64 0 V
64 0 V
64 0 V
63 0 V
64 0 V
64 0 V
63 0 V
64 0 V
64 0 V
64 0 V
63 0 V
64 0 V
64 0 V
64 0 V
63 0 V
64 0 V
64 0 V
63 0 V
64 0 V
64 0 V
64 0 V
63 0 V
64 0 V
64 0 V
64 0 V
63 0 V
64 0 V
stroke
1100 5519 N
0 -4879 V
6319 0 V
0 4879 V
-6319 0 V
Z stroke
1.000 UP
1.000 UL
LTb
stroke
grestore
end
showpage
  }}%
  \put(6516,903){\makebox(0,0)[r]{\strut{}Reference}}%
  \put(6516,1103){\makebox(0,0)[r]{\strut{}Entropy}}%
  \put(6516,1303){\makebox(0,0)[r]{\strut{}$\lambda_0$}}%
  \put(4259,140){\makebox(0,0){\strut{}Histories}}%
  \put(8238,3079){%
  \special{ps: gsave currentpoint currentpoint translate
630 rotate neg exch neg exch translate}%
  \makebox(0,0){\strut{}Shannon Entropy}%
  \special{ps: currentpoint grestore moveto}%
  }%
  \put(280,3079){%
  \special{ps: gsave currentpoint currentpoint translate
630 rotate neg exch neg exch translate}%
  \makebox(0,0){\strut{}Eigenvalue Estimate}%
  \special{ps: currentpoint grestore moveto}%
  }%
  \put(7539,5519){\makebox(0,0)[l]{\strut{} 4.5}}%
  \put(7539,4706){\makebox(0,0)[l]{\strut{} 4}}%
  \put(7539,3893){\makebox(0,0)[l]{\strut{} 3.5}}%
  \put(7539,3080){\makebox(0,0)[l]{\strut{} 3}}%
  \put(7539,2266){\makebox(0,0)[l]{\strut{} 2.5}}%
  \put(7539,1453){\makebox(0,0)[l]{\strut{} 2}}%
  \put(7539,640){\makebox(0,0)[l]{\strut{} 1.5}}%
  \put(7118,440){\makebox(0,0){\strut{} 1.2e+08}}%
  \put(6115,440){\makebox(0,0){\strut{} 1e+08}}%
  \put(5112,440){\makebox(0,0){\strut{} 8e+07}}%
  \put(4109,440){\makebox(0,0){\strut{} 6e+07}}%
  \put(3106,440){\makebox(0,0){\strut{} 4e+07}}%
  \put(2103,440){\makebox(0,0){\strut{} 2e+07}}%
  \put(1100,440){\makebox(0,0){\strut{} 0}}%
  \put(980,5519){\makebox(0,0)[r]{\strut{} 5}}%
  \put(980,5031){\makebox(0,0)[r]{\strut{} 4.8}}%
  \put(980,4543){\makebox(0,0)[r]{\strut{} 4.6}}%
  \put(980,4055){\makebox(0,0)[r]{\strut{} 4.4}}%
  \put(980,3567){\makebox(0,0)[r]{\strut{} 4.2}}%
  \put(980,3080){\makebox(0,0)[r]{\strut{} 4}}%
  \put(980,2592){\makebox(0,0)[r]{\strut{} 3.8}}%
  \put(980,2104){\makebox(0,0)[r]{\strut{} 3.6}}%
  \put(980,1616){\makebox(0,0)[r]{\strut{} 3.4}}%
  \put(980,1128){\makebox(0,0)[r]{\strut{} 3.2}}%
  \put(980,640){\makebox(0,0)[r]{\strut{} 3}}%
\end{picture}%
\endgroup
\endinput
}

    \subfloat[Eigenvector]{\label{fig:N1ArnoldiVectors}\input{Conclusions/Data/N1ArnoldiVectors}}
    \caption{Preliminary calculation of eigenvalue estimates, eigenvector, and Shanon entropy from an Arnoldi's method with just 2 iterations per restart and only saving the fundamental eigenmode.  The black line is the reference solution.}
\end{figure}

These preliminary results show two important things.  First, if we are interested in just one eigenmode we should use a smaller Krylov subspace (fewer iterations per restart) and have more active restarts; this reduces the variance and increases the figure of merit. Second, we see that computing an eigenvalue estimate more frequently can reduce the variance for the mean eigenvalue.

\subsection{Condensing Arnoldi's Method}
One suggestion that has been made\footnote{very determinedly, in fact} for Arnoldi's method is to eliminate restarts and just do a few highly accurate iterations.  We have seen that Arnoldi's method does not need many inactive restarts to converge the fission source.  The only remaining reason to use many restarts is to obtain an estimation of the statistical uncertainty of the eigenvalue estimate.  However we also know that the estimate of the uncertainty is wrong.

It has been proposed that rather than performing many Arnoldi restarts, it may be beneficial to track many more particles during just one or two Arnoldi restarts with sufficient Krylov subspace size.  The application of \A{} would be performed much more accurately and the eigenvalue estimates would be much better.

With only one or two Arnoldi restarts the statistical uncertainty could not be calculated as described in this dissertation; the variance of just two estimates isn't helpful.  In principle the statistical uncertainty could be calculated by propagating the statistical error through the iterations of Arnoldi's method.  This would be a profound change in how statistical uncertainties are calculated in Monte Carlo eigenvalue computations, no one has done this before.  

A preliminary simulation has been done using this idea.  We return to the 20 mfp slab geometry introduced in \Fref{ch:ArnoldiMethod}.  We will use the same total number of particles, but will put all of them into one inactive restart and one active restart; each restart has 10 iterations with 6.25\e{6} particles tracked in each iteration.  

The eigenvalues from this simulation are given in \Fref{tab:CondensedArnoldi} and the estimated eigenvectors in \Fref{fig:CondensedArnoldiVectors}; the ``Condensed'' Arnoldi is Arnoldi's method with just two restarts, but many particles tracked in each iteration.  We see that the eigenvalue estimates are not exactly the same as the reference solution but that the eigenvector is an excellent estimate as with the regular Arnoldi's method.  However we do not have a way yet of estimating the statistical uncertainty of these values so we do not know if the eigenvalue estimates are within statistical uncertainty of the reference solution.
\begin{table}[h] \centering
    \begin{tabular}{rccc}
        \toprule
         & $\lambda_0$ & $\lambda_1$ & $\lambda_2$ \\
         \midrule
        Condensed Arnoldi & 4.8309 & 4.3824 & 3.8131 \\
        Reference & 4.8278 & 4.3831 & 3.8174 \\
        \bottomrule
\end{tabular}
    \caption{Eigenvalue estimates for 20 mfp thick slab geometry from a condensed Arnoldi's method and Reference eigenvalues from \cite{Garis:1991One-s-0}, and \cite{Dahl:1979Eigen-0}.}
    \label{tab:CondensedArnoldi}
\end{table}

\begin{figure} \centering
    \input{Conclusions/Data/CondensedArnoldiVectors}
    \caption{Preliminary calculation of eigenvectors from condensed Arnoldi.}
    \label{fig:CondensedArnoldiVectors}
\end{figure}

\subsection{Multi-dimensional and Real-world Problems}
All of the simulations demonstrated in this dissertation have been one-dimensional.  Restricting a prolem to one-dimension is sufficient for a proof-of-concept but it certainly does not represent a real-world problem.  As of yet, there have been no attempts at using Arnoldi's method in a three-dimensional, production code such as MCNP; any estimates of how Arnoldi's method may operate in three-dimensions is just speculation.

Most likely the biggest issue when moving to three-dimensions is the length of the Arnoldi vectors.  It is expected that the number of discretization bins necessary for three dimensions would be at least $N^3$ where $N$ is the number of bins for a one-dimensional problem.  Increasing the size of the Arnoldi vectors by this magnitude would increase the computational expense of orthogonalizing the Arnoldi vectors as well as the expense of sampling and scoring in and from a three-dimensional source.  

The size of the Krylov subspace (number of iterations in a restart) necessary for a three-dimensional problem would seem to be greater than for a one-dimensional problem.  In this dissertation, the more difficult problems (i.e. problems with larger dominance ratios) required additional iterations for an accurate eigenvalue estimate.  A three-dimensional problem would almost certainly require more iterations than a one-dimensional problem and would therefore require more time.  

Arnoldi's method, while having been used extensively in the numerical analysis community has yet to be incorporated in Monte Carlo particle transport algorithms.  This dissertation represents the first work in this area.  Arnoldi's method has many promising qualities; further investigation will determine the ability of Arnoldi's method to be used in production Monte Carlo codes.
