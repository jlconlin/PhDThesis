%!TEX root = ../Thesis.tex

\chapter{Relaxed Arnoldi \label{ch:RelaxedArnoldi}}
In Monte Carlo particle transport the figure of merit (FOM) is a measure of efficiency of a particular algorithm or simulation.  FOM was given in \Fref{eq:FOM} and again here for clarity
\begin{equation}
    \mathrm{FOM} \equiv \frac{1}{\sigma_{\lambda}^2T}.
    \label{eq:FOMRelaxed}
\end{equation}
The variance of an estimated eigenvalue $\lambda$ in a Monte Carlo approach is
\begin{equation}
    \sigma_{\lambda}^2 = \frac{1}{N}\sigma^2,
    \label{eq:SampleVariance}
\end{equation}
where $N$ is the number of estimates of the eigenvalue and $\sigma^2$ is the variance of the distribution of the estimates.  A larger FOM indicates a more efficient calculation, i.e. the variance is smaller for a given amount of computer time or number of particles tracked.  To increase the FOM, one must increase $N$ while keeping $T$ as small as possible which, in turn, will decrease $\sigma$.

In this thesis the focus has been on estimating the eigenvalues of the fission-transport operator \A. In \Fref{ch:ArnoldiMethod}, I compared the figure of merit for Arnoldi's method to that produced by the power method for estimating the fundamental eigenvalue.  The power method had a FOM that was much larger than that of Arnoldi's method, even though both methods tracked the same number of particles in each iteration.  

The power method calculates an estimate of the fundamental eigenvalues at the end of every iteration, while in Arnoldi's method it was chosen to calculate an eigenvalue estimate at the end of an Arnoldi restart.  The simulations used in \Fref{ch:ArnoldiMethod} had ten iterations in each Arnoldi restart.  Even though the power method and Arnoldi's method used the same number of particles to apply the linear operator at each iteration, the power method had ten times more eigenvalue estimates than Arnoldi's method---for the same number of particles tracked.  The FOM shown in \Fref{tab:BasicResults} is nearly exactly ten times larger for the power method than for Arnoldi's method.  This indicates that the variance is driven primarily by the number of eigenvalue estimates.  

\section{Relaxed Arnoldi's Method}
Recently \cite{Eshof:2004Inexa-0,Sleijpen:2005Resta-0,Simoncini:2003Theor-0,Freitag:2007Inner-0} there has been some interest in Krylov subspace methods where the application of the linear operator is performed inexactly; for example when the application of the operator \A{} is an iterative process and the iterations are terminated before the calculation converges. By terminating early, computation time is saved at the expense of the precision of the application of \A{}.

A group of researchers \citep{Bouras:2000A-rel-1} discovered experimentally that under certain circumstances the inexact application of the linear operator has little or no effect upon the convergence of the eigenpair calculation. They discovered that, as Arnoldi's method proceeds and the size of the Krylov subspace increases, the precision to which the matrix-vector product is calculated can be decreased, or relaxed.  Their results have been further investigated leading to more formal theories and proofs (see \cite{Bouras:2005Inexa-0} and \cite{Simoncini:2005Varia-0}).

When a linear operator is applied inexactly  we can obtain the inexact vector $\hat{v}_k$ by computing
\begin{equation}
    \hat{v}_k = \left(\A + \DA_k\right)\hat{v}_{k-1}, 
    \label{eq:InexactApplication}
\end{equation}
where $\DA_k$ is some perturbation matrix.  Of course, when the linear operator is applied inexactly, we are no longer forming a true Krylov subspace as shown in \Fref{eq:KrylovSubspace} but have rather
\begin{equation}
    \hat{\mathcal{K}} = \mathspan\left\{v_0, \hat{v}_1, \hat{v}_2, \ldots, \hat{v}_{m-1}\right\},
    \label{eq:RelaxedKrylovVectors}
\end{equation}
where $\hat{v}_k$ is given in \Fref{eq:InexactApplication} and is orthogonalized against the previously calculated Arnoldi vectors $\hat{V}_k = \left[v_0\, \hat{v}_1\, \ldots\, \hat{v}_{k-1}\right]$.

The basic idea presented by Bouras and Frayss\'{e} is to reduce the precision to which the linear operator is applied in an iteration of Arnoldi's method, but limit the size of $\DA_k$ to some fraction of \A.  To show how this is done, let $\eta$ be the final tolerance required and let $\alpha_k$ be a scalar defined by
\begin{equation}
    \alpha_k = \frac{1}{\min\left(\left\|r_{k-1}\right\|,1\right)},
    \label{eq:RelaxedAlpha}
\end{equation}
where $\left\|r_{k-1}\right\|$ is the magnitude of the residual as given in \Fref{eq:Residual}.  The limit to $\DA_k$'s size is defined as
\begin{gather}
    \left\|\DA_k\right\| = \varepsilon_k\left\|\A\right\| \label{eq:PerturbationMatrix}
    \intertext{where}
    \varepsilon_k = \min\left(\alpha_k\eta, 1\right). \label{eq:RelaxedEpsilon}
\end{gather}.

If we put together \Fref{eq:RelaxedAlpha}, \Fref{eq:PerturbationMatrix}, and \Fref{eq:RelaxedEpsilon} we see that as the residual decreases the size of $\DA_k$ is allowed to become larger---the precision to which the linear operator \A{} is applied is \emph{relaxed}.

\subsubsection{Relaxed Monte Carlo Application of Fission-Transport Operator}
In Monte Carlo particle transport the magnitude of the perturbation matrix, $\left\|\DA\right\|$ is proportional to $1/\sqrt{N}$ where $N$ is the number of particles tracked when applying the fission-transport operator.  Relaxing the application of the fission-transport operator means simply reducing the number of particles used in applying the operator
\begin{equation}
    N_k = f\left(\left\|r_{k-1}\right\|\right) N_0,
    \label{eq:RelaxedN}
\end{equation}
where $N_k$ is the number of particles to be used in iteration $k$ and $N_0$ is the number of particles tracked when not relaxed.  The scalar \mbox{$f\left(\left\|r_{k-1}\right\|\right)$} is the fraction of the number of particles $N_0$ that should be tracked in iteration $k$.  

In Monte Carlo particle transport the parameter $\eta < 1$ is some input value defining when relaxing can occur.  We want to relax the application of \A{} when the residual is smaller than $\eta$ and to track $N_0$ particles when the residual is larger than $\eta$.  To do this we define 
\begin{equation}
    \varepsilon_k = \begin{cases}
        \eta/\left\|r_{k-1}\right\| & \left\|r_{k-1}\right\| < \eta \\
        1 & \left\|r_{k-1}\right\| \geq \eta.
    \end{cases}
    \label{eq:epsilonMC}
\end{equation}
and insert into \Fref{eq:PerturbationMatrix}.  We note that \[\left\|\DA\right\| = \frac{\bm{C}}{\sqrt{N_k}}\] where $\bm{C}$ is just some constant of proportionality, we can obtain an equation for $N_k$,
\begin{equation}
    N_k = \frac{1}{\varepsilon_k^2}\left(\frac{\bm{C}}{\left\|\A\right\|}\right)^2.
    \label{eq:Nk}
\end{equation}

To determine the fraction of particles to be tracked in iteration $k$ we equate the right-hand sides of \Fref{eq:Nk} and \Fref{eq:RelaxedN}, 
\begin{equation}
    f\left(\left\|r_{k-1}\right\|\right) N_0 = \frac{1}{\varepsilon_k^2} \left(\frac{\bm{C}}{\left\|\A\right\|}\right)^2 .
\end{equation}
The term $\left(\sfrac{\bm{C}}{\left\|\A\right\|}\right)^2$ is just some constant which we conveniently choose to be $N_0$.  Therefore the fraction we are seeking must be
\begin{equation}
    f\left(\left\|r_{k-1}\right\|\right) = \frac{1}{\varepsilon_k^2} = 
    \begin{cases}
        \left(\left\|r_{k-1}\right\|/\eta\right)^2 & \left\|r_{k-1}\right\| < \eta \\
        1 & \left\|r_{k-1}\right\| \geq \eta.
    \end{cases}
    \label{eq:NFraction}
\end{equation}
So if the residual is less than $\eta$ the number of particles tracked in an iteration is reduced, but the number of particles tracked in an iteration will never be more than $N_0$.  From \Fref{eq:RelaxedN} we then have
\begin{equation}
    N_k = 
    \begin{cases}
        \left(\left\|r_{k-1}\right\|/\eta\right)^2N_0 & \left\|r_{k-1}\right\| < \eta \\
        N_0 & \left\|r_{k-1}\right\| \geq \eta.
    \end{cases}
    \label{eq:NkFinal}
\end{equation}

This method of relaxation is not the only strategy that can be used.  Here we have a quadratic function.  A less aggressive strategy would be a linear function where
\begin{equation}
    f\left(\left\|r_{k-1}\right\|\right) = \frac{1}{\varepsilon_k} = 
    \begin{cases}
        \left\|r_{k-1}\right\|/\eta & \left\|r_{k-1}\right\| < \eta \\
        1 & \left\|r_{k-1}\right\| \geq \eta.
    \end{cases}
    \label{eq:NkFinalLessAggressive}
\end{equation}
What is important is that as the residual decreases, relaxation increases and fewer particles are tracked in that iteration.  

With the number of particles used in an iteration defined by \Fref{eq:NkFinal} we have a procedure for relaxing the precision to which the fission-transport operator is applied; as the residual decreases---the estimate of the eigenvalue improves---fewer particles are tracked in an iteration.  We expect this relaxation to have no negative effect on the convergence of the eigenvalue estimate and therefore have a positive effect on the figure of merit as the computational expense required to apply the linear operator is decreased with the decrease in the number of particles tracked.

\section{Numerical Results} \label{sec:RelaxedNumericalResults}
To demonstrate the effect of relaxing Arnoldi's method, a similar problem to those already displayed in this thesis is shown.  I will keep the same cross sections (\mbox{$\nu\Sigma_f = 1.0$}, \mbox{$\Sigma_a = 0.2$}, and \mbox{$\Sigma_s = 0.8$}; \mbox{$\Sigma_t = 1.0$}) but the width of the slab will be 50 mfp.  This is a more difficult problem than the thinner slabs because the dominance ratio is 0.9924, which is larger than for the thinner geometries previously used.  

For these simulations the more aggressive relaxation shown in \Fref{eq:NkFinal}  is used and the  relaxation parameter, $\eta$, is varied from 1E-8 to 1.0.  As $\eta$ increases, the number of particles tracked in an iteration/restart decreases.  To keep the total number of particles the same among all simulations, the number of active restarts was increased when necessary.  (The number of inactive restarts is 100 for all the simulations.  Relaxation is done during inactive restarts as well as active restarts.)  This means that for larger values of $\eta$ there will be more eigenvalue estimates and a lower uncertainty in the eigenvalue estimate.  When the uncertainty decreases the figure of merit increases.  

\begin{sidewaysfigure}\centering
    % GNUPLOT: LaTeX picture with Postscript
\begingroup%
\makeatletter%
\newcommand{\GNUPLOTspecial}{%
  \@sanitize\catcode`\%=14\relax\special}%
\setlength{\unitlength}{0.0500bp}%
\begin{picture}(12960,8640)(0,0)%
  {\GNUPLOTspecial{"
%!PS-Adobe-2.0 EPSF-2.0
%%Title: Relaxed.tex
%%Creator: gnuplot 4.3 patchlevel 0
%%CreationDate: Sat Jul 18 21:07:56 2009
%%DocumentFonts: 
%%BoundingBox: 0 0 648 432
%%EndComments
%%BeginProlog
/gnudict 256 dict def
gnudict begin
%
% The following true/false flags may be edited by hand if desired.
% The unit line width and grayscale image gamma correction may also be changed.
%
/Color true def
/Blacktext true def
/Solid false def
/Dashlength 1 def
/Landscape false def
/Level1 false def
/Rounded false def
/ClipToBoundingBox false def
/TransparentPatterns false def
/gnulinewidth 5.000 def
/userlinewidth gnulinewidth def
/Gamma 1.0 def
%
/vshift -66 def
/dl1 {
  10.0 Dashlength mul mul
  Rounded { currentlinewidth 0.75 mul sub dup 0 le { pop 0.01 } if } if
} def
/dl2 {
  10.0 Dashlength mul mul
  Rounded { currentlinewidth 0.75 mul add } if
} def
/hpt_ 31.5 def
/vpt_ 31.5 def
/hpt hpt_ def
/vpt vpt_ def
Level1 {} {
/SDict 10 dict def
systemdict /pdfmark known not {
  userdict /pdfmark systemdict /cleartomark get put
} if
SDict begin [
  /Title (Relaxed.tex)
  /Subject (gnuplot plot)
  /Creator (gnuplot 4.3 patchlevel 0)
  /Author (Jeremy Conlin)
%  /Producer (gnuplot)
%  /Keywords ()
  /CreationDate (Sat Jul 18 21:07:56 2009)
  /DOCINFO pdfmark
end
} ifelse
/doclip {
  ClipToBoundingBox {
    newpath 0 0 moveto 648 0 lineto 648 432 lineto 0 432 lineto closepath
    clip
  } if
} def
%
% Gnuplot Prolog Version 4.2 (November 2007)
%
/M {moveto} bind def
/L {lineto} bind def
/R {rmoveto} bind def
/V {rlineto} bind def
/N {newpath moveto} bind def
/Z {closepath} bind def
/C {setrgbcolor} bind def
/f {rlineto fill} bind def
/Gshow {show} def   % May be redefined later in the file to support UTF-8
/vpt2 vpt 2 mul def
/hpt2 hpt 2 mul def
/Lshow {currentpoint stroke M 0 vshift R 
	Blacktext {gsave 0 setgray show grestore} {show} ifelse} def
/Rshow {currentpoint stroke M dup stringwidth pop neg vshift R
	Blacktext {gsave 0 setgray show grestore} {show} ifelse} def
/Cshow {currentpoint stroke M dup stringwidth pop -2 div vshift R 
	Blacktext {gsave 0 setgray show grestore} {show} ifelse} def
/UP {dup vpt_ mul /vpt exch def hpt_ mul /hpt exch def
  /hpt2 hpt 2 mul def /vpt2 vpt 2 mul def} def
/DL {Color {setrgbcolor Solid {pop []} if 0 setdash}
 {pop pop pop 0 setgray Solid {pop []} if 0 setdash} ifelse} def
/BL {stroke userlinewidth 2 mul setlinewidth
	Rounded {1 setlinejoin 1 setlinecap} if} def
/AL {stroke userlinewidth 2 div setlinewidth
	Rounded {1 setlinejoin 1 setlinecap} if} def
/UL {dup gnulinewidth mul /userlinewidth exch def
	dup 1 lt {pop 1} if 10 mul /udl exch def} def
/PL {stroke userlinewidth setlinewidth
	Rounded {1 setlinejoin 1 setlinecap} if} def
% Default Line colors
/LCw {1 1 1} def
/LCb {0 0 0} def
/LCa {0 0 0} def
/LC0 {1 0 0} def
/LC1 {0 1 0} def
/LC2 {0 0 1} def
/LC3 {1 0 1} def
/LC4 {0 1 1} def
/LC5 {1 1 0} def
/LC6 {0 0 0} def
/LC7 {1 0.3 0} def
/LC8 {0.5 0.5 0.5} def
% Default Line Types
/LTw {PL [] 1 setgray} def
/LTb {BL [] LCb DL} def
/LTa {AL [1 udl mul 2 udl mul] 0 setdash LCa setrgbcolor} def
/LT0 {PL [] LC0 DL} def
/LT1 {PL [4 dl1 2 dl2] LC1 DL} def
/LT2 {PL [2 dl1 3 dl2] LC2 DL} def
/LT3 {PL [1 dl1 1.5 dl2] LC3 DL} def
/LT4 {PL [6 dl1 2 dl2 1 dl1 2 dl2] LC4 DL} def
/LT5 {PL [3 dl1 3 dl2 1 dl1 3 dl2] LC5 DL} def
/LT6 {PL [2 dl1 2 dl2 2 dl1 6 dl2] LC6 DL} def
/LT7 {PL [1 dl1 2 dl2 6 dl1 2 dl2 1 dl1 2 dl2] LC7 DL} def
/LT8 {PL [2 dl1 2 dl2 2 dl1 2 dl2 2 dl1 2 dl2 2 dl1 4 dl2] LC8 DL} def
/Pnt {stroke [] 0 setdash gsave 1 setlinecap M 0 0 V stroke grestore} def
/Dia {stroke [] 0 setdash 2 copy vpt add M
  hpt neg vpt neg V hpt vpt neg V
  hpt vpt V hpt neg vpt V closepath stroke
  Pnt} def
/Pls {stroke [] 0 setdash vpt sub M 0 vpt2 V
  currentpoint stroke M
  hpt neg vpt neg R hpt2 0 V stroke
 } def
/Box {stroke [] 0 setdash 2 copy exch hpt sub exch vpt add M
  0 vpt2 neg V hpt2 0 V 0 vpt2 V
  hpt2 neg 0 V closepath stroke
  Pnt} def
/Crs {stroke [] 0 setdash exch hpt sub exch vpt add M
  hpt2 vpt2 neg V currentpoint stroke M
  hpt2 neg 0 R hpt2 vpt2 V stroke} def
/TriU {stroke [] 0 setdash 2 copy vpt 1.12 mul add M
  hpt neg vpt -1.62 mul V
  hpt 2 mul 0 V
  hpt neg vpt 1.62 mul V closepath stroke
  Pnt} def
/Star {2 copy Pls Crs} def
/BoxF {stroke [] 0 setdash exch hpt sub exch vpt add M
  0 vpt2 neg V hpt2 0 V 0 vpt2 V
  hpt2 neg 0 V closepath fill} def
/TriUF {stroke [] 0 setdash vpt 1.12 mul add M
  hpt neg vpt -1.62 mul V
  hpt 2 mul 0 V
  hpt neg vpt 1.62 mul V closepath fill} def
/TriD {stroke [] 0 setdash 2 copy vpt 1.12 mul sub M
  hpt neg vpt 1.62 mul V
  hpt 2 mul 0 V
  hpt neg vpt -1.62 mul V closepath stroke
  Pnt} def
/TriDF {stroke [] 0 setdash vpt 1.12 mul sub M
  hpt neg vpt 1.62 mul V
  hpt 2 mul 0 V
  hpt neg vpt -1.62 mul V closepath fill} def
/DiaF {stroke [] 0 setdash vpt add M
  hpt neg vpt neg V hpt vpt neg V
  hpt vpt V hpt neg vpt V closepath fill} def
/Pent {stroke [] 0 setdash 2 copy gsave
  translate 0 hpt M 4 {72 rotate 0 hpt L} repeat
  closepath stroke grestore Pnt} def
/PentF {stroke [] 0 setdash gsave
  translate 0 hpt M 4 {72 rotate 0 hpt L} repeat
  closepath fill grestore} def
/Circle {stroke [] 0 setdash 2 copy
  hpt 0 360 arc stroke Pnt} def
/CircleF {stroke [] 0 setdash hpt 0 360 arc fill} def
/C0 {BL [] 0 setdash 2 copy moveto vpt 90 450 arc} bind def
/C1 {BL [] 0 setdash 2 copy moveto
	2 copy vpt 0 90 arc closepath fill
	vpt 0 360 arc closepath} bind def
/C2 {BL [] 0 setdash 2 copy moveto
	2 copy vpt 90 180 arc closepath fill
	vpt 0 360 arc closepath} bind def
/C3 {BL [] 0 setdash 2 copy moveto
	2 copy vpt 0 180 arc closepath fill
	vpt 0 360 arc closepath} bind def
/C4 {BL [] 0 setdash 2 copy moveto
	2 copy vpt 180 270 arc closepath fill
	vpt 0 360 arc closepath} bind def
/C5 {BL [] 0 setdash 2 copy moveto
	2 copy vpt 0 90 arc
	2 copy moveto
	2 copy vpt 180 270 arc closepath fill
	vpt 0 360 arc} bind def
/C6 {BL [] 0 setdash 2 copy moveto
	2 copy vpt 90 270 arc closepath fill
	vpt 0 360 arc closepath} bind def
/C7 {BL [] 0 setdash 2 copy moveto
	2 copy vpt 0 270 arc closepath fill
	vpt 0 360 arc closepath} bind def
/C8 {BL [] 0 setdash 2 copy moveto
	2 copy vpt 270 360 arc closepath fill
	vpt 0 360 arc closepath} bind def
/C9 {BL [] 0 setdash 2 copy moveto
	2 copy vpt 270 450 arc closepath fill
	vpt 0 360 arc closepath} bind def
/C10 {BL [] 0 setdash 2 copy 2 copy moveto vpt 270 360 arc closepath fill
	2 copy moveto
	2 copy vpt 90 180 arc closepath fill
	vpt 0 360 arc closepath} bind def
/C11 {BL [] 0 setdash 2 copy moveto
	2 copy vpt 0 180 arc closepath fill
	2 copy moveto
	2 copy vpt 270 360 arc closepath fill
	vpt 0 360 arc closepath} bind def
/C12 {BL [] 0 setdash 2 copy moveto
	2 copy vpt 180 360 arc closepath fill
	vpt 0 360 arc closepath} bind def
/C13 {BL [] 0 setdash 2 copy moveto
	2 copy vpt 0 90 arc closepath fill
	2 copy moveto
	2 copy vpt 180 360 arc closepath fill
	vpt 0 360 arc closepath} bind def
/C14 {BL [] 0 setdash 2 copy moveto
	2 copy vpt 90 360 arc closepath fill
	vpt 0 360 arc} bind def
/C15 {BL [] 0 setdash 2 copy vpt 0 360 arc closepath fill
	vpt 0 360 arc closepath} bind def
/Rec {newpath 4 2 roll moveto 1 index 0 rlineto 0 exch rlineto
	neg 0 rlineto closepath} bind def
/Square {dup Rec} bind def
/Bsquare {vpt sub exch vpt sub exch vpt2 Square} bind def
/S0 {BL [] 0 setdash 2 copy moveto 0 vpt rlineto BL Bsquare} bind def
/S1 {BL [] 0 setdash 2 copy vpt Square fill Bsquare} bind def
/S2 {BL [] 0 setdash 2 copy exch vpt sub exch vpt Square fill Bsquare} bind def
/S3 {BL [] 0 setdash 2 copy exch vpt sub exch vpt2 vpt Rec fill Bsquare} bind def
/S4 {BL [] 0 setdash 2 copy exch vpt sub exch vpt sub vpt Square fill Bsquare} bind def
/S5 {BL [] 0 setdash 2 copy 2 copy vpt Square fill
	exch vpt sub exch vpt sub vpt Square fill Bsquare} bind def
/S6 {BL [] 0 setdash 2 copy exch vpt sub exch vpt sub vpt vpt2 Rec fill Bsquare} bind def
/S7 {BL [] 0 setdash 2 copy exch vpt sub exch vpt sub vpt vpt2 Rec fill
	2 copy vpt Square fill Bsquare} bind def
/S8 {BL [] 0 setdash 2 copy vpt sub vpt Square fill Bsquare} bind def
/S9 {BL [] 0 setdash 2 copy vpt sub vpt vpt2 Rec fill Bsquare} bind def
/S10 {BL [] 0 setdash 2 copy vpt sub vpt Square fill 2 copy exch vpt sub exch vpt Square fill
	Bsquare} bind def
/S11 {BL [] 0 setdash 2 copy vpt sub vpt Square fill 2 copy exch vpt sub exch vpt2 vpt Rec fill
	Bsquare} bind def
/S12 {BL [] 0 setdash 2 copy exch vpt sub exch vpt sub vpt2 vpt Rec fill Bsquare} bind def
/S13 {BL [] 0 setdash 2 copy exch vpt sub exch vpt sub vpt2 vpt Rec fill
	2 copy vpt Square fill Bsquare} bind def
/S14 {BL [] 0 setdash 2 copy exch vpt sub exch vpt sub vpt2 vpt Rec fill
	2 copy exch vpt sub exch vpt Square fill Bsquare} bind def
/S15 {BL [] 0 setdash 2 copy Bsquare fill Bsquare} bind def
/D0 {gsave translate 45 rotate 0 0 S0 stroke grestore} bind def
/D1 {gsave translate 45 rotate 0 0 S1 stroke grestore} bind def
/D2 {gsave translate 45 rotate 0 0 S2 stroke grestore} bind def
/D3 {gsave translate 45 rotate 0 0 S3 stroke grestore} bind def
/D4 {gsave translate 45 rotate 0 0 S4 stroke grestore} bind def
/D5 {gsave translate 45 rotate 0 0 S5 stroke grestore} bind def
/D6 {gsave translate 45 rotate 0 0 S6 stroke grestore} bind def
/D7 {gsave translate 45 rotate 0 0 S7 stroke grestore} bind def
/D8 {gsave translate 45 rotate 0 0 S8 stroke grestore} bind def
/D9 {gsave translate 45 rotate 0 0 S9 stroke grestore} bind def
/D10 {gsave translate 45 rotate 0 0 S10 stroke grestore} bind def
/D11 {gsave translate 45 rotate 0 0 S11 stroke grestore} bind def
/D12 {gsave translate 45 rotate 0 0 S12 stroke grestore} bind def
/D13 {gsave translate 45 rotate 0 0 S13 stroke grestore} bind def
/D14 {gsave translate 45 rotate 0 0 S14 stroke grestore} bind def
/D15 {gsave translate 45 rotate 0 0 S15 stroke grestore} bind def
/DiaE {stroke [] 0 setdash vpt add M
  hpt neg vpt neg V hpt vpt neg V
  hpt vpt V hpt neg vpt V closepath stroke} def
/BoxE {stroke [] 0 setdash exch hpt sub exch vpt add M
  0 vpt2 neg V hpt2 0 V 0 vpt2 V
  hpt2 neg 0 V closepath stroke} def
/TriUE {stroke [] 0 setdash vpt 1.12 mul add M
  hpt neg vpt -1.62 mul V
  hpt 2 mul 0 V
  hpt neg vpt 1.62 mul V closepath stroke} def
/TriDE {stroke [] 0 setdash vpt 1.12 mul sub M
  hpt neg vpt 1.62 mul V
  hpt 2 mul 0 V
  hpt neg vpt -1.62 mul V closepath stroke} def
/PentE {stroke [] 0 setdash gsave
  translate 0 hpt M 4 {72 rotate 0 hpt L} repeat
  closepath stroke grestore} def
/CircE {stroke [] 0 setdash 
  hpt 0 360 arc stroke} def
/Opaque {gsave closepath 1 setgray fill grestore 0 setgray closepath} def
/DiaW {stroke [] 0 setdash vpt add M
  hpt neg vpt neg V hpt vpt neg V
  hpt vpt V hpt neg vpt V Opaque stroke} def
/BoxW {stroke [] 0 setdash exch hpt sub exch vpt add M
  0 vpt2 neg V hpt2 0 V 0 vpt2 V
  hpt2 neg 0 V Opaque stroke} def
/TriUW {stroke [] 0 setdash vpt 1.12 mul add M
  hpt neg vpt -1.62 mul V
  hpt 2 mul 0 V
  hpt neg vpt 1.62 mul V Opaque stroke} def
/TriDW {stroke [] 0 setdash vpt 1.12 mul sub M
  hpt neg vpt 1.62 mul V
  hpt 2 mul 0 V
  hpt neg vpt -1.62 mul V Opaque stroke} def
/PentW {stroke [] 0 setdash gsave
  translate 0 hpt M 4 {72 rotate 0 hpt L} repeat
  Opaque stroke grestore} def
/CircW {stroke [] 0 setdash 
  hpt 0 360 arc Opaque stroke} def
/BoxFill {gsave Rec 1 setgray fill grestore} def
/Density {
  /Fillden exch def
  currentrgbcolor
  /ColB exch def /ColG exch def /ColR exch def
  /ColR ColR Fillden mul Fillden sub 1 add def
  /ColG ColG Fillden mul Fillden sub 1 add def
  /ColB ColB Fillden mul Fillden sub 1 add def
  ColR ColG ColB setrgbcolor} def
/BoxColFill {gsave Rec PolyFill} def
/PolyFill {gsave Density fill grestore grestore} def
/h {rlineto rlineto rlineto gsave closepath fill grestore} bind def
%
% PostScript Level 1 Pattern Fill routine for rectangles
% Usage: x y w h s a XX PatternFill
%	x,y = lower left corner of box to be filled
%	w,h = width and height of box
%	  a = angle in degrees between lines and x-axis
%	 XX = 0/1 for no/yes cross-hatch
%
/PatternFill {gsave /PFa [ 9 2 roll ] def
  PFa 0 get PFa 2 get 2 div add PFa 1 get PFa 3 get 2 div add translate
  PFa 2 get -2 div PFa 3 get -2 div PFa 2 get PFa 3 get Rec
  gsave 1 setgray fill grestore clip
  currentlinewidth 0.5 mul setlinewidth
  /PFs PFa 2 get dup mul PFa 3 get dup mul add sqrt def
  0 0 M PFa 5 get rotate PFs -2 div dup translate
  0 1 PFs PFa 4 get div 1 add floor cvi
	{PFa 4 get mul 0 M 0 PFs V} for
  0 PFa 6 get ne {
	0 1 PFs PFa 4 get div 1 add floor cvi
	{PFa 4 get mul 0 2 1 roll M PFs 0 V} for
 } if
  stroke grestore} def
%
/languagelevel where
 {pop languagelevel} {1} ifelse
 2 lt
	{/InterpretLevel1 true def}
	{/InterpretLevel1 Level1 def}
 ifelse
%
% PostScript level 2 pattern fill definitions
%
/Level2PatternFill {
/Tile8x8 {/PaintType 2 /PatternType 1 /TilingType 1 /BBox [0 0 8 8] /XStep 8 /YStep 8}
	bind def
/KeepColor {currentrgbcolor [/Pattern /DeviceRGB] setcolorspace} bind def
<< Tile8x8
 /PaintProc {0.5 setlinewidth pop 0 0 M 8 8 L 0 8 M 8 0 L stroke} 
>> matrix makepattern
/Pat1 exch def
<< Tile8x8
 /PaintProc {0.5 setlinewidth pop 0 0 M 8 8 L 0 8 M 8 0 L stroke
	0 4 M 4 8 L 8 4 L 4 0 L 0 4 L stroke}
>> matrix makepattern
/Pat2 exch def
<< Tile8x8
 /PaintProc {0.5 setlinewidth pop 0 0 M 0 8 L
	8 8 L 8 0 L 0 0 L fill}
>> matrix makepattern
/Pat3 exch def
<< Tile8x8
 /PaintProc {0.5 setlinewidth pop -4 8 M 8 -4 L
	0 12 M 12 0 L stroke}
>> matrix makepattern
/Pat4 exch def
<< Tile8x8
 /PaintProc {0.5 setlinewidth pop -4 0 M 8 12 L
	0 -4 M 12 8 L stroke}
>> matrix makepattern
/Pat5 exch def
<< Tile8x8
 /PaintProc {0.5 setlinewidth pop -2 8 M 4 -4 L
	0 12 M 8 -4 L 4 12 M 10 0 L stroke}
>> matrix makepattern
/Pat6 exch def
<< Tile8x8
 /PaintProc {0.5 setlinewidth pop -2 0 M 4 12 L
	0 -4 M 8 12 L 4 -4 M 10 8 L stroke}
>> matrix makepattern
/Pat7 exch def
<< Tile8x8
 /PaintProc {0.5 setlinewidth pop 8 -2 M -4 4 L
	12 0 M -4 8 L 12 4 M 0 10 L stroke}
>> matrix makepattern
/Pat8 exch def
<< Tile8x8
 /PaintProc {0.5 setlinewidth pop 0 -2 M 12 4 L
	-4 0 M 12 8 L -4 4 M 8 10 L stroke}
>> matrix makepattern
/Pat9 exch def
/Pattern1 {PatternBgnd KeepColor Pat1 setpattern} bind def
/Pattern2 {PatternBgnd KeepColor Pat2 setpattern} bind def
/Pattern3 {PatternBgnd KeepColor Pat3 setpattern} bind def
/Pattern4 {PatternBgnd KeepColor Landscape {Pat5} {Pat4} ifelse setpattern} bind def
/Pattern5 {PatternBgnd KeepColor Landscape {Pat4} {Pat5} ifelse setpattern} bind def
/Pattern6 {PatternBgnd KeepColor Landscape {Pat9} {Pat6} ifelse setpattern} bind def
/Pattern7 {PatternBgnd KeepColor Landscape {Pat8} {Pat7} ifelse setpattern} bind def
} def
%
%
%End of PostScript Level 2 code
%
/PatternBgnd {
  TransparentPatterns {} {gsave 1 setgray fill grestore} ifelse
} def
%
% Substitute for Level 2 pattern fill codes with
% grayscale if Level 2 support is not selected.
%
/Level1PatternFill {
/Pattern1 {0.250 Density} bind def
/Pattern2 {0.500 Density} bind def
/Pattern3 {0.750 Density} bind def
/Pattern4 {0.125 Density} bind def
/Pattern5 {0.375 Density} bind def
/Pattern6 {0.625 Density} bind def
/Pattern7 {0.875 Density} bind def
} def
%
% Now test for support of Level 2 code
%
Level1 {Level1PatternFill} {Level2PatternFill} ifelse
%
/Symbol-Oblique /Symbol findfont [1 0 .167 1 0 0] makefont
dup length dict begin {1 index /FID eq {pop pop} {def} ifelse} forall
currentdict end definefont pop
end
%%EndProlog
gnudict begin
gsave
doclip
0 0 translate
0.050 0.050 scale
0 setgray
newpath
1.000 UL
LTb
1220 640 M
63 0 V
11276 0 R
-63 0 V
1220 1610 M
63 0 V
11276 0 R
-63 0 V
1220 2580 M
63 0 V
11276 0 R
-63 0 V
1220 3550 M
63 0 V
11276 0 R
-63 0 V
1220 4520 M
63 0 V
11276 0 R
-63 0 V
1220 5489 M
63 0 V
11276 0 R
-63 0 V
1220 6459 M
63 0 V
11276 0 R
-63 0 V
1220 7429 M
63 0 V
11276 0 R
-63 0 V
1220 8399 M
63 0 V
11276 0 R
-63 0 V
1220 640 M
0 63 V
0 7696 R
0 -63 V
1641 640 M
0 31 V
0 7728 R
0 -31 V
2197 640 M
0 31 V
0 7728 R
0 -31 V
2482 640 M
0 31 V
0 7728 R
0 -31 V
2617 640 M
0 63 V
0 7696 R
0 -63 V
3038 640 M
0 31 V
0 7728 R
0 -31 V
3594 640 M
0 31 V
0 7728 R
0 -31 V
3880 640 M
0 31 V
0 7728 R
0 -31 V
4015 640 M
0 63 V
0 7696 R
0 -63 V
4436 640 M
0 31 V
0 7728 R
0 -31 V
4992 640 M
0 31 V
0 7728 R
0 -31 V
5277 640 M
0 31 V
0 7728 R
0 -31 V
5412 640 M
0 63 V
0 7696 R
0 -63 V
5833 640 M
0 31 V
0 7728 R
0 -31 V
6389 640 M
0 31 V
0 7728 R
0 -31 V
6674 640 M
0 31 V
0 7728 R
0 -31 V
6810 640 M
0 63 V
0 7696 R
0 -63 V
7231 640 M
0 31 V
stroke 7231 671 M
0 7728 R
0 -31 V
7787 640 M
0 31 V
0 7728 R
0 -31 V
8072 640 M
0 31 V
0 7728 R
0 -31 V
8207 640 M
0 63 V
0 7696 R
0 -63 V
8628 640 M
0 31 V
0 7728 R
0 -31 V
9184 640 M
0 31 V
0 7728 R
0 -31 V
9469 640 M
0 31 V
0 7728 R
0 -31 V
9605 640 M
0 63 V
0 7696 R
0 -63 V
10026 640 M
0 31 V
0 7728 R
0 -31 V
10582 640 M
0 31 V
0 7728 R
0 -31 V
10867 640 M
0 31 V
0 7728 R
0 -31 V
11002 640 M
0 63 V
0 7696 R
0 -63 V
11423 640 M
0 31 V
0 7728 R
0 -31 V
11979 640 M
0 31 V
0 7728 R
0 -31 V
12264 640 M
0 31 V
0 7728 R
0 -31 V
12400 640 M
0 63 V
0 7696 R
0 -63 V
stroke
1220 8399 N
0 -7759 V
11339 0 V
0 7759 V
-11339 0 V
Z stroke
LCb setrgbcolor
LTb
LCb setrgbcolor
LTb
LCb setrgbcolor
LTb
LCb setrgbcolor
LTb
1.000 UP
1.000 UL
LTb
1.000 UL
LTb
1340 7416 N
0 920 V
1383 0 V
0 -920 V
-1383 0 V
Z stroke
1340 8336 M
1383 0 V
1.000 UP
stroke
LT0
LCb setrgbcolor
LT0
2060 8116 M
543 0 V
-543 31 R
0 -62 V
543 62 R
0 -62 V
1220 3298 M
556 -2 V
421 1 V
246 -4 V
174 3 V
557 13 V
420 -9 V
246 -5 V
731 5 V
667 -6 V
174 10 V
557 3 V
420 -19 V
246 -8 V
175 12 V
556 0 V
421 4 V
246 -10 V
174 7 V
556 -2 V
421 45 V
246 15 V
175 32 V
977 438 V
420 543 V
977 1911 V
421 1736 V
1220 3289 M
0 19 V
-31 -19 R
62 0 V
-62 19 R
62 0 V
525 -24 R
0 25 V
-31 -25 R
62 0 V
-62 25 R
62 0 V
390 -23 R
0 22 V
-31 -22 R
62 0 V
-62 22 R
62 0 V
215 -25 R
0 20 V
-31 -20 R
62 0 V
-62 20 R
62 0 V
143 -17 R
0 19 V
-31 -19 R
62 0 V
-62 19 R
62 0 V
526 -6 R
0 20 V
-31 -20 R
62 0 V
-62 20 R
62 0 V
389 -29 R
0 21 V
-31 -21 R
62 0 V
-62 21 R
62 0 V
215 -26 R
0 19 V
-31 -19 R
62 0 V
-62 19 R
62 0 V
700 -14 R
0 20 V
-31 -20 R
62 0 V
-62 20 R
62 0 V
636 -26 R
0 19 V
-31 -19 R
62 0 V
-62 19 R
62 0 V
143 -10 R
0 22 V
-31 -22 R
62 0 V
-62 22 R
62 0 V
526 -18 R
0 21 V
-31 -21 R
62 0 V
-62 21 R
62 0 V
stroke 6000 3318 M
389 -38 R
0 16 V
-31 -16 R
62 0 V
-62 16 R
62 0 V
215 -26 R
0 20 V
-31 -20 R
62 0 V
-62 20 R
62 0 V
144 -9 R
0 21 V
-31 -21 R
62 0 V
-62 21 R
62 0 V
525 -18 R
0 16 V
-31 -16 R
62 0 V
-62 16 R
62 0 V
390 -14 R
0 19 V
-31 -19 R
62 0 V
-62 19 R
62 0 V
215 -28 R
0 17 V
-31 -17 R
62 0 V
-62 17 R
62 0 V
143 -9 R
0 15 V
-31 -15 R
62 0 V
-62 15 R
62 0 V
525 -17 R
0 15 V
-31 -15 R
62 0 V
-62 15 R
62 0 V
390 5 R
0 67 V
-31 -67 R
62 0 V
-62 67 R
62 0 V
215 -41 R
0 44 V
-31 -44 R
62 0 V
-62 44 R
62 0 V
144 -25 R
0 70 V
-31 -70 R
62 0 V
-62 70 R
62 0 V
946 369 R
0 68 V
-31 -68 R
62 0 V
-62 68 R
62 0 V
389 472 R
0 75 V
-31 -75 R
62 0 V
-62 75 R
62 0 V
946 1832 R
0 82 V
-31 -82 R
62 0 V
-62 82 R
62 0 V
390 1654 R
0 82 V
-31 -82 R
62 0 V
-62 82 R
62 0 V
1220 3298 Pls
1776 3296 Pls
2197 3297 Pls
2443 3293 Pls
2617 3296 Pls
3174 3309 Pls
3594 3300 Pls
3840 3295 Pls
4571 3300 Pls
5238 3294 Pls
5412 3304 Pls
5969 3307 Pls
6389 3288 Pls
6635 3280 Pls
6810 3292 Pls
7366 3292 Pls
7787 3296 Pls
8033 3286 Pls
8207 3293 Pls
8763 3291 Pls
9184 3336 Pls
9430 3351 Pls
9605 3383 Pls
10582 3821 Pls
11002 4364 Pls
11979 6275 Pls
12400 8011 Pls
2331 8116 Pls
1.000 UP
1.000 UL
LT0
LC1 setrgbcolor
LCb setrgbcolor
LT0
LC1 setrgbcolor
2060 7876 M
543 0 V
-543 31 R
0 -62 V
543 62 R
0 -62 V
1220 2562 M
556 -7 V
421 1 V
246 -15 V
174 15 V
557 -4 V
420 3 V
246 -12 V
731 -8 V
667 -9 V
174 28 V
557 6 V
420 -9 V
246 19 V
175 -25 V
556 15 V
421 -19 V
246 13 V
174 -12 V
556 9 V
421 8 V
246 5 V
175 31 V
977 186 V
420 207 V
977 1221 V
421 1135 V
1220 2552 M
0 20 V
-31 -20 R
62 0 V
-62 20 R
62 0 V
525 -28 R
0 21 V
-31 -21 R
62 0 V
-62 21 R
62 0 V
390 -18 R
0 17 V
-31 -17 R
62 0 V
-62 17 R
62 0 V
215 -33 R
0 20 V
-31 -20 R
62 0 V
-62 20 R
62 0 V
143 -5 R
0 19 V
-31 -19 R
62 0 V
-62 19 R
62 0 V
526 -24 R
0 21 V
-31 -21 R
62 0 V
-62 21 R
62 0 V
389 -18 R
0 21 V
-31 -21 R
62 0 V
-62 21 R
62 0 V
215 -33 R
0 22 V
-31 -22 R
62 0 V
-62 22 R
62 0 V
700 -30 R
0 22 V
-31 -22 R
62 0 V
-62 22 R
62 0 V
636 -29 R
0 19 V
-31 -19 R
62 0 V
-62 19 R
62 0 V
143 6 R
0 23 V
-31 -23 R
62 0 V
-62 23 R
62 0 V
526 -15 R
0 21 V
-31 -21 R
62 0 V
-62 21 R
62 0 V
stroke 6000 2571 M
389 -30 R
0 19 V
-31 -19 R
62 0 V
-62 19 R
62 0 V
215 0 R
0 21 V
-31 -21 R
62 0 V
-62 21 R
62 0 V
144 -46 R
0 20 V
-31 -20 R
62 0 V
-62 20 R
62 0 V
525 -5 R
0 20 V
-31 -20 R
62 0 V
-62 20 R
62 0 V
390 -39 R
0 20 V
-31 -20 R
62 0 V
-62 20 R
62 0 V
215 -6 R
0 17 V
-31 -17 R
62 0 V
-62 17 R
62 0 V
143 -28 R
0 16 V
-31 -16 R
62 0 V
-62 16 R
62 0 V
525 -8 R
0 18 V
-31 -18 R
62 0 V
-62 18 R
62 0 V
390 -11 R
0 21 V
-31 -21 R
62 0 V
-62 21 R
62 0 V
215 -22 R
0 32 V
-31 -32 R
62 0 V
-62 32 R
62 0 V
144 2 R
0 27 V
-31 -27 R
62 0 V
-62 27 R
62 0 V
946 163 R
0 17 V
-31 -17 R
62 0 V
-62 17 R
62 0 V
389 192 R
0 14 V
-31 -14 R
62 0 V
-62 14 R
62 0 V
946 1203 R
0 22 V
-31 -22 R
62 0 V
-62 22 R
62 0 V
390 1106 R
0 36 V
-31 -36 R
62 0 V
-62 36 R
62 0 V
1220 2562 Crs
1776 2555 Crs
2197 2556 Crs
2443 2541 Crs
2617 2556 Crs
3174 2552 Crs
3594 2555 Crs
3840 2543 Crs
4571 2535 Crs
5238 2526 Crs
5412 2554 Crs
5969 2560 Crs
6389 2551 Crs
6635 2570 Crs
6810 2545 Crs
7366 2560 Crs
7787 2541 Crs
8033 2554 Crs
8207 2542 Crs
8763 2551 Crs
9184 2559 Crs
9430 2564 Crs
9605 2595 Crs
10582 2781 Crs
11002 2988 Crs
11979 4209 Crs
12400 5344 Crs
2331 7876 Crs
1.000 UP
1.000 UL
LT0
LC2 setrgbcolor
LCb setrgbcolor
LT0
LC2 setrgbcolor
2060 7636 M
543 0 V
-543 31 R
0 -62 V
543 62 R
0 -62 V
1220 1340 M
556 -2 V
421 -6 V
246 7 V
174 -7 V
557 7 V
420 -14 V
246 21 V
731 -13 V
667 -3 V
174 18 V
557 -39 V
420 33 V
246 -8 V
175 10 V
556 -1 V
421 -8 V
246 8 V
174 -11 V
556 13 V
421 16 V
246 46 V
175 9 V
977 66 V
420 98 V
977 817 V
421 733 V
1220 1332 M
0 17 V
-31 -17 R
62 0 V
-62 17 R
62 0 V
525 -20 R
0 18 V
-31 -18 R
62 0 V
-62 18 R
62 0 V
390 -24 R
0 19 V
-31 -19 R
62 0 V
-62 19 R
62 0 V
215 -13 R
0 20 V
-31 -20 R
62 0 V
-62 20 R
62 0 V
143 -27 R
0 19 V
-31 -19 R
62 0 V
-62 19 R
62 0 V
526 -12 R
0 20 V
-31 -20 R
62 0 V
-62 20 R
62 0 V
389 -35 R
0 22 V
-31 -22 R
62 0 V
-62 22 R
62 0 V
215 1 R
0 19 V
-31 -19 R
62 0 V
-62 19 R
62 0 V
700 -34 R
0 23 V
-31 -23 R
62 0 V
-62 23 R
62 0 V
636 -24 R
0 17 V
-31 -17 R
62 0 V
-62 17 R
62 0 V
143 0 R
0 20 V
-31 -20 R
62 0 V
-62 20 R
62 0 V
526 -58 R
0 19 V
-31 -19 R
62 0 V
-62 19 R
62 0 V
stroke 6000 1319 M
389 13 R
0 20 V
-31 -20 R
62 0 V
-62 20 R
62 0 V
215 -27 R
0 19 V
-31 -19 R
62 0 V
-62 19 R
62 0 V
144 -8 R
0 16 V
-31 -16 R
62 0 V
-62 16 R
62 0 V
525 -19 R
0 20 V
-31 -20 R
62 0 V
-62 20 R
62 0 V
390 -30 R
0 24 V
-31 -24 R
62 0 V
-62 24 R
62 0 V
215 -12 R
0 16 V
-31 -16 R
62 0 V
-62 16 R
62 0 V
143 -26 R
0 15 V
-31 -15 R
62 0 V
-62 15 R
62 0 V
525 -5 R
0 20 V
-31 -20 R
62 0 V
-62 20 R
62 0 V
390 -11 R
0 33 V
-31 -33 R
62 0 V
-62 33 R
62 0 V
215 11 R
0 38 V
-31 -38 R
62 0 V
-62 38 R
62 0 V
144 -30 R
0 40 V
-31 -40 R
62 0 V
-62 40 R
62 0 V
946 36 R
0 20 V
-31 -20 R
62 0 V
-62 20 R
62 0 V
389 75 R
0 26 V
-31 -26 R
62 0 V
-62 26 R
62 0 V
946 790 R
0 29 V
-31 -29 R
62 0 V
-62 29 R
62 0 V
390 704 R
0 28 V
-31 -28 R
62 0 V
-62 28 R
62 0 V
1220 1340 Star
1776 1338 Star
2197 1332 Star
2443 1339 Star
2617 1332 Star
3174 1339 Star
3594 1325 Star
3840 1346 Star
4571 1333 Star
5238 1330 Star
5412 1348 Star
5969 1309 Star
6389 1342 Star
6635 1334 Star
6810 1344 Star
7366 1343 Star
7787 1335 Star
8033 1343 Star
8207 1332 Star
8763 1345 Star
9184 1361 Star
9430 1407 Star
9605 1416 Star
10582 1482 Star
11002 1580 Star
11979 2397 Star
12400 3130 Star
2331 7636 Star
2.000 UL
LTb
1220 3298 M
115 0 V
114 0 V
115 0 V
114 0 V
115 0 V
114 0 V
115 0 V
114 0 V
115 0 V
114 0 V
115 0 V
114 0 V
115 0 V
114 0 V
115 0 V
115 0 V
114 0 V
115 0 V
114 0 V
115 0 V
114 0 V
115 0 V
114 0 V
115 0 V
114 0 V
115 0 V
114 0 V
115 0 V
115 0 V
114 0 V
115 0 V
114 0 V
115 0 V
114 0 V
115 0 V
114 0 V
115 0 V
114 0 V
115 0 V
114 0 V
115 0 V
114 0 V
115 0 V
115 0 V
114 0 V
115 0 V
114 0 V
115 0 V
114 0 V
115 0 V
114 0 V
115 0 V
114 0 V
115 0 V
114 0 V
115 0 V
115 0 V
114 0 V
115 0 V
114 0 V
115 0 V
114 0 V
115 0 V
114 0 V
115 0 V
114 0 V
115 0 V
114 0 V
115 0 V
114 0 V
115 0 V
115 0 V
114 0 V
115 0 V
114 0 V
115 0 V
114 0 V
115 0 V
114 0 V
115 0 V
114 0 V
115 0 V
114 0 V
115 0 V
115 0 V
114 0 V
115 0 V
114 0 V
115 0 V
114 0 V
115 0 V
114 0 V
115 0 V
114 0 V
115 0 V
114 0 V
115 0 V
114 0 V
115 0 V
1220 2562 M
115 0 V
114 0 V
115 0 V
114 0 V
stroke 1678 2562 M
115 0 V
114 0 V
115 0 V
114 0 V
115 0 V
114 0 V
115 0 V
114 0 V
115 0 V
114 0 V
115 0 V
115 0 V
114 0 V
115 0 V
114 0 V
115 0 V
114 0 V
115 0 V
114 0 V
115 0 V
114 0 V
115 0 V
114 0 V
115 0 V
115 0 V
114 0 V
115 0 V
114 0 V
115 0 V
114 0 V
115 0 V
114 0 V
115 0 V
114 0 V
115 0 V
114 0 V
115 0 V
114 0 V
115 0 V
115 0 V
114 0 V
115 0 V
114 0 V
115 0 V
114 0 V
115 0 V
114 0 V
115 0 V
114 0 V
115 0 V
114 0 V
115 0 V
115 0 V
114 0 V
115 0 V
114 0 V
115 0 V
114 0 V
115 0 V
114 0 V
115 0 V
114 0 V
115 0 V
114 0 V
115 0 V
114 0 V
115 0 V
115 0 V
114 0 V
115 0 V
114 0 V
115 0 V
114 0 V
115 0 V
114 0 V
115 0 V
114 0 V
115 0 V
114 0 V
115 0 V
115 0 V
114 0 V
115 0 V
114 0 V
115 0 V
114 0 V
115 0 V
114 0 V
115 0 V
114 0 V
115 0 V
114 0 V
115 0 V
114 0 V
115 0 V
1220 1340 M
115 0 V
114 0 V
115 0 V
114 0 V
115 0 V
114 0 V
115 0 V
114 0 V
stroke 2136 1340 M
115 0 V
114 0 V
115 0 V
114 0 V
115 0 V
114 0 V
115 0 V
115 0 V
114 0 V
115 0 V
114 0 V
115 0 V
114 0 V
115 0 V
114 0 V
115 0 V
114 0 V
115 0 V
114 0 V
115 0 V
115 0 V
114 0 V
115 0 V
114 0 V
115 0 V
114 0 V
115 0 V
114 0 V
115 0 V
114 0 V
115 0 V
114 0 V
115 0 V
114 0 V
115 0 V
115 0 V
114 0 V
115 0 V
114 0 V
115 0 V
114 0 V
115 0 V
114 0 V
115 0 V
114 0 V
115 0 V
114 0 V
115 0 V
115 0 V
114 0 V
115 0 V
114 0 V
115 0 V
114 0 V
115 0 V
114 0 V
115 0 V
114 0 V
115 0 V
114 0 V
115 0 V
114 0 V
115 0 V
115 0 V
114 0 V
115 0 V
114 0 V
115 0 V
114 0 V
115 0 V
114 0 V
115 0 V
114 0 V
115 0 V
114 0 V
115 0 V
115 0 V
114 0 V
115 0 V
114 0 V
115 0 V
114 0 V
115 0 V
114 0 V
115 0 V
114 0 V
115 0 V
114 0 V
115 0 V
114 0 V
115 0 V
stroke
1.000 UL
LTb
1220 8399 N
0 -7759 V
11339 0 V
0 7759 V
-11339 0 V
Z stroke
1.000 UP
1.000 UL
LTb
stroke
grestore
end
showpage
  }}%
  \put(1940,7636){\makebox(0,0)[r]{\strut{}$\lambda_2$}}%
  \put(1940,7876){\makebox(0,0)[r]{\strut{}$\lambda_1$}}%
  \put(1940,8116){\makebox(0,0)[r]{\strut{}$\lambda_0$}}%
  \put(6889,140){\makebox(0,0){\strut{}Relaxation Parameter $\eta$}}%
  \put(280,4519){%
  \special{ps: gsave currentpoint currentpoint translate
630 rotate neg exch neg exch translate}%
  \makebox(0,0){\strut{}Eigenvalue Estimate}%
  \special{ps: currentpoint grestore moveto}%
  }%
  \put(12400,440){\makebox(0,0){\strut{} 1}}%
  \put(11002,440){\makebox(0,0){\strut{} 0.1}}%
  \put(9605,440){\makebox(0,0){\strut{} 0.01}}%
  \put(8207,440){\makebox(0,0){\strut{} 0.001}}%
  \put(6810,440){\makebox(0,0){\strut{} 0.0001}}%
  \put(5412,440){\makebox(0,0){\strut{} 1e-05}}%
  \put(4015,440){\makebox(0,0){\strut{} 1e-06}}%
  \put(2617,440){\makebox(0,0){\strut{} 1e-07}}%
  \put(1220,440){\makebox(0,0){\strut{} 1e-08}}%
  \put(1100,8399){\makebox(0,0)[r]{\strut{} 1.05}}%
  \put(1100,7429){\makebox(0,0)[r]{\strut{} 1.04}}%
  \put(1100,6459){\makebox(0,0)[r]{\strut{} 1.03}}%
  \put(1100,5489){\makebox(0,0)[r]{\strut{} 1.02}}%
  \put(1100,4520){\makebox(0,0)[r]{\strut{} 1.01}}%
  \put(1100,3550){\makebox(0,0)[r]{\strut{} 1}}%
  \put(1100,2580){\makebox(0,0)[r]{\strut{} 0.99}}%
  \put(1100,1610){\makebox(0,0)[r]{\strut{} 0.98}}%
  \put(1100,640){\makebox(0,0)[r]{\strut{} 0.97}}%
\end{picture}%
\endgroup
\endinput

    \caption{Eigenvalue estimates for the fundamental and first two harmonics for varying values of the relaxation parameter $\eta$.  The number of particles tracked in a non-relaxed iteration is 5E5.  The heavy lines are the reference eigenvalues from \cite{Garis:1991One-s-0} and \cite{Dahl:1979Eigen-0}.}
    \label{fig:RelaxedArnoldi}
\end{sidewaysfigure}

If what Bouras and Frayss\'{e} suggest for relaxed Arnoldi is valid for Monte Carlo criticality calculations, we expect to see the estimates of the eigenvalue unaffected by relaxation.  In addition, we expect the figure of merit to increase when relaxation increases (i.e. $\eta$ becomes larger) because fewer particles are being tracked in an iteration and therefore less time required to calculate an eigenvalue estimate.  This doesn't hold for every value of $\eta$; eventually it will become so large the application of \A{} is so imprecise---too relaxed---the eigenvalue estimate is just wrong.  

The eigenvalue estimates as a function of relaxation parameter are graphed in \Fref{fig:RelaxedArnoldi}.  The heavy lines are the eigenvalue estimates when no relaxation is used.    We see that the eigenvalue estimates all fall within one standard deviation of the estimate calculated without any relaxation (shown in the dark heavy lines) until the relaxation becomes too great at around \mbox{$\eta = 0.005$}.  

In \Fref{fig:MoreAggFOM} the fundamental eigenvalue estimates are plotted along with the FOM for varying values of $\eta$.  The dashed line in the \Fref{fig:MoreAggFOM} is the figure of merit when Arnoldi's method is not relaxed.  It appears that no relaxation ($\eta = 0.0$) is better than a little bit of relaxation ($\eta = $ small).  There are a few values for $\eta$ which show about a 50\% increase in the figure of merit.  However the range of $\eta$ where there is an improvement over no relaxation is small.  Also, at the same value of $\eta$ where the eigenvalue estimate starts to diverge from the true value we see the figure of merit decrease by 1--2 orders of magnitude.  
\begin{sidewaysfigure}\centering
    % GNUPLOT: LaTeX picture with Postscript
\begingroup%
\makeatletter%
\newcommand{\GNUPLOTspecial}{%
  \@sanitize\catcode`\%=14\relax\special}%
\setlength{\unitlength}{0.0500bp}%
\begin{picture}(12960,8640)(0,0)%
  {\GNUPLOTspecial{"
%!PS-Adobe-2.0 EPSF-2.0
%%Title: MoreAggFOM.tex
%%Creator: gnuplot 4.3 patchlevel 0
%%CreationDate: Wed Aug 12 17:35:08 2009
%%DocumentFonts: 
%%BoundingBox: 0 0 648 432
%%EndComments
%%BeginProlog
/gnudict 256 dict def
gnudict begin
%
% The following true/false flags may be edited by hand if desired.
% The unit line width and grayscale image gamma correction may also be changed.
%
/Color true def
/Blacktext true def
/Solid false def
/Dashlength 1 def
/Landscape false def
/Level1 false def
/Rounded false def
/ClipToBoundingBox false def
/TransparentPatterns false def
/gnulinewidth 5.000 def
/userlinewidth gnulinewidth def
/Gamma 1.0 def
%
/vshift -66 def
/dl1 {
  10.0 Dashlength mul mul
  Rounded { currentlinewidth 0.75 mul sub dup 0 le { pop 0.01 } if } if
} def
/dl2 {
  10.0 Dashlength mul mul
  Rounded { currentlinewidth 0.75 mul add } if
} def
/hpt_ 31.5 def
/vpt_ 31.5 def
/hpt hpt_ def
/vpt vpt_ def
Level1 {} {
/SDict 10 dict def
systemdict /pdfmark known not {
  userdict /pdfmark systemdict /cleartomark get put
} if
SDict begin [
  /Title (MoreAggFOM.tex)
  /Subject (gnuplot plot)
  /Creator (gnuplot 4.3 patchlevel 0)
  /Author (Jeremy Conlin)
%  /Producer (gnuplot)
%  /Keywords ()
  /CreationDate (Wed Aug 12 17:35:08 2009)
  /DOCINFO pdfmark
end
} ifelse
/doclip {
  ClipToBoundingBox {
    newpath 0 0 moveto 648 0 lineto 648 432 lineto 0 432 lineto closepath
    clip
  } if
} def
%
% Gnuplot Prolog Version 4.2 (November 2007)
%
/M {moveto} bind def
/L {lineto} bind def
/R {rmoveto} bind def
/V {rlineto} bind def
/N {newpath moveto} bind def
/Z {closepath} bind def
/C {setrgbcolor} bind def
/f {rlineto fill} bind def
/Gshow {show} def   % May be redefined later in the file to support UTF-8
/vpt2 vpt 2 mul def
/hpt2 hpt 2 mul def
/Lshow {currentpoint stroke M 0 vshift R 
	Blacktext {gsave 0 setgray show grestore} {show} ifelse} def
/Rshow {currentpoint stroke M dup stringwidth pop neg vshift R
	Blacktext {gsave 0 setgray show grestore} {show} ifelse} def
/Cshow {currentpoint stroke M dup stringwidth pop -2 div vshift R 
	Blacktext {gsave 0 setgray show grestore} {show} ifelse} def
/UP {dup vpt_ mul /vpt exch def hpt_ mul /hpt exch def
  /hpt2 hpt 2 mul def /vpt2 vpt 2 mul def} def
/DL {Color {setrgbcolor Solid {pop []} if 0 setdash}
 {pop pop pop 0 setgray Solid {pop []} if 0 setdash} ifelse} def
/BL {stroke userlinewidth 2 mul setlinewidth
	Rounded {1 setlinejoin 1 setlinecap} if} def
/AL {stroke userlinewidth 2 div setlinewidth
	Rounded {1 setlinejoin 1 setlinecap} if} def
/UL {dup gnulinewidth mul /userlinewidth exch def
	dup 1 lt {pop 1} if 10 mul /udl exch def} def
/PL {stroke userlinewidth setlinewidth
	Rounded {1 setlinejoin 1 setlinecap} if} def
% Default Line colors
/LCw {1 1 1} def
/LCb {0 0 0} def
/LCa {0 0 0} def
/LC0 {1 0 0} def
/LC1 {0 1 0} def
/LC2 {0 0 1} def
/LC3 {1 0 1} def
/LC4 {0 1 1} def
/LC5 {1 1 0} def
/LC6 {0 0 0} def
/LC7 {1 0.3 0} def
/LC8 {0.5 0.5 0.5} def
% Default Line Types
/LTw {PL [] 1 setgray} def
/LTb {BL [] LCb DL} def
/LTa {AL [1 udl mul 2 udl mul] 0 setdash LCa setrgbcolor} def
/LT0 {PL [] LC0 DL} def
/LT1 {PL [4 dl1 2 dl2] LC1 DL} def
/LT2 {PL [2 dl1 3 dl2] LC2 DL} def
/LT3 {PL [1 dl1 1.5 dl2] LC3 DL} def
/LT4 {PL [6 dl1 2 dl2 1 dl1 2 dl2] LC4 DL} def
/LT5 {PL [3 dl1 3 dl2 1 dl1 3 dl2] LC5 DL} def
/LT6 {PL [2 dl1 2 dl2 2 dl1 6 dl2] LC6 DL} def
/LT7 {PL [1 dl1 2 dl2 6 dl1 2 dl2 1 dl1 2 dl2] LC7 DL} def
/LT8 {PL [2 dl1 2 dl2 2 dl1 2 dl2 2 dl1 2 dl2 2 dl1 4 dl2] LC8 DL} def
/Pnt {stroke [] 0 setdash gsave 1 setlinecap M 0 0 V stroke grestore} def
/Dia {stroke [] 0 setdash 2 copy vpt add M
  hpt neg vpt neg V hpt vpt neg V
  hpt vpt V hpt neg vpt V closepath stroke
  Pnt} def
/Pls {stroke [] 0 setdash vpt sub M 0 vpt2 V
  currentpoint stroke M
  hpt neg vpt neg R hpt2 0 V stroke
 } def
/Box {stroke [] 0 setdash 2 copy exch hpt sub exch vpt add M
  0 vpt2 neg V hpt2 0 V 0 vpt2 V
  hpt2 neg 0 V closepath stroke
  Pnt} def
/Crs {stroke [] 0 setdash exch hpt sub exch vpt add M
  hpt2 vpt2 neg V currentpoint stroke M
  hpt2 neg 0 R hpt2 vpt2 V stroke} def
/TriU {stroke [] 0 setdash 2 copy vpt 1.12 mul add M
  hpt neg vpt -1.62 mul V
  hpt 2 mul 0 V
  hpt neg vpt 1.62 mul V closepath stroke
  Pnt} def
/Star {2 copy Pls Crs} def
/BoxF {stroke [] 0 setdash exch hpt sub exch vpt add M
  0 vpt2 neg V hpt2 0 V 0 vpt2 V
  hpt2 neg 0 V closepath fill} def
/TriUF {stroke [] 0 setdash vpt 1.12 mul add M
  hpt neg vpt -1.62 mul V
  hpt 2 mul 0 V
  hpt neg vpt 1.62 mul V closepath fill} def
/TriD {stroke [] 0 setdash 2 copy vpt 1.12 mul sub M
  hpt neg vpt 1.62 mul V
  hpt 2 mul 0 V
  hpt neg vpt -1.62 mul V closepath stroke
  Pnt} def
/TriDF {stroke [] 0 setdash vpt 1.12 mul sub M
  hpt neg vpt 1.62 mul V
  hpt 2 mul 0 V
  hpt neg vpt -1.62 mul V closepath fill} def
/DiaF {stroke [] 0 setdash vpt add M
  hpt neg vpt neg V hpt vpt neg V
  hpt vpt V hpt neg vpt V closepath fill} def
/Pent {stroke [] 0 setdash 2 copy gsave
  translate 0 hpt M 4 {72 rotate 0 hpt L} repeat
  closepath stroke grestore Pnt} def
/PentF {stroke [] 0 setdash gsave
  translate 0 hpt M 4 {72 rotate 0 hpt L} repeat
  closepath fill grestore} def
/Circle {stroke [] 0 setdash 2 copy
  hpt 0 360 arc stroke Pnt} def
/CircleF {stroke [] 0 setdash hpt 0 360 arc fill} def
/C0 {BL [] 0 setdash 2 copy moveto vpt 90 450 arc} bind def
/C1 {BL [] 0 setdash 2 copy moveto
	2 copy vpt 0 90 arc closepath fill
	vpt 0 360 arc closepath} bind def
/C2 {BL [] 0 setdash 2 copy moveto
	2 copy vpt 90 180 arc closepath fill
	vpt 0 360 arc closepath} bind def
/C3 {BL [] 0 setdash 2 copy moveto
	2 copy vpt 0 180 arc closepath fill
	vpt 0 360 arc closepath} bind def
/C4 {BL [] 0 setdash 2 copy moveto
	2 copy vpt 180 270 arc closepath fill
	vpt 0 360 arc closepath} bind def
/C5 {BL [] 0 setdash 2 copy moveto
	2 copy vpt 0 90 arc
	2 copy moveto
	2 copy vpt 180 270 arc closepath fill
	vpt 0 360 arc} bind def
/C6 {BL [] 0 setdash 2 copy moveto
	2 copy vpt 90 270 arc closepath fill
	vpt 0 360 arc closepath} bind def
/C7 {BL [] 0 setdash 2 copy moveto
	2 copy vpt 0 270 arc closepath fill
	vpt 0 360 arc closepath} bind def
/C8 {BL [] 0 setdash 2 copy moveto
	2 copy vpt 270 360 arc closepath fill
	vpt 0 360 arc closepath} bind def
/C9 {BL [] 0 setdash 2 copy moveto
	2 copy vpt 270 450 arc closepath fill
	vpt 0 360 arc closepath} bind def
/C10 {BL [] 0 setdash 2 copy 2 copy moveto vpt 270 360 arc closepath fill
	2 copy moveto
	2 copy vpt 90 180 arc closepath fill
	vpt 0 360 arc closepath} bind def
/C11 {BL [] 0 setdash 2 copy moveto
	2 copy vpt 0 180 arc closepath fill
	2 copy moveto
	2 copy vpt 270 360 arc closepath fill
	vpt 0 360 arc closepath} bind def
/C12 {BL [] 0 setdash 2 copy moveto
	2 copy vpt 180 360 arc closepath fill
	vpt 0 360 arc closepath} bind def
/C13 {BL [] 0 setdash 2 copy moveto
	2 copy vpt 0 90 arc closepath fill
	2 copy moveto
	2 copy vpt 180 360 arc closepath fill
	vpt 0 360 arc closepath} bind def
/C14 {BL [] 0 setdash 2 copy moveto
	2 copy vpt 90 360 arc closepath fill
	vpt 0 360 arc} bind def
/C15 {BL [] 0 setdash 2 copy vpt 0 360 arc closepath fill
	vpt 0 360 arc closepath} bind def
/Rec {newpath 4 2 roll moveto 1 index 0 rlineto 0 exch rlineto
	neg 0 rlineto closepath} bind def
/Square {dup Rec} bind def
/Bsquare {vpt sub exch vpt sub exch vpt2 Square} bind def
/S0 {BL [] 0 setdash 2 copy moveto 0 vpt rlineto BL Bsquare} bind def
/S1 {BL [] 0 setdash 2 copy vpt Square fill Bsquare} bind def
/S2 {BL [] 0 setdash 2 copy exch vpt sub exch vpt Square fill Bsquare} bind def
/S3 {BL [] 0 setdash 2 copy exch vpt sub exch vpt2 vpt Rec fill Bsquare} bind def
/S4 {BL [] 0 setdash 2 copy exch vpt sub exch vpt sub vpt Square fill Bsquare} bind def
/S5 {BL [] 0 setdash 2 copy 2 copy vpt Square fill
	exch vpt sub exch vpt sub vpt Square fill Bsquare} bind def
/S6 {BL [] 0 setdash 2 copy exch vpt sub exch vpt sub vpt vpt2 Rec fill Bsquare} bind def
/S7 {BL [] 0 setdash 2 copy exch vpt sub exch vpt sub vpt vpt2 Rec fill
	2 copy vpt Square fill Bsquare} bind def
/S8 {BL [] 0 setdash 2 copy vpt sub vpt Square fill Bsquare} bind def
/S9 {BL [] 0 setdash 2 copy vpt sub vpt vpt2 Rec fill Bsquare} bind def
/S10 {BL [] 0 setdash 2 copy vpt sub vpt Square fill 2 copy exch vpt sub exch vpt Square fill
	Bsquare} bind def
/S11 {BL [] 0 setdash 2 copy vpt sub vpt Square fill 2 copy exch vpt sub exch vpt2 vpt Rec fill
	Bsquare} bind def
/S12 {BL [] 0 setdash 2 copy exch vpt sub exch vpt sub vpt2 vpt Rec fill Bsquare} bind def
/S13 {BL [] 0 setdash 2 copy exch vpt sub exch vpt sub vpt2 vpt Rec fill
	2 copy vpt Square fill Bsquare} bind def
/S14 {BL [] 0 setdash 2 copy exch vpt sub exch vpt sub vpt2 vpt Rec fill
	2 copy exch vpt sub exch vpt Square fill Bsquare} bind def
/S15 {BL [] 0 setdash 2 copy Bsquare fill Bsquare} bind def
/D0 {gsave translate 45 rotate 0 0 S0 stroke grestore} bind def
/D1 {gsave translate 45 rotate 0 0 S1 stroke grestore} bind def
/D2 {gsave translate 45 rotate 0 0 S2 stroke grestore} bind def
/D3 {gsave translate 45 rotate 0 0 S3 stroke grestore} bind def
/D4 {gsave translate 45 rotate 0 0 S4 stroke grestore} bind def
/D5 {gsave translate 45 rotate 0 0 S5 stroke grestore} bind def
/D6 {gsave translate 45 rotate 0 0 S6 stroke grestore} bind def
/D7 {gsave translate 45 rotate 0 0 S7 stroke grestore} bind def
/D8 {gsave translate 45 rotate 0 0 S8 stroke grestore} bind def
/D9 {gsave translate 45 rotate 0 0 S9 stroke grestore} bind def
/D10 {gsave translate 45 rotate 0 0 S10 stroke grestore} bind def
/D11 {gsave translate 45 rotate 0 0 S11 stroke grestore} bind def
/D12 {gsave translate 45 rotate 0 0 S12 stroke grestore} bind def
/D13 {gsave translate 45 rotate 0 0 S13 stroke grestore} bind def
/D14 {gsave translate 45 rotate 0 0 S14 stroke grestore} bind def
/D15 {gsave translate 45 rotate 0 0 S15 stroke grestore} bind def
/DiaE {stroke [] 0 setdash vpt add M
  hpt neg vpt neg V hpt vpt neg V
  hpt vpt V hpt neg vpt V closepath stroke} def
/BoxE {stroke [] 0 setdash exch hpt sub exch vpt add M
  0 vpt2 neg V hpt2 0 V 0 vpt2 V
  hpt2 neg 0 V closepath stroke} def
/TriUE {stroke [] 0 setdash vpt 1.12 mul add M
  hpt neg vpt -1.62 mul V
  hpt 2 mul 0 V
  hpt neg vpt 1.62 mul V closepath stroke} def
/TriDE {stroke [] 0 setdash vpt 1.12 mul sub M
  hpt neg vpt 1.62 mul V
  hpt 2 mul 0 V
  hpt neg vpt -1.62 mul V closepath stroke} def
/PentE {stroke [] 0 setdash gsave
  translate 0 hpt M 4 {72 rotate 0 hpt L} repeat
  closepath stroke grestore} def
/CircE {stroke [] 0 setdash 
  hpt 0 360 arc stroke} def
/Opaque {gsave closepath 1 setgray fill grestore 0 setgray closepath} def
/DiaW {stroke [] 0 setdash vpt add M
  hpt neg vpt neg V hpt vpt neg V
  hpt vpt V hpt neg vpt V Opaque stroke} def
/BoxW {stroke [] 0 setdash exch hpt sub exch vpt add M
  0 vpt2 neg V hpt2 0 V 0 vpt2 V
  hpt2 neg 0 V Opaque stroke} def
/TriUW {stroke [] 0 setdash vpt 1.12 mul add M
  hpt neg vpt -1.62 mul V
  hpt 2 mul 0 V
  hpt neg vpt 1.62 mul V Opaque stroke} def
/TriDW {stroke [] 0 setdash vpt 1.12 mul sub M
  hpt neg vpt 1.62 mul V
  hpt 2 mul 0 V
  hpt neg vpt -1.62 mul V Opaque stroke} def
/PentW {stroke [] 0 setdash gsave
  translate 0 hpt M 4 {72 rotate 0 hpt L} repeat
  Opaque stroke grestore} def
/CircW {stroke [] 0 setdash 
  hpt 0 360 arc Opaque stroke} def
/BoxFill {gsave Rec 1 setgray fill grestore} def
/Density {
  /Fillden exch def
  currentrgbcolor
  /ColB exch def /ColG exch def /ColR exch def
  /ColR ColR Fillden mul Fillden sub 1 add def
  /ColG ColG Fillden mul Fillden sub 1 add def
  /ColB ColB Fillden mul Fillden sub 1 add def
  ColR ColG ColB setrgbcolor} def
/BoxColFill {gsave Rec PolyFill} def
/PolyFill {gsave Density fill grestore grestore} def
/h {rlineto rlineto rlineto gsave closepath fill grestore} bind def
%
% PostScript Level 1 Pattern Fill routine for rectangles
% Usage: x y w h s a XX PatternFill
%	x,y = lower left corner of box to be filled
%	w,h = width and height of box
%	  a = angle in degrees between lines and x-axis
%	 XX = 0/1 for no/yes cross-hatch
%
/PatternFill {gsave /PFa [ 9 2 roll ] def
  PFa 0 get PFa 2 get 2 div add PFa 1 get PFa 3 get 2 div add translate
  PFa 2 get -2 div PFa 3 get -2 div PFa 2 get PFa 3 get Rec
  gsave 1 setgray fill grestore clip
  currentlinewidth 0.5 mul setlinewidth
  /PFs PFa 2 get dup mul PFa 3 get dup mul add sqrt def
  0 0 M PFa 5 get rotate PFs -2 div dup translate
  0 1 PFs PFa 4 get div 1 add floor cvi
	{PFa 4 get mul 0 M 0 PFs V} for
  0 PFa 6 get ne {
	0 1 PFs PFa 4 get div 1 add floor cvi
	{PFa 4 get mul 0 2 1 roll M PFs 0 V} for
 } if
  stroke grestore} def
%
/languagelevel where
 {pop languagelevel} {1} ifelse
 2 lt
	{/InterpretLevel1 true def}
	{/InterpretLevel1 Level1 def}
 ifelse
%
% PostScript level 2 pattern fill definitions
%
/Level2PatternFill {
/Tile8x8 {/PaintType 2 /PatternType 1 /TilingType 1 /BBox [0 0 8 8] /XStep 8 /YStep 8}
	bind def
/KeepColor {currentrgbcolor [/Pattern /DeviceRGB] setcolorspace} bind def
<< Tile8x8
 /PaintProc {0.5 setlinewidth pop 0 0 M 8 8 L 0 8 M 8 0 L stroke} 
>> matrix makepattern
/Pat1 exch def
<< Tile8x8
 /PaintProc {0.5 setlinewidth pop 0 0 M 8 8 L 0 8 M 8 0 L stroke
	0 4 M 4 8 L 8 4 L 4 0 L 0 4 L stroke}
>> matrix makepattern
/Pat2 exch def
<< Tile8x8
 /PaintProc {0.5 setlinewidth pop 0 0 M 0 8 L
	8 8 L 8 0 L 0 0 L fill}
>> matrix makepattern
/Pat3 exch def
<< Tile8x8
 /PaintProc {0.5 setlinewidth pop -4 8 M 8 -4 L
	0 12 M 12 0 L stroke}
>> matrix makepattern
/Pat4 exch def
<< Tile8x8
 /PaintProc {0.5 setlinewidth pop -4 0 M 8 12 L
	0 -4 M 12 8 L stroke}
>> matrix makepattern
/Pat5 exch def
<< Tile8x8
 /PaintProc {0.5 setlinewidth pop -2 8 M 4 -4 L
	0 12 M 8 -4 L 4 12 M 10 0 L stroke}
>> matrix makepattern
/Pat6 exch def
<< Tile8x8
 /PaintProc {0.5 setlinewidth pop -2 0 M 4 12 L
	0 -4 M 8 12 L 4 -4 M 10 8 L stroke}
>> matrix makepattern
/Pat7 exch def
<< Tile8x8
 /PaintProc {0.5 setlinewidth pop 8 -2 M -4 4 L
	12 0 M -4 8 L 12 4 M 0 10 L stroke}
>> matrix makepattern
/Pat8 exch def
<< Tile8x8
 /PaintProc {0.5 setlinewidth pop 0 -2 M 12 4 L
	-4 0 M 12 8 L -4 4 M 8 10 L stroke}
>> matrix makepattern
/Pat9 exch def
/Pattern1 {PatternBgnd KeepColor Pat1 setpattern} bind def
/Pattern2 {PatternBgnd KeepColor Pat2 setpattern} bind def
/Pattern3 {PatternBgnd KeepColor Pat3 setpattern} bind def
/Pattern4 {PatternBgnd KeepColor Landscape {Pat5} {Pat4} ifelse setpattern} bind def
/Pattern5 {PatternBgnd KeepColor Landscape {Pat4} {Pat5} ifelse setpattern} bind def
/Pattern6 {PatternBgnd KeepColor Landscape {Pat9} {Pat6} ifelse setpattern} bind def
/Pattern7 {PatternBgnd KeepColor Landscape {Pat8} {Pat7} ifelse setpattern} bind def
} def
%
%
%End of PostScript Level 2 code
%
/PatternBgnd {
  TransparentPatterns {} {gsave 1 setgray fill grestore} ifelse
} def
%
% Substitute for Level 2 pattern fill codes with
% grayscale if Level 2 support is not selected.
%
/Level1PatternFill {
/Pattern1 {0.250 Density} bind def
/Pattern2 {0.500 Density} bind def
/Pattern3 {0.750 Density} bind def
/Pattern4 {0.125 Density} bind def
/Pattern5 {0.375 Density} bind def
/Pattern6 {0.625 Density} bind def
/Pattern7 {0.875 Density} bind def
} def
%
% Now test for support of Level 2 code
%
Level1 {Level1PatternFill} {Level2PatternFill} ifelse
%
/Symbol-Oblique /Symbol findfont [1 0 .167 1 0 0] makefont
dup length dict begin {1 index /FID eq {pop pop} {def} ifelse} forall
currentdict end definefont pop
end
%%EndProlog
gnudict begin
gsave
doclip
0 0 translate
0.050 0.050 scale
0 setgray
newpath
1.000 UL
LTb
1340 640 M
63 0 V
-63 705 R
63 0 V
-63 706 R
63 0 V
-63 705 R
63 0 V
-63 705 R
63 0 V
-63 706 R
63 0 V
-63 705 R
63 0 V
-63 706 R
63 0 V
-63 705 R
63 0 V
-63 705 R
63 0 V
-63 706 R
63 0 V
-63 705 R
63 0 V
1340 640 M
0 63 V
0 7696 R
0 -63 V
1717 640 M
0 31 V
0 7728 R
0 -31 V
2215 640 M
0 31 V
0 7728 R
0 -31 V
2471 640 M
0 31 V
0 7728 R
0 -31 V
2592 640 M
0 63 V
0 7696 R
0 -63 V
2969 640 M
0 31 V
0 7728 R
0 -31 V
3467 640 M
0 31 V
0 7728 R
0 -31 V
3723 640 M
0 31 V
0 7728 R
0 -31 V
3844 640 M
0 63 V
0 7696 R
0 -63 V
4221 640 M
0 31 V
0 7728 R
0 -31 V
4719 640 M
0 31 V
0 7728 R
0 -31 V
4975 640 M
0 31 V
0 7728 R
0 -31 V
5096 640 M
0 63 V
0 7696 R
0 -63 V
5473 640 M
0 31 V
0 7728 R
0 -31 V
5971 640 M
0 31 V
0 7728 R
0 -31 V
6227 640 M
0 31 V
0 7728 R
0 -31 V
6348 640 M
0 63 V
0 7696 R
0 -63 V
6725 640 M
0 31 V
0 7728 R
0 -31 V
7223 640 M
0 31 V
0 7728 R
0 -31 V
7479 640 M
0 31 V
0 7728 R
0 -31 V
7600 640 M
0 63 V
stroke 7600 703 M
0 7696 R
0 -63 V
7977 640 M
0 31 V
0 7728 R
0 -31 V
8475 640 M
0 31 V
0 7728 R
0 -31 V
8731 640 M
0 31 V
0 7728 R
0 -31 V
8852 640 M
0 63 V
0 7696 R
0 -63 V
9229 640 M
0 31 V
0 7728 R
0 -31 V
9727 640 M
0 31 V
0 7728 R
0 -31 V
9983 640 M
0 31 V
0 7728 R
0 -31 V
10104 640 M
0 63 V
0 7696 R
0 -63 V
10481 640 M
0 31 V
0 7728 R
0 -31 V
10979 640 M
0 31 V
0 7728 R
0 -31 V
11235 640 M
0 31 V
0 7728 R
0 -31 V
11356 640 M
0 63 V
0 7696 R
0 -63 V
11499 640 M
-63 0 V
63 1552 R
-63 0 V
63 1552 R
-63 0 V
63 1551 R
-63 0 V
63 1552 R
-63 0 V
63 1552 R
-63 0 V
stroke
1340 8399 N
0 -7759 V
10159 0 V
0 7759 V
-10159 0 V
Z stroke
LCb setrgbcolor
LTb
LCb setrgbcolor
LTb
LCb setrgbcolor
LTb
LCb setrgbcolor
LTb
1.000 UP
1.000 UL
LTb
1.000 UL
LTb
1460 7656 N
0 680 V
1383 0 V
0 -680 V
-1383 0 V
Z stroke
1460 8336 M
1383 0 V
1.000 UP
stroke
LT0
LCb setrgbcolor
LT0
2180 8116 M
543 0 V
-543 31 R
0 -62 V
543 62 R
0 -62 V
1340 980 M
498 -3 V
377 2 V
221 -7 V
156 4 V
498 20 V
377 -13 V
221 -9 V
654 9 V
598 -10 V
156 15 V
498 5 V
377 -29 V
221 -11 V
156 17 V
498 0 V
377 6 V
221 -14 V
156 10 V
498 -3 V
377 66 V
221 21 V
156 47 V
875 637 V
377 790 V
875 2780 V
377 2525 V
1340 966 M
0 28 V
-31 -28 R
62 0 V
-62 28 R
62 0 V
467 -34 R
0 35 V
-31 -35 R
62 0 V
-62 35 R
62 0 V
346 -32 R
0 32 V
-31 -32 R
62 0 V
-62 32 R
62 0 V
190 -38 R
0 29 V
-31 -29 R
62 0 V
-62 29 R
62 0 V
125 -25 R
0 29 V
-31 -29 R
62 0 V
-62 29 R
62 0 V
467 -10 R
0 31 V
-31 -31 R
62 0 V
-62 31 R
62 0 V
346 -43 R
0 30 V
-31 -30 R
62 0 V
-62 30 R
62 0 V
190 -38 R
0 28 V
-31 -28 R
62 0 V
-62 28 R
62 0 V
623 -20 R
0 29 V
-31 -29 R
62 0 V
-62 29 R
62 0 V
567 -37 R
0 26 V
-31 -26 R
62 0 V
-62 26 R
62 0 V
125 -13 R
0 31 V
-31 -31 R
62 0 V
-62 31 R
62 0 V
467 -27 R
0 31 V
-31 -31 R
62 0 V
-62 31 R
62 0 V
stroke 5625 1008 M
346 -55 R
0 23 V
-31 -23 R
62 0 V
-62 23 R
62 0 V
190 -37 R
0 28 V
-31 -28 R
62 0 V
-62 28 R
62 0 V
125 -12 R
0 30 V
-31 -30 R
62 0 V
-62 30 R
62 0 V
467 -26 R
0 23 V
-31 -23 R
62 0 V
-62 23 R
62 0 V
346 -20 R
0 28 V
-31 -28 R
62 0 V
-62 28 R
62 0 V
190 -41 R
0 25 V
-31 -25 R
62 0 V
-62 25 R
62 0 V
125 -14 R
0 23 V
-31 -23 R
62 0 V
-62 23 R
62 0 V
467 -25 R
0 22 V
-31 -22 R
62 0 V
-62 22 R
62 0 V
346 6 R
0 98 V
-31 -98 R
62 0 V
-62 98 R
62 0 V
190 -60 R
0 64 V
-31 -64 R
62 0 V
-62 64 R
62 0 V
125 -37 R
0 103 V
-31 -103 R
62 0 V
-62 103 R
62 0 V
844 537 R
0 99 V
-31 -99 R
62 0 V
-62 99 R
62 0 V
346 686 R
0 109 V
-31 -109 R
62 0 V
-62 109 R
62 0 V
844 2665 R
0 119 V
-31 -119 R
62 0 V
-62 119 R
62 0 V
346 2406 R
0 120 V
-31 -120 R
62 0 V
-62 120 R
62 0 V
1340 980 Pls
1838 977 Pls
2215 979 Pls
2436 972 Pls
2592 976 Pls
3090 996 Pls
3467 983 Pls
3688 974 Pls
4342 983 Pls
4940 973 Pls
5096 988 Pls
5594 993 Pls
5971 964 Pls
6192 953 Pls
6348 970 Pls
6846 970 Pls
7223 976 Pls
7444 962 Pls
7600 972 Pls
8098 969 Pls
8475 1035 Pls
8696 1056 Pls
8852 1103 Pls
9727 1740 Pls
10104 2530 Pls
10979 5310 Pls
11356 7835 Pls
2451 8116 Pls
1.000 UL
LT0
LC2 setrgbcolor
LCb setrgbcolor
LT0
LC2 setrgbcolor
2180 7876 M
543 0 V
1340 4632 M
1838 3238 L
377 528 V
221 553 V
156 123 V
498 -315 V
377 -35 V
221 651 V
654 -252 V
598 533 V
5096 3877 L
498 22 V
377 2446 V
6192 4502 L
156 -361 V
498 2418 V
7223 4499 L
221 1069 V
156 974 V
498 750 V
8475 972 L
221 424 V
8852 939 L
875 25 V
377 -60 V
875 -43 V
377 -1 V
stroke
LT1
LC2 setrgbcolor
1340 4497 M
103 0 V
102 0 V
103 0 V
102 0 V
103 0 V
103 0 V
102 0 V
103 0 V
103 0 V
102 0 V
103 0 V
102 0 V
103 0 V
103 0 V
102 0 V
103 0 V
102 0 V
103 0 V
103 0 V
102 0 V
103 0 V
103 0 V
102 0 V
103 0 V
102 0 V
103 0 V
103 0 V
102 0 V
103 0 V
102 0 V
103 0 V
103 0 V
102 0 V
103 0 V
103 0 V
102 0 V
103 0 V
102 0 V
103 0 V
103 0 V
102 0 V
103 0 V
102 0 V
103 0 V
103 0 V
102 0 V
103 0 V
103 0 V
102 0 V
103 0 V
102 0 V
103 0 V
103 0 V
102 0 V
103 0 V
103 0 V
102 0 V
103 0 V
102 0 V
103 0 V
103 0 V
102 0 V
103 0 V
102 0 V
103 0 V
103 0 V
102 0 V
103 0 V
103 0 V
102 0 V
103 0 V
102 0 V
103 0 V
103 0 V
102 0 V
103 0 V
102 0 V
103 0 V
103 0 V
102 0 V
103 0 V
103 0 V
102 0 V
103 0 V
102 0 V
103 0 V
103 0 V
102 0 V
103 0 V
102 0 V
103 0 V
103 0 V
102 0 V
103 0 V
103 0 V
102 0 V
103 0 V
102 0 V
103 0 V
stroke
LTb
1340 8399 N
0 -7759 V
10159 0 V
0 7759 V
-10159 0 V
Z stroke
1.000 UP
1.000 UL
LTb
stroke
grestore
end
showpage
  }}%
  \put(2060,7876){\makebox(0,0)[r]{\strut{}FOM}}%
  \put(2060,8116){\makebox(0,0)[r]{\strut{}$\lambda_0$}}%
  \put(6419,140){\makebox(0,0){\strut{}Relaxation Parameter $\eta$}}%
  \put(12558,4519){%
  \special{ps: gsave currentpoint currentpoint translate
630 rotate neg exch neg exch translate}%
  \makebox(0,0){\strut{}FOM $\left(1/\sigma^2T\right)$}%
  \special{ps: currentpoint grestore moveto}%
  }%
  \put(280,4519){%
  \special{ps: gsave currentpoint currentpoint translate
630 rotate neg exch neg exch translate}%
  \makebox(0,0){\strut{}Eigenvalue Estimate}%
  \special{ps: currentpoint grestore moveto}%
  }%
  \put(11619,8399){\makebox(0,0)[l]{\strut{} 25000}}%
  \put(11619,6847){\makebox(0,0)[l]{\strut{} 20000}}%
  \put(11619,5295){\makebox(0,0)[l]{\strut{} 15000}}%
  \put(11619,3744){\makebox(0,0)[l]{\strut{} 10000}}%
  \put(11619,2192){\makebox(0,0)[l]{\strut{} 5000}}%
  \put(11619,640){\makebox(0,0)[l]{\strut{} 0}}%
  \put(11356,440){\makebox(0,0){\strut{} 1}}%
  \put(10104,440){\makebox(0,0){\strut{} 0.1}}%
  \put(8852,440){\makebox(0,0){\strut{} 0.01}}%
  \put(7600,440){\makebox(0,0){\strut{} 0.001}}%
  \put(6348,440){\makebox(0,0){\strut{} 0.0001}}%
  \put(5096,440){\makebox(0,0){\strut{} 1e-05}}%
  \put(3844,440){\makebox(0,0){\strut{} 1e-06}}%
  \put(2592,440){\makebox(0,0){\strut{} 1e-07}}%
  \put(1340,440){\makebox(0,0){\strut{} 1e-08}}%
  \put(1220,8399){\makebox(0,0)[r]{\strut{} 1.05}}%
  \put(1220,7694){\makebox(0,0)[r]{\strut{} 1.045}}%
  \put(1220,6988){\makebox(0,0)[r]{\strut{} 1.04}}%
  \put(1220,6283){\makebox(0,0)[r]{\strut{} 1.035}}%
  \put(1220,5578){\makebox(0,0)[r]{\strut{} 1.03}}%
  \put(1220,4872){\makebox(0,0)[r]{\strut{} 1.025}}%
  \put(1220,4167){\makebox(0,0)[r]{\strut{} 1.02}}%
  \put(1220,3461){\makebox(0,0)[r]{\strut{} 1.015}}%
  \put(1220,2756){\makebox(0,0)[r]{\strut{} 1.01}}%
  \put(1220,2051){\makebox(0,0)[r]{\strut{} 1.005}}%
  \put(1220,1345){\makebox(0,0)[r]{\strut{} 1}}%
  \put(1220,640){\makebox(0,0)[r]{\strut{} 0.995}}%
\end{picture}%
\endgroup
\endinput

    \caption{Fundamental eigenvalue estimate and figure of merit for varying values of the relaxation parameter $\eta$.  The dashed line is the value of the figure of merit when there is no relaxation.  The number of particles tracked in a non-relaxed iteration is 5E5.}
    \label{fig:MoreAggFOM}
\end{sidewaysfigure}

In \Fref{tab:Relaxed0} the eigenvalue estimates, standard deviation and the figure of merit are shown for these simulations.  For \mbox{$\eta \leq 0.0025$} the standard deviation is constant, \mbox{$\sim 1\e{-4}$}.  When \mbox{$\eta > 0.0025$} the eigenvalue estimates diverge and the standard deviation increases.  This is what causes the figure of merit to drop.  

% More aggressive
\begin{table}\centering
    \begin{tabular}{ccccrr}
        \toprule
        \multirow{2}{*}{$\eta$} & Active & Total \# & \multirow{2}{*}{$\lambda$} & \multicolumn{1}{c}{\multirow{2}{*}{FOM}} & \multicolumn{1}{c}{Time} \\
        & Restarts & Particles & & & \multicolumn{1}{c}{(s)} \\
        \midrule
               0 &   100 & 2.500e+09 & 0.9973 $\pm$  1.0\e{-4} &  12427.9 & 7776.7 \\
           1E-08 &   100 & 2.500e+09 & 0.9974 $\pm$  1.0\e{-4} &  12861.4 & 7741.5 \\
         2.5E-08 &   100 & 2.500e+09 & 0.9974 $\pm$  1.2\e{-4} &   8371.5 & 7732.8 \\
           5E-08 &   100 & 2.500e+09 & 0.9974 $\pm$  1.1\e{-4} &  10072.6 & 7734.8 \\
         7.5E-08 &   100 & 2.500e+09 & 0.9974 $\pm$  1.0\e{-4} &  11854.2 & 7809.2 \\
           1E-07 &   100 & 2.500e+09 & 0.9974 $\pm$  1.0\e{-4} &  12249.2 & 7727.9 \\
         2.5E-07 &   100 & 2.500e+09 & 0.9975 $\pm$  1.1\e{-4} &  11235.8 & 7734.2 \\
           5E-07 &   100 & 2.500e+09 & 0.9974 $\pm$  1.1\e{-4} &  11123.8 & 7757.4 \\
         7.5E-07 &   100 & 2.500e+09 & 0.9974 $\pm$  9.8\e{-5} &  13221.0 & 7835.0 \\
         2.5E-06 &   100 & 2.500e+09 & 0.9974 $\pm$  1.0\e{-4} &  12407.1 & 7753.4 \\
         7.5E-06 &   100 & 2.499e+09 & 0.9974 $\pm$  9.5\e{-5} &  14125.4 & 7817.1 \\
           1E-05 &   100 & 2.500e+09 & 0.9975 $\pm$  1.1\e{-4} &  10430.4 & 7741.5 \\
         2.5E-05 &   105 & 2.555e+09 & 0.9975 $\pm$  1.1\e{-4} &  10501.4 & 7910.3 \\
           5E-05 &   112 & 2.621e+09 & 0.9973 $\pm$  8.2\e{-5} &  18382.2 & 8122.4 \\
         7.5E-05 &   109 & 2.565e+09 & 0.9972 $\pm$  1.0\e{-4} &  12444.6 & 8006.8 \\
          0.0001 &   110 & 2.544e+09 & 0.9973 $\pm$  1.1\e{-4} &  11280.2 & 7871.7 \\
         0.00025 &   124 & 2.541e+09 & 0.9973 $\pm$  8.2\e{-5} &  19072.5 & 7853.4 \\
          0.0005 &   140 & 2.554e+09 & 0.9974 $\pm$  1.0\e{-4} &  12433.1 & 7909.4 \\
         0.00075 &   150 & 2.562e+09 & 0.9973 $\pm$  8.9\e{-5} &  15879.3 & 7945.2 \\
           0.001 &   160 & 2.585e+09 & 0.9973 $\pm$  8.1\e{-5} &  19017.0 & 8010.1 \\
          0.0025 &   187 & 2.551e+09 & 0.9973 $\pm$  7.7\e{-5} &  21433.5 & 7899.2 \\
           0.005 &   220 & 2.498e+09 & 0.9978 $\pm$  3.5\e{-4} &   1069.1 & 7754.7 \\
          0.0075 &   270 & 2.559e+09 & 0.9980 $\pm$  2.3\e{-4} &   2437.4 & 7938.2 \\
            0.01 &   316 & 2.537e+09 & 0.9983 $\pm$  3.6\e{-4} &    962.5 & 7879.3 \\
            0.05 &  1250 & 2.476e+09 & 1.0028 $\pm$  3.5\e{-4} &   1042.8 & 7724.8 \\
             0.1 &  1800 & 2.520e+09 & 1.0084 $\pm$  3.9\e{-4} &    851.8 & 7872.8 \\
             0.5 &  2300 & 2.506e+09 & 1.0281 $\pm$  4.2\e{-4} &    710.5 & 7854.4 \\
       1 &  2350 & 2.502e+09 & 1.0460 $\pm$  4.2\e{-4} &    710.3 & 7824.0 \\
                \bottomrule
    \end{tabular}
    \caption{Eigenvalue estimates for fundamental eigenvalue, figure of merit, and time for a relaxed Arnoldi simulation of a 20mfp thick slab.  Also shown is the number of  active restarts and the total number of particles tracked in the simulation.  5E5 particles were tracked in each non-relaxed iteration.}
    \label{tab:Relaxed0}
\end{table}

% Harmonics
\begin{comment}
\begin{table}[ht]\centering
    \begin{tabular}{ccccrr}
        \toprule
        \multirow{2}{*}{$\eta$} & Active & Total \# & \multirow{2}{*}{$\lambda$} & \multicolumn{1}{c}{\multirow{2}{*}{FOM}} & \multicolumn{1}{c}{Time} \\
        & Restarts & Particles & & & \multicolumn{1}{c}{(s)} \\
        \midrule
               0 &   100 & 2.500e+09 & 0.9899 $\pm$  1.0\e{-4} &  11703.7 & 7776.7 \\
           1E-08 &   100 & 2.500e+09 & 0.9898 $\pm$  1.0\e{-4} &  11769.3 & 7741.5 \\
         2.5E-08 &   100 & 2.500e+09 & 0.9897 $\pm$  1.0\e{-4} &  11864.9 & 7732.8 \\
           5E-08 &   100 & 2.500e+09 & 0.9898 $\pm$  9.1\e{-5} &  15485.8 & 7734.8 \\
         7.5E-08 &   100 & 2.500e+09 & 0.9896 $\pm$  1.0\e{-4} &  11768.9 & 7809.2 \\
           1E-07 &   100 & 2.500e+09 & 0.9898 $\pm$  9.9\e{-5} &  13117.8 & 7727.9 \\
         2.5E-07 &   100 & 2.500e+09 & 0.9897 $\pm$  1.1\e{-4} &  11301.2 & 7734.2 \\
           5E-07 &   100 & 2.500e+09 & 0.9897 $\pm$  1.1\e{-4} &  10732.1 & 7757.4 \\
         7.5E-07 &   100 & 2.500e+09 & 0.9896 $\pm$  1.2\e{-4} &   9545.0 & 7835.0 \\
         2.5E-06 &   100 & 2.500e+09 & 0.9895 $\pm$  1.1\e{-4} &  10118.3 & 7753.4 \\
         7.5E-06 &   100 & 2.499e+09 & 0.9895 $\pm$  9.9\e{-5} &  12927.3 & 7817.1 \\
           1E-05 &   100 & 2.500e+09 & 0.9897 $\pm$  1.2\e{-4} &   9526.8 & 7741.5 \\
         2.5E-05 &   105 & 2.555e+09 & 0.9898 $\pm$  1.1\e{-4} &  11105.9 & 7910.3 \\
           5E-05 &   112 & 2.621e+09 & 0.9897 $\pm$  1.0\e{-4} &  12146.8 & 8122.4 \\
         7.5E-05 &   109 & 2.565e+09 & 0.9899 $\pm$  1.1\e{-4} &  10722.2 & 8006.8 \\
          0.0001 &   110 & 2.544e+09 & 0.9896 $\pm$  1.0\e{-4} &  11645.9 & 7871.7 \\
         0.00025 &   124 & 2.541e+09 & 0.9898 $\pm$  1.0\e{-4} &  11710.7 & 7853.4 \\
          0.0005 &   140 & 2.554e+09 & 0.9896 $\pm$  1.0\e{-4} &  12022.0 & 7909.4 \\
         0.00075 &   150 & 2.562e+09 & 0.9897 $\pm$  9.2\e{-5} &  14981.4 & 7945.2 \\
           0.001 &   160 & 2.585e+09 & 0.9896 $\pm$  7.9\e{-5} &  20123.4 & 8010.1 \\
          0.0025 &   187 & 2.551e+09 & 0.9897 $\pm$  9.4\e{-5} &  14469.3 & 7899.2 \\
           0.005 &   220 & 2.498e+09 & 0.9898 $\pm$  1.1\e{-4} &  10404.2 & 7754.7 \\
          0.0075 &   270 & 2.559e+09 & 0.9898 $\pm$  1.7\e{-4} &   4557.0 & 7938.2 \\
            0.01 &   316 & 2.537e+09 & 0.9902 $\pm$  1.4\e{-4} &   6526.0 & 7879.3 \\
            0.05 &  1250 & 2.476e+09 & 0.9921 $\pm$  8.5\e{-5} &  18012.3 & 7724.8 \\
             0.1 &  1800 & 2.520e+09 & 0.9942 $\pm$  7.6\e{-5} &  21848.3 & 7872.8 \\
             0.5 &  2300 & 2.506e+09 & 1.0068 $\pm$  1.1\e{-4} &  10089.1 & 7854.4 \\
       1 &  2350 & 2.502e+09 & 1.0185 $\pm$  1.9\e{-4} &   3714.0 & 7824.0 \\
        \bottomrule
    \end{tabular}
    \caption{First higher order eigenvalue estimates with relaxation.}
    \label{tab:Relaxed1}
\end{table}

\begin{table}[ht]\centering
    \begin{tabular}{ccccrr}
        \toprule
        \multirow{2}{*}{$\eta$} & Active & Total \# & \multirow{2}{*}{$\lambda$} & \multicolumn{1}{c}{\multirow{2}{*}{FOM}} & \multicolumn{1}{c}{Time} \\
        & Restarts & Particles & & & \multicolumn{1}{c}{(s)} \\
        \midrule
               0 &   100 & 2.500e+09 & 0.9773 $\pm$  1.1\e{-4} &  10398.5 & 7776.7 \\
           1E-08 &   100 & 2.500e+09 & 0.9772 $\pm$  9.0\e{-5} &  15897.9 & 7741.5 \\
         2.5E-08 &   100 & 2.500e+09 & 0.9772 $\pm$  9.4\e{-5} &  14496.4 & 7732.8 \\
           5E-08 &   100 & 2.500e+09 & 0.9771 $\pm$  1.0\e{-4} &  12311.8 & 7734.8 \\
         7.5E-08 &   100 & 2.500e+09 & 0.9772 $\pm$  1.1\e{-4} &  11541.0 & 7809.2 \\
           1E-07 &   100 & 2.500e+09 & 0.9771 $\pm$  9.7\e{-5} &  13886.4 & 7727.9 \\
         2.5E-07 &   100 & 2.500e+09 & 0.9772 $\pm$  1.0\e{-4} &  12340.2 & 7734.2 \\
           5E-07 &   100 & 2.500e+09 & 0.9771 $\pm$  1.1\e{-4} &  10260.1 & 7757.4 \\
         7.5E-07 &   100 & 2.500e+09 & 0.9773 $\pm$  9.8\e{-5} &  13304.6 & 7835.0 \\
         2.5E-06 &   100 & 2.500e+09 & 0.9771 $\pm$  1.2\e{-4} &   9143.5 & 7753.4 \\
         7.5E-06 &   100 & 2.499e+09 & 0.9771 $\pm$  8.8\e{-5} &  16636.0 & 7817.1 \\
           1E-05 &   100 & 2.500e+09 & 0.9773 $\pm$  1.0\e{-4} &  12113.0 & 7741.5 \\
         2.5E-05 &   105 & 2.555e+09 & 0.9769 $\pm$  9.7\e{-5} &  13327.1 & 7910.3 \\
           5E-05 &   112 & 2.621e+09 & 0.9772 $\pm$  1.0\e{-4} &  11808.5 & 8122.4 \\
         7.5E-05 &   109 & 2.565e+09 & 0.9772 $\pm$  9.5\e{-5} &  13856.7 & 8006.8 \\
          0.0001 &   110 & 2.544e+09 & 0.9773 $\pm$  7.9\e{-5} &  20303.5 & 7871.7 \\
         0.00025 &   124 & 2.541e+09 & 0.9772 $\pm$  1.1\e{-4} &  11242.6 & 7853.4 \\
          0.0005 &   140 & 2.554e+09 & 0.9772 $\pm$  1.2\e{-4} &   8134.5 & 7909.4 \\
         0.00075 &   150 & 2.562e+09 & 0.9773 $\pm$  8.5\e{-5} &  17608.8 & 7945.2 \\
           0.001 &   160 & 2.585e+09 & 0.9771 $\pm$  8.0\e{-5} &  19293.4 & 8010.1 \\
          0.0025 &   187 & 2.551e+09 & 0.9773 $\pm$  1.0\e{-4} &  12740.4 & 7899.2 \\
           0.005 &   220 & 2.498e+09 & 0.9774 $\pm$  1.7\e{-4} &   4415.6 & 7754.7 \\
          0.0075 &   270 & 2.559e+09 & 0.9779 $\pm$  1.9\e{-4} &   3350.6 & 7938.2 \\
            0.01 &   316 & 2.537e+09 & 0.9780 $\pm$  2.0\e{-4} &   3050.0 & 7879.3 \\
            0.05 &  1250 & 2.476e+09 & 0.9787 $\pm$  1.1\e{-4} &  11365.8 & 7724.8 \\
             0.1 &  1800 & 2.520e+09 & 0.9797 $\pm$  1.3\e{-4} &   7252.4 & 7872.8 \\
             0.5 &  2300 & 2.506e+09 & 0.9881 $\pm$  1.5\e{-4} &   5872.1 & 7854.4 \\
               1 &  2350 & 2.502e+09 & 0.9957 $\pm$  1.4\e{-4} &   6202.1 & 7824.0 \\
        \bottomrule
    \end{tabular}
    \caption{Second higher order eigenvalue estimates with relaxation.}
    \label{tab:Relaxed2}
\end{table}
\end{comment}

\citet{Bouras:2000A-rel-1} suggest that the first few Arnoldi vectors need to be known with ``full accuracy'' and that Arnoldi vectors calculated in later iterations can be relaxed.  Given that Monte Carlo Arnoldi can never really apply \A{} with full precision it is important to know how accurately the first Arnoldi vectors really need to be applied.  In the next set of results, an attempt to come closer to full accuracy for the initial applications of the fission-transport operator by repeating the simulations but increasing the number of particles tracked in a non-relaxed iteration to 5E6---ten times more than in the previous simulations.

In \Fref{fig:RelaxedArnoldi5E6} and in \Fref{tab:Relaxed5E60} the results of these simulations are shown.  We see a similar pattern for these as we did with the simulations tracking 5E5 particles per iteration; the eigenvalue estimates are unaffected by relaxation until the relaxation parameter becomes too large at which point the eigenvalue estimate diverges.  

\begin{sidewaysfigure}\centering
    % GNUPLOT: LaTeX picture with Postscript
\begingroup%
\makeatletter%
\newcommand{\GNUPLOTspecial}{%
  \@sanitize\catcode`\%=14\relax\special}%
\setlength{\unitlength}{0.0500bp}%
\begin{picture}(12960,8640)(0,0)%
  {\GNUPLOTspecial{"
%!PS-Adobe-2.0 EPSF-2.0
%%Title: Relaxed5E6.tex
%%Creator: gnuplot 4.3 patchlevel 0
%%CreationDate: Sat Jul 18 21:07:56 2009
%%DocumentFonts: 
%%BoundingBox: 0 0 648 432
%%EndComments
%%BeginProlog
/gnudict 256 dict def
gnudict begin
%
% The following true/false flags may be edited by hand if desired.
% The unit line width and grayscale image gamma correction may also be changed.
%
/Color true def
/Blacktext true def
/Solid false def
/Dashlength 1 def
/Landscape false def
/Level1 false def
/Rounded false def
/ClipToBoundingBox false def
/TransparentPatterns false def
/gnulinewidth 5.000 def
/userlinewidth gnulinewidth def
/Gamma 1.0 def
%
/vshift -66 def
/dl1 {
  10.0 Dashlength mul mul
  Rounded { currentlinewidth 0.75 mul sub dup 0 le { pop 0.01 } if } if
} def
/dl2 {
  10.0 Dashlength mul mul
  Rounded { currentlinewidth 0.75 mul add } if
} def
/hpt_ 31.5 def
/vpt_ 31.5 def
/hpt hpt_ def
/vpt vpt_ def
Level1 {} {
/SDict 10 dict def
systemdict /pdfmark known not {
  userdict /pdfmark systemdict /cleartomark get put
} if
SDict begin [
  /Title (Relaxed5E6.tex)
  /Subject (gnuplot plot)
  /Creator (gnuplot 4.3 patchlevel 0)
  /Author (Jeremy Conlin)
%  /Producer (gnuplot)
%  /Keywords ()
  /CreationDate (Sat Jul 18 21:07:56 2009)
  /DOCINFO pdfmark
end
} ifelse
/doclip {
  ClipToBoundingBox {
    newpath 0 0 moveto 648 0 lineto 648 432 lineto 0 432 lineto closepath
    clip
  } if
} def
%
% Gnuplot Prolog Version 4.2 (November 2007)
%
/M {moveto} bind def
/L {lineto} bind def
/R {rmoveto} bind def
/V {rlineto} bind def
/N {newpath moveto} bind def
/Z {closepath} bind def
/C {setrgbcolor} bind def
/f {rlineto fill} bind def
/Gshow {show} def   % May be redefined later in the file to support UTF-8
/vpt2 vpt 2 mul def
/hpt2 hpt 2 mul def
/Lshow {currentpoint stroke M 0 vshift R 
	Blacktext {gsave 0 setgray show grestore} {show} ifelse} def
/Rshow {currentpoint stroke M dup stringwidth pop neg vshift R
	Blacktext {gsave 0 setgray show grestore} {show} ifelse} def
/Cshow {currentpoint stroke M dup stringwidth pop -2 div vshift R 
	Blacktext {gsave 0 setgray show grestore} {show} ifelse} def
/UP {dup vpt_ mul /vpt exch def hpt_ mul /hpt exch def
  /hpt2 hpt 2 mul def /vpt2 vpt 2 mul def} def
/DL {Color {setrgbcolor Solid {pop []} if 0 setdash}
 {pop pop pop 0 setgray Solid {pop []} if 0 setdash} ifelse} def
/BL {stroke userlinewidth 2 mul setlinewidth
	Rounded {1 setlinejoin 1 setlinecap} if} def
/AL {stroke userlinewidth 2 div setlinewidth
	Rounded {1 setlinejoin 1 setlinecap} if} def
/UL {dup gnulinewidth mul /userlinewidth exch def
	dup 1 lt {pop 1} if 10 mul /udl exch def} def
/PL {stroke userlinewidth setlinewidth
	Rounded {1 setlinejoin 1 setlinecap} if} def
% Default Line colors
/LCw {1 1 1} def
/LCb {0 0 0} def
/LCa {0 0 0} def
/LC0 {1 0 0} def
/LC1 {0 1 0} def
/LC2 {0 0 1} def
/LC3 {1 0 1} def
/LC4 {0 1 1} def
/LC5 {1 1 0} def
/LC6 {0 0 0} def
/LC7 {1 0.3 0} def
/LC8 {0.5 0.5 0.5} def
% Default Line Types
/LTw {PL [] 1 setgray} def
/LTb {BL [] LCb DL} def
/LTa {AL [1 udl mul 2 udl mul] 0 setdash LCa setrgbcolor} def
/LT0 {PL [] LC0 DL} def
/LT1 {PL [4 dl1 2 dl2] LC1 DL} def
/LT2 {PL [2 dl1 3 dl2] LC2 DL} def
/LT3 {PL [1 dl1 1.5 dl2] LC3 DL} def
/LT4 {PL [6 dl1 2 dl2 1 dl1 2 dl2] LC4 DL} def
/LT5 {PL [3 dl1 3 dl2 1 dl1 3 dl2] LC5 DL} def
/LT6 {PL [2 dl1 2 dl2 2 dl1 6 dl2] LC6 DL} def
/LT7 {PL [1 dl1 2 dl2 6 dl1 2 dl2 1 dl1 2 dl2] LC7 DL} def
/LT8 {PL [2 dl1 2 dl2 2 dl1 2 dl2 2 dl1 2 dl2 2 dl1 4 dl2] LC8 DL} def
/Pnt {stroke [] 0 setdash gsave 1 setlinecap M 0 0 V stroke grestore} def
/Dia {stroke [] 0 setdash 2 copy vpt add M
  hpt neg vpt neg V hpt vpt neg V
  hpt vpt V hpt neg vpt V closepath stroke
  Pnt} def
/Pls {stroke [] 0 setdash vpt sub M 0 vpt2 V
  currentpoint stroke M
  hpt neg vpt neg R hpt2 0 V stroke
 } def
/Box {stroke [] 0 setdash 2 copy exch hpt sub exch vpt add M
  0 vpt2 neg V hpt2 0 V 0 vpt2 V
  hpt2 neg 0 V closepath stroke
  Pnt} def
/Crs {stroke [] 0 setdash exch hpt sub exch vpt add M
  hpt2 vpt2 neg V currentpoint stroke M
  hpt2 neg 0 R hpt2 vpt2 V stroke} def
/TriU {stroke [] 0 setdash 2 copy vpt 1.12 mul add M
  hpt neg vpt -1.62 mul V
  hpt 2 mul 0 V
  hpt neg vpt 1.62 mul V closepath stroke
  Pnt} def
/Star {2 copy Pls Crs} def
/BoxF {stroke [] 0 setdash exch hpt sub exch vpt add M
  0 vpt2 neg V hpt2 0 V 0 vpt2 V
  hpt2 neg 0 V closepath fill} def
/TriUF {stroke [] 0 setdash vpt 1.12 mul add M
  hpt neg vpt -1.62 mul V
  hpt 2 mul 0 V
  hpt neg vpt 1.62 mul V closepath fill} def
/TriD {stroke [] 0 setdash 2 copy vpt 1.12 mul sub M
  hpt neg vpt 1.62 mul V
  hpt 2 mul 0 V
  hpt neg vpt -1.62 mul V closepath stroke
  Pnt} def
/TriDF {stroke [] 0 setdash vpt 1.12 mul sub M
  hpt neg vpt 1.62 mul V
  hpt 2 mul 0 V
  hpt neg vpt -1.62 mul V closepath fill} def
/DiaF {stroke [] 0 setdash vpt add M
  hpt neg vpt neg V hpt vpt neg V
  hpt vpt V hpt neg vpt V closepath fill} def
/Pent {stroke [] 0 setdash 2 copy gsave
  translate 0 hpt M 4 {72 rotate 0 hpt L} repeat
  closepath stroke grestore Pnt} def
/PentF {stroke [] 0 setdash gsave
  translate 0 hpt M 4 {72 rotate 0 hpt L} repeat
  closepath fill grestore} def
/Circle {stroke [] 0 setdash 2 copy
  hpt 0 360 arc stroke Pnt} def
/CircleF {stroke [] 0 setdash hpt 0 360 arc fill} def
/C0 {BL [] 0 setdash 2 copy moveto vpt 90 450 arc} bind def
/C1 {BL [] 0 setdash 2 copy moveto
	2 copy vpt 0 90 arc closepath fill
	vpt 0 360 arc closepath} bind def
/C2 {BL [] 0 setdash 2 copy moveto
	2 copy vpt 90 180 arc closepath fill
	vpt 0 360 arc closepath} bind def
/C3 {BL [] 0 setdash 2 copy moveto
	2 copy vpt 0 180 arc closepath fill
	vpt 0 360 arc closepath} bind def
/C4 {BL [] 0 setdash 2 copy moveto
	2 copy vpt 180 270 arc closepath fill
	vpt 0 360 arc closepath} bind def
/C5 {BL [] 0 setdash 2 copy moveto
	2 copy vpt 0 90 arc
	2 copy moveto
	2 copy vpt 180 270 arc closepath fill
	vpt 0 360 arc} bind def
/C6 {BL [] 0 setdash 2 copy moveto
	2 copy vpt 90 270 arc closepath fill
	vpt 0 360 arc closepath} bind def
/C7 {BL [] 0 setdash 2 copy moveto
	2 copy vpt 0 270 arc closepath fill
	vpt 0 360 arc closepath} bind def
/C8 {BL [] 0 setdash 2 copy moveto
	2 copy vpt 270 360 arc closepath fill
	vpt 0 360 arc closepath} bind def
/C9 {BL [] 0 setdash 2 copy moveto
	2 copy vpt 270 450 arc closepath fill
	vpt 0 360 arc closepath} bind def
/C10 {BL [] 0 setdash 2 copy 2 copy moveto vpt 270 360 arc closepath fill
	2 copy moveto
	2 copy vpt 90 180 arc closepath fill
	vpt 0 360 arc closepath} bind def
/C11 {BL [] 0 setdash 2 copy moveto
	2 copy vpt 0 180 arc closepath fill
	2 copy moveto
	2 copy vpt 270 360 arc closepath fill
	vpt 0 360 arc closepath} bind def
/C12 {BL [] 0 setdash 2 copy moveto
	2 copy vpt 180 360 arc closepath fill
	vpt 0 360 arc closepath} bind def
/C13 {BL [] 0 setdash 2 copy moveto
	2 copy vpt 0 90 arc closepath fill
	2 copy moveto
	2 copy vpt 180 360 arc closepath fill
	vpt 0 360 arc closepath} bind def
/C14 {BL [] 0 setdash 2 copy moveto
	2 copy vpt 90 360 arc closepath fill
	vpt 0 360 arc} bind def
/C15 {BL [] 0 setdash 2 copy vpt 0 360 arc closepath fill
	vpt 0 360 arc closepath} bind def
/Rec {newpath 4 2 roll moveto 1 index 0 rlineto 0 exch rlineto
	neg 0 rlineto closepath} bind def
/Square {dup Rec} bind def
/Bsquare {vpt sub exch vpt sub exch vpt2 Square} bind def
/S0 {BL [] 0 setdash 2 copy moveto 0 vpt rlineto BL Bsquare} bind def
/S1 {BL [] 0 setdash 2 copy vpt Square fill Bsquare} bind def
/S2 {BL [] 0 setdash 2 copy exch vpt sub exch vpt Square fill Bsquare} bind def
/S3 {BL [] 0 setdash 2 copy exch vpt sub exch vpt2 vpt Rec fill Bsquare} bind def
/S4 {BL [] 0 setdash 2 copy exch vpt sub exch vpt sub vpt Square fill Bsquare} bind def
/S5 {BL [] 0 setdash 2 copy 2 copy vpt Square fill
	exch vpt sub exch vpt sub vpt Square fill Bsquare} bind def
/S6 {BL [] 0 setdash 2 copy exch vpt sub exch vpt sub vpt vpt2 Rec fill Bsquare} bind def
/S7 {BL [] 0 setdash 2 copy exch vpt sub exch vpt sub vpt vpt2 Rec fill
	2 copy vpt Square fill Bsquare} bind def
/S8 {BL [] 0 setdash 2 copy vpt sub vpt Square fill Bsquare} bind def
/S9 {BL [] 0 setdash 2 copy vpt sub vpt vpt2 Rec fill Bsquare} bind def
/S10 {BL [] 0 setdash 2 copy vpt sub vpt Square fill 2 copy exch vpt sub exch vpt Square fill
	Bsquare} bind def
/S11 {BL [] 0 setdash 2 copy vpt sub vpt Square fill 2 copy exch vpt sub exch vpt2 vpt Rec fill
	Bsquare} bind def
/S12 {BL [] 0 setdash 2 copy exch vpt sub exch vpt sub vpt2 vpt Rec fill Bsquare} bind def
/S13 {BL [] 0 setdash 2 copy exch vpt sub exch vpt sub vpt2 vpt Rec fill
	2 copy vpt Square fill Bsquare} bind def
/S14 {BL [] 0 setdash 2 copy exch vpt sub exch vpt sub vpt2 vpt Rec fill
	2 copy exch vpt sub exch vpt Square fill Bsquare} bind def
/S15 {BL [] 0 setdash 2 copy Bsquare fill Bsquare} bind def
/D0 {gsave translate 45 rotate 0 0 S0 stroke grestore} bind def
/D1 {gsave translate 45 rotate 0 0 S1 stroke grestore} bind def
/D2 {gsave translate 45 rotate 0 0 S2 stroke grestore} bind def
/D3 {gsave translate 45 rotate 0 0 S3 stroke grestore} bind def
/D4 {gsave translate 45 rotate 0 0 S4 stroke grestore} bind def
/D5 {gsave translate 45 rotate 0 0 S5 stroke grestore} bind def
/D6 {gsave translate 45 rotate 0 0 S6 stroke grestore} bind def
/D7 {gsave translate 45 rotate 0 0 S7 stroke grestore} bind def
/D8 {gsave translate 45 rotate 0 0 S8 stroke grestore} bind def
/D9 {gsave translate 45 rotate 0 0 S9 stroke grestore} bind def
/D10 {gsave translate 45 rotate 0 0 S10 stroke grestore} bind def
/D11 {gsave translate 45 rotate 0 0 S11 stroke grestore} bind def
/D12 {gsave translate 45 rotate 0 0 S12 stroke grestore} bind def
/D13 {gsave translate 45 rotate 0 0 S13 stroke grestore} bind def
/D14 {gsave translate 45 rotate 0 0 S14 stroke grestore} bind def
/D15 {gsave translate 45 rotate 0 0 S15 stroke grestore} bind def
/DiaE {stroke [] 0 setdash vpt add M
  hpt neg vpt neg V hpt vpt neg V
  hpt vpt V hpt neg vpt V closepath stroke} def
/BoxE {stroke [] 0 setdash exch hpt sub exch vpt add M
  0 vpt2 neg V hpt2 0 V 0 vpt2 V
  hpt2 neg 0 V closepath stroke} def
/TriUE {stroke [] 0 setdash vpt 1.12 mul add M
  hpt neg vpt -1.62 mul V
  hpt 2 mul 0 V
  hpt neg vpt 1.62 mul V closepath stroke} def
/TriDE {stroke [] 0 setdash vpt 1.12 mul sub M
  hpt neg vpt 1.62 mul V
  hpt 2 mul 0 V
  hpt neg vpt -1.62 mul V closepath stroke} def
/PentE {stroke [] 0 setdash gsave
  translate 0 hpt M 4 {72 rotate 0 hpt L} repeat
  closepath stroke grestore} def
/CircE {stroke [] 0 setdash 
  hpt 0 360 arc stroke} def
/Opaque {gsave closepath 1 setgray fill grestore 0 setgray closepath} def
/DiaW {stroke [] 0 setdash vpt add M
  hpt neg vpt neg V hpt vpt neg V
  hpt vpt V hpt neg vpt V Opaque stroke} def
/BoxW {stroke [] 0 setdash exch hpt sub exch vpt add M
  0 vpt2 neg V hpt2 0 V 0 vpt2 V
  hpt2 neg 0 V Opaque stroke} def
/TriUW {stroke [] 0 setdash vpt 1.12 mul add M
  hpt neg vpt -1.62 mul V
  hpt 2 mul 0 V
  hpt neg vpt 1.62 mul V Opaque stroke} def
/TriDW {stroke [] 0 setdash vpt 1.12 mul sub M
  hpt neg vpt 1.62 mul V
  hpt 2 mul 0 V
  hpt neg vpt -1.62 mul V Opaque stroke} def
/PentW {stroke [] 0 setdash gsave
  translate 0 hpt M 4 {72 rotate 0 hpt L} repeat
  Opaque stroke grestore} def
/CircW {stroke [] 0 setdash 
  hpt 0 360 arc Opaque stroke} def
/BoxFill {gsave Rec 1 setgray fill grestore} def
/Density {
  /Fillden exch def
  currentrgbcolor
  /ColB exch def /ColG exch def /ColR exch def
  /ColR ColR Fillden mul Fillden sub 1 add def
  /ColG ColG Fillden mul Fillden sub 1 add def
  /ColB ColB Fillden mul Fillden sub 1 add def
  ColR ColG ColB setrgbcolor} def
/BoxColFill {gsave Rec PolyFill} def
/PolyFill {gsave Density fill grestore grestore} def
/h {rlineto rlineto rlineto gsave closepath fill grestore} bind def
%
% PostScript Level 1 Pattern Fill routine for rectangles
% Usage: x y w h s a XX PatternFill
%	x,y = lower left corner of box to be filled
%	w,h = width and height of box
%	  a = angle in degrees between lines and x-axis
%	 XX = 0/1 for no/yes cross-hatch
%
/PatternFill {gsave /PFa [ 9 2 roll ] def
  PFa 0 get PFa 2 get 2 div add PFa 1 get PFa 3 get 2 div add translate
  PFa 2 get -2 div PFa 3 get -2 div PFa 2 get PFa 3 get Rec
  gsave 1 setgray fill grestore clip
  currentlinewidth 0.5 mul setlinewidth
  /PFs PFa 2 get dup mul PFa 3 get dup mul add sqrt def
  0 0 M PFa 5 get rotate PFs -2 div dup translate
  0 1 PFs PFa 4 get div 1 add floor cvi
	{PFa 4 get mul 0 M 0 PFs V} for
  0 PFa 6 get ne {
	0 1 PFs PFa 4 get div 1 add floor cvi
	{PFa 4 get mul 0 2 1 roll M PFs 0 V} for
 } if
  stroke grestore} def
%
/languagelevel where
 {pop languagelevel} {1} ifelse
 2 lt
	{/InterpretLevel1 true def}
	{/InterpretLevel1 Level1 def}
 ifelse
%
% PostScript level 2 pattern fill definitions
%
/Level2PatternFill {
/Tile8x8 {/PaintType 2 /PatternType 1 /TilingType 1 /BBox [0 0 8 8] /XStep 8 /YStep 8}
	bind def
/KeepColor {currentrgbcolor [/Pattern /DeviceRGB] setcolorspace} bind def
<< Tile8x8
 /PaintProc {0.5 setlinewidth pop 0 0 M 8 8 L 0 8 M 8 0 L stroke} 
>> matrix makepattern
/Pat1 exch def
<< Tile8x8
 /PaintProc {0.5 setlinewidth pop 0 0 M 8 8 L 0 8 M 8 0 L stroke
	0 4 M 4 8 L 8 4 L 4 0 L 0 4 L stroke}
>> matrix makepattern
/Pat2 exch def
<< Tile8x8
 /PaintProc {0.5 setlinewidth pop 0 0 M 0 8 L
	8 8 L 8 0 L 0 0 L fill}
>> matrix makepattern
/Pat3 exch def
<< Tile8x8
 /PaintProc {0.5 setlinewidth pop -4 8 M 8 -4 L
	0 12 M 12 0 L stroke}
>> matrix makepattern
/Pat4 exch def
<< Tile8x8
 /PaintProc {0.5 setlinewidth pop -4 0 M 8 12 L
	0 -4 M 12 8 L stroke}
>> matrix makepattern
/Pat5 exch def
<< Tile8x8
 /PaintProc {0.5 setlinewidth pop -2 8 M 4 -4 L
	0 12 M 8 -4 L 4 12 M 10 0 L stroke}
>> matrix makepattern
/Pat6 exch def
<< Tile8x8
 /PaintProc {0.5 setlinewidth pop -2 0 M 4 12 L
	0 -4 M 8 12 L 4 -4 M 10 8 L stroke}
>> matrix makepattern
/Pat7 exch def
<< Tile8x8
 /PaintProc {0.5 setlinewidth pop 8 -2 M -4 4 L
	12 0 M -4 8 L 12 4 M 0 10 L stroke}
>> matrix makepattern
/Pat8 exch def
<< Tile8x8
 /PaintProc {0.5 setlinewidth pop 0 -2 M 12 4 L
	-4 0 M 12 8 L -4 4 M 8 10 L stroke}
>> matrix makepattern
/Pat9 exch def
/Pattern1 {PatternBgnd KeepColor Pat1 setpattern} bind def
/Pattern2 {PatternBgnd KeepColor Pat2 setpattern} bind def
/Pattern3 {PatternBgnd KeepColor Pat3 setpattern} bind def
/Pattern4 {PatternBgnd KeepColor Landscape {Pat5} {Pat4} ifelse setpattern} bind def
/Pattern5 {PatternBgnd KeepColor Landscape {Pat4} {Pat5} ifelse setpattern} bind def
/Pattern6 {PatternBgnd KeepColor Landscape {Pat9} {Pat6} ifelse setpattern} bind def
/Pattern7 {PatternBgnd KeepColor Landscape {Pat8} {Pat7} ifelse setpattern} bind def
} def
%
%
%End of PostScript Level 2 code
%
/PatternBgnd {
  TransparentPatterns {} {gsave 1 setgray fill grestore} ifelse
} def
%
% Substitute for Level 2 pattern fill codes with
% grayscale if Level 2 support is not selected.
%
/Level1PatternFill {
/Pattern1 {0.250 Density} bind def
/Pattern2 {0.500 Density} bind def
/Pattern3 {0.750 Density} bind def
/Pattern4 {0.125 Density} bind def
/Pattern5 {0.375 Density} bind def
/Pattern6 {0.625 Density} bind def
/Pattern7 {0.875 Density} bind def
} def
%
% Now test for support of Level 2 code
%
Level1 {Level1PatternFill} {Level2PatternFill} ifelse
%
/Symbol-Oblique /Symbol findfont [1 0 .167 1 0 0] makefont
dup length dict begin {1 index /FID eq {pop pop} {def} ifelse} forall
currentdict end definefont pop
end
%%EndProlog
gnudict begin
gsave
doclip
0 0 translate
0.050 0.050 scale
0 setgray
newpath
1.000 UL
LTb
1340 640 M
63 0 V
11156 0 R
-63 0 V
1340 1416 M
63 0 V
11156 0 R
-63 0 V
1340 2192 M
63 0 V
11156 0 R
-63 0 V
1340 2968 M
63 0 V
11156 0 R
-63 0 V
1340 3744 M
63 0 V
11156 0 R
-63 0 V
1340 4520 M
63 0 V
11156 0 R
-63 0 V
1340 5295 M
63 0 V
11156 0 R
-63 0 V
1340 6071 M
63 0 V
11156 0 R
-63 0 V
1340 6847 M
63 0 V
11156 0 R
-63 0 V
1340 7623 M
63 0 V
11156 0 R
-63 0 V
1340 8399 M
63 0 V
11156 0 R
-63 0 V
1340 640 M
0 63 V
0 7696 R
0 -63 V
1756 640 M
0 31 V
0 7728 R
0 -31 V
2306 640 M
0 31 V
0 7728 R
0 -31 V
2589 640 M
0 31 V
0 7728 R
0 -31 V
2723 640 M
0 63 V
0 7696 R
0 -63 V
3139 640 M
0 31 V
0 7728 R
0 -31 V
3689 640 M
0 31 V
0 7728 R
0 -31 V
3971 640 M
0 31 V
0 7728 R
0 -31 V
4105 640 M
0 63 V
0 7696 R
0 -63 V
4522 640 M
0 31 V
0 7728 R
0 -31 V
5072 640 M
0 31 V
0 7728 R
0 -31 V
5354 640 M
0 31 V
0 7728 R
0 -31 V
5488 640 M
0 63 V
0 7696 R
0 -63 V
5904 640 M
0 31 V
0 7728 R
0 -31 V
6454 640 M
0 31 V
0 7728 R
0 -31 V
6737 640 M
0 31 V
stroke 6737 671 M
0 7728 R
0 -31 V
6871 640 M
0 63 V
0 7696 R
0 -63 V
7287 640 M
0 31 V
0 7728 R
0 -31 V
7837 640 M
0 31 V
0 7728 R
0 -31 V
8119 640 M
0 31 V
0 7728 R
0 -31 V
8253 640 M
0 63 V
0 7696 R
0 -63 V
8670 640 M
0 31 V
0 7728 R
0 -31 V
9220 640 M
0 31 V
0 7728 R
0 -31 V
9502 640 M
0 31 V
0 7728 R
0 -31 V
9636 640 M
0 63 V
0 7696 R
0 -63 V
10052 640 M
0 31 V
0 7728 R
0 -31 V
10603 640 M
0 31 V
0 7728 R
0 -31 V
10885 640 M
0 31 V
0 7728 R
0 -31 V
11019 640 M
0 63 V
0 7696 R
0 -63 V
11435 640 M
0 31 V
0 7728 R
0 -31 V
11985 640 M
0 31 V
0 7728 R
0 -31 V
12267 640 M
0 31 V
0 7728 R
0 -31 V
12401 640 M
0 63 V
0 7696 R
0 -63 V
stroke
1340 8399 N
0 -7759 V
11219 0 V
0 7759 V
-11219 0 V
Z stroke
LCb setrgbcolor
LTb
LCb setrgbcolor
LTb
LCb setrgbcolor
LTb
LCb setrgbcolor
LTb
1.000 UP
1.000 UL
LTb
1.000 UL
LTb
1460 7416 N
0 920 V
1383 0 V
0 -920 V
-1383 0 V
Z stroke
1460 8336 M
1383 0 V
1.000 UP
stroke
LT0
LCb setrgbcolor
LT0
2180 8116 M
543 0 V
-543 31 R
0 -62 V
543 62 R
0 -62 V
1340 4110 M
550 3 V
416 3 V
244 -5 V
173 0 V
550 7 V
416 -2 V
244 0 V
723 -2 V
659 11 V
173 -18 V
550 11 V
416 6 V
244 0 V
173 -6 V
550 -2 V
416 -2 V
244 8 V
172 -3 V
551 -8 V
416 67 V
243 -45 V
173 23 V
967 218 V
416 332 V
550 822 V
416 900 V
244 652 V
172 559 V
1340 4104 M
0 11 V
-31 -11 R
62 0 V
-62 11 R
62 0 V
519 -7 R
0 10 V
-31 -10 R
62 0 V
-62 10 R
62 0 V
385 -7 R
0 10 V
-31 -10 R
62 0 V
-62 10 R
62 0 V
213 -14 R
0 9 V
-31 -9 R
62 0 V
-62 9 R
62 0 V
142 -9 R
0 9 V
-31 -9 R
62 0 V
-62 9 R
62 0 V
519 -3 R
0 9 V
-31 -9 R
62 0 V
-62 9 R
62 0 V
385 -11 R
0 10 V
-31 -10 R
62 0 V
-62 10 R
62 0 V
213 -11 R
0 12 V
-31 -12 R
62 0 V
-62 12 R
62 0 V
692 -13 R
0 11 V
-31 -11 R
62 0 V
-62 11 R
62 0 V
628 1 R
0 9 V
-31 -9 R
62 0 V
-62 9 R
62 0 V
142 -28 R
0 10 V
-31 -10 R
62 0 V
-62 10 R
62 0 V
519 1 R
0 10 V
-31 -10 R
62 0 V
stroke 6069 4113 M
-62 10 R
62 0 V
385 -3 R
0 8 V
-31 -8 R
62 0 V
-62 8 R
62 0 V
213 -9 R
0 10 V
-31 -10 R
62 0 V
-62 10 R
62 0 V
142 -16 R
0 9 V
-31 -9 R
62 0 V
-62 9 R
62 0 V
519 -10 R
0 8 V
-31 -8 R
62 0 V
-62 8 R
62 0 V
385 -10 R
0 8 V
-31 -8 R
62 0 V
-62 8 R
62 0 V
213 1 R
0 7 V
-31 -7 R
62 0 V
-62 7 R
62 0 V
141 -10 R
0 6 V
-31 -6 R
62 0 V
-62 6 R
62 0 V
520 -14 R
0 6 V
-31 -6 R
62 0 V
-62 6 R
62 0 V
385 14 R
0 100 V
-31 -100 R
62 0 V
-62 100 R
62 0 V
212 -108 R
0 26 V
-31 -26 R
62 0 V
-62 26 R
62 0 V
142 -10 R
0 40 V
-31 -40 R
62 0 V
-62 40 R
62 0 V
936 164 R
0 68 V
-31 -68 R
62 0 V
-62 68 R
62 0 V
385 269 R
0 57 V
-31 -57 R
62 0 V
-62 57 R
62 0 V
519 742 R
0 104 V
-31 -104 R
62 0 V
-62 104 R
62 0 V
385 800 R
0 96 V
-31 -96 R
62 0 V
-62 96 R
62 0 V
213 562 R
0 84 V
-31 -84 R
62 0 V
-62 84 R
62 0 V
141 455 R
0 123 V
-31 -123 R
62 0 V
-62 123 R
62 0 V
stroke 12432 7700 M
1340 4110 Pls
1890 4113 Pls
2306 4116 Pls
2550 4111 Pls
2723 4111 Pls
3273 4118 Pls
3689 4116 Pls
3933 4116 Pls
4656 4114 Pls
5315 4125 Pls
5488 4107 Pls
6038 4118 Pls
6454 4124 Pls
6698 4124 Pls
6871 4118 Pls
7421 4116 Pls
7837 4114 Pls
8081 4122 Pls
8253 4119 Pls
8804 4111 Pls
9220 4178 Pls
9463 4133 Pls
9636 4156 Pls
10603 4374 Pls
11019 4706 Pls
11569 5528 Pls
11985 6428 Pls
12229 7080 Pls
12401 7639 Pls
2451 8116 Pls
1.000 UP
1.000 UL
LT0
LC1 setrgbcolor
LCb setrgbcolor
LT0
LC1 setrgbcolor
2180 7876 M
543 0 V
-543 31 R
0 -62 V
543 62 R
0 -62 V
1340 2915 M
550 5 V
416 0 V
244 4 V
173 -3 V
550 -11 V
416 8 V
244 5 V
723 -7 V
659 11 V
173 -7 V
550 3 V
416 -7 V
244 5 V
173 -6 V
550 9 V
416 -12 V
244 9 V
172 -1 V
551 3 V
416 15 V
243 -9 V
173 8 V
967 128 V
416 106 V
550 307 V
416 488 V
244 375 V
172 334 V
1340 2909 M
0 11 V
-31 -11 R
62 0 V
-62 11 R
62 0 V
519 -6 R
0 11 V
-31 -11 R
62 0 V
-62 11 R
62 0 V
385 -11 R
0 11 V
-31 -11 R
62 0 V
-62 11 R
62 0 V
213 -6 R
0 10 V
-31 -10 R
62 0 V
-62 10 R
62 0 V
142 -13 R
0 10 V
-31 -10 R
62 0 V
-62 10 R
62 0 V
519 -21 R
0 10 V
-31 -10 R
62 0 V
-62 10 R
62 0 V
385 -2 R
0 10 V
-31 -10 R
62 0 V
-62 10 R
62 0 V
213 -6 R
0 11 V
-31 -11 R
62 0 V
-62 11 R
62 0 V
692 -17 R
0 11 V
-31 -11 R
62 0 V
-62 11 R
62 0 V
628 0 R
0 11 V
-31 -11 R
62 0 V
-62 11 R
62 0 V
142 -19 R
0 11 V
-31 -11 R
62 0 V
-62 11 R
62 0 V
519 -7 R
0 10 V
-31 -10 R
62 0 V
stroke 6069 2918 M
-62 10 R
62 0 V
385 -16 R
0 9 V
-31 -9 R
62 0 V
-62 9 R
62 0 V
213 -4 R
0 8 V
-31 -8 R
62 0 V
-62 8 R
62 0 V
142 -14 R
0 8 V
-31 -8 R
62 0 V
-62 8 R
62 0 V
519 1 R
0 8 V
-31 -8 R
62 0 V
-62 8 R
62 0 V
385 -20 R
0 8 V
-31 -8 R
62 0 V
-62 8 R
62 0 V
213 2 R
0 7 V
-31 -7 R
62 0 V
-62 7 R
62 0 V
141 -9 R
0 7 V
-31 -7 R
62 0 V
-62 7 R
62 0 V
520 -4 R
0 7 V
-31 -7 R
62 0 V
-62 7 R
62 0 V
385 2 R
0 20 V
-31 -20 R
62 0 V
-62 20 R
62 0 V
212 -26 R
0 14 V
-31 -14 R
62 0 V
-62 14 R
62 0 V
142 -9 R
0 19 V
-31 -19 R
62 0 V
-62 19 R
62 0 V
936 106 R
0 27 V
-31 -27 R
62 0 V
-62 27 R
62 0 V
385 84 R
0 16 V
-31 -16 R
62 0 V
-62 16 R
62 0 V
519 289 R
0 21 V
-31 -21 R
62 0 V
-62 21 R
62 0 V
385 467 R
0 19 V
-31 -19 R
62 0 V
-62 19 R
62 0 V
213 354 R
0 24 V
-31 -24 R
62 0 V
-62 24 R
62 0 V
141 308 R
0 27 V
-31 -27 R
62 0 V
-62 27 R
62 0 V
stroke 12432 4688 M
1340 2915 Crs
1890 2920 Crs
2306 2920 Crs
2550 2924 Crs
2723 2921 Crs
3273 2910 Crs
3689 2918 Crs
3933 2923 Crs
4656 2916 Crs
5315 2927 Crs
5488 2920 Crs
6038 2923 Crs
6454 2916 Crs
6698 2921 Crs
6871 2915 Crs
7421 2924 Crs
7837 2912 Crs
8081 2921 Crs
8253 2920 Crs
8804 2923 Crs
9220 2938 Crs
9463 2929 Crs
9636 2937 Crs
10603 3065 Crs
11019 3171 Crs
11569 3478 Crs
11985 3966 Crs
12229 4341 Crs
12401 4675 Crs
2451 7876 Crs
1.000 UP
1.000 UL
LT0
LC2 setrgbcolor
LCb setrgbcolor
LT0
LC2 setrgbcolor
2180 7636 M
543 0 V
-543 31 R
0 -62 V
543 62 R
0 -62 V
1340 975 M
550 2 V
416 4 V
244 -3 V
173 0 V
550 0 V
416 -6 V
244 0 V
723 5 V
659 1 V
173 -1 V
550 -5 V
416 6 V
244 -3 V
173 6 V
550 -3 V
416 5 V
244 -5 V
172 0 V
551 10 V
416 52 V
243 27 V
173 -5 V
967 68 V
416 33 V
550 169 V
416 379 V
244 184 V
172 287 V
1340 970 M
0 10 V
-31 -10 R
62 0 V
-62 10 R
62 0 V
519 -7 R
0 8 V
-31 -8 R
62 0 V
-62 8 R
62 0 V
385 -4 R
0 9 V
-31 -9 R
62 0 V
-62 9 R
62 0 V
213 -12 R
0 8 V
-31 -8 R
62 0 V
-62 8 R
62 0 V
142 -8 R
0 9 V
-31 -9 R
62 0 V
-62 9 R
62 0 V
519 -10 R
0 10 V
-31 -10 R
62 0 V
-62 10 R
62 0 V
385 -15 R
0 8 V
-31 -8 R
62 0 V
-62 8 R
62 0 V
213 -9 R
0 10 V
-31 -10 R
62 0 V
-62 10 R
62 0 V
692 -5 R
0 9 V
-31 -9 R
62 0 V
-62 9 R
62 0 V
628 -7 R
0 9 V
-31 -9 R
62 0 V
-62 9 R
62 0 V
142 -11 R
0 9 V
-31 -9 R
62 0 V
-62 9 R
62 0 V
519 -14 R
0 10 V
-31 -10 R
62 0 V
stroke 6069 967 M
-62 10 R
62 0 V
385 -3 R
0 8 V
-31 -8 R
62 0 V
-62 8 R
62 0 V
213 -11 R
0 8 V
-31 -8 R
62 0 V
-62 8 R
62 0 V
142 -2 R
0 8 V
-31 -8 R
62 0 V
-62 8 R
62 0 V
519 -11 R
0 8 V
-31 -8 R
62 0 V
-62 8 R
62 0 V
385 -3 R
0 8 V
-31 -8 R
62 0 V
-62 8 R
62 0 V
213 -12 R
0 7 V
-31 -7 R
62 0 V
-62 7 R
62 0 V
141 -7 R
0 7 V
-31 -7 R
62 0 V
-62 7 R
62 0 V
520 -4 R
0 19 V
-31 -19 R
62 0 V
-62 19 R
62 0 V
385 24 R
0 39 V
-31 -39 R
62 0 V
-62 39 R
62 0 V
212 -15 R
0 43 V
-31 -43 R
62 0 V
-62 43 R
62 0 V
142 -47 R
0 42 V
-31 -42 R
62 0 V
-62 42 R
62 0 V
936 31 R
0 33 V
-31 -33 R
62 0 V
-62 33 R
62 0 V
385 6 R
0 19 V
-31 -19 R
62 0 V
-62 19 R
62 0 V
519 146 R
0 29 V
-31 -29 R
62 0 V
-62 29 R
62 0 V
385 344 R
0 39 V
-31 -39 R
62 0 V
-62 39 R
62 0 V
213 146 R
0 39 V
-31 -39 R
62 0 V
-62 39 R
62 0 V
141 248 R
0 39 V
-31 -39 R
62 0 V
-62 39 R
62 0 V
stroke 12432 2202 M
1340 975 Star
1890 977 Star
2306 981 Star
2550 978 Star
2723 978 Star
3273 978 Star
3689 972 Star
3933 972 Star
4656 977 Star
5315 978 Star
5488 977 Star
6038 972 Star
6454 978 Star
6698 975 Star
6871 981 Star
7421 978 Star
7837 983 Star
8081 978 Star
8253 978 Star
8804 988 Star
9220 1040 Star
9463 1067 Star
9636 1062 Star
10603 1130 Star
11019 1163 Star
11569 1332 Star
11985 1711 Star
12229 1895 Star
12401 2182 Star
2451 7636 Star
2.000 UL
LTb
1340 4110 M
113 0 V
114 0 V
113 0 V
113 0 V
114 0 V
113 0 V
113 0 V
114 0 V
113 0 V
113 0 V
114 0 V
113 0 V
113 0 V
114 0 V
113 0 V
113 0 V
113 0 V
114 0 V
113 0 V
113 0 V
114 0 V
113 0 V
113 0 V
114 0 V
113 0 V
113 0 V
114 0 V
113 0 V
113 0 V
114 0 V
113 0 V
113 0 V
114 0 V
113 0 V
113 0 V
114 0 V
113 0 V
113 0 V
114 0 V
113 0 V
113 0 V
114 0 V
113 0 V
113 0 V
114 0 V
113 0 V
113 0 V
114 0 V
113 0 V
113 0 V
113 0 V
114 0 V
113 0 V
113 0 V
114 0 V
113 0 V
113 0 V
114 0 V
113 0 V
113 0 V
114 0 V
113 0 V
113 0 V
114 0 V
113 0 V
113 0 V
114 0 V
113 0 V
113 0 V
114 0 V
113 0 V
113 0 V
114 0 V
113 0 V
113 0 V
114 0 V
113 0 V
113 0 V
114 0 V
113 0 V
113 0 V
114 0 V
113 0 V
113 0 V
113 0 V
114 0 V
113 0 V
113 0 V
114 0 V
113 0 V
113 0 V
114 0 V
113 0 V
113 0 V
114 0 V
113 0 V
113 0 V
114 0 V
113 0 V
1340 2915 M
113 0 V
114 0 V
113 0 V
113 0 V
stroke 1793 2915 M
114 0 V
113 0 V
113 0 V
114 0 V
113 0 V
113 0 V
114 0 V
113 0 V
113 0 V
114 0 V
113 0 V
113 0 V
113 0 V
114 0 V
113 0 V
113 0 V
114 0 V
113 0 V
113 0 V
114 0 V
113 0 V
113 0 V
114 0 V
113 0 V
113 0 V
114 0 V
113 0 V
113 0 V
114 0 V
113 0 V
113 0 V
114 0 V
113 0 V
113 0 V
114 0 V
113 0 V
113 0 V
114 0 V
113 0 V
113 0 V
114 0 V
113 0 V
113 0 V
114 0 V
113 0 V
113 0 V
113 0 V
114 0 V
113 0 V
113 0 V
114 0 V
113 0 V
113 0 V
114 0 V
113 0 V
113 0 V
114 0 V
113 0 V
113 0 V
114 0 V
113 0 V
113 0 V
114 0 V
113 0 V
113 0 V
114 0 V
113 0 V
113 0 V
114 0 V
113 0 V
113 0 V
114 0 V
113 0 V
113 0 V
114 0 V
113 0 V
113 0 V
114 0 V
113 0 V
113 0 V
113 0 V
114 0 V
113 0 V
113 0 V
114 0 V
113 0 V
113 0 V
114 0 V
113 0 V
113 0 V
114 0 V
113 0 V
113 0 V
114 0 V
113 0 V
1340 975 M
113 0 V
114 0 V
113 0 V
113 0 V
114 0 V
113 0 V
113 0 V
114 0 V
stroke 2247 975 M
113 0 V
113 0 V
114 0 V
113 0 V
113 0 V
114 0 V
113 0 V
113 0 V
113 0 V
114 0 V
113 0 V
113 0 V
114 0 V
113 0 V
113 0 V
114 0 V
113 0 V
113 0 V
114 0 V
113 0 V
113 0 V
114 0 V
113 0 V
113 0 V
114 0 V
113 0 V
113 0 V
114 0 V
113 0 V
113 0 V
114 0 V
113 0 V
113 0 V
114 0 V
113 0 V
113 0 V
114 0 V
113 0 V
113 0 V
114 0 V
113 0 V
113 0 V
113 0 V
114 0 V
113 0 V
113 0 V
114 0 V
113 0 V
113 0 V
114 0 V
113 0 V
113 0 V
114 0 V
113 0 V
113 0 V
114 0 V
113 0 V
113 0 V
114 0 V
113 0 V
113 0 V
114 0 V
113 0 V
113 0 V
114 0 V
113 0 V
113 0 V
114 0 V
113 0 V
113 0 V
114 0 V
113 0 V
113 0 V
114 0 V
113 0 V
113 0 V
113 0 V
114 0 V
113 0 V
113 0 V
114 0 V
113 0 V
113 0 V
114 0 V
113 0 V
113 0 V
114 0 V
113 0 V
113 0 V
114 0 V
113 0 V
stroke
1.000 UL
LTb
1340 8399 N
0 -7759 V
11219 0 V
0 7759 V
-11219 0 V
Z stroke
1.000 UP
1.000 UL
LTb
stroke
grestore
end
showpage
  }}%
  \put(2060,7636){\makebox(0,0)[r]{\strut{}$\lambda_2$}}%
  \put(2060,7876){\makebox(0,0)[r]{\strut{}$\lambda_1$}}%
  \put(2060,8116){\makebox(0,0)[r]{\strut{}$\lambda_0$}}%
  \put(6949,140){\makebox(0,0){\strut{}Relaxation Parameter $\eta$}}%
  \put(280,4519){%
  \special{ps: gsave currentpoint currentpoint translate
630 rotate neg exch neg exch translate}%
  \makebox(0,0){\strut{}Eigenvalue Estimate}%
  \special{ps: currentpoint grestore moveto}%
  }%
  \put(12401,440){\makebox(0,0){\strut{} 1}}%
  \put(11019,440){\makebox(0,0){\strut{} 0.1}}%
  \put(9636,440){\makebox(0,0){\strut{} 0.01}}%
  \put(8253,440){\makebox(0,0){\strut{} 0.001}}%
  \put(6871,440){\makebox(0,0){\strut{} 0.0001}}%
  \put(5488,440){\makebox(0,0){\strut{} 1e-05}}%
  \put(4105,440){\makebox(0,0){\strut{} 1e-06}}%
  \put(2723,440){\makebox(0,0){\strut{} 1e-07}}%
  \put(1340,440){\makebox(0,0){\strut{} 1e-08}}%
  \put(1220,8399){\makebox(0,0)[r]{\strut{} 1.025}}%
  \put(1220,7623){\makebox(0,0)[r]{\strut{} 1.02}}%
  \put(1220,6847){\makebox(0,0)[r]{\strut{} 1.015}}%
  \put(1220,6071){\makebox(0,0)[r]{\strut{} 1.01}}%
  \put(1220,5295){\makebox(0,0)[r]{\strut{} 1.005}}%
  \put(1220,4520){\makebox(0,0)[r]{\strut{} 1}}%
  \put(1220,3744){\makebox(0,0)[r]{\strut{} 0.995}}%
  \put(1220,2968){\makebox(0,0)[r]{\strut{} 0.99}}%
  \put(1220,2192){\makebox(0,0)[r]{\strut{} 0.985}}%
  \put(1220,1416){\makebox(0,0)[r]{\strut{} 0.98}}%
  \put(1220,640){\makebox(0,0)[r]{\strut{} 0.975}}%
\end{picture}%
\endgroup
\endinput

    \caption{Eigenvalue estimates for the fundamental and first two harmonics for varying values of the relaxation parameter $\eta$.  The number of particles tracked in a non-relaxed iteration is 5E6.  The heavy lines are the reference eigenvalues from \cite{Garis:1991One-s-0} and \cite{Dahl:1979Eigen-0}.}
    \label{fig:RelaxedArnoldi5E6}
\end{sidewaysfigure}

\begin{sidewaysfigure}\centering
    % GNUPLOT: LaTeX picture with Postscript
\begingroup%
\makeatletter%
\newcommand{\GNUPLOTspecial}{%
  \@sanitize\catcode`\%=14\relax\special}%
\setlength{\unitlength}{0.0500bp}%
\begin{picture}(12960,8640)(0,0)%
  {\GNUPLOTspecial{"
%!PS-Adobe-2.0 EPSF-2.0
%%Title: MoreAggFOM5E6.tex
%%Creator: gnuplot 4.3 patchlevel 0
%%CreationDate: Sat Jul 18 21:07:54 2009
%%DocumentFonts: 
%%BoundingBox: 0 0 648 432
%%EndComments
%%BeginProlog
/gnudict 256 dict def
gnudict begin
%
% The following true/false flags may be edited by hand if desired.
% The unit line width and grayscale image gamma correction may also be changed.
%
/Color true def
/Blacktext true def
/Solid false def
/Dashlength 1 def
/Landscape false def
/Level1 false def
/Rounded false def
/ClipToBoundingBox false def
/TransparentPatterns false def
/gnulinewidth 5.000 def
/userlinewidth gnulinewidth def
/Gamma 1.0 def
%
/vshift -66 def
/dl1 {
  10.0 Dashlength mul mul
  Rounded { currentlinewidth 0.75 mul sub dup 0 le { pop 0.01 } if } if
} def
/dl2 {
  10.0 Dashlength mul mul
  Rounded { currentlinewidth 0.75 mul add } if
} def
/hpt_ 31.5 def
/vpt_ 31.5 def
/hpt hpt_ def
/vpt vpt_ def
Level1 {} {
/SDict 10 dict def
systemdict /pdfmark known not {
  userdict /pdfmark systemdict /cleartomark get put
} if
SDict begin [
  /Title (MoreAggFOM5E6.tex)
  /Subject (gnuplot plot)
  /Creator (gnuplot 4.3 patchlevel 0)
  /Author (Jeremy Conlin)
%  /Producer (gnuplot)
%  /Keywords ()
  /CreationDate (Sat Jul 18 21:07:54 2009)
  /DOCINFO pdfmark
end
} ifelse
/doclip {
  ClipToBoundingBox {
    newpath 0 0 moveto 648 0 lineto 648 432 lineto 0 432 lineto closepath
    clip
  } if
} def
%
% Gnuplot Prolog Version 4.2 (November 2007)
%
/M {moveto} bind def
/L {lineto} bind def
/R {rmoveto} bind def
/V {rlineto} bind def
/N {newpath moveto} bind def
/Z {closepath} bind def
/C {setrgbcolor} bind def
/f {rlineto fill} bind def
/Gshow {show} def   % May be redefined later in the file to support UTF-8
/vpt2 vpt 2 mul def
/hpt2 hpt 2 mul def
/Lshow {currentpoint stroke M 0 vshift R 
	Blacktext {gsave 0 setgray show grestore} {show} ifelse} def
/Rshow {currentpoint stroke M dup stringwidth pop neg vshift R
	Blacktext {gsave 0 setgray show grestore} {show} ifelse} def
/Cshow {currentpoint stroke M dup stringwidth pop -2 div vshift R 
	Blacktext {gsave 0 setgray show grestore} {show} ifelse} def
/UP {dup vpt_ mul /vpt exch def hpt_ mul /hpt exch def
  /hpt2 hpt 2 mul def /vpt2 vpt 2 mul def} def
/DL {Color {setrgbcolor Solid {pop []} if 0 setdash}
 {pop pop pop 0 setgray Solid {pop []} if 0 setdash} ifelse} def
/BL {stroke userlinewidth 2 mul setlinewidth
	Rounded {1 setlinejoin 1 setlinecap} if} def
/AL {stroke userlinewidth 2 div setlinewidth
	Rounded {1 setlinejoin 1 setlinecap} if} def
/UL {dup gnulinewidth mul /userlinewidth exch def
	dup 1 lt {pop 1} if 10 mul /udl exch def} def
/PL {stroke userlinewidth setlinewidth
	Rounded {1 setlinejoin 1 setlinecap} if} def
% Default Line colors
/LCw {1 1 1} def
/LCb {0 0 0} def
/LCa {0 0 0} def
/LC0 {1 0 0} def
/LC1 {0 1 0} def
/LC2 {0 0 1} def
/LC3 {1 0 1} def
/LC4 {0 1 1} def
/LC5 {1 1 0} def
/LC6 {0 0 0} def
/LC7 {1 0.3 0} def
/LC8 {0.5 0.5 0.5} def
% Default Line Types
/LTw {PL [] 1 setgray} def
/LTb {BL [] LCb DL} def
/LTa {AL [1 udl mul 2 udl mul] 0 setdash LCa setrgbcolor} def
/LT0 {PL [] LC0 DL} def
/LT1 {PL [4 dl1 2 dl2] LC1 DL} def
/LT2 {PL [2 dl1 3 dl2] LC2 DL} def
/LT3 {PL [1 dl1 1.5 dl2] LC3 DL} def
/LT4 {PL [6 dl1 2 dl2 1 dl1 2 dl2] LC4 DL} def
/LT5 {PL [3 dl1 3 dl2 1 dl1 3 dl2] LC5 DL} def
/LT6 {PL [2 dl1 2 dl2 2 dl1 6 dl2] LC6 DL} def
/LT7 {PL [1 dl1 2 dl2 6 dl1 2 dl2 1 dl1 2 dl2] LC7 DL} def
/LT8 {PL [2 dl1 2 dl2 2 dl1 2 dl2 2 dl1 2 dl2 2 dl1 4 dl2] LC8 DL} def
/Pnt {stroke [] 0 setdash gsave 1 setlinecap M 0 0 V stroke grestore} def
/Dia {stroke [] 0 setdash 2 copy vpt add M
  hpt neg vpt neg V hpt vpt neg V
  hpt vpt V hpt neg vpt V closepath stroke
  Pnt} def
/Pls {stroke [] 0 setdash vpt sub M 0 vpt2 V
  currentpoint stroke M
  hpt neg vpt neg R hpt2 0 V stroke
 } def
/Box {stroke [] 0 setdash 2 copy exch hpt sub exch vpt add M
  0 vpt2 neg V hpt2 0 V 0 vpt2 V
  hpt2 neg 0 V closepath stroke
  Pnt} def
/Crs {stroke [] 0 setdash exch hpt sub exch vpt add M
  hpt2 vpt2 neg V currentpoint stroke M
  hpt2 neg 0 R hpt2 vpt2 V stroke} def
/TriU {stroke [] 0 setdash 2 copy vpt 1.12 mul add M
  hpt neg vpt -1.62 mul V
  hpt 2 mul 0 V
  hpt neg vpt 1.62 mul V closepath stroke
  Pnt} def
/Star {2 copy Pls Crs} def
/BoxF {stroke [] 0 setdash exch hpt sub exch vpt add M
  0 vpt2 neg V hpt2 0 V 0 vpt2 V
  hpt2 neg 0 V closepath fill} def
/TriUF {stroke [] 0 setdash vpt 1.12 mul add M
  hpt neg vpt -1.62 mul V
  hpt 2 mul 0 V
  hpt neg vpt 1.62 mul V closepath fill} def
/TriD {stroke [] 0 setdash 2 copy vpt 1.12 mul sub M
  hpt neg vpt 1.62 mul V
  hpt 2 mul 0 V
  hpt neg vpt -1.62 mul V closepath stroke
  Pnt} def
/TriDF {stroke [] 0 setdash vpt 1.12 mul sub M
  hpt neg vpt 1.62 mul V
  hpt 2 mul 0 V
  hpt neg vpt -1.62 mul V closepath fill} def
/DiaF {stroke [] 0 setdash vpt add M
  hpt neg vpt neg V hpt vpt neg V
  hpt vpt V hpt neg vpt V closepath fill} def
/Pent {stroke [] 0 setdash 2 copy gsave
  translate 0 hpt M 4 {72 rotate 0 hpt L} repeat
  closepath stroke grestore Pnt} def
/PentF {stroke [] 0 setdash gsave
  translate 0 hpt M 4 {72 rotate 0 hpt L} repeat
  closepath fill grestore} def
/Circle {stroke [] 0 setdash 2 copy
  hpt 0 360 arc stroke Pnt} def
/CircleF {stroke [] 0 setdash hpt 0 360 arc fill} def
/C0 {BL [] 0 setdash 2 copy moveto vpt 90 450 arc} bind def
/C1 {BL [] 0 setdash 2 copy moveto
	2 copy vpt 0 90 arc closepath fill
	vpt 0 360 arc closepath} bind def
/C2 {BL [] 0 setdash 2 copy moveto
	2 copy vpt 90 180 arc closepath fill
	vpt 0 360 arc closepath} bind def
/C3 {BL [] 0 setdash 2 copy moveto
	2 copy vpt 0 180 arc closepath fill
	vpt 0 360 arc closepath} bind def
/C4 {BL [] 0 setdash 2 copy moveto
	2 copy vpt 180 270 arc closepath fill
	vpt 0 360 arc closepath} bind def
/C5 {BL [] 0 setdash 2 copy moveto
	2 copy vpt 0 90 arc
	2 copy moveto
	2 copy vpt 180 270 arc closepath fill
	vpt 0 360 arc} bind def
/C6 {BL [] 0 setdash 2 copy moveto
	2 copy vpt 90 270 arc closepath fill
	vpt 0 360 arc closepath} bind def
/C7 {BL [] 0 setdash 2 copy moveto
	2 copy vpt 0 270 arc closepath fill
	vpt 0 360 arc closepath} bind def
/C8 {BL [] 0 setdash 2 copy moveto
	2 copy vpt 270 360 arc closepath fill
	vpt 0 360 arc closepath} bind def
/C9 {BL [] 0 setdash 2 copy moveto
	2 copy vpt 270 450 arc closepath fill
	vpt 0 360 arc closepath} bind def
/C10 {BL [] 0 setdash 2 copy 2 copy moveto vpt 270 360 arc closepath fill
	2 copy moveto
	2 copy vpt 90 180 arc closepath fill
	vpt 0 360 arc closepath} bind def
/C11 {BL [] 0 setdash 2 copy moveto
	2 copy vpt 0 180 arc closepath fill
	2 copy moveto
	2 copy vpt 270 360 arc closepath fill
	vpt 0 360 arc closepath} bind def
/C12 {BL [] 0 setdash 2 copy moveto
	2 copy vpt 180 360 arc closepath fill
	vpt 0 360 arc closepath} bind def
/C13 {BL [] 0 setdash 2 copy moveto
	2 copy vpt 0 90 arc closepath fill
	2 copy moveto
	2 copy vpt 180 360 arc closepath fill
	vpt 0 360 arc closepath} bind def
/C14 {BL [] 0 setdash 2 copy moveto
	2 copy vpt 90 360 arc closepath fill
	vpt 0 360 arc} bind def
/C15 {BL [] 0 setdash 2 copy vpt 0 360 arc closepath fill
	vpt 0 360 arc closepath} bind def
/Rec {newpath 4 2 roll moveto 1 index 0 rlineto 0 exch rlineto
	neg 0 rlineto closepath} bind def
/Square {dup Rec} bind def
/Bsquare {vpt sub exch vpt sub exch vpt2 Square} bind def
/S0 {BL [] 0 setdash 2 copy moveto 0 vpt rlineto BL Bsquare} bind def
/S1 {BL [] 0 setdash 2 copy vpt Square fill Bsquare} bind def
/S2 {BL [] 0 setdash 2 copy exch vpt sub exch vpt Square fill Bsquare} bind def
/S3 {BL [] 0 setdash 2 copy exch vpt sub exch vpt2 vpt Rec fill Bsquare} bind def
/S4 {BL [] 0 setdash 2 copy exch vpt sub exch vpt sub vpt Square fill Bsquare} bind def
/S5 {BL [] 0 setdash 2 copy 2 copy vpt Square fill
	exch vpt sub exch vpt sub vpt Square fill Bsquare} bind def
/S6 {BL [] 0 setdash 2 copy exch vpt sub exch vpt sub vpt vpt2 Rec fill Bsquare} bind def
/S7 {BL [] 0 setdash 2 copy exch vpt sub exch vpt sub vpt vpt2 Rec fill
	2 copy vpt Square fill Bsquare} bind def
/S8 {BL [] 0 setdash 2 copy vpt sub vpt Square fill Bsquare} bind def
/S9 {BL [] 0 setdash 2 copy vpt sub vpt vpt2 Rec fill Bsquare} bind def
/S10 {BL [] 0 setdash 2 copy vpt sub vpt Square fill 2 copy exch vpt sub exch vpt Square fill
	Bsquare} bind def
/S11 {BL [] 0 setdash 2 copy vpt sub vpt Square fill 2 copy exch vpt sub exch vpt2 vpt Rec fill
	Bsquare} bind def
/S12 {BL [] 0 setdash 2 copy exch vpt sub exch vpt sub vpt2 vpt Rec fill Bsquare} bind def
/S13 {BL [] 0 setdash 2 copy exch vpt sub exch vpt sub vpt2 vpt Rec fill
	2 copy vpt Square fill Bsquare} bind def
/S14 {BL [] 0 setdash 2 copy exch vpt sub exch vpt sub vpt2 vpt Rec fill
	2 copy exch vpt sub exch vpt Square fill Bsquare} bind def
/S15 {BL [] 0 setdash 2 copy Bsquare fill Bsquare} bind def
/D0 {gsave translate 45 rotate 0 0 S0 stroke grestore} bind def
/D1 {gsave translate 45 rotate 0 0 S1 stroke grestore} bind def
/D2 {gsave translate 45 rotate 0 0 S2 stroke grestore} bind def
/D3 {gsave translate 45 rotate 0 0 S3 stroke grestore} bind def
/D4 {gsave translate 45 rotate 0 0 S4 stroke grestore} bind def
/D5 {gsave translate 45 rotate 0 0 S5 stroke grestore} bind def
/D6 {gsave translate 45 rotate 0 0 S6 stroke grestore} bind def
/D7 {gsave translate 45 rotate 0 0 S7 stroke grestore} bind def
/D8 {gsave translate 45 rotate 0 0 S8 stroke grestore} bind def
/D9 {gsave translate 45 rotate 0 0 S9 stroke grestore} bind def
/D10 {gsave translate 45 rotate 0 0 S10 stroke grestore} bind def
/D11 {gsave translate 45 rotate 0 0 S11 stroke grestore} bind def
/D12 {gsave translate 45 rotate 0 0 S12 stroke grestore} bind def
/D13 {gsave translate 45 rotate 0 0 S13 stroke grestore} bind def
/D14 {gsave translate 45 rotate 0 0 S14 stroke grestore} bind def
/D15 {gsave translate 45 rotate 0 0 S15 stroke grestore} bind def
/DiaE {stroke [] 0 setdash vpt add M
  hpt neg vpt neg V hpt vpt neg V
  hpt vpt V hpt neg vpt V closepath stroke} def
/BoxE {stroke [] 0 setdash exch hpt sub exch vpt add M
  0 vpt2 neg V hpt2 0 V 0 vpt2 V
  hpt2 neg 0 V closepath stroke} def
/TriUE {stroke [] 0 setdash vpt 1.12 mul add M
  hpt neg vpt -1.62 mul V
  hpt 2 mul 0 V
  hpt neg vpt 1.62 mul V closepath stroke} def
/TriDE {stroke [] 0 setdash vpt 1.12 mul sub M
  hpt neg vpt 1.62 mul V
  hpt 2 mul 0 V
  hpt neg vpt -1.62 mul V closepath stroke} def
/PentE {stroke [] 0 setdash gsave
  translate 0 hpt M 4 {72 rotate 0 hpt L} repeat
  closepath stroke grestore} def
/CircE {stroke [] 0 setdash 
  hpt 0 360 arc stroke} def
/Opaque {gsave closepath 1 setgray fill grestore 0 setgray closepath} def
/DiaW {stroke [] 0 setdash vpt add M
  hpt neg vpt neg V hpt vpt neg V
  hpt vpt V hpt neg vpt V Opaque stroke} def
/BoxW {stroke [] 0 setdash exch hpt sub exch vpt add M
  0 vpt2 neg V hpt2 0 V 0 vpt2 V
  hpt2 neg 0 V Opaque stroke} def
/TriUW {stroke [] 0 setdash vpt 1.12 mul add M
  hpt neg vpt -1.62 mul V
  hpt 2 mul 0 V
  hpt neg vpt 1.62 mul V Opaque stroke} def
/TriDW {stroke [] 0 setdash vpt 1.12 mul sub M
  hpt neg vpt 1.62 mul V
  hpt 2 mul 0 V
  hpt neg vpt -1.62 mul V Opaque stroke} def
/PentW {stroke [] 0 setdash gsave
  translate 0 hpt M 4 {72 rotate 0 hpt L} repeat
  Opaque stroke grestore} def
/CircW {stroke [] 0 setdash 
  hpt 0 360 arc Opaque stroke} def
/BoxFill {gsave Rec 1 setgray fill grestore} def
/Density {
  /Fillden exch def
  currentrgbcolor
  /ColB exch def /ColG exch def /ColR exch def
  /ColR ColR Fillden mul Fillden sub 1 add def
  /ColG ColG Fillden mul Fillden sub 1 add def
  /ColB ColB Fillden mul Fillden sub 1 add def
  ColR ColG ColB setrgbcolor} def
/BoxColFill {gsave Rec PolyFill} def
/PolyFill {gsave Density fill grestore grestore} def
/h {rlineto rlineto rlineto gsave closepath fill grestore} bind def
%
% PostScript Level 1 Pattern Fill routine for rectangles
% Usage: x y w h s a XX PatternFill
%	x,y = lower left corner of box to be filled
%	w,h = width and height of box
%	  a = angle in degrees between lines and x-axis
%	 XX = 0/1 for no/yes cross-hatch
%
/PatternFill {gsave /PFa [ 9 2 roll ] def
  PFa 0 get PFa 2 get 2 div add PFa 1 get PFa 3 get 2 div add translate
  PFa 2 get -2 div PFa 3 get -2 div PFa 2 get PFa 3 get Rec
  gsave 1 setgray fill grestore clip
  currentlinewidth 0.5 mul setlinewidth
  /PFs PFa 2 get dup mul PFa 3 get dup mul add sqrt def
  0 0 M PFa 5 get rotate PFs -2 div dup translate
  0 1 PFs PFa 4 get div 1 add floor cvi
	{PFa 4 get mul 0 M 0 PFs V} for
  0 PFa 6 get ne {
	0 1 PFs PFa 4 get div 1 add floor cvi
	{PFa 4 get mul 0 2 1 roll M PFs 0 V} for
 } if
  stroke grestore} def
%
/languagelevel where
 {pop languagelevel} {1} ifelse
 2 lt
	{/InterpretLevel1 true def}
	{/InterpretLevel1 Level1 def}
 ifelse
%
% PostScript level 2 pattern fill definitions
%
/Level2PatternFill {
/Tile8x8 {/PaintType 2 /PatternType 1 /TilingType 1 /BBox [0 0 8 8] /XStep 8 /YStep 8}
	bind def
/KeepColor {currentrgbcolor [/Pattern /DeviceRGB] setcolorspace} bind def
<< Tile8x8
 /PaintProc {0.5 setlinewidth pop 0 0 M 8 8 L 0 8 M 8 0 L stroke} 
>> matrix makepattern
/Pat1 exch def
<< Tile8x8
 /PaintProc {0.5 setlinewidth pop 0 0 M 8 8 L 0 8 M 8 0 L stroke
	0 4 M 4 8 L 8 4 L 4 0 L 0 4 L stroke}
>> matrix makepattern
/Pat2 exch def
<< Tile8x8
 /PaintProc {0.5 setlinewidth pop 0 0 M 0 8 L
	8 8 L 8 0 L 0 0 L fill}
>> matrix makepattern
/Pat3 exch def
<< Tile8x8
 /PaintProc {0.5 setlinewidth pop -4 8 M 8 -4 L
	0 12 M 12 0 L stroke}
>> matrix makepattern
/Pat4 exch def
<< Tile8x8
 /PaintProc {0.5 setlinewidth pop -4 0 M 8 12 L
	0 -4 M 12 8 L stroke}
>> matrix makepattern
/Pat5 exch def
<< Tile8x8
 /PaintProc {0.5 setlinewidth pop -2 8 M 4 -4 L
	0 12 M 8 -4 L 4 12 M 10 0 L stroke}
>> matrix makepattern
/Pat6 exch def
<< Tile8x8
 /PaintProc {0.5 setlinewidth pop -2 0 M 4 12 L
	0 -4 M 8 12 L 4 -4 M 10 8 L stroke}
>> matrix makepattern
/Pat7 exch def
<< Tile8x8
 /PaintProc {0.5 setlinewidth pop 8 -2 M -4 4 L
	12 0 M -4 8 L 12 4 M 0 10 L stroke}
>> matrix makepattern
/Pat8 exch def
<< Tile8x8
 /PaintProc {0.5 setlinewidth pop 0 -2 M 12 4 L
	-4 0 M 12 8 L -4 4 M 8 10 L stroke}
>> matrix makepattern
/Pat9 exch def
/Pattern1 {PatternBgnd KeepColor Pat1 setpattern} bind def
/Pattern2 {PatternBgnd KeepColor Pat2 setpattern} bind def
/Pattern3 {PatternBgnd KeepColor Pat3 setpattern} bind def
/Pattern4 {PatternBgnd KeepColor Landscape {Pat5} {Pat4} ifelse setpattern} bind def
/Pattern5 {PatternBgnd KeepColor Landscape {Pat4} {Pat5} ifelse setpattern} bind def
/Pattern6 {PatternBgnd KeepColor Landscape {Pat9} {Pat6} ifelse setpattern} bind def
/Pattern7 {PatternBgnd KeepColor Landscape {Pat8} {Pat7} ifelse setpattern} bind def
} def
%
%
%End of PostScript Level 2 code
%
/PatternBgnd {
  TransparentPatterns {} {gsave 1 setgray fill grestore} ifelse
} def
%
% Substitute for Level 2 pattern fill codes with
% grayscale if Level 2 support is not selected.
%
/Level1PatternFill {
/Pattern1 {0.250 Density} bind def
/Pattern2 {0.500 Density} bind def
/Pattern3 {0.750 Density} bind def
/Pattern4 {0.125 Density} bind def
/Pattern5 {0.375 Density} bind def
/Pattern6 {0.625 Density} bind def
/Pattern7 {0.875 Density} bind def
} def
%
% Now test for support of Level 2 code
%
Level1 {Level1PatternFill} {Level2PatternFill} ifelse
%
/Symbol-Oblique /Symbol findfont [1 0 .167 1 0 0] makefont
dup length dict begin {1 index /FID eq {pop pop} {def} ifelse} forall
currentdict end definefont pop
end
%%EndProlog
gnudict begin
gsave
doclip
0 0 translate
0.050 0.050 scale
0 setgray
newpath
1.000 UL
LTb
1340 640 M
63 0 V
-63 1293 R
63 0 V
-63 1293 R
63 0 V
-63 1293 R
63 0 V
-63 1294 R
63 0 V
-63 1293 R
63 0 V
-63 1293 R
63 0 V
1340 640 M
0 63 V
0 7696 R
0 -63 V
1717 640 M
0 31 V
0 7728 R
0 -31 V
2215 640 M
0 31 V
0 7728 R
0 -31 V
2471 640 M
0 31 V
0 7728 R
0 -31 V
2592 640 M
0 63 V
0 7696 R
0 -63 V
2969 640 M
0 31 V
0 7728 R
0 -31 V
3467 640 M
0 31 V
0 7728 R
0 -31 V
3723 640 M
0 31 V
0 7728 R
0 -31 V
3844 640 M
0 63 V
0 7696 R
0 -63 V
4221 640 M
0 31 V
0 7728 R
0 -31 V
4719 640 M
0 31 V
0 7728 R
0 -31 V
4975 640 M
0 31 V
0 7728 R
0 -31 V
5096 640 M
0 63 V
0 7696 R
0 -63 V
5473 640 M
0 31 V
0 7728 R
0 -31 V
5971 640 M
0 31 V
0 7728 R
0 -31 V
6227 640 M
0 31 V
0 7728 R
0 -31 V
6348 640 M
0 63 V
0 7696 R
0 -63 V
6725 640 M
0 31 V
0 7728 R
0 -31 V
7223 640 M
0 31 V
0 7728 R
0 -31 V
7479 640 M
0 31 V
0 7728 R
0 -31 V
7600 640 M
0 63 V
0 7696 R
0 -63 V
7977 640 M
0 31 V
0 7728 R
0 -31 V
8475 640 M
0 31 V
0 7728 R
0 -31 V
stroke 8475 8368 M
8731 640 M
0 31 V
0 7728 R
0 -31 V
8852 640 M
0 63 V
0 7696 R
0 -63 V
9229 640 M
0 31 V
0 7728 R
0 -31 V
9727 640 M
0 31 V
0 7728 R
0 -31 V
9983 640 M
0 31 V
0 7728 R
0 -31 V
10104 640 M
0 63 V
0 7696 R
0 -63 V
10481 640 M
0 31 V
0 7728 R
0 -31 V
10979 640 M
0 31 V
0 7728 R
0 -31 V
11235 640 M
0 31 V
0 7728 R
0 -31 V
11356 640 M
0 63 V
0 7696 R
0 -63 V
11499 640 M
-63 0 V
63 1108 R
-63 0 V
63 1109 R
-63 0 V
63 1108 R
-63 0 V
63 1109 R
-63 0 V
63 1108 R
-63 0 V
63 1109 R
-63 0 V
63 1108 R
-63 0 V
stroke
1340 8399 N
0 -7759 V
10159 0 V
0 7759 V
-10159 0 V
Z stroke
LCb setrgbcolor
LTb
LCb setrgbcolor
LTb
LCb setrgbcolor
LTb
LCb setrgbcolor
LTb
1.000 UP
1.000 UL
LTb
1.000 UL
LTb
1460 7656 N
0 680 V
1383 0 V
0 -680 V
-1383 0 V
Z stroke
1460 8336 M
1383 0 V
1.000 UP
stroke
LT0
LCb setrgbcolor
LT0
2180 8116 M
543 0 V
-543 31 R
0 -62 V
543 62 R
0 -62 V
1340 1250 M
498 6 V
377 5 V
221 -8 V
156 0 V
498 10 V
377 -2 V
221 0 V
654 -3 V
598 18 V
156 -31 V
498 18 V
377 11 V
221 0 V
156 -11 V
498 -2 V
377 -3 V
221 13 V
156 -5 V
498 -13 V
377 111 V
221 -75 V
156 39 V
875 362 V
377 554 V
499 1370 V
376 1500 V
221 1087 V
156 931 V
1340 1241 M
0 18 V
-31 -18 R
62 0 V
-62 18 R
62 0 V
467 -11 R
0 16 V
-31 -16 R
62 0 V
-62 16 R
62 0 V
346 -12 R
0 17 V
-31 -17 R
62 0 V
-62 17 R
62 0 V
190 -24 R
0 16 V
-31 -16 R
62 0 V
-62 16 R
62 0 V
125 -16 R
0 16 V
-31 -16 R
62 0 V
-62 16 R
62 0 V
467 -6 R
0 16 V
-31 -16 R
62 0 V
-62 16 R
62 0 V
346 -18 R
0 16 V
-31 -16 R
62 0 V
-62 16 R
62 0 V
190 -18 R
0 20 V
-31 -20 R
62 0 V
-62 20 R
62 0 V
623 -21 R
0 17 V
-31 -17 R
62 0 V
-62 17 R
62 0 V
567 1 R
0 16 V
-31 -16 R
62 0 V
-62 16 R
62 0 V
125 -47 R
0 16 V
-31 -16 R
62 0 V
-62 16 R
62 0 V
467 2 R
0 17 V
-31 -17 R
62 0 V
stroke 5625 1255 M
-62 17 R
62 0 V
346 -5 R
0 14 V
-31 -14 R
62 0 V
-62 14 R
62 0 V
190 -15 R
0 16 V
-31 -16 R
62 0 V
-62 16 R
62 0 V
125 -26 R
0 15 V
-31 -15 R
62 0 V
-62 15 R
62 0 V
467 -16 R
0 12 V
-31 -12 R
62 0 V
-62 12 R
62 0 V
346 -16 R
0 14 V
-31 -14 R
62 0 V
-62 14 R
62 0 V
190 0 R
0 12 V
-31 -12 R
62 0 V
-62 12 R
62 0 V
125 -17 R
0 11 V
-31 -11 R
62 0 V
-62 11 R
62 0 V
467 -23 R
0 10 V
-31 -10 R
62 0 V
-62 10 R
62 0 V
346 22 R
0 168 V
-31 -168 R
62 0 V
-62 168 R
62 0 V
190 -180 R
0 43 V
-31 -43 R
62 0 V
-62 43 R
62 0 V
125 -17 R
0 67 V
-31 -67 R
62 0 V
-62 67 R
62 0 V
844 272 R
0 114 V
-31 -114 R
62 0 V
-62 114 R
62 0 V
346 449 R
0 95 V
-31 -95 R
62 0 V
-62 95 R
62 0 V
468 1237 R
0 173 V
-31 -173 R
62 0 V
-62 173 R
62 0 V
345 1333 R
0 161 V
-31 -161 R
62 0 V
-62 161 R
62 0 V
190 936 R
0 139 V
-31 -139 R
62 0 V
-62 139 R
62 0 V
125 760 R
0 204 V
-31 -204 R
62 0 V
-62 204 R
62 0 V
stroke 11387 7234 M
1340 1250 Pls
1838 1256 Pls
2215 1261 Pls
2436 1253 Pls
2592 1253 Pls
3090 1263 Pls
3467 1261 Pls
3688 1261 Pls
4342 1258 Pls
4940 1276 Pls
5096 1245 Pls
5594 1263 Pls
5971 1274 Pls
6192 1274 Pls
6348 1263 Pls
6846 1261 Pls
7223 1258 Pls
7444 1271 Pls
7600 1266 Pls
8098 1253 Pls
8475 1364 Pls
8696 1289 Pls
8852 1328 Pls
9727 1690 Pls
10104 2244 Pls
10603 3614 Pls
10979 5114 Pls
11200 6201 Pls
11356 7132 Pls
2451 8116 Pls
1.000 UL
LT0
LC2 setrgbcolor
LCb setrgbcolor
LT0
LC2 setrgbcolor
2180 7876 M
543 0 V
1340 2957 M
498 669 V
377 -191 V
221 120 V
156 106 V
498 -158 V
377 -20 V
221 -882 V
654 668 V
598 462 V
156 -135 V
498 -165 V
377 861 V
221 -833 V
156 632 V
498 1882 V
7223 4856 L
221 725 V
156 1335 V
498 1175 V
8475 667 L
221 396 V
8852 808 L
9727 697 L
377 29 V
499 -61 V
376 4 V
221 10 V
156 -21 V
stroke
LT1
LC2 setrgbcolor
1340 3233 M
103 0 V
102 0 V
103 0 V
102 0 V
103 0 V
103 0 V
102 0 V
103 0 V
103 0 V
102 0 V
103 0 V
102 0 V
103 0 V
103 0 V
102 0 V
103 0 V
102 0 V
103 0 V
103 0 V
102 0 V
103 0 V
103 0 V
102 0 V
103 0 V
102 0 V
103 0 V
103 0 V
102 0 V
103 0 V
102 0 V
103 0 V
103 0 V
102 0 V
103 0 V
103 0 V
102 0 V
103 0 V
102 0 V
103 0 V
103 0 V
102 0 V
103 0 V
102 0 V
103 0 V
103 0 V
102 0 V
103 0 V
103 0 V
102 0 V
103 0 V
102 0 V
103 0 V
103 0 V
102 0 V
103 0 V
103 0 V
102 0 V
103 0 V
102 0 V
103 0 V
103 0 V
102 0 V
103 0 V
102 0 V
103 0 V
103 0 V
102 0 V
103 0 V
103 0 V
102 0 V
103 0 V
102 0 V
103 0 V
103 0 V
102 0 V
103 0 V
102 0 V
103 0 V
103 0 V
102 0 V
103 0 V
103 0 V
102 0 V
103 0 V
102 0 V
103 0 V
103 0 V
102 0 V
103 0 V
102 0 V
103 0 V
103 0 V
102 0 V
103 0 V
103 0 V
102 0 V
103 0 V
102 0 V
103 0 V
stroke
LTb
1340 8399 N
0 -7759 V
10159 0 V
0 7759 V
-10159 0 V
Z stroke
1.000 UP
1.000 UL
LTb
stroke
grestore
end
showpage
  }}%
  \put(2060,7876){\makebox(0,0)[r]{\strut{}FOM}}%
  \put(2060,8116){\makebox(0,0)[r]{\strut{}$\lambda_0$}}%
  \put(6419,140){\makebox(0,0){\strut{}Relaxation Parameter $\eta$}}%
  \put(12558,4519){%
  \special{ps: gsave currentpoint currentpoint translate
630 rotate neg exch neg exch translate}%
  \makebox(0,0){\strut{}FOM $\left(1/\sigma^2T\right)$}%
  \special{ps: currentpoint grestore moveto}%
  }%
  \put(280,4519){%
  \special{ps: gsave currentpoint currentpoint translate
630 rotate neg exch neg exch translate}%
  \makebox(0,0){\strut{}Eigenvalue Estimate}%
  \special{ps: currentpoint grestore moveto}%
  }%
  \put(11619,8399){\makebox(0,0)[l]{\strut{} 35000}}%
  \put(11619,7291){\makebox(0,0)[l]{\strut{} 30000}}%
  \put(11619,6182){\makebox(0,0)[l]{\strut{} 25000}}%
  \put(11619,5074){\makebox(0,0)[l]{\strut{} 20000}}%
  \put(11619,3965){\makebox(0,0)[l]{\strut{} 15000}}%
  \put(11619,2857){\makebox(0,0)[l]{\strut{} 10000}}%
  \put(11619,1748){\makebox(0,0)[l]{\strut{} 5000}}%
  \put(11619,640){\makebox(0,0)[l]{\strut{} 0}}%
  \put(11356,440){\makebox(0,0){\strut{} 1}}%
  \put(10104,440){\makebox(0,0){\strut{} 0.1}}%
  \put(8852,440){\makebox(0,0){\strut{} 0.01}}%
  \put(7600,440){\makebox(0,0){\strut{} 0.001}}%
  \put(6348,440){\makebox(0,0){\strut{} 0.0001}}%
  \put(5096,440){\makebox(0,0){\strut{} 1e-05}}%
  \put(3844,440){\makebox(0,0){\strut{} 1e-06}}%
  \put(2592,440){\makebox(0,0){\strut{} 1e-07}}%
  \put(1340,440){\makebox(0,0){\strut{} 1e-08}}%
  \put(1220,8399){\makebox(0,0)[r]{\strut{} 1.025}}%
  \put(1220,7106){\makebox(0,0)[r]{\strut{} 1.02}}%
  \put(1220,5813){\makebox(0,0)[r]{\strut{} 1.015}}%
  \put(1220,4519){\makebox(0,0)[r]{\strut{} 1.01}}%
  \put(1220,3226){\makebox(0,0)[r]{\strut{} 1.005}}%
  \put(1220,1933){\makebox(0,0)[r]{\strut{} 1}}%
  \put(1220,640){\makebox(0,0)[r]{\strut{} 0.995}}%
\end{picture}%
\endgroup
\endinput

    \caption{Fundamental eigenvalue estimate and figure of merit for varying values of the relaxation parameter $\eta$.  The dashed line is the value of the figure of merit when there is no relaxation.  The number of particles tracked in a non-relaxed iteration is 5E6.}
    \label{fig:MoreAggFOM5E6}
\end{sidewaysfigure}

% More aggressive
\begin{table}\centering
    \begin{tabular}{ccccrr}
        \toprule
        \multirow{2}{*}{$\eta$} & Active & Total \# & \multirow{2}{*}{$\lambda$} & \multicolumn{1}{c}{\multirow{2}{*}{FOM}} & \multicolumn{1}{c}{Time} \\
        & Restarts & Particles & & & \multicolumn{1}{c}{(s)} \\
        \midrule
               0 &   100 & 2.500e+10 & 0.9974 $\pm$  3.3\e{-5} &  11697.8 & 78476.2 \\
           1E-08 &   100 & 2.500e+10 & 0.9974 $\pm$  3.5\e{-5} &  10452.8 & 78779.5 \\
         2.5E-08 &   100 & 2.500e+10 & 0.9974 $\pm$  3.1\e{-5} &  13470.2 & 78307.7 \\
           5E-08 &   100 & 2.500e+10 & 0.9974 $\pm$  3.2\e{-5} &  12606.0 & 77916.5 \\
         7.5E-08 &   100 & 2.500e+10 & 0.9974 $\pm$  3.1\e{-5} &  13150.2 & 78104.0 \\
           1E-07 &   100 & 2.500e+10 & 0.9974 $\pm$  3.1\e{-5} &  13628.3 & 78784.8 \\
         2.5E-07 &   100 & 2.500e+10 & 0.9974 $\pm$  3.1\e{-5} &  12913.8 & 78219.4 \\
           5E-07 &   100 & 2.500e+10 & 0.9974 $\pm$  3.2\e{-5} &  12823.6 & 77835.0 \\
         7.5E-07 &   100 & 2.500e+10 & 0.9974 $\pm$  3.8\e{-5} &   8847.1 & 78099.9 \\
         2.5E-06 &   100 & 2.500e+10 & 0.9974 $\pm$  3.3\e{-5} &  11857.0 & 78717.5 \\
         7.5E-06 &   100 & 2.497e+10 & 0.9975 $\pm$  3.0\e{-5} &  13944.0 & 78135.7 \\
           1E-05 &   100 & 2.492e+10 & 0.9973 $\pm$  3.1\e{-5} &  13334.8 & 77898.4 \\
         2.5E-05 &   100 & 2.452e+10 & 0.9974 $\pm$  3.2\e{-5} &  12592.5 & 76967.0 \\
           5E-05 &   118 & 2.544e+10 & 0.9975 $\pm$  2.8\e{-5} &  16472.7 & 79677.3 \\
         7.5E-05 &   125 & 2.550e+10 & 0.9974 $\pm$  3.1\e{-5} &  12717.2 & 79890.8 \\
          0.0001 &   129 & 2.544e+10 & 0.9974 $\pm$  2.9\e{-5} &  15565.5 & 78825.1 \\
         0.00025 &   153 & 2.560e+10 & 0.9974 $\pm$  2.3\e{-5} &  24056.5 & 79514.8 \\
          0.0005 &   172 & 2.556e+10 & 0.9974 $\pm$  2.6\e{-5} &  19015.5 & 79022.0 \\
         0.00075 &   186 & 2.548e+10 & 0.9974 $\pm$  2.4\e{-5} &  22290.5 & 78981.0 \\
           0.001 &   194 & 2.515e+10 & 0.9974 $\pm$  2.1\e{-5} &  28311.4 & 78137.4 \\
          0.0025 &   237 & 2.523e+10 & 0.9974 $\pm$  2.0\e{-5} &  33610.1 & 77832.7 \\
           0.005 &   272 & 2.534e+10 & 0.9978 $\pm$  3.2\e{-4} &    121.7 & 78369.7 \\
          0.0075 &   307 & 2.470e+10 & 0.9975 $\pm$  8.3\e{-5} &   1906.6 & 76614.9 \\
            0.01 &   395 & 2.525e+10 & 0.9977 $\pm$  1.3\e{-4} &    758.7 & 78486.7 \\
            0.05 &  1571 & 2.515e+10 & 0.9991 $\pm$  2.2\e{-4} &    256.2 & 81338.9 \\
             0.1 &  2024 & 2.487e+10 & 1.0012 $\pm$  1.8\e{-4} &    389.9 & 77344.4 \\
            0.25 &  2291 & 2.497e+10 & 1.0065 $\pm$  3.3\e{-4} &    113.5 & 78881.1 \\
             0.5 &  2361 & 2.504e+10 & 1.0123 $\pm$  3.1\e{-4} &    132.7 & 78254.0 \\
            0.75 &  2377 & 2.503e+10 & 1.0165 $\pm$  2.7\e{-4} &    175.2 & 78251.1 \\
               1 &  2385 & 2.503e+10 & 1.0201 $\pm$  3.9\e{-4} &     81.1 & 79328.2 \\
        \bottomrule
    \end{tabular}
    \caption{Eigenvalue estimates for fundamental eigenvalue, figure of merit, and time for a relaxed Arnoldi simulation of a 20mfp thick slab.  Also shown is the number of  active restarts and the total number of particles tracked in the simulation.  In non-relaxed iterations, 5E6 particles were tracked.}
    \label{tab:Relaxed5E60}
\end{table}

% Harmonics 5E6
\begin{comment}
\begin{table}[ht]\centering
    \begin{tabular}{ccccrr}
        \toprule
        \multirow{2}{*}{$\eta$} & Active & Total \# & \multirow{2}{*}{$\lambda$} & \multicolumn{1}{c}{\multirow{2}{*}{FOM}} & \multicolumn{1}{c}{Time} \\
        & Restarts & Particles & & & \multicolumn{1}{c}{(s)} \\
        \midrule
               0 &   100 & 2.500e+10 & 0.9897 $\pm$  3.5\e{-5} &  10203.7 & 78476.2 \\
           1E-08 &   100 & 2.500e+10 & 0.9897 $\pm$  3.6\e{-5} &   9928.0 & 78779.5 \\
         2.5E-08 &   100 & 2.500e+10 & 0.9897 $\pm$  3.5\e{-5} &  10709.4 & 78307.7 \\
           5E-08 &   100 & 2.500e+10 & 0.9897 $\pm$  3.3\e{-5} &  11706.9 & 77916.5 \\
         7.5E-08 &   100 & 2.500e+10 & 0.9897 $\pm$  3.3\e{-5} &  12000.6 & 78104.0 \\
           1E-07 &   100 & 2.500e+10 & 0.9897 $\pm$  3.2\e{-5} &  12430.3 & 78784.8 \\
         2.5E-07 &   100 & 2.500e+10 & 0.9896 $\pm$  3.1\e{-5} &  13423.0 & 78219.4 \\
           5E-07 &   100 & 2.500e+10 & 0.9897 $\pm$  3.3\e{-5} &  11913.4 & 77835.0 \\
         7.5E-07 &   100 & 2.500e+10 & 0.9897 $\pm$  3.4\e{-5} &  10990.8 & 78099.9 \\
         2.5E-06 &   100 & 2.500e+10 & 0.9897 $\pm$  3.5\e{-5} &  10623.5 & 78717.5 \\
         7.5E-06 &   100 & 2.497e+10 & 0.9897 $\pm$  3.6\e{-5} &  10132.8 & 78135.7 \\
           1E-05 &   100 & 2.492e+10 & 0.9897 $\pm$  3.6\e{-5} &   9673.3 & 77898.4 \\
         2.5E-05 &   100 & 2.452e+10 & 0.9897 $\pm$  3.3\e{-5} &  11754.6 & 76967.0 \\
           5E-05 &   118 & 2.544e+10 & 0.9897 $\pm$  3.0\e{-5} &  13696.3 & 79677.3 \\
         7.5E-05 &   125 & 2.550e+10 & 0.9897 $\pm$  2.7\e{-5} &  16722.2 & 79890.8 \\
          0.0001 &   129 & 2.544e+10 & 0.9897 $\pm$  2.8\e{-5} &  16545.6 & 78825.1 \\
         0.00025 &   153 & 2.560e+10 & 0.9897 $\pm$  2.5\e{-5} &  20918.1 & 79514.8 \\
          0.0005 &   172 & 2.556e+10 & 0.9896 $\pm$  2.5\e{-5} &  19990.5 & 79022.0 \\
         0.00075 &   186 & 2.548e+10 & 0.9897 $\pm$  2.3\e{-5} &  23668.1 & 78981.0 \\
           0.001 &   194 & 2.515e+10 & 0.9897 $\pm$  2.3\e{-5} &  23767.6 & 78137.4 \\
          0.0025 &   237 & 2.523e+10 & 0.9897 $\pm$  2.3\e{-5} &  24527.5 & 77832.7 \\
           0.005 &   272 & 2.534e+10 & 0.9898 $\pm$  6.5\e{-5} &   2990.9 & 78369.7 \\
          0.0075 &   307 & 2.470e+10 & 0.9897 $\pm$  4.6\e{-5} &   6096.3 & 76614.9 \\
            0.01 &   395 & 2.525e+10 & 0.9898 $\pm$  6.1\e{-5} &   3402.1 & 78486.7 \\
            0.05 &  1571 & 2.515e+10 & 0.9906 $\pm$  8.8\e{-5} &   1600.9 & 81338.9 \\
             0.1 &  2024 & 2.487e+10 & 0.9913 $\pm$  5.2\e{-5} &   4872.7 & 77344.4 \\
            0.25 &  2291 & 2.497e+10 & 0.9933 $\pm$  6.8\e{-5} &   2758.2 & 78881.1 \\
             0.5 &  2361 & 2.504e+10 & 0.9964 $\pm$  6.3\e{-5} &   3196.7 & 78254.0 \\
            0.75 &  2377 & 2.503e+10 & 0.9989 $\pm$  8.0\e{-5} &   2001.2 & 78251.1 \\
               1 &  2385 & 2.503e+10 & 1.0010 $\pm$  8.6\e{-5} &   1698.6 & 79328.2 \\
        \bottomrule
    \end{tabular}
    \caption{First higher order eigenvalue estimates with relaxation tracking 5E6 particles.}
    \label{tab:Relaxed5E61}
\end{table}

\begin{table}[ht]\centering
    \begin{tabular}{ccccrr}
        \toprule
        \multirow{2}{*}{$\eta$} & Active & Total \# & \multirow{2}{*}{$\lambda$} & \multicolumn{1}{c}{\multirow{2}{*}{FOM}} & \multicolumn{1}{c}{Time} \\
        & Restarts & Particles & & & \multicolumn{1}{c}{(s)} \\
        \midrule
           0 &   100 & 2.500e+10 & 0.9772 $\pm$  3.0\e{-5} &  13952.9 & 78476.2 \\
       1E-08 &   100 & 2.500e+10 & 0.9772 $\pm$  3.2\e{-5} &  12249.8 & 78779.5 \\
     2.5E-08 &   100 & 2.500e+10 & 0.9772 $\pm$  2.7\e{-5} &  17217.9 & 78307.7 \\
       5E-08 &   100 & 2.500e+10 & 0.9772 $\pm$  2.9\e{-5} &  15078.3 & 77916.5 \\
     7.5E-08 &   100 & 2.500e+10 & 0.9772 $\pm$  2.7\e{-5} &  18139.6 & 78104.0 \\
       1E-07 &   100 & 2.500e+10 & 0.9772 $\pm$  3.0\e{-5} &  14585.1 & 78784.8 \\
     2.5E-07 &   100 & 2.500e+10 & 0.9772 $\pm$  3.1\e{-5} &  13397.8 & 78219.4 \\
       5E-07 &   100 & 2.500e+10 & 0.9771 $\pm$  2.6\e{-5} &  19720.2 & 77835.0 \\
     7.5E-07 &   100 & 2.500e+10 & 0.9771 $\pm$  3.1\e{-5} &  13764.1 & 78099.9 \\
     2.5E-06 &   100 & 2.500e+10 & 0.9772 $\pm$  2.9\e{-5} &  15404.6 & 78717.5 \\
     7.5E-06 &   100 & 2.497e+10 & 0.9772 $\pm$  2.8\e{-5} &  16578.1 & 78135.7 \\
       1E-05 &   100 & 2.492e+10 & 0.9772 $\pm$  3.0\e{-5} &  14525.6 & 77898.4 \\
     2.5E-05 &   100 & 2.452e+10 & 0.9771 $\pm$  3.1\e{-5} &  13807.7 & 76967.0 \\
       5E-05 &   118 & 2.544e+10 & 0.9772 $\pm$  2.6\e{-5} &  18452.0 & 79677.3 \\
     7.5E-05 &   125 & 2.550e+10 & 0.9772 $\pm$  2.4\e{-5} &  20911.3 & 79890.8 \\
      0.0001 &   129 & 2.544e+10 & 0.9772 $\pm$  2.6\e{-5} &  18588.6 & 78825.1 \\
     0.00025 &   153 & 2.560e+10 & 0.9772 $\pm$  2.5\e{-5} &  19495.1 & 79514.8 \\
      0.0005 &   172 & 2.556e+10 & 0.9772 $\pm$  2.4\e{-5} &  22469.2 & 79022.0 \\
     0.00075 &   186 & 2.548e+10 & 0.9772 $\pm$  2.4\e{-5} &  21277.2 & 78981.0 \\
       0.001 &   194 & 2.515e+10 & 0.9772 $\pm$  2.1\e{-5} &  27720.5 & 78137.4 \\
      0.0025 &   237 & 2.523e+10 & 0.9772 $\pm$  6.1\e{-5} &   3404.0 & 77832.7 \\
       0.005 &   272 & 2.534e+10 & 0.9776 $\pm$  1.3\e{-4} &    786.0 & 78369.7 \\
      0.0075 &   307 & 2.470e+10 & 0.9778 $\pm$  1.4\e{-4} &    681.0 & 76614.9 \\
        0.01 &   395 & 2.525e+10 & 0.9777 $\pm$  1.3\e{-4} &    707.5 & 78486.7 \\
        0.05 &  1571 & 2.515e+10 & 0.9782 $\pm$  1.0\e{-4} &   1122.4 & 81338.9 \\
         0.1 &  2024 & 2.487e+10 & 0.9784 $\pm$  6.1\e{-5} &   3467.4 & 77344.4 \\
        0.25 &  2291 & 2.497e+10 & 0.9795 $\pm$  9.3\e{-5} &   1461.2 & 78881.1 \\
         0.5 &  2361 & 2.504e+10 & 0.9819 $\pm$  1.2\e{-4} &    818.2 & 78254.0 \\
        0.75 &  2377 & 2.503e+10 & 0.9831 $\pm$  1.3\e{-4} &    792.8 & 78251.1 \\
           1 &  2385 & 2.503e+10 & 0.9849 $\pm$  1.3\e{-4} &    765.6 & 79328.2 \\
        \bottomrule
    \end{tabular}
    \caption{Second higher order eigenvalue estimates with relaxation tracking 5E6 particles.}
    \label{tab:Relaxed5E62}
\end{table}
\end{comment}

\begin{comment}     % Less aggressive
\begin{table}[ht]\centering
    \begin{tabular}{ccccrr}
        \toprule
        \multirow{2}{*}{$\eta$} & Active & Total \# & \multirow{2}{*}{$\lambda$} & \multicolumn{1}{c}{\multirow{2}{*}{FOM}} & \multicolumn{1}{c}{Time} \\
        & Restarts & Particles & & & \multicolumn{1}{c}{(s)} \\
        \midrule
               0 &   100 & 2.500e+10 & 0.9974 $\pm$  2.9\e{-5} &  14588.8 & 81199.5 \\
           1E-08 &   100 & 2.500e+10 & 0.9974 $\pm$  3.3\e{-5} &  11793.6 & 78304.7 \\
         2.5E-08 &   100 & 2.500e+10 & 0.9974 $\pm$  2.9\e{-5} &  14992.3 & 78540.6 \\
           5E-08 &   100 & 2.500e+10 & 0.9975 $\pm$  3.2\e{-5} &  12544.8 & 78433.4 \\
         7.5E-08 &   100 & 2.500e+10 & 0.9973 $\pm$  3.5\e{-5} &  10337.1 & 78848.5 \\
           1E-07 &   100 & 2.500e+10 & 0.9974 $\pm$  2.9\e{-5} &  14828.1 & 78526.8 \\
         2.5E-07 &   100 & 2.500e+10 & 0.9974 $\pm$  3.2\e{-5} &  12258.9 & 78066.4 \\
           5E-07 &   100 & 2.500e+10 & 0.9974 $\pm$  3.3\e{-5} &  12228.5 & 77405.3 \\
         7.5E-07 &   100 & 2.500e+10 & 0.9974 $\pm$  3.1\e{-5} &  13344.0 & 79346.7 \\
         2.5E-06 &   100 & 2.500e+10 & 0.9974 $\pm$  3.2\e{-5} &  12278.4 & 78020.9 \\
         7.5E-06 &   100 & 2.496e+10 & 0.9974 $\pm$  3.2\e{-5} &  12790.1 & 78677.4 \\
           1E-05 &   100 & 2.495e+10 & 0.9974 $\pm$  3.3\e{-5} &  11382.5 & 78321.1 \\
         2.5E-05 &   100 & 2.464e+10 & 0.9974 $\pm$  3.3\e{-5} &  12069.1 & 77446.8 \\
           5E-05 &   100 & 2.390e+10 & 0.9974 $\pm$  3.5\e{-5} &  11043.7 & 75064.0 \\
         7.5E-05 &   114 & 2.490e+10 & 0.9974 $\pm$  3.1\e{-5} &  13771.0 & 77773.5 \\
          0.0001 &   118 & 2.498e+10 & 0.9974 $\pm$  3.0\e{-5} &  14004.8 & 77735.2 \\
         0.00025 &   135 & 2.485e+10 & 0.9974 $\pm$  3.0\e{-5} &  14748.3 & 77769.5 \\
          0.0005 &   152 & 2.487e+10 & 0.9974 $\pm$  2.4\e{-5} &  22222.6 & 77335.2 \\
         0.00075 &   163 & 2.478e+10 & 0.9974 $\pm$  2.2\e{-5} &  26096.1 & 77428.2 \\
           0.001 &   172 & 2.481e+10 & 0.9974 $\pm$  2.4\e{-5} &  20251.0 & 83281.9 \\
          0.0025 &   209 & 2.496e+10 & 0.9974 $\pm$  2.2\e{-5} &  26145.8 & 77802.5 \\
           0.005 &   246 & 2.519e+10 & 0.9974 $\pm$  2.0\e{-5} &  30368.7 & 78527.0 \\
          0.0075 &   277 & 2.493e+10 & 0.9974 $\pm$  1.9\e{-5} &  32501.4 & 83692.5 \\
            0.01 &   326 & 2.486e+10 & 0.9974 $\pm$  1.9\e{-5} &  36465.4 & 77694.0 \\
            0.05 &   940 & 2.484e+10 & 0.9974 $\pm$  1.9\e{-5} &  35696.0 & 77564.5 \\
             0.1 &  1348 & 2.511e+10 & 0.9974 $\pm$  2.2\e{-5} &  26515.8 & 78792.5 \\
            0.25 &  1793 & 2.498e+10 & 0.9974 $\pm$  3.2\e{-5} &  12268.6 & 78116.6 \\
             0.5 &  2022 & 2.484e+10 & 0.9975 $\pm$  4.9\e{-5} &   5406.4 & 77177.2 \\
            0.75 &  2124 & 2.489e+10 & 0.9976 $\pm$  6.1\e{-5} &   3466.4 & 77363.6 \\
               1 &  2190 & 2.503e+10 & 0.9978 $\pm$  8.2\e{-5} &   1817.0 & 82681.7 \\
        \bottomrule
    \end{tabular}
    \caption{Fundamental eigenvalue estimates with relaxation tracking 5E6 particles.}
    \label{tab:Relaxed5E60}
\end{table}

\begin{table}[ht]\centering
    \begin{tabular}{ccccrr}
        \toprule
        \multirow{2}{*}{$\eta$} & Active & Total \# & \multirow{2}{*}{$\lambda$} & \multicolumn{1}{c}{\multirow{2}{*}{FOM}} & \multicolumn{1}{c}{Time} \\
        & Restarts & Particles & & & \multicolumn{1}{c}{(s)} \\
        \midrule
               0 &   100 & 2.500e+10 & 0.9897 $\pm$  3.5\e{-5} &   9805.6 & 81199.5 \\
           1E-08 &   100 & 2.500e+10 & 0.9897 $\pm$  3.1\e{-5} &  13352.1 & 78304.7 \\
         2.5E-08 &   100 & 2.500e+10 & 0.9897 $\pm$  2.9\e{-5} &  15343.7 & 78540.6 \\
           5E-08 &   100 & 2.500e+10 & 0.9897 $\pm$  3.0\e{-5} &  13708.2 & 78433.4 \\
         7.5E-08 &   100 & 2.500e+10 & 0.9897 $\pm$  3.3\e{-5} &  11648.9 & 78848.5 \\
           1E-07 &   100 & 2.500e+10 & 0.9897 $\pm$  3.2\e{-5} &  12170.2 & 78526.8 \\
         2.5E-07 &   100 & 2.500e+10 & 0.9897 $\pm$  3.3\e{-5} &  12015.4 & 78066.4 \\
           5E-07 &   100 & 2.500e+10 & 0.9897 $\pm$  3.3\e{-5} &  11598.0 & 77405.3 \\
         7.5E-07 &   100 & 2.500e+10 & 0.9897 $\pm$  3.4\e{-5} &  10670.7 & 79346.7 \\
         2.5E-06 &   100 & 2.500e+10 & 0.9897 $\pm$  3.6\e{-5} &  10124.9 & 78020.9 \\
         7.5E-06 &   100 & 2.496e+10 & 0.9897 $\pm$  3.3\e{-5} &  11831.9 & 78677.4 \\
           1E-05 &   100 & 2.495e+10 & 0.9897 $\pm$  3.2\e{-5} &  12677.8 & 78321.1 \\
         2.5E-05 &   100 & 2.464e+10 & 0.9896 $\pm$  2.9\e{-5} &  15304.8 & 77446.8 \\
           5E-05 &   100 & 2.390e+10 & 0.9897 $\pm$  3.0\e{-5} &  14510.2 & 75064.0 \\
         7.5E-05 &   114 & 2.490e+10 & 0.9897 $\pm$  2.8\e{-5} &  16190.9 & 77773.5 \\
          0.0001 &   118 & 2.498e+10 & 0.9897 $\pm$  3.1\e{-5} &  13036.2 & 77735.2 \\
         0.00025 &   135 & 2.485e+10 & 0.9897 $\pm$  2.8\e{-5} &  16892.1 & 77769.5 \\
          0.0005 &   152 & 2.487e+10 & 0.9897 $\pm$  2.6\e{-5} &  19591.2 & 77335.2 \\
         0.00075 &   163 & 2.478e+10 & 0.9896 $\pm$  2.4\e{-5} &  22133.0 & 77428.2 \\
           0.001 &   172 & 2.481e+10 & 0.9897 $\pm$  2.6\e{-5} &  17426.9 & 83281.9 \\
          0.0025 &   209 & 2.496e+10 & 0.9897 $\pm$  2.3\e{-5} &  24879.5 & 77802.5 \\
           0.005 &   246 & 2.519e+10 & 0.9897 $\pm$  1.9\e{-5} &  34125.2 & 78527.0 \\
          0.0075 &   277 & 2.493e+10 & 0.9897 $\pm$  2.2\e{-5} &  25467.6 & 83692.5 \\
            0.01 &   326 & 2.486e+10 & 0.9897 $\pm$  2.1\e{-5} &  29011.9 & 77694.0 \\
            0.05 &   940 & 2.484e+10 & 0.9897 $\pm$  2.6\e{-5} &  19482.0 & 77564.5 \\
             0.1 &  1348 & 2.511e+10 & 0.9896 $\pm$  3.0\e{-5} &  14078.9 & 78792.5 \\
            0.25 &  1793 & 2.498e+10 & 0.9897 $\pm$  4.8\e{-5} &   5589.8 & 78116.6 \\
             0.5 &  2022 & 2.484e+10 & 0.9897 $\pm$  7.5\e{-5} &   2333.8 & 77177.2 \\
            0.75 &  2124 & 2.489e+10 & 0.9897 $\pm$  8.5\e{-5} &   1776.0 & 77363.6 \\
               1 &  2190 & 2.503e+10 & 0.9896 $\pm$  9.1\e{-5} &   1464.7 & 82681.7 \\
        \bottomrule
    \end{tabular}
    \caption{First higher order eigenvalue estimates with relaxation tracking 5E6 particles.}
    \label{tab:Relaxed5E61}
\end{table}

\begin{table}[ht]\centering
    \begin{tabular}{ccccrr}
        \toprule
        \multirow{2}{*}{$\eta$} & Active & Total \# & \multirow{2}{*}{$\lambda$} & \multicolumn{1}{c}{\multirow{2}{*}{FOM}} & \multicolumn{1}{c}{Time} \\
        & Restarts & Particles & & & \multicolumn{1}{c}{(s)} \\
        \midrule
               0 &   100 & 2.500e+10 & 0.9772 $\pm$  2.9\e{-5} &  14528.6 & 81199.5 \\
           1E-08 &   100 & 2.500e+10 & 0.9772 $\pm$  2.8\e{-5} &  15915.5 & 78304.7 \\
         2.5E-08 &   100 & 2.500e+10 & 0.9771 $\pm$  2.9\e{-5} &  14812.2 & 78540.6 \\
           5E-08 &   100 & 2.500e+10 & 0.9771 $\pm$  3.3\e{-5} &  11756.7 & 78433.4 \\
         7.5E-08 &   100 & 2.500e+10 & 0.9772 $\pm$  3.1\e{-5} &  13367.4 & 78848.5 \\
           1E-07 &   100 & 2.500e+10 & 0.9772 $\pm$  2.6\e{-5} &  19080.7 & 78526.8 \\
         2.5E-07 &   100 & 2.500e+10 & 0.9772 $\pm$  2.8\e{-5} &  16309.2 & 78066.4 \\
           5E-07 &   100 & 2.500e+10 & 0.9772 $\pm$  3.2\e{-5} &  12518.9 & 77405.3 \\
         7.5E-07 &   100 & 2.500e+10 & 0.9772 $\pm$  2.9\e{-5} &  15143.0 & 79346.7 \\
         2.5E-06 &   100 & 2.500e+10 & 0.9771 $\pm$  3.0\e{-5} &  14608.4 & 78020.9 \\
         7.5E-06 &   100 & 2.496e+10 & 0.9772 $\pm$  3.0\e{-5} &  14007.1 & 78677.4 \\
           1E-05 &   100 & 2.495e+10 & 0.9772 $\pm$  2.7\e{-5} &  16977.8 & 78321.1 \\
         2.5E-05 &   100 & 2.464e+10 & 0.9772 $\pm$  2.9\e{-5} &  14846.6 & 77446.8 \\
           5E-05 &   100 & 2.390e+10 & 0.9772 $\pm$  2.5\e{-5} &  21554.5 & 75064.0 \\
         7.5E-05 &   114 & 2.490e+10 & 0.9772 $\pm$  3.0\e{-5} &  13981.5 & 77773.5 \\
          0.0001 &   118 & 2.498e+10 & 0.9772 $\pm$  2.6\e{-5} &  19106.1 & 77735.2 \\
         0.00025 &   135 & 2.485e+10 & 0.9772 $\pm$  2.7\e{-5} &  17691.3 & 77769.5 \\
          0.0005 &   152 & 2.487e+10 & 0.9772 $\pm$  2.4\e{-5} &  21919.0 & 77335.2 \\
         0.00075 &   163 & 2.478e+10 & 0.9772 $\pm$  2.5\e{-5} &  21508.5 & 77428.2 \\
           0.001 &   172 & 2.481e+10 & 0.9772 $\pm$  2.2\e{-5} &  25879.4 & 83281.9 \\
          0.0025 &   209 & 2.496e+10 & 0.9772 $\pm$  1.9\e{-5} &  34091.5 & 77802.5 \\
           0.005 &   246 & 2.519e+10 & 0.9772 $\pm$  2.0\e{-5} &  31219.7 & 78527.0 \\
          0.0075 &   277 & 2.493e+10 & 0.9772 $\pm$  1.8\e{-5} &  36500.0 & 83692.5 \\
            0.01 &   326 & 2.486e+10 & 0.9772 $\pm$  1.7\e{-5} &  43478.8 & 77694.0 \\
            0.05 &   940 & 2.484e+10 & 0.9772 $\pm$  1.7\e{-5} &  45754.9 & 77564.5 \\
             0.1 &  1348 & 2.511e+10 & 0.9772 $\pm$  2.2\e{-5} &  27421.1 & 78792.5 \\
            0.25 &  1793 & 2.498e+10 & 0.9773 $\pm$  3.7\e{-5} &   9560.7 & 78116.6 \\
             0.5 &  2022 & 2.484e+10 & 0.9772 $\pm$  5.5\e{-5} &   4217.4 & 77177.2 \\
            0.75 &  2124 & 2.489e+10 & 0.9769 $\pm$  7.1\e{-5} &   2596.2 & 77363.6 \\
               1 &  2190 & 2.503e+10 & 0.9768 $\pm$  9.0\e{-5} &   1495.3 & 82681.7 \\
        \bottomrule
    \end{tabular}
    \caption{Second higher order eigenvalue estimates with relaxation tracking 5E6 particles.}
    \label{tab:Relaxed5E62}
\end{table}
\end{comment}

These simulations have been repeated using 1E5 and 5E4 particles per iteration.  The data is not given here in tables, but it has been plotted.  In \Fref{fig:RelaxedArnoldiComboValues} I show the effect of relaxation on the eigenvalue estimates similar to \Fref{fig:RelaxedArnoldi} and \Fref{fig:RelaxedArnoldi5E6}, but have plotted for the different numbers of particles per iteration each with a different line dashing.  Plotting these together helps to identify the effect of running more or fewer particles per iteration.  We see that all cases can accurately estimate the eigenvalues for small values of $\eta$, but as $\eta$ becomes large the eigenvalue estimate will diverge.  The value of $\eta$ at the point where the eigenvalue estimate diverges ($\eta \approx 0.005$) depends upon the number of particles tracked in an iteration; the greater the number of particles tracked, the larger $\eta$ can be and still obtain a good estimate of the eigenvalue.  

In \Fref{fig:RelaxedArnoldiComboFOM} the figures of merit for calculating the fundamental eigenvalue for the four cases of different particles tracked per iteration have been plotted.  We see what we have already seen previously, that at roughly the same value of $\eta$ where the eigenvalue estimate begins to diverge, the figure of merit becomes small.  The highest figure of merit occurs for values of $\eta$ just smaller than that for which the eigenvalue estimate diverges.  This is not too surprising; as $\eta$ increases, the number of particles tracked in an iteration decreases and the computational time decreases.  Eventually the increase in variance due to poor estimates of the eigenvalue become the dominate factor in the figure of merit.  

\begin{sidewaysfigure}\centering
    % GNUPLOT: LaTeX picture with Postscript
\begingroup%
\makeatletter%
\newcommand{\GNUPLOTspecial}{%
  \@sanitize\catcode`\%=14\relax\special}%
\setlength{\unitlength}{0.0500bp}%
\begin{picture}(12960,8640)(0,0)%
  {\GNUPLOTspecial{"
%!PS-Adobe-2.0 EPSF-2.0
%%Title: RelaxedComboValues.tex
%%Creator: gnuplot 4.3 patchlevel 0
%%CreationDate: Wed Aug 12 17:50:54 2009
%%DocumentFonts: 
%%BoundingBox: 0 0 648 432
%%EndComments
%%BeginProlog
/gnudict 256 dict def
gnudict begin
%
% The following true/false flags may be edited by hand if desired.
% The unit line width and grayscale image gamma correction may also be changed.
%
/Color true def
/Blacktext true def
/Solid false def
/Dashlength 1 def
/Landscape false def
/Level1 false def
/Rounded false def
/ClipToBoundingBox false def
/TransparentPatterns false def
/gnulinewidth 5.000 def
/userlinewidth gnulinewidth def
/Gamma 1.0 def
%
/vshift -66 def
/dl1 {
  10.0 Dashlength mul mul
  Rounded { currentlinewidth 0.75 mul sub dup 0 le { pop 0.01 } if } if
} def
/dl2 {
  10.0 Dashlength mul mul
  Rounded { currentlinewidth 0.75 mul add } if
} def
/hpt_ 31.5 def
/vpt_ 31.5 def
/hpt hpt_ def
/vpt vpt_ def
Level1 {} {
/SDict 10 dict def
systemdict /pdfmark known not {
  userdict /pdfmark systemdict /cleartomark get put
} if
SDict begin [
  /Title (RelaxedComboValues.tex)
  /Subject (gnuplot plot)
  /Creator (gnuplot 4.3 patchlevel 0)
  /Author (Jeremy Conlin)
%  /Producer (gnuplot)
%  /Keywords ()
  /CreationDate (Wed Aug 12 17:50:54 2009)
  /DOCINFO pdfmark
end
} ifelse
/doclip {
  ClipToBoundingBox {
    newpath 0 0 moveto 648 0 lineto 648 432 lineto 0 432 lineto closepath
    clip
  } if
} def
%
% Gnuplot Prolog Version 4.2 (November 2007)
%
/M {moveto} bind def
/L {lineto} bind def
/R {rmoveto} bind def
/V {rlineto} bind def
/N {newpath moveto} bind def
/Z {closepath} bind def
/C {setrgbcolor} bind def
/f {rlineto fill} bind def
/Gshow {show} def   % May be redefined later in the file to support UTF-8
/vpt2 vpt 2 mul def
/hpt2 hpt 2 mul def
/Lshow {currentpoint stroke M 0 vshift R 
	Blacktext {gsave 0 setgray show grestore} {show} ifelse} def
/Rshow {currentpoint stroke M dup stringwidth pop neg vshift R
	Blacktext {gsave 0 setgray show grestore} {show} ifelse} def
/Cshow {currentpoint stroke M dup stringwidth pop -2 div vshift R 
	Blacktext {gsave 0 setgray show grestore} {show} ifelse} def
/UP {dup vpt_ mul /vpt exch def hpt_ mul /hpt exch def
  /hpt2 hpt 2 mul def /vpt2 vpt 2 mul def} def
/DL {Color {setrgbcolor Solid {pop []} if 0 setdash}
 {pop pop pop 0 setgray Solid {pop []} if 0 setdash} ifelse} def
/BL {stroke userlinewidth 2 mul setlinewidth
	Rounded {1 setlinejoin 1 setlinecap} if} def
/AL {stroke userlinewidth 2 div setlinewidth
	Rounded {1 setlinejoin 1 setlinecap} if} def
/UL {dup gnulinewidth mul /userlinewidth exch def
	dup 1 lt {pop 1} if 10 mul /udl exch def} def
/PL {stroke userlinewidth setlinewidth
	Rounded {1 setlinejoin 1 setlinecap} if} def
% Default Line colors
/LCw {1 1 1} def
/LCb {0 0 0} def
/LCa {0 0 0} def
/LC0 {1 0 0} def
/LC1 {0 1 0} def
/LC2 {0 0 1} def
/LC3 {1 0 1} def
/LC4 {0 1 1} def
/LC5 {1 1 0} def
/LC6 {0 0 0} def
/LC7 {1 0.3 0} def
/LC8 {0.5 0.5 0.5} def
% Default Line Types
/LTw {PL [] 1 setgray} def
/LTb {BL [] LCb DL} def
/LTa {AL [1 udl mul 2 udl mul] 0 setdash LCa setrgbcolor} def
/LT0 {PL [] LC0 DL} def
/LT1 {PL [4 dl1 2 dl2] LC1 DL} def
/LT2 {PL [2 dl1 3 dl2] LC2 DL} def
/LT3 {PL [1 dl1 1.5 dl2] LC3 DL} def
/LT4 {PL [6 dl1 2 dl2 1 dl1 2 dl2] LC4 DL} def
/LT5 {PL [3 dl1 3 dl2 1 dl1 3 dl2] LC5 DL} def
/LT6 {PL [2 dl1 2 dl2 2 dl1 6 dl2] LC6 DL} def
/LT7 {PL [1 dl1 2 dl2 6 dl1 2 dl2 1 dl1 2 dl2] LC7 DL} def
/LT8 {PL [2 dl1 2 dl2 2 dl1 2 dl2 2 dl1 2 dl2 2 dl1 4 dl2] LC8 DL} def
/Pnt {stroke [] 0 setdash gsave 1 setlinecap M 0 0 V stroke grestore} def
/Dia {stroke [] 0 setdash 2 copy vpt add M
  hpt neg vpt neg V hpt vpt neg V
  hpt vpt V hpt neg vpt V closepath stroke
  Pnt} def
/Pls {stroke [] 0 setdash vpt sub M 0 vpt2 V
  currentpoint stroke M
  hpt neg vpt neg R hpt2 0 V stroke
 } def
/Box {stroke [] 0 setdash 2 copy exch hpt sub exch vpt add M
  0 vpt2 neg V hpt2 0 V 0 vpt2 V
  hpt2 neg 0 V closepath stroke
  Pnt} def
/Crs {stroke [] 0 setdash exch hpt sub exch vpt add M
  hpt2 vpt2 neg V currentpoint stroke M
  hpt2 neg 0 R hpt2 vpt2 V stroke} def
/TriU {stroke [] 0 setdash 2 copy vpt 1.12 mul add M
  hpt neg vpt -1.62 mul V
  hpt 2 mul 0 V
  hpt neg vpt 1.62 mul V closepath stroke
  Pnt} def
/Star {2 copy Pls Crs} def
/BoxF {stroke [] 0 setdash exch hpt sub exch vpt add M
  0 vpt2 neg V hpt2 0 V 0 vpt2 V
  hpt2 neg 0 V closepath fill} def
/TriUF {stroke [] 0 setdash vpt 1.12 mul add M
  hpt neg vpt -1.62 mul V
  hpt 2 mul 0 V
  hpt neg vpt 1.62 mul V closepath fill} def
/TriD {stroke [] 0 setdash 2 copy vpt 1.12 mul sub M
  hpt neg vpt 1.62 mul V
  hpt 2 mul 0 V
  hpt neg vpt -1.62 mul V closepath stroke
  Pnt} def
/TriDF {stroke [] 0 setdash vpt 1.12 mul sub M
  hpt neg vpt 1.62 mul V
  hpt 2 mul 0 V
  hpt neg vpt -1.62 mul V closepath fill} def
/DiaF {stroke [] 0 setdash vpt add M
  hpt neg vpt neg V hpt vpt neg V
  hpt vpt V hpt neg vpt V closepath fill} def
/Pent {stroke [] 0 setdash 2 copy gsave
  translate 0 hpt M 4 {72 rotate 0 hpt L} repeat
  closepath stroke grestore Pnt} def
/PentF {stroke [] 0 setdash gsave
  translate 0 hpt M 4 {72 rotate 0 hpt L} repeat
  closepath fill grestore} def
/Circle {stroke [] 0 setdash 2 copy
  hpt 0 360 arc stroke Pnt} def
/CircleF {stroke [] 0 setdash hpt 0 360 arc fill} def
/C0 {BL [] 0 setdash 2 copy moveto vpt 90 450 arc} bind def
/C1 {BL [] 0 setdash 2 copy moveto
	2 copy vpt 0 90 arc closepath fill
	vpt 0 360 arc closepath} bind def
/C2 {BL [] 0 setdash 2 copy moveto
	2 copy vpt 90 180 arc closepath fill
	vpt 0 360 arc closepath} bind def
/C3 {BL [] 0 setdash 2 copy moveto
	2 copy vpt 0 180 arc closepath fill
	vpt 0 360 arc closepath} bind def
/C4 {BL [] 0 setdash 2 copy moveto
	2 copy vpt 180 270 arc closepath fill
	vpt 0 360 arc closepath} bind def
/C5 {BL [] 0 setdash 2 copy moveto
	2 copy vpt 0 90 arc
	2 copy moveto
	2 copy vpt 180 270 arc closepath fill
	vpt 0 360 arc} bind def
/C6 {BL [] 0 setdash 2 copy moveto
	2 copy vpt 90 270 arc closepath fill
	vpt 0 360 arc closepath} bind def
/C7 {BL [] 0 setdash 2 copy moveto
	2 copy vpt 0 270 arc closepath fill
	vpt 0 360 arc closepath} bind def
/C8 {BL [] 0 setdash 2 copy moveto
	2 copy vpt 270 360 arc closepath fill
	vpt 0 360 arc closepath} bind def
/C9 {BL [] 0 setdash 2 copy moveto
	2 copy vpt 270 450 arc closepath fill
	vpt 0 360 arc closepath} bind def
/C10 {BL [] 0 setdash 2 copy 2 copy moveto vpt 270 360 arc closepath fill
	2 copy moveto
	2 copy vpt 90 180 arc closepath fill
	vpt 0 360 arc closepath} bind def
/C11 {BL [] 0 setdash 2 copy moveto
	2 copy vpt 0 180 arc closepath fill
	2 copy moveto
	2 copy vpt 270 360 arc closepath fill
	vpt 0 360 arc closepath} bind def
/C12 {BL [] 0 setdash 2 copy moveto
	2 copy vpt 180 360 arc closepath fill
	vpt 0 360 arc closepath} bind def
/C13 {BL [] 0 setdash 2 copy moveto
	2 copy vpt 0 90 arc closepath fill
	2 copy moveto
	2 copy vpt 180 360 arc closepath fill
	vpt 0 360 arc closepath} bind def
/C14 {BL [] 0 setdash 2 copy moveto
	2 copy vpt 90 360 arc closepath fill
	vpt 0 360 arc} bind def
/C15 {BL [] 0 setdash 2 copy vpt 0 360 arc closepath fill
	vpt 0 360 arc closepath} bind def
/Rec {newpath 4 2 roll moveto 1 index 0 rlineto 0 exch rlineto
	neg 0 rlineto closepath} bind def
/Square {dup Rec} bind def
/Bsquare {vpt sub exch vpt sub exch vpt2 Square} bind def
/S0 {BL [] 0 setdash 2 copy moveto 0 vpt rlineto BL Bsquare} bind def
/S1 {BL [] 0 setdash 2 copy vpt Square fill Bsquare} bind def
/S2 {BL [] 0 setdash 2 copy exch vpt sub exch vpt Square fill Bsquare} bind def
/S3 {BL [] 0 setdash 2 copy exch vpt sub exch vpt2 vpt Rec fill Bsquare} bind def
/S4 {BL [] 0 setdash 2 copy exch vpt sub exch vpt sub vpt Square fill Bsquare} bind def
/S5 {BL [] 0 setdash 2 copy 2 copy vpt Square fill
	exch vpt sub exch vpt sub vpt Square fill Bsquare} bind def
/S6 {BL [] 0 setdash 2 copy exch vpt sub exch vpt sub vpt vpt2 Rec fill Bsquare} bind def
/S7 {BL [] 0 setdash 2 copy exch vpt sub exch vpt sub vpt vpt2 Rec fill
	2 copy vpt Square fill Bsquare} bind def
/S8 {BL [] 0 setdash 2 copy vpt sub vpt Square fill Bsquare} bind def
/S9 {BL [] 0 setdash 2 copy vpt sub vpt vpt2 Rec fill Bsquare} bind def
/S10 {BL [] 0 setdash 2 copy vpt sub vpt Square fill 2 copy exch vpt sub exch vpt Square fill
	Bsquare} bind def
/S11 {BL [] 0 setdash 2 copy vpt sub vpt Square fill 2 copy exch vpt sub exch vpt2 vpt Rec fill
	Bsquare} bind def
/S12 {BL [] 0 setdash 2 copy exch vpt sub exch vpt sub vpt2 vpt Rec fill Bsquare} bind def
/S13 {BL [] 0 setdash 2 copy exch vpt sub exch vpt sub vpt2 vpt Rec fill
	2 copy vpt Square fill Bsquare} bind def
/S14 {BL [] 0 setdash 2 copy exch vpt sub exch vpt sub vpt2 vpt Rec fill
	2 copy exch vpt sub exch vpt Square fill Bsquare} bind def
/S15 {BL [] 0 setdash 2 copy Bsquare fill Bsquare} bind def
/D0 {gsave translate 45 rotate 0 0 S0 stroke grestore} bind def
/D1 {gsave translate 45 rotate 0 0 S1 stroke grestore} bind def
/D2 {gsave translate 45 rotate 0 0 S2 stroke grestore} bind def
/D3 {gsave translate 45 rotate 0 0 S3 stroke grestore} bind def
/D4 {gsave translate 45 rotate 0 0 S4 stroke grestore} bind def
/D5 {gsave translate 45 rotate 0 0 S5 stroke grestore} bind def
/D6 {gsave translate 45 rotate 0 0 S6 stroke grestore} bind def
/D7 {gsave translate 45 rotate 0 0 S7 stroke grestore} bind def
/D8 {gsave translate 45 rotate 0 0 S8 stroke grestore} bind def
/D9 {gsave translate 45 rotate 0 0 S9 stroke grestore} bind def
/D10 {gsave translate 45 rotate 0 0 S10 stroke grestore} bind def
/D11 {gsave translate 45 rotate 0 0 S11 stroke grestore} bind def
/D12 {gsave translate 45 rotate 0 0 S12 stroke grestore} bind def
/D13 {gsave translate 45 rotate 0 0 S13 stroke grestore} bind def
/D14 {gsave translate 45 rotate 0 0 S14 stroke grestore} bind def
/D15 {gsave translate 45 rotate 0 0 S15 stroke grestore} bind def
/DiaE {stroke [] 0 setdash vpt add M
  hpt neg vpt neg V hpt vpt neg V
  hpt vpt V hpt neg vpt V closepath stroke} def
/BoxE {stroke [] 0 setdash exch hpt sub exch vpt add M
  0 vpt2 neg V hpt2 0 V 0 vpt2 V
  hpt2 neg 0 V closepath stroke} def
/TriUE {stroke [] 0 setdash vpt 1.12 mul add M
  hpt neg vpt -1.62 mul V
  hpt 2 mul 0 V
  hpt neg vpt 1.62 mul V closepath stroke} def
/TriDE {stroke [] 0 setdash vpt 1.12 mul sub M
  hpt neg vpt 1.62 mul V
  hpt 2 mul 0 V
  hpt neg vpt -1.62 mul V closepath stroke} def
/PentE {stroke [] 0 setdash gsave
  translate 0 hpt M 4 {72 rotate 0 hpt L} repeat
  closepath stroke grestore} def
/CircE {stroke [] 0 setdash 
  hpt 0 360 arc stroke} def
/Opaque {gsave closepath 1 setgray fill grestore 0 setgray closepath} def
/DiaW {stroke [] 0 setdash vpt add M
  hpt neg vpt neg V hpt vpt neg V
  hpt vpt V hpt neg vpt V Opaque stroke} def
/BoxW {stroke [] 0 setdash exch hpt sub exch vpt add M
  0 vpt2 neg V hpt2 0 V 0 vpt2 V
  hpt2 neg 0 V Opaque stroke} def
/TriUW {stroke [] 0 setdash vpt 1.12 mul add M
  hpt neg vpt -1.62 mul V
  hpt 2 mul 0 V
  hpt neg vpt 1.62 mul V Opaque stroke} def
/TriDW {stroke [] 0 setdash vpt 1.12 mul sub M
  hpt neg vpt 1.62 mul V
  hpt 2 mul 0 V
  hpt neg vpt -1.62 mul V Opaque stroke} def
/PentW {stroke [] 0 setdash gsave
  translate 0 hpt M 4 {72 rotate 0 hpt L} repeat
  Opaque stroke grestore} def
/CircW {stroke [] 0 setdash 
  hpt 0 360 arc Opaque stroke} def
/BoxFill {gsave Rec 1 setgray fill grestore} def
/Density {
  /Fillden exch def
  currentrgbcolor
  /ColB exch def /ColG exch def /ColR exch def
  /ColR ColR Fillden mul Fillden sub 1 add def
  /ColG ColG Fillden mul Fillden sub 1 add def
  /ColB ColB Fillden mul Fillden sub 1 add def
  ColR ColG ColB setrgbcolor} def
/BoxColFill {gsave Rec PolyFill} def
/PolyFill {gsave Density fill grestore grestore} def
/h {rlineto rlineto rlineto gsave closepath fill grestore} bind def
%
% PostScript Level 1 Pattern Fill routine for rectangles
% Usage: x y w h s a XX PatternFill
%	x,y = lower left corner of box to be filled
%	w,h = width and height of box
%	  a = angle in degrees between lines and x-axis
%	 XX = 0/1 for no/yes cross-hatch
%
/PatternFill {gsave /PFa [ 9 2 roll ] def
  PFa 0 get PFa 2 get 2 div add PFa 1 get PFa 3 get 2 div add translate
  PFa 2 get -2 div PFa 3 get -2 div PFa 2 get PFa 3 get Rec
  gsave 1 setgray fill grestore clip
  currentlinewidth 0.5 mul setlinewidth
  /PFs PFa 2 get dup mul PFa 3 get dup mul add sqrt def
  0 0 M PFa 5 get rotate PFs -2 div dup translate
  0 1 PFs PFa 4 get div 1 add floor cvi
	{PFa 4 get mul 0 M 0 PFs V} for
  0 PFa 6 get ne {
	0 1 PFs PFa 4 get div 1 add floor cvi
	{PFa 4 get mul 0 2 1 roll M PFs 0 V} for
 } if
  stroke grestore} def
%
/languagelevel where
 {pop languagelevel} {1} ifelse
 2 lt
	{/InterpretLevel1 true def}
	{/InterpretLevel1 Level1 def}
 ifelse
%
% PostScript level 2 pattern fill definitions
%
/Level2PatternFill {
/Tile8x8 {/PaintType 2 /PatternType 1 /TilingType 1 /BBox [0 0 8 8] /XStep 8 /YStep 8}
	bind def
/KeepColor {currentrgbcolor [/Pattern /DeviceRGB] setcolorspace} bind def
<< Tile8x8
 /PaintProc {0.5 setlinewidth pop 0 0 M 8 8 L 0 8 M 8 0 L stroke} 
>> matrix makepattern
/Pat1 exch def
<< Tile8x8
 /PaintProc {0.5 setlinewidth pop 0 0 M 8 8 L 0 8 M 8 0 L stroke
	0 4 M 4 8 L 8 4 L 4 0 L 0 4 L stroke}
>> matrix makepattern
/Pat2 exch def
<< Tile8x8
 /PaintProc {0.5 setlinewidth pop 0 0 M 0 8 L
	8 8 L 8 0 L 0 0 L fill}
>> matrix makepattern
/Pat3 exch def
<< Tile8x8
 /PaintProc {0.5 setlinewidth pop -4 8 M 8 -4 L
	0 12 M 12 0 L stroke}
>> matrix makepattern
/Pat4 exch def
<< Tile8x8
 /PaintProc {0.5 setlinewidth pop -4 0 M 8 12 L
	0 -4 M 12 8 L stroke}
>> matrix makepattern
/Pat5 exch def
<< Tile8x8
 /PaintProc {0.5 setlinewidth pop -2 8 M 4 -4 L
	0 12 M 8 -4 L 4 12 M 10 0 L stroke}
>> matrix makepattern
/Pat6 exch def
<< Tile8x8
 /PaintProc {0.5 setlinewidth pop -2 0 M 4 12 L
	0 -4 M 8 12 L 4 -4 M 10 8 L stroke}
>> matrix makepattern
/Pat7 exch def
<< Tile8x8
 /PaintProc {0.5 setlinewidth pop 8 -2 M -4 4 L
	12 0 M -4 8 L 12 4 M 0 10 L stroke}
>> matrix makepattern
/Pat8 exch def
<< Tile8x8
 /PaintProc {0.5 setlinewidth pop 0 -2 M 12 4 L
	-4 0 M 12 8 L -4 4 M 8 10 L stroke}
>> matrix makepattern
/Pat9 exch def
/Pattern1 {PatternBgnd KeepColor Pat1 setpattern} bind def
/Pattern2 {PatternBgnd KeepColor Pat2 setpattern} bind def
/Pattern3 {PatternBgnd KeepColor Pat3 setpattern} bind def
/Pattern4 {PatternBgnd KeepColor Landscape {Pat5} {Pat4} ifelse setpattern} bind def
/Pattern5 {PatternBgnd KeepColor Landscape {Pat4} {Pat5} ifelse setpattern} bind def
/Pattern6 {PatternBgnd KeepColor Landscape {Pat9} {Pat6} ifelse setpattern} bind def
/Pattern7 {PatternBgnd KeepColor Landscape {Pat8} {Pat7} ifelse setpattern} bind def
} def
%
%
%End of PostScript Level 2 code
%
/PatternBgnd {
  TransparentPatterns {} {gsave 1 setgray fill grestore} ifelse
} def
%
% Substitute for Level 2 pattern fill codes with
% grayscale if Level 2 support is not selected.
%
/Level1PatternFill {
/Pattern1 {0.250 Density} bind def
/Pattern2 {0.500 Density} bind def
/Pattern3 {0.750 Density} bind def
/Pattern4 {0.125 Density} bind def
/Pattern5 {0.375 Density} bind def
/Pattern6 {0.625 Density} bind def
/Pattern7 {0.875 Density} bind def
} def
%
% Now test for support of Level 2 code
%
Level1 {Level1PatternFill} {Level2PatternFill} ifelse
%
/Symbol-Oblique /Symbol findfont [1 0 .167 1 0 0] makefont
dup length dict begin {1 index /FID eq {pop pop} {def} ifelse} forall
currentdict end definefont pop
end
%%EndProlog
gnudict begin
gsave
doclip
0 0 translate
0.050 0.050 scale
0 setgray
newpath
1.000 UL
LTb
1220 640 M
63 0 V
11276 0 R
-63 0 V
1220 1748 M
63 0 V
11276 0 R
-63 0 V
1220 2857 M
63 0 V
11276 0 R
-63 0 V
1220 3965 M
63 0 V
11276 0 R
-63 0 V
1220 5074 M
63 0 V
11276 0 R
-63 0 V
1220 6182 M
63 0 V
11276 0 R
-63 0 V
1220 7291 M
63 0 V
11276 0 R
-63 0 V
1220 8399 M
63 0 V
11276 0 R
-63 0 V
1220 640 M
0 63 V
0 7696 R
0 -63 V
1641 640 M
0 31 V
0 7728 R
0 -31 V
2197 640 M
0 31 V
0 7728 R
0 -31 V
2482 640 M
0 31 V
0 7728 R
0 -31 V
2617 640 M
0 63 V
0 7696 R
0 -63 V
3038 640 M
0 31 V
0 7728 R
0 -31 V
3594 640 M
0 31 V
0 7728 R
0 -31 V
3880 640 M
0 31 V
0 7728 R
0 -31 V
4015 640 M
0 63 V
0 7696 R
0 -63 V
4436 640 M
0 31 V
0 7728 R
0 -31 V
4992 640 M
0 31 V
0 7728 R
0 -31 V
5277 640 M
0 31 V
0 7728 R
0 -31 V
5412 640 M
0 63 V
0 7696 R
0 -63 V
5833 640 M
0 31 V
0 7728 R
0 -31 V
6389 640 M
0 31 V
0 7728 R
0 -31 V
6674 640 M
0 31 V
0 7728 R
0 -31 V
6810 640 M
0 63 V
0 7696 R
0 -63 V
7231 640 M
0 31 V
0 7728 R
0 -31 V
7787 640 M
0 31 V
stroke 7787 671 M
0 7728 R
0 -31 V
8072 640 M
0 31 V
0 7728 R
0 -31 V
8207 640 M
0 63 V
0 7696 R
0 -63 V
8628 640 M
0 31 V
0 7728 R
0 -31 V
9184 640 M
0 31 V
0 7728 R
0 -31 V
9469 640 M
0 31 V
0 7728 R
0 -31 V
9605 640 M
0 63 V
0 7696 R
0 -63 V
10026 640 M
0 31 V
0 7728 R
0 -31 V
10582 640 M
0 31 V
0 7728 R
0 -31 V
10867 640 M
0 31 V
0 7728 R
0 -31 V
11002 640 M
0 63 V
0 7696 R
0 -63 V
11423 640 M
0 31 V
0 7728 R
0 -31 V
11979 640 M
0 31 V
0 7728 R
0 -31 V
12264 640 M
0 31 V
0 7728 R
0 -31 V
12400 640 M
0 63 V
0 7696 R
0 -63 V
stroke
1220 8399 N
0 -7759 V
11339 0 V
0 7759 V
-11339 0 V
Z stroke
LCb setrgbcolor
LTb
LCb setrgbcolor
LTb
LCb setrgbcolor
LTb
LCb setrgbcolor
LTb
1.000 UP
1.000 UL
LTb
1.000 UL
LTb
1340 7336 N
0 1000 V
1263 0 V
0 -1000 V
-1263 0 V
Z stroke
1340 8336 M
1263 0 V
stroke
LT0
LC0 setrgbcolor
LCb setrgbcolor
LT0
LC0 setrgbcolor
1940 8136 M
543 0 V
1220 2711 M
556 1 V
421 1 V
246 -2 V
174 0 V
557 2 V
420 0 V
246 0 V
731 -1 V
667 4 V
174 -7 V
557 4 V
420 3 V
246 0 V
175 -3 V
556 0 V
421 -1 V
246 3 V
174 -1 V
556 -3 V
421 24 V
246 -16 V
175 8 V
977 78 V
420 118 V
556 294 V
421 322 V
246 232 V
175 200 V
stroke
LT7
LC0 setrgbcolor
LCb setrgbcolor
LT7
LC0 setrgbcolor
1940 7936 M
543 0 V
1220 2713 M
556 -1 V
421 1 V
246 -3 V
174 2 V
557 7 V
420 -5 V
246 -3 V
731 3 V
667 -3 V
174 6 V
557 1 V
420 -11 V
246 -4 V
175 6 V
556 0 V
421 3 V
246 -6 V
174 4 V
556 -1 V
421 26 V
246 8 V
175 19 V
977 250 V
420 310 V
977 1092 V
421 992 V
stroke
LT2
LC0 setrgbcolor
LCb setrgbcolor
LT2
LC0 setrgbcolor
1940 7736 M
543 0 V
1220 2729 M
556 -2 V
421 -26 V
246 -5 V
174 7 V
557 -2 V
420 18 V
246 -6 V
731 -15 V
667 22 V
174 -30 V
557 -7 V
420 33 V
246 -15 V
175 -8 V
556 14 V
421 33 V
246 -34 V
174 -24 V
556 56 V
421 58 V
246 22 V
175 89 V
977 382 V
420 410 V
556 887 V
421 1031 V
246 787 V
175 759 V
stroke
LT4
LC0 setrgbcolor
LCb setrgbcolor
LT4
LC0 setrgbcolor
1940 7536 M
543 0 V
1220 2720 M
556 -14 V
421 -2 V
246 16 V
174 -8 V
557 16 V
420 6 V
246 -36 V
731 -30 V
667 65 V
174 -42 V
557 20 V
420 6 V
246 -5 V
175 13 V
556 -11 V
421 13 V
246 -2 V
174 -17 V
556 87 V
421 -16 V
246 56 V
175 133 V
977 576 V
420 455 V
556 1053 V
421 1202 V
246 1037 V
175 947 V
stroke
LT0
LC1 setrgbcolor
1220 2284 M
556 1 V
421 0 V
246 2 V
174 -1 V
557 -4 V
420 3 V
246 2 V
731 -3 V
667 4 V
174 -3 V
557 2 V
420 -3 V
246 2 V
175 -2 V
556 3 V
421 -4 V
246 3 V
174 -1 V
556 2 V
421 5 V
246 -3 V
175 3 V
977 46 V
420 37 V
556 110 V
421 174 V
246 134 V
175 119 V
stroke
LT7
LC1 setrgbcolor
1220 2293 M
556 -5 V
421 1 V
246 -9 V
174 9 V
557 -2 V
420 1 V
246 -6 V
731 -5 V
667 -5 V
174 16 V
557 4 V
420 -6 V
246 11 V
175 -14 V
556 9 V
421 -12 V
246 8 V
174 -7 V
556 5 V
421 5 V
246 3 V
175 18 V
977 105 V
420 119 V
977 698 V
421 648 V
stroke
LT2
LC1 setrgbcolor
1220 2261 M
556 31 V
421 -3 V
246 -9 V
174 2 V
557 -16 V
420 1 V
246 -3 V
731 1 V
667 30 V
174 6 V
557 -22 V
420 3 V
246 -13 V
175 0 V
556 4 V
421 9 V
246 3 V
174 -1 V
556 8 V
421 6 V
246 4 V
175 88 V
977 141 V
420 235 V
556 573 V
421 671 V
246 665 V
175 565 V
stroke
LT4
LC1 setrgbcolor
1220 2260 M
556 14 V
421 -2 V
246 9 V
174 23 V
557 -9 V
420 -16 V
246 -28 V
731 3 V
667 -29 V
174 10 V
557 26 V
420 34 V
246 -58 V
175 41 V
556 0 V
421 32 V
246 -55 V
174 23 V
556 46 V
421 -16 V
246 59 V
175 25 V
977 253 V
420 273 V
556 687 V
421 920 V
246 920 V
175 737 V
stroke
LT0
LC2 setrgbcolor
1220 1591 M
556 1 V
421 1 V
246 -1 V
174 0 V
557 0 V
420 -2 V
246 0 V
731 2 V
667 0 V
174 0 V
557 -2 V
420 2 V
246 -1 V
175 2 V
556 -1 V
421 2 V
246 -2 V
174 0 V
556 3 V
421 19 V
246 10 V
175 -2 V
977 24 V
420 12 V
556 61 V
421 135 V
246 66 V
175 102 V
stroke
LT7
LC2 setrgbcolor
1220 1594 M
556 -1 V
421 -3 V
246 4 V
174 -5 V
557 5 V
420 -9 V
246 13 V
731 -8 V
667 -2 V
174 11 V
557 -22 V
420 18 V
246 -4 V
175 6 V
556 -1 V
421 -4 V
246 4 V
174 -6 V
556 7 V
421 9 V
246 27 V
175 5 V
977 37 V
420 56 V
977 467 V
421 419 V
stroke
LT2
LC2 setrgbcolor
1220 1619 M
556 -53 V
421 20 V
246 17 V
174 -6 V
557 3 V
420 -13 V
246 24 V
731 -4 V
667 -20 V
174 9 V
557 -24 V
420 12 V
246 13 V
175 -32 V
556 14 V
421 -12 V
246 23 V
174 19 V
556 -32 V
421 2 V
246 50 V
175 -17 V
977 159 V
420 113 V
556 393 V
421 424 V
246 533 V
175 316 V
stroke
LT4
LC2 setrgbcolor
1220 1581 M
556 -4 V
421 -12 V
246 4 V
174 22 V
557 -26 V
420 45 V
246 -32 V
731 51 V
667 -27 V
174 -30 V
557 -1 V
420 -28 V
246 20 V
175 -11 V
556 47 V
421 -7 V
246 -20 V
174 4 V
556 16 V
421 36 V
246 -5 V
175 41 V
977 193 V
420 144 V
556 435 V
421 698 V
246 571 V
175 598 V
stroke
LTb
1220 8399 N
0 -7759 V
11339 0 V
0 7759 V
-11339 0 V
Z stroke
1.000 UP
1.000 UL
LTb
stroke
grestore
end
showpage
  }}%
  \put(1820,7536){\makebox(0,0)[r]{\strut{}5E4}}%
  \put(1820,7736){\makebox(0,0)[r]{\strut{}1E5}}%
  \put(1820,7936){\makebox(0,0)[r]{\strut{}5E5}}%
  \put(1820,8136){\makebox(0,0)[r]{\strut{}5E6}}%
  \put(6889,140){\makebox(0,0){\strut{}Relaxation Parameter $\eta$}}%
  \put(280,4519){%
  \special{ps: gsave currentpoint currentpoint translate
630 rotate neg exch neg exch translate}%
  \makebox(0,0){\strut{}Eigenvalue Estimate}%
  \special{ps: currentpoint grestore moveto}%
  }%
  \put(12400,440){\makebox(0,0){\strut{} 1}}%
  \put(11002,440){\makebox(0,0){\strut{} 0.1}}%
  \put(9605,440){\makebox(0,0){\strut{} 0.01}}%
  \put(8207,440){\makebox(0,0){\strut{} 0.001}}%
  \put(6810,440){\makebox(0,0){\strut{} 0.0001}}%
  \put(5412,440){\makebox(0,0){\strut{} 1e-05}}%
  \put(4015,440){\makebox(0,0){\strut{} 1e-06}}%
  \put(2617,440){\makebox(0,0){\strut{} 1e-07}}%
  \put(1220,440){\makebox(0,0){\strut{} 1e-08}}%
  \put(1100,8399){\makebox(0,0)[r]{\strut{} 1.1}}%
  \put(1100,7291){\makebox(0,0)[r]{\strut{} 1.08}}%
  \put(1100,6182){\makebox(0,0)[r]{\strut{} 1.06}}%
  \put(1100,5074){\makebox(0,0)[r]{\strut{} 1.04}}%
  \put(1100,3965){\makebox(0,0)[r]{\strut{} 1.02}}%
  \put(1100,2857){\makebox(0,0)[r]{\strut{} 1}}%
  \put(1100,1748){\makebox(0,0)[r]{\strut{} 0.98}}%
  \put(1100,640){\makebox(0,0)[r]{\strut{} 0.96}}%
\end{picture}%
\endgroup
\endinput

    \caption{Eigenvalue estimates for the fundamental and first two harmonics for varying values of the relaxation parameter $\eta$.  The different curves indicate the number of particles tracked in a non-relaxed iteration.}
    \label{fig:RelaxedArnoldiComboValues}
\end{sidewaysfigure}

\begin{sidewaysfigure}\centering
    % GNUPLOT: LaTeX picture with Postscript
\begingroup%
\makeatletter%
\newcommand{\GNUPLOTspecial}{%
  \@sanitize\catcode`\%=14\relax\special}%
\setlength{\unitlength}{0.0500bp}%
\begin{picture}(12960,8640)(0,0)%
  {\GNUPLOTspecial{"
%!PS-Adobe-2.0 EPSF-2.0
%%Title: RelaxedComboFOM.tex
%%Creator: gnuplot 4.3 patchlevel 0
%%CreationDate: Sat Jul 18 21:07:56 2009
%%DocumentFonts: 
%%BoundingBox: 0 0 648 432
%%EndComments
%%BeginProlog
/gnudict 256 dict def
gnudict begin
%
% The following true/false flags may be edited by hand if desired.
% The unit line width and grayscale image gamma correction may also be changed.
%
/Color true def
/Blacktext true def
/Solid false def
/Dashlength 1 def
/Landscape false def
/Level1 false def
/Rounded false def
/ClipToBoundingBox false def
/TransparentPatterns false def
/gnulinewidth 5.000 def
/userlinewidth gnulinewidth def
/Gamma 1.0 def
%
/vshift -66 def
/dl1 {
  10.0 Dashlength mul mul
  Rounded { currentlinewidth 0.75 mul sub dup 0 le { pop 0.01 } if } if
} def
/dl2 {
  10.0 Dashlength mul mul
  Rounded { currentlinewidth 0.75 mul add } if
} def
/hpt_ 31.5 def
/vpt_ 31.5 def
/hpt hpt_ def
/vpt vpt_ def
Level1 {} {
/SDict 10 dict def
systemdict /pdfmark known not {
  userdict /pdfmark systemdict /cleartomark get put
} if
SDict begin [
  /Title (RelaxedComboFOM.tex)
  /Subject (gnuplot plot)
  /Creator (gnuplot 4.3 patchlevel 0)
  /Author (Jeremy Conlin)
%  /Producer (gnuplot)
%  /Keywords ()
  /CreationDate (Sat Jul 18 21:07:56 2009)
  /DOCINFO pdfmark
end
} ifelse
/doclip {
  ClipToBoundingBox {
    newpath 0 0 moveto 648 0 lineto 648 432 lineto 0 432 lineto closepath
    clip
  } if
} def
%
% Gnuplot Prolog Version 4.2 (November 2007)
%
/M {moveto} bind def
/L {lineto} bind def
/R {rmoveto} bind def
/V {rlineto} bind def
/N {newpath moveto} bind def
/Z {closepath} bind def
/C {setrgbcolor} bind def
/f {rlineto fill} bind def
/Gshow {show} def   % May be redefined later in the file to support UTF-8
/vpt2 vpt 2 mul def
/hpt2 hpt 2 mul def
/Lshow {currentpoint stroke M 0 vshift R 
	Blacktext {gsave 0 setgray show grestore} {show} ifelse} def
/Rshow {currentpoint stroke M dup stringwidth pop neg vshift R
	Blacktext {gsave 0 setgray show grestore} {show} ifelse} def
/Cshow {currentpoint stroke M dup stringwidth pop -2 div vshift R 
	Blacktext {gsave 0 setgray show grestore} {show} ifelse} def
/UP {dup vpt_ mul /vpt exch def hpt_ mul /hpt exch def
  /hpt2 hpt 2 mul def /vpt2 vpt 2 mul def} def
/DL {Color {setrgbcolor Solid {pop []} if 0 setdash}
 {pop pop pop 0 setgray Solid {pop []} if 0 setdash} ifelse} def
/BL {stroke userlinewidth 2 mul setlinewidth
	Rounded {1 setlinejoin 1 setlinecap} if} def
/AL {stroke userlinewidth 2 div setlinewidth
	Rounded {1 setlinejoin 1 setlinecap} if} def
/UL {dup gnulinewidth mul /userlinewidth exch def
	dup 1 lt {pop 1} if 10 mul /udl exch def} def
/PL {stroke userlinewidth setlinewidth
	Rounded {1 setlinejoin 1 setlinecap} if} def
% Default Line colors
/LCw {1 1 1} def
/LCb {0 0 0} def
/LCa {0 0 0} def
/LC0 {1 0 0} def
/LC1 {0 1 0} def
/LC2 {0 0 1} def
/LC3 {1 0 1} def
/LC4 {0 1 1} def
/LC5 {1 1 0} def
/LC6 {0 0 0} def
/LC7 {1 0.3 0} def
/LC8 {0.5 0.5 0.5} def
% Default Line Types
/LTw {PL [] 1 setgray} def
/LTb {BL [] LCb DL} def
/LTa {AL [1 udl mul 2 udl mul] 0 setdash LCa setrgbcolor} def
/LT0 {PL [] LC0 DL} def
/LT1 {PL [4 dl1 2 dl2] LC1 DL} def
/LT2 {PL [2 dl1 3 dl2] LC2 DL} def
/LT3 {PL [1 dl1 1.5 dl2] LC3 DL} def
/LT4 {PL [6 dl1 2 dl2 1 dl1 2 dl2] LC4 DL} def
/LT5 {PL [3 dl1 3 dl2 1 dl1 3 dl2] LC5 DL} def
/LT6 {PL [2 dl1 2 dl2 2 dl1 6 dl2] LC6 DL} def
/LT7 {PL [1 dl1 2 dl2 6 dl1 2 dl2 1 dl1 2 dl2] LC7 DL} def
/LT8 {PL [2 dl1 2 dl2 2 dl1 2 dl2 2 dl1 2 dl2 2 dl1 4 dl2] LC8 DL} def
/Pnt {stroke [] 0 setdash gsave 1 setlinecap M 0 0 V stroke grestore} def
/Dia {stroke [] 0 setdash 2 copy vpt add M
  hpt neg vpt neg V hpt vpt neg V
  hpt vpt V hpt neg vpt V closepath stroke
  Pnt} def
/Pls {stroke [] 0 setdash vpt sub M 0 vpt2 V
  currentpoint stroke M
  hpt neg vpt neg R hpt2 0 V stroke
 } def
/Box {stroke [] 0 setdash 2 copy exch hpt sub exch vpt add M
  0 vpt2 neg V hpt2 0 V 0 vpt2 V
  hpt2 neg 0 V closepath stroke
  Pnt} def
/Crs {stroke [] 0 setdash exch hpt sub exch vpt add M
  hpt2 vpt2 neg V currentpoint stroke M
  hpt2 neg 0 R hpt2 vpt2 V stroke} def
/TriU {stroke [] 0 setdash 2 copy vpt 1.12 mul add M
  hpt neg vpt -1.62 mul V
  hpt 2 mul 0 V
  hpt neg vpt 1.62 mul V closepath stroke
  Pnt} def
/Star {2 copy Pls Crs} def
/BoxF {stroke [] 0 setdash exch hpt sub exch vpt add M
  0 vpt2 neg V hpt2 0 V 0 vpt2 V
  hpt2 neg 0 V closepath fill} def
/TriUF {stroke [] 0 setdash vpt 1.12 mul add M
  hpt neg vpt -1.62 mul V
  hpt 2 mul 0 V
  hpt neg vpt 1.62 mul V closepath fill} def
/TriD {stroke [] 0 setdash 2 copy vpt 1.12 mul sub M
  hpt neg vpt 1.62 mul V
  hpt 2 mul 0 V
  hpt neg vpt -1.62 mul V closepath stroke
  Pnt} def
/TriDF {stroke [] 0 setdash vpt 1.12 mul sub M
  hpt neg vpt 1.62 mul V
  hpt 2 mul 0 V
  hpt neg vpt -1.62 mul V closepath fill} def
/DiaF {stroke [] 0 setdash vpt add M
  hpt neg vpt neg V hpt vpt neg V
  hpt vpt V hpt neg vpt V closepath fill} def
/Pent {stroke [] 0 setdash 2 copy gsave
  translate 0 hpt M 4 {72 rotate 0 hpt L} repeat
  closepath stroke grestore Pnt} def
/PentF {stroke [] 0 setdash gsave
  translate 0 hpt M 4 {72 rotate 0 hpt L} repeat
  closepath fill grestore} def
/Circle {stroke [] 0 setdash 2 copy
  hpt 0 360 arc stroke Pnt} def
/CircleF {stroke [] 0 setdash hpt 0 360 arc fill} def
/C0 {BL [] 0 setdash 2 copy moveto vpt 90 450 arc} bind def
/C1 {BL [] 0 setdash 2 copy moveto
	2 copy vpt 0 90 arc closepath fill
	vpt 0 360 arc closepath} bind def
/C2 {BL [] 0 setdash 2 copy moveto
	2 copy vpt 90 180 arc closepath fill
	vpt 0 360 arc closepath} bind def
/C3 {BL [] 0 setdash 2 copy moveto
	2 copy vpt 0 180 arc closepath fill
	vpt 0 360 arc closepath} bind def
/C4 {BL [] 0 setdash 2 copy moveto
	2 copy vpt 180 270 arc closepath fill
	vpt 0 360 arc closepath} bind def
/C5 {BL [] 0 setdash 2 copy moveto
	2 copy vpt 0 90 arc
	2 copy moveto
	2 copy vpt 180 270 arc closepath fill
	vpt 0 360 arc} bind def
/C6 {BL [] 0 setdash 2 copy moveto
	2 copy vpt 90 270 arc closepath fill
	vpt 0 360 arc closepath} bind def
/C7 {BL [] 0 setdash 2 copy moveto
	2 copy vpt 0 270 arc closepath fill
	vpt 0 360 arc closepath} bind def
/C8 {BL [] 0 setdash 2 copy moveto
	2 copy vpt 270 360 arc closepath fill
	vpt 0 360 arc closepath} bind def
/C9 {BL [] 0 setdash 2 copy moveto
	2 copy vpt 270 450 arc closepath fill
	vpt 0 360 arc closepath} bind def
/C10 {BL [] 0 setdash 2 copy 2 copy moveto vpt 270 360 arc closepath fill
	2 copy moveto
	2 copy vpt 90 180 arc closepath fill
	vpt 0 360 arc closepath} bind def
/C11 {BL [] 0 setdash 2 copy moveto
	2 copy vpt 0 180 arc closepath fill
	2 copy moveto
	2 copy vpt 270 360 arc closepath fill
	vpt 0 360 arc closepath} bind def
/C12 {BL [] 0 setdash 2 copy moveto
	2 copy vpt 180 360 arc closepath fill
	vpt 0 360 arc closepath} bind def
/C13 {BL [] 0 setdash 2 copy moveto
	2 copy vpt 0 90 arc closepath fill
	2 copy moveto
	2 copy vpt 180 360 arc closepath fill
	vpt 0 360 arc closepath} bind def
/C14 {BL [] 0 setdash 2 copy moveto
	2 copy vpt 90 360 arc closepath fill
	vpt 0 360 arc} bind def
/C15 {BL [] 0 setdash 2 copy vpt 0 360 arc closepath fill
	vpt 0 360 arc closepath} bind def
/Rec {newpath 4 2 roll moveto 1 index 0 rlineto 0 exch rlineto
	neg 0 rlineto closepath} bind def
/Square {dup Rec} bind def
/Bsquare {vpt sub exch vpt sub exch vpt2 Square} bind def
/S0 {BL [] 0 setdash 2 copy moveto 0 vpt rlineto BL Bsquare} bind def
/S1 {BL [] 0 setdash 2 copy vpt Square fill Bsquare} bind def
/S2 {BL [] 0 setdash 2 copy exch vpt sub exch vpt Square fill Bsquare} bind def
/S3 {BL [] 0 setdash 2 copy exch vpt sub exch vpt2 vpt Rec fill Bsquare} bind def
/S4 {BL [] 0 setdash 2 copy exch vpt sub exch vpt sub vpt Square fill Bsquare} bind def
/S5 {BL [] 0 setdash 2 copy 2 copy vpt Square fill
	exch vpt sub exch vpt sub vpt Square fill Bsquare} bind def
/S6 {BL [] 0 setdash 2 copy exch vpt sub exch vpt sub vpt vpt2 Rec fill Bsquare} bind def
/S7 {BL [] 0 setdash 2 copy exch vpt sub exch vpt sub vpt vpt2 Rec fill
	2 copy vpt Square fill Bsquare} bind def
/S8 {BL [] 0 setdash 2 copy vpt sub vpt Square fill Bsquare} bind def
/S9 {BL [] 0 setdash 2 copy vpt sub vpt vpt2 Rec fill Bsquare} bind def
/S10 {BL [] 0 setdash 2 copy vpt sub vpt Square fill 2 copy exch vpt sub exch vpt Square fill
	Bsquare} bind def
/S11 {BL [] 0 setdash 2 copy vpt sub vpt Square fill 2 copy exch vpt sub exch vpt2 vpt Rec fill
	Bsquare} bind def
/S12 {BL [] 0 setdash 2 copy exch vpt sub exch vpt sub vpt2 vpt Rec fill Bsquare} bind def
/S13 {BL [] 0 setdash 2 copy exch vpt sub exch vpt sub vpt2 vpt Rec fill
	2 copy vpt Square fill Bsquare} bind def
/S14 {BL [] 0 setdash 2 copy exch vpt sub exch vpt sub vpt2 vpt Rec fill
	2 copy exch vpt sub exch vpt Square fill Bsquare} bind def
/S15 {BL [] 0 setdash 2 copy Bsquare fill Bsquare} bind def
/D0 {gsave translate 45 rotate 0 0 S0 stroke grestore} bind def
/D1 {gsave translate 45 rotate 0 0 S1 stroke grestore} bind def
/D2 {gsave translate 45 rotate 0 0 S2 stroke grestore} bind def
/D3 {gsave translate 45 rotate 0 0 S3 stroke grestore} bind def
/D4 {gsave translate 45 rotate 0 0 S4 stroke grestore} bind def
/D5 {gsave translate 45 rotate 0 0 S5 stroke grestore} bind def
/D6 {gsave translate 45 rotate 0 0 S6 stroke grestore} bind def
/D7 {gsave translate 45 rotate 0 0 S7 stroke grestore} bind def
/D8 {gsave translate 45 rotate 0 0 S8 stroke grestore} bind def
/D9 {gsave translate 45 rotate 0 0 S9 stroke grestore} bind def
/D10 {gsave translate 45 rotate 0 0 S10 stroke grestore} bind def
/D11 {gsave translate 45 rotate 0 0 S11 stroke grestore} bind def
/D12 {gsave translate 45 rotate 0 0 S12 stroke grestore} bind def
/D13 {gsave translate 45 rotate 0 0 S13 stroke grestore} bind def
/D14 {gsave translate 45 rotate 0 0 S14 stroke grestore} bind def
/D15 {gsave translate 45 rotate 0 0 S15 stroke grestore} bind def
/DiaE {stroke [] 0 setdash vpt add M
  hpt neg vpt neg V hpt vpt neg V
  hpt vpt V hpt neg vpt V closepath stroke} def
/BoxE {stroke [] 0 setdash exch hpt sub exch vpt add M
  0 vpt2 neg V hpt2 0 V 0 vpt2 V
  hpt2 neg 0 V closepath stroke} def
/TriUE {stroke [] 0 setdash vpt 1.12 mul add M
  hpt neg vpt -1.62 mul V
  hpt 2 mul 0 V
  hpt neg vpt 1.62 mul V closepath stroke} def
/TriDE {stroke [] 0 setdash vpt 1.12 mul sub M
  hpt neg vpt 1.62 mul V
  hpt 2 mul 0 V
  hpt neg vpt -1.62 mul V closepath stroke} def
/PentE {stroke [] 0 setdash gsave
  translate 0 hpt M 4 {72 rotate 0 hpt L} repeat
  closepath stroke grestore} def
/CircE {stroke [] 0 setdash 
  hpt 0 360 arc stroke} def
/Opaque {gsave closepath 1 setgray fill grestore 0 setgray closepath} def
/DiaW {stroke [] 0 setdash vpt add M
  hpt neg vpt neg V hpt vpt neg V
  hpt vpt V hpt neg vpt V Opaque stroke} def
/BoxW {stroke [] 0 setdash exch hpt sub exch vpt add M
  0 vpt2 neg V hpt2 0 V 0 vpt2 V
  hpt2 neg 0 V Opaque stroke} def
/TriUW {stroke [] 0 setdash vpt 1.12 mul add M
  hpt neg vpt -1.62 mul V
  hpt 2 mul 0 V
  hpt neg vpt 1.62 mul V Opaque stroke} def
/TriDW {stroke [] 0 setdash vpt 1.12 mul sub M
  hpt neg vpt 1.62 mul V
  hpt 2 mul 0 V
  hpt neg vpt -1.62 mul V Opaque stroke} def
/PentW {stroke [] 0 setdash gsave
  translate 0 hpt M 4 {72 rotate 0 hpt L} repeat
  Opaque stroke grestore} def
/CircW {stroke [] 0 setdash 
  hpt 0 360 arc Opaque stroke} def
/BoxFill {gsave Rec 1 setgray fill grestore} def
/Density {
  /Fillden exch def
  currentrgbcolor
  /ColB exch def /ColG exch def /ColR exch def
  /ColR ColR Fillden mul Fillden sub 1 add def
  /ColG ColG Fillden mul Fillden sub 1 add def
  /ColB ColB Fillden mul Fillden sub 1 add def
  ColR ColG ColB setrgbcolor} def
/BoxColFill {gsave Rec PolyFill} def
/PolyFill {gsave Density fill grestore grestore} def
/h {rlineto rlineto rlineto gsave closepath fill grestore} bind def
%
% PostScript Level 1 Pattern Fill routine for rectangles
% Usage: x y w h s a XX PatternFill
%	x,y = lower left corner of box to be filled
%	w,h = width and height of box
%	  a = angle in degrees between lines and x-axis
%	 XX = 0/1 for no/yes cross-hatch
%
/PatternFill {gsave /PFa [ 9 2 roll ] def
  PFa 0 get PFa 2 get 2 div add PFa 1 get PFa 3 get 2 div add translate
  PFa 2 get -2 div PFa 3 get -2 div PFa 2 get PFa 3 get Rec
  gsave 1 setgray fill grestore clip
  currentlinewidth 0.5 mul setlinewidth
  /PFs PFa 2 get dup mul PFa 3 get dup mul add sqrt def
  0 0 M PFa 5 get rotate PFs -2 div dup translate
  0 1 PFs PFa 4 get div 1 add floor cvi
	{PFa 4 get mul 0 M 0 PFs V} for
  0 PFa 6 get ne {
	0 1 PFs PFa 4 get div 1 add floor cvi
	{PFa 4 get mul 0 2 1 roll M PFs 0 V} for
 } if
  stroke grestore} def
%
/languagelevel where
 {pop languagelevel} {1} ifelse
 2 lt
	{/InterpretLevel1 true def}
	{/InterpretLevel1 Level1 def}
 ifelse
%
% PostScript level 2 pattern fill definitions
%
/Level2PatternFill {
/Tile8x8 {/PaintType 2 /PatternType 1 /TilingType 1 /BBox [0 0 8 8] /XStep 8 /YStep 8}
	bind def
/KeepColor {currentrgbcolor [/Pattern /DeviceRGB] setcolorspace} bind def
<< Tile8x8
 /PaintProc {0.5 setlinewidth pop 0 0 M 8 8 L 0 8 M 8 0 L stroke} 
>> matrix makepattern
/Pat1 exch def
<< Tile8x8
 /PaintProc {0.5 setlinewidth pop 0 0 M 8 8 L 0 8 M 8 0 L stroke
	0 4 M 4 8 L 8 4 L 4 0 L 0 4 L stroke}
>> matrix makepattern
/Pat2 exch def
<< Tile8x8
 /PaintProc {0.5 setlinewidth pop 0 0 M 0 8 L
	8 8 L 8 0 L 0 0 L fill}
>> matrix makepattern
/Pat3 exch def
<< Tile8x8
 /PaintProc {0.5 setlinewidth pop -4 8 M 8 -4 L
	0 12 M 12 0 L stroke}
>> matrix makepattern
/Pat4 exch def
<< Tile8x8
 /PaintProc {0.5 setlinewidth pop -4 0 M 8 12 L
	0 -4 M 12 8 L stroke}
>> matrix makepattern
/Pat5 exch def
<< Tile8x8
 /PaintProc {0.5 setlinewidth pop -2 8 M 4 -4 L
	0 12 M 8 -4 L 4 12 M 10 0 L stroke}
>> matrix makepattern
/Pat6 exch def
<< Tile8x8
 /PaintProc {0.5 setlinewidth pop -2 0 M 4 12 L
	0 -4 M 8 12 L 4 -4 M 10 8 L stroke}
>> matrix makepattern
/Pat7 exch def
<< Tile8x8
 /PaintProc {0.5 setlinewidth pop 8 -2 M -4 4 L
	12 0 M -4 8 L 12 4 M 0 10 L stroke}
>> matrix makepattern
/Pat8 exch def
<< Tile8x8
 /PaintProc {0.5 setlinewidth pop 0 -2 M 12 4 L
	-4 0 M 12 8 L -4 4 M 8 10 L stroke}
>> matrix makepattern
/Pat9 exch def
/Pattern1 {PatternBgnd KeepColor Pat1 setpattern} bind def
/Pattern2 {PatternBgnd KeepColor Pat2 setpattern} bind def
/Pattern3 {PatternBgnd KeepColor Pat3 setpattern} bind def
/Pattern4 {PatternBgnd KeepColor Landscape {Pat5} {Pat4} ifelse setpattern} bind def
/Pattern5 {PatternBgnd KeepColor Landscape {Pat4} {Pat5} ifelse setpattern} bind def
/Pattern6 {PatternBgnd KeepColor Landscape {Pat9} {Pat6} ifelse setpattern} bind def
/Pattern7 {PatternBgnd KeepColor Landscape {Pat8} {Pat7} ifelse setpattern} bind def
} def
%
%
%End of PostScript Level 2 code
%
/PatternBgnd {
  TransparentPatterns {} {gsave 1 setgray fill grestore} ifelse
} def
%
% Substitute for Level 2 pattern fill codes with
% grayscale if Level 2 support is not selected.
%
/Level1PatternFill {
/Pattern1 {0.250 Density} bind def
/Pattern2 {0.500 Density} bind def
/Pattern3 {0.750 Density} bind def
/Pattern4 {0.125 Density} bind def
/Pattern5 {0.375 Density} bind def
/Pattern6 {0.625 Density} bind def
/Pattern7 {0.875 Density} bind def
} def
%
% Now test for support of Level 2 code
%
Level1 {Level1PatternFill} {Level2PatternFill} ifelse
%
/Symbol-Oblique /Symbol findfont [1 0 .167 1 0 0] makefont
dup length dict begin {1 index /FID eq {pop pop} {def} ifelse} forall
currentdict end definefont pop
end
%%EndProlog
gnudict begin
gsave
doclip
0 0 translate
0.050 0.050 scale
0 setgray
newpath
1.000 UL
LTb
1340 640 M
63 0 V
11156 0 R
-63 0 V
1340 1748 M
63 0 V
11156 0 R
-63 0 V
1340 2857 M
63 0 V
11156 0 R
-63 0 V
1340 3965 M
63 0 V
11156 0 R
-63 0 V
1340 5074 M
63 0 V
11156 0 R
-63 0 V
1340 6182 M
63 0 V
11156 0 R
-63 0 V
1340 7291 M
63 0 V
11156 0 R
-63 0 V
1340 8399 M
63 0 V
11156 0 R
-63 0 V
1340 640 M
0 63 V
0 7696 R
0 -63 V
1756 640 M
0 31 V
0 7728 R
0 -31 V
2306 640 M
0 31 V
0 7728 R
0 -31 V
2589 640 M
0 31 V
0 7728 R
0 -31 V
2723 640 M
0 63 V
0 7696 R
0 -63 V
3139 640 M
0 31 V
0 7728 R
0 -31 V
3689 640 M
0 31 V
0 7728 R
0 -31 V
3971 640 M
0 31 V
0 7728 R
0 -31 V
4105 640 M
0 63 V
0 7696 R
0 -63 V
4522 640 M
0 31 V
0 7728 R
0 -31 V
5072 640 M
0 31 V
0 7728 R
0 -31 V
5354 640 M
0 31 V
0 7728 R
0 -31 V
5488 640 M
0 63 V
0 7696 R
0 -63 V
5904 640 M
0 31 V
0 7728 R
0 -31 V
6454 640 M
0 31 V
0 7728 R
0 -31 V
6737 640 M
0 31 V
0 7728 R
0 -31 V
6871 640 M
0 63 V
0 7696 R
0 -63 V
7287 640 M
0 31 V
0 7728 R
0 -31 V
7837 640 M
0 31 V
stroke 7837 671 M
0 7728 R
0 -31 V
8119 640 M
0 31 V
0 7728 R
0 -31 V
8253 640 M
0 63 V
0 7696 R
0 -63 V
8670 640 M
0 31 V
0 7728 R
0 -31 V
9220 640 M
0 31 V
0 7728 R
0 -31 V
9502 640 M
0 31 V
0 7728 R
0 -31 V
9636 640 M
0 63 V
0 7696 R
0 -63 V
10052 640 M
0 31 V
0 7728 R
0 -31 V
10603 640 M
0 31 V
0 7728 R
0 -31 V
10885 640 M
0 31 V
0 7728 R
0 -31 V
11019 640 M
0 63 V
0 7696 R
0 -63 V
11435 640 M
0 31 V
0 7728 R
0 -31 V
11985 640 M
0 31 V
0 7728 R
0 -31 V
12267 640 M
0 31 V
0 7728 R
0 -31 V
12401 640 M
0 63 V
0 7696 R
0 -63 V
stroke
1340 8399 N
0 -7759 V
11219 0 V
0 7759 V
-11219 0 V
Z stroke
LCb setrgbcolor
LTb
LCb setrgbcolor
LTb
LCb setrgbcolor
LTb
LCb setrgbcolor
LTb
1.000 UP
1.000 UL
LTb
1.000 UL
LTb
1460 7336 N
0 1000 V
1263 0 V
0 -1000 V
-1263 0 V
Z stroke
1460 8336 M
1263 0 V
stroke
LT0
LC2 setrgbcolor
LCb setrgbcolor
LT0
LC2 setrgbcolor
2060 8136 M
543 0 V
1340 2957 M
550 669 V
416 -191 V
244 120 V
173 106 V
550 -158 V
416 -20 V
244 -882 V
723 668 V
659 462 V
173 -135 V
550 -165 V
416 861 V
244 -833 V
173 632 V
550 1882 V
7837 4856 L
244 725 V
172 1335 V
551 1175 V
9220 667 L
243 396 V
9636 808 L
967 -111 V
416 29 V
550 -61 V
416 4 V
244 10 V
172 -21 V
stroke
LT7
LC2 setrgbcolor
LCb setrgbcolor
LT7
LC2 setrgbcolor
2060 7936 M
543 0 V
1340 3491 M
550 -995 V
416 377 V
244 395 V
173 87 V
550 -224 V
416 -25 V
244 465 V
723 -181 V
659 381 V
173 -819 V
550 16 V
416 1747 V
6698 3399 L
173 -258 V
550 1727 V
7837 3396 L
244 764 V
172 696 V
551 536 V
9220 877 L
243 303 V
9636 853 L
967 18 V
416 -42 V
966 -31 V
416 -1 V
stroke
LT2
LC2 setrgbcolor
LCb setrgbcolor
LT2
LC2 setrgbcolor
2060 7736 M
543 0 V
1340 2400 M
550 222 V
416 469 V
244 227 V
173 -82 V
550 -461 V
416 538 V
244 -58 V
723 -254 V
659 1158 V
173 -963 V
550 -110 V
416 430 V
244 -368 V
173 167 V
550 149 V
416 -353 V
244 674 V
172 173 V
8804 2167 L
9220 1125 L
243 -22 V
173 -62 V
967 877 V
416 -664 V
550 589 V
416 -433 V
244 -263 V
172 9 V
stroke
LT4
LC2 setrgbcolor
LCb setrgbcolor
LT4
LC2 setrgbcolor
2060 7536 M
543 0 V
1340 2789 M
550 658 V
416 -260 V
244 694 V
173 -495 V
550 -567 V
416 242 V
244 228 V
723 -373 V
659 -371 V
173 868 V
550 -604 V
416 1278 V
6698 2876 L
173 1351 V
7421 2850 L
416 366 V
244 -577 V
172 672 V
8804 1378 L
416 21 V
243 -212 V
173 -26 V
967 81 V
416 551 V
550 267 V
416 166 V
244 314 V
172 -697 V
stroke
LTb
1340 8399 N
0 -7759 V
11219 0 V
0 7759 V
-11219 0 V
Z stroke
1.000 UP
1.000 UL
LTb
stroke
grestore
end
showpage
  }}%
  \put(1940,7536){\makebox(0,0)[r]{\strut{}5E4}}%
  \put(1940,7736){\makebox(0,0)[r]{\strut{}1E5}}%
  \put(1940,7936){\makebox(0,0)[r]{\strut{}5E5}}%
  \put(1940,8136){\makebox(0,0)[r]{\strut{}5E6}}%
  \put(6949,140){\makebox(0,0){\strut{}Relaxation Parameter $\eta$}}%
  \put(280,4519){%
  \special{ps: gsave currentpoint currentpoint translate
630 rotate neg exch neg exch translate}%
  \makebox(0,0){\strut{}FOM $\left(1/\sigma^2T\right)$}%
  \special{ps: currentpoint grestore moveto}%
  }%
  \put(12401,440){\makebox(0,0){\strut{} 1}}%
  \put(11019,440){\makebox(0,0){\strut{} 0.1}}%
  \put(9636,440){\makebox(0,0){\strut{} 0.01}}%
  \put(8253,440){\makebox(0,0){\strut{} 0.001}}%
  \put(6871,440){\makebox(0,0){\strut{} 0.0001}}%
  \put(5488,440){\makebox(0,0){\strut{} 1e-05}}%
  \put(4105,440){\makebox(0,0){\strut{} 1e-06}}%
  \put(2723,440){\makebox(0,0){\strut{} 1e-07}}%
  \put(1340,440){\makebox(0,0){\strut{} 1e-08}}%
  \put(1220,8399){\makebox(0,0)[r]{\strut{} 35000}}%
  \put(1220,7291){\makebox(0,0)[r]{\strut{} 30000}}%
  \put(1220,6182){\makebox(0,0)[r]{\strut{} 25000}}%
  \put(1220,5074){\makebox(0,0)[r]{\strut{} 20000}}%
  \put(1220,3965){\makebox(0,0)[r]{\strut{} 15000}}%
  \put(1220,2857){\makebox(0,0)[r]{\strut{} 10000}}%
  \put(1220,1748){\makebox(0,0)[r]{\strut{} 5000}}%
  \put(1220,640){\makebox(0,0)[r]{\strut{} 0}}%
\end{picture}%
\endgroup
\endinput

    \caption{Figure of merit for varying values of the relaxation parameter $\eta$.  The different curves indicate the number of particles tracked in a non-relaxed iteration.}
    \label{fig:RelaxedArnoldiComboFOM}
\end{sidewaysfigure}

\clearpage
\subsection{Linear relaxation \label{sec:LinearRelaxation} }
With \Fref{eq:NkFinalLessAggressive} it was mentioned that different relaxation strategies could be employed.  All the results shown so far have used a quadratic relaxation as given in \Fref{eq:NkFinal}.  To demonstrate the effect using a different relaxation strategy, I have performed simulations using a linear relaxation strategy for the same problems that have already been shown using the quadratic relaxation.  (See also \citet{Conlin:2009Relax-0}.)  In this section is shown the results of this strategy and a brief comparison between the two methods.  

Any relaxation strategy begins to relax at the same time, whenever the residual is less than $\eta$.  A quadratic strategy will reduce the number of particles tracked in an iteration more aggressively than will a linear strategy.  \Fref{fig:MoreandLessValue} shows the eigenvalue estimates for both linear and quadratic relaxation strategies.  (The data for the linear strategy is given in \Fref{tab:Linear0}.)  The quadratic results are shown with the solid, colored lines while the linear results used the dashed lines.  We see that for the quadratic strategy for relaxation the eigenvalue estimates begin to diverge from the true eigenvalue for smaller values of $\eta$ as compared to the linear strategy.  
\begin{sidewaysfigure}\centering
    % GNUPLOT: LaTeX picture with Postscript
\begingroup%
\makeatletter%
\newcommand{\GNUPLOTspecial}{%
  \@sanitize\catcode`\%=14\relax\special}%
\setlength{\unitlength}{0.0500bp}%
\begin{picture}(12960,8640)(0,0)%
  {\GNUPLOTspecial{"
%!PS-Adobe-2.0 EPSF-2.0
%%Title: MoreandLessValue.tex
%%Creator: gnuplot 4.3 patchlevel 0
%%CreationDate: Sat Jul 18 21:07:55 2009
%%DocumentFonts: 
%%BoundingBox: 0 0 648 432
%%EndComments
%%BeginProlog
/gnudict 256 dict def
gnudict begin
%
% The following true/false flags may be edited by hand if desired.
% The unit line width and grayscale image gamma correction may also be changed.
%
/Color true def
/Blacktext true def
/Solid false def
/Dashlength 1 def
/Landscape false def
/Level1 false def
/Rounded false def
/ClipToBoundingBox false def
/TransparentPatterns false def
/gnulinewidth 5.000 def
/userlinewidth gnulinewidth def
/Gamma 1.0 def
%
/vshift -66 def
/dl1 {
  10.0 Dashlength mul mul
  Rounded { currentlinewidth 0.75 mul sub dup 0 le { pop 0.01 } if } if
} def
/dl2 {
  10.0 Dashlength mul mul
  Rounded { currentlinewidth 0.75 mul add } if
} def
/hpt_ 31.5 def
/vpt_ 31.5 def
/hpt hpt_ def
/vpt vpt_ def
Level1 {} {
/SDict 10 dict def
systemdict /pdfmark known not {
  userdict /pdfmark systemdict /cleartomark get put
} if
SDict begin [
  /Title (MoreandLessValue.tex)
  /Subject (gnuplot plot)
  /Creator (gnuplot 4.3 patchlevel 0)
  /Author (Jeremy Conlin)
%  /Producer (gnuplot)
%  /Keywords ()
  /CreationDate (Sat Jul 18 21:07:55 2009)
  /DOCINFO pdfmark
end
} ifelse
/doclip {
  ClipToBoundingBox {
    newpath 0 0 moveto 648 0 lineto 648 432 lineto 0 432 lineto closepath
    clip
  } if
} def
%
% Gnuplot Prolog Version 4.2 (November 2007)
%
/M {moveto} bind def
/L {lineto} bind def
/R {rmoveto} bind def
/V {rlineto} bind def
/N {newpath moveto} bind def
/Z {closepath} bind def
/C {setrgbcolor} bind def
/f {rlineto fill} bind def
/Gshow {show} def   % May be redefined later in the file to support UTF-8
/vpt2 vpt 2 mul def
/hpt2 hpt 2 mul def
/Lshow {currentpoint stroke M 0 vshift R 
	Blacktext {gsave 0 setgray show grestore} {show} ifelse} def
/Rshow {currentpoint stroke M dup stringwidth pop neg vshift R
	Blacktext {gsave 0 setgray show grestore} {show} ifelse} def
/Cshow {currentpoint stroke M dup stringwidth pop -2 div vshift R 
	Blacktext {gsave 0 setgray show grestore} {show} ifelse} def
/UP {dup vpt_ mul /vpt exch def hpt_ mul /hpt exch def
  /hpt2 hpt 2 mul def /vpt2 vpt 2 mul def} def
/DL {Color {setrgbcolor Solid {pop []} if 0 setdash}
 {pop pop pop 0 setgray Solid {pop []} if 0 setdash} ifelse} def
/BL {stroke userlinewidth 2 mul setlinewidth
	Rounded {1 setlinejoin 1 setlinecap} if} def
/AL {stroke userlinewidth 2 div setlinewidth
	Rounded {1 setlinejoin 1 setlinecap} if} def
/UL {dup gnulinewidth mul /userlinewidth exch def
	dup 1 lt {pop 1} if 10 mul /udl exch def} def
/PL {stroke userlinewidth setlinewidth
	Rounded {1 setlinejoin 1 setlinecap} if} def
% Default Line colors
/LCw {1 1 1} def
/LCb {0 0 0} def
/LCa {0 0 0} def
/LC0 {1 0 0} def
/LC1 {0 1 0} def
/LC2 {0 0 1} def
/LC3 {1 0 1} def
/LC4 {0 1 1} def
/LC5 {1 1 0} def
/LC6 {0 0 0} def
/LC7 {1 0.3 0} def
/LC8 {0.5 0.5 0.5} def
% Default Line Types
/LTw {PL [] 1 setgray} def
/LTb {BL [] LCb DL} def
/LTa {AL [1 udl mul 2 udl mul] 0 setdash LCa setrgbcolor} def
/LT0 {PL [] LC0 DL} def
/LT1 {PL [4 dl1 2 dl2] LC1 DL} def
/LT2 {PL [2 dl1 3 dl2] LC2 DL} def
/LT3 {PL [1 dl1 1.5 dl2] LC3 DL} def
/LT4 {PL [6 dl1 2 dl2 1 dl1 2 dl2] LC4 DL} def
/LT5 {PL [3 dl1 3 dl2 1 dl1 3 dl2] LC5 DL} def
/LT6 {PL [2 dl1 2 dl2 2 dl1 6 dl2] LC6 DL} def
/LT7 {PL [1 dl1 2 dl2 6 dl1 2 dl2 1 dl1 2 dl2] LC7 DL} def
/LT8 {PL [2 dl1 2 dl2 2 dl1 2 dl2 2 dl1 2 dl2 2 dl1 4 dl2] LC8 DL} def
/Pnt {stroke [] 0 setdash gsave 1 setlinecap M 0 0 V stroke grestore} def
/Dia {stroke [] 0 setdash 2 copy vpt add M
  hpt neg vpt neg V hpt vpt neg V
  hpt vpt V hpt neg vpt V closepath stroke
  Pnt} def
/Pls {stroke [] 0 setdash vpt sub M 0 vpt2 V
  currentpoint stroke M
  hpt neg vpt neg R hpt2 0 V stroke
 } def
/Box {stroke [] 0 setdash 2 copy exch hpt sub exch vpt add M
  0 vpt2 neg V hpt2 0 V 0 vpt2 V
  hpt2 neg 0 V closepath stroke
  Pnt} def
/Crs {stroke [] 0 setdash exch hpt sub exch vpt add M
  hpt2 vpt2 neg V currentpoint stroke M
  hpt2 neg 0 R hpt2 vpt2 V stroke} def
/TriU {stroke [] 0 setdash 2 copy vpt 1.12 mul add M
  hpt neg vpt -1.62 mul V
  hpt 2 mul 0 V
  hpt neg vpt 1.62 mul V closepath stroke
  Pnt} def
/Star {2 copy Pls Crs} def
/BoxF {stroke [] 0 setdash exch hpt sub exch vpt add M
  0 vpt2 neg V hpt2 0 V 0 vpt2 V
  hpt2 neg 0 V closepath fill} def
/TriUF {stroke [] 0 setdash vpt 1.12 mul add M
  hpt neg vpt -1.62 mul V
  hpt 2 mul 0 V
  hpt neg vpt 1.62 mul V closepath fill} def
/TriD {stroke [] 0 setdash 2 copy vpt 1.12 mul sub M
  hpt neg vpt 1.62 mul V
  hpt 2 mul 0 V
  hpt neg vpt -1.62 mul V closepath stroke
  Pnt} def
/TriDF {stroke [] 0 setdash vpt 1.12 mul sub M
  hpt neg vpt 1.62 mul V
  hpt 2 mul 0 V
  hpt neg vpt -1.62 mul V closepath fill} def
/DiaF {stroke [] 0 setdash vpt add M
  hpt neg vpt neg V hpt vpt neg V
  hpt vpt V hpt neg vpt V closepath fill} def
/Pent {stroke [] 0 setdash 2 copy gsave
  translate 0 hpt M 4 {72 rotate 0 hpt L} repeat
  closepath stroke grestore Pnt} def
/PentF {stroke [] 0 setdash gsave
  translate 0 hpt M 4 {72 rotate 0 hpt L} repeat
  closepath fill grestore} def
/Circle {stroke [] 0 setdash 2 copy
  hpt 0 360 arc stroke Pnt} def
/CircleF {stroke [] 0 setdash hpt 0 360 arc fill} def
/C0 {BL [] 0 setdash 2 copy moveto vpt 90 450 arc} bind def
/C1 {BL [] 0 setdash 2 copy moveto
	2 copy vpt 0 90 arc closepath fill
	vpt 0 360 arc closepath} bind def
/C2 {BL [] 0 setdash 2 copy moveto
	2 copy vpt 90 180 arc closepath fill
	vpt 0 360 arc closepath} bind def
/C3 {BL [] 0 setdash 2 copy moveto
	2 copy vpt 0 180 arc closepath fill
	vpt 0 360 arc closepath} bind def
/C4 {BL [] 0 setdash 2 copy moveto
	2 copy vpt 180 270 arc closepath fill
	vpt 0 360 arc closepath} bind def
/C5 {BL [] 0 setdash 2 copy moveto
	2 copy vpt 0 90 arc
	2 copy moveto
	2 copy vpt 180 270 arc closepath fill
	vpt 0 360 arc} bind def
/C6 {BL [] 0 setdash 2 copy moveto
	2 copy vpt 90 270 arc closepath fill
	vpt 0 360 arc closepath} bind def
/C7 {BL [] 0 setdash 2 copy moveto
	2 copy vpt 0 270 arc closepath fill
	vpt 0 360 arc closepath} bind def
/C8 {BL [] 0 setdash 2 copy moveto
	2 copy vpt 270 360 arc closepath fill
	vpt 0 360 arc closepath} bind def
/C9 {BL [] 0 setdash 2 copy moveto
	2 copy vpt 270 450 arc closepath fill
	vpt 0 360 arc closepath} bind def
/C10 {BL [] 0 setdash 2 copy 2 copy moveto vpt 270 360 arc closepath fill
	2 copy moveto
	2 copy vpt 90 180 arc closepath fill
	vpt 0 360 arc closepath} bind def
/C11 {BL [] 0 setdash 2 copy moveto
	2 copy vpt 0 180 arc closepath fill
	2 copy moveto
	2 copy vpt 270 360 arc closepath fill
	vpt 0 360 arc closepath} bind def
/C12 {BL [] 0 setdash 2 copy moveto
	2 copy vpt 180 360 arc closepath fill
	vpt 0 360 arc closepath} bind def
/C13 {BL [] 0 setdash 2 copy moveto
	2 copy vpt 0 90 arc closepath fill
	2 copy moveto
	2 copy vpt 180 360 arc closepath fill
	vpt 0 360 arc closepath} bind def
/C14 {BL [] 0 setdash 2 copy moveto
	2 copy vpt 90 360 arc closepath fill
	vpt 0 360 arc} bind def
/C15 {BL [] 0 setdash 2 copy vpt 0 360 arc closepath fill
	vpt 0 360 arc closepath} bind def
/Rec {newpath 4 2 roll moveto 1 index 0 rlineto 0 exch rlineto
	neg 0 rlineto closepath} bind def
/Square {dup Rec} bind def
/Bsquare {vpt sub exch vpt sub exch vpt2 Square} bind def
/S0 {BL [] 0 setdash 2 copy moveto 0 vpt rlineto BL Bsquare} bind def
/S1 {BL [] 0 setdash 2 copy vpt Square fill Bsquare} bind def
/S2 {BL [] 0 setdash 2 copy exch vpt sub exch vpt Square fill Bsquare} bind def
/S3 {BL [] 0 setdash 2 copy exch vpt sub exch vpt2 vpt Rec fill Bsquare} bind def
/S4 {BL [] 0 setdash 2 copy exch vpt sub exch vpt sub vpt Square fill Bsquare} bind def
/S5 {BL [] 0 setdash 2 copy 2 copy vpt Square fill
	exch vpt sub exch vpt sub vpt Square fill Bsquare} bind def
/S6 {BL [] 0 setdash 2 copy exch vpt sub exch vpt sub vpt vpt2 Rec fill Bsquare} bind def
/S7 {BL [] 0 setdash 2 copy exch vpt sub exch vpt sub vpt vpt2 Rec fill
	2 copy vpt Square fill Bsquare} bind def
/S8 {BL [] 0 setdash 2 copy vpt sub vpt Square fill Bsquare} bind def
/S9 {BL [] 0 setdash 2 copy vpt sub vpt vpt2 Rec fill Bsquare} bind def
/S10 {BL [] 0 setdash 2 copy vpt sub vpt Square fill 2 copy exch vpt sub exch vpt Square fill
	Bsquare} bind def
/S11 {BL [] 0 setdash 2 copy vpt sub vpt Square fill 2 copy exch vpt sub exch vpt2 vpt Rec fill
	Bsquare} bind def
/S12 {BL [] 0 setdash 2 copy exch vpt sub exch vpt sub vpt2 vpt Rec fill Bsquare} bind def
/S13 {BL [] 0 setdash 2 copy exch vpt sub exch vpt sub vpt2 vpt Rec fill
	2 copy vpt Square fill Bsquare} bind def
/S14 {BL [] 0 setdash 2 copy exch vpt sub exch vpt sub vpt2 vpt Rec fill
	2 copy exch vpt sub exch vpt Square fill Bsquare} bind def
/S15 {BL [] 0 setdash 2 copy Bsquare fill Bsquare} bind def
/D0 {gsave translate 45 rotate 0 0 S0 stroke grestore} bind def
/D1 {gsave translate 45 rotate 0 0 S1 stroke grestore} bind def
/D2 {gsave translate 45 rotate 0 0 S2 stroke grestore} bind def
/D3 {gsave translate 45 rotate 0 0 S3 stroke grestore} bind def
/D4 {gsave translate 45 rotate 0 0 S4 stroke grestore} bind def
/D5 {gsave translate 45 rotate 0 0 S5 stroke grestore} bind def
/D6 {gsave translate 45 rotate 0 0 S6 stroke grestore} bind def
/D7 {gsave translate 45 rotate 0 0 S7 stroke grestore} bind def
/D8 {gsave translate 45 rotate 0 0 S8 stroke grestore} bind def
/D9 {gsave translate 45 rotate 0 0 S9 stroke grestore} bind def
/D10 {gsave translate 45 rotate 0 0 S10 stroke grestore} bind def
/D11 {gsave translate 45 rotate 0 0 S11 stroke grestore} bind def
/D12 {gsave translate 45 rotate 0 0 S12 stroke grestore} bind def
/D13 {gsave translate 45 rotate 0 0 S13 stroke grestore} bind def
/D14 {gsave translate 45 rotate 0 0 S14 stroke grestore} bind def
/D15 {gsave translate 45 rotate 0 0 S15 stroke grestore} bind def
/DiaE {stroke [] 0 setdash vpt add M
  hpt neg vpt neg V hpt vpt neg V
  hpt vpt V hpt neg vpt V closepath stroke} def
/BoxE {stroke [] 0 setdash exch hpt sub exch vpt add M
  0 vpt2 neg V hpt2 0 V 0 vpt2 V
  hpt2 neg 0 V closepath stroke} def
/TriUE {stroke [] 0 setdash vpt 1.12 mul add M
  hpt neg vpt -1.62 mul V
  hpt 2 mul 0 V
  hpt neg vpt 1.62 mul V closepath stroke} def
/TriDE {stroke [] 0 setdash vpt 1.12 mul sub M
  hpt neg vpt 1.62 mul V
  hpt 2 mul 0 V
  hpt neg vpt -1.62 mul V closepath stroke} def
/PentE {stroke [] 0 setdash gsave
  translate 0 hpt M 4 {72 rotate 0 hpt L} repeat
  closepath stroke grestore} def
/CircE {stroke [] 0 setdash 
  hpt 0 360 arc stroke} def
/Opaque {gsave closepath 1 setgray fill grestore 0 setgray closepath} def
/DiaW {stroke [] 0 setdash vpt add M
  hpt neg vpt neg V hpt vpt neg V
  hpt vpt V hpt neg vpt V Opaque stroke} def
/BoxW {stroke [] 0 setdash exch hpt sub exch vpt add M
  0 vpt2 neg V hpt2 0 V 0 vpt2 V
  hpt2 neg 0 V Opaque stroke} def
/TriUW {stroke [] 0 setdash vpt 1.12 mul add M
  hpt neg vpt -1.62 mul V
  hpt 2 mul 0 V
  hpt neg vpt 1.62 mul V Opaque stroke} def
/TriDW {stroke [] 0 setdash vpt 1.12 mul sub M
  hpt neg vpt 1.62 mul V
  hpt 2 mul 0 V
  hpt neg vpt -1.62 mul V Opaque stroke} def
/PentW {stroke [] 0 setdash gsave
  translate 0 hpt M 4 {72 rotate 0 hpt L} repeat
  Opaque stroke grestore} def
/CircW {stroke [] 0 setdash 
  hpt 0 360 arc Opaque stroke} def
/BoxFill {gsave Rec 1 setgray fill grestore} def
/Density {
  /Fillden exch def
  currentrgbcolor
  /ColB exch def /ColG exch def /ColR exch def
  /ColR ColR Fillden mul Fillden sub 1 add def
  /ColG ColG Fillden mul Fillden sub 1 add def
  /ColB ColB Fillden mul Fillden sub 1 add def
  ColR ColG ColB setrgbcolor} def
/BoxColFill {gsave Rec PolyFill} def
/PolyFill {gsave Density fill grestore grestore} def
/h {rlineto rlineto rlineto gsave closepath fill grestore} bind def
%
% PostScript Level 1 Pattern Fill routine for rectangles
% Usage: x y w h s a XX PatternFill
%	x,y = lower left corner of box to be filled
%	w,h = width and height of box
%	  a = angle in degrees between lines and x-axis
%	 XX = 0/1 for no/yes cross-hatch
%
/PatternFill {gsave /PFa [ 9 2 roll ] def
  PFa 0 get PFa 2 get 2 div add PFa 1 get PFa 3 get 2 div add translate
  PFa 2 get -2 div PFa 3 get -2 div PFa 2 get PFa 3 get Rec
  gsave 1 setgray fill grestore clip
  currentlinewidth 0.5 mul setlinewidth
  /PFs PFa 2 get dup mul PFa 3 get dup mul add sqrt def
  0 0 M PFa 5 get rotate PFs -2 div dup translate
  0 1 PFs PFa 4 get div 1 add floor cvi
	{PFa 4 get mul 0 M 0 PFs V} for
  0 PFa 6 get ne {
	0 1 PFs PFa 4 get div 1 add floor cvi
	{PFa 4 get mul 0 2 1 roll M PFs 0 V} for
 } if
  stroke grestore} def
%
/languagelevel where
 {pop languagelevel} {1} ifelse
 2 lt
	{/InterpretLevel1 true def}
	{/InterpretLevel1 Level1 def}
 ifelse
%
% PostScript level 2 pattern fill definitions
%
/Level2PatternFill {
/Tile8x8 {/PaintType 2 /PatternType 1 /TilingType 1 /BBox [0 0 8 8] /XStep 8 /YStep 8}
	bind def
/KeepColor {currentrgbcolor [/Pattern /DeviceRGB] setcolorspace} bind def
<< Tile8x8
 /PaintProc {0.5 setlinewidth pop 0 0 M 8 8 L 0 8 M 8 0 L stroke} 
>> matrix makepattern
/Pat1 exch def
<< Tile8x8
 /PaintProc {0.5 setlinewidth pop 0 0 M 8 8 L 0 8 M 8 0 L stroke
	0 4 M 4 8 L 8 4 L 4 0 L 0 4 L stroke}
>> matrix makepattern
/Pat2 exch def
<< Tile8x8
 /PaintProc {0.5 setlinewidth pop 0 0 M 0 8 L
	8 8 L 8 0 L 0 0 L fill}
>> matrix makepattern
/Pat3 exch def
<< Tile8x8
 /PaintProc {0.5 setlinewidth pop -4 8 M 8 -4 L
	0 12 M 12 0 L stroke}
>> matrix makepattern
/Pat4 exch def
<< Tile8x8
 /PaintProc {0.5 setlinewidth pop -4 0 M 8 12 L
	0 -4 M 12 8 L stroke}
>> matrix makepattern
/Pat5 exch def
<< Tile8x8
 /PaintProc {0.5 setlinewidth pop -2 8 M 4 -4 L
	0 12 M 8 -4 L 4 12 M 10 0 L stroke}
>> matrix makepattern
/Pat6 exch def
<< Tile8x8
 /PaintProc {0.5 setlinewidth pop -2 0 M 4 12 L
	0 -4 M 8 12 L 4 -4 M 10 8 L stroke}
>> matrix makepattern
/Pat7 exch def
<< Tile8x8
 /PaintProc {0.5 setlinewidth pop 8 -2 M -4 4 L
	12 0 M -4 8 L 12 4 M 0 10 L stroke}
>> matrix makepattern
/Pat8 exch def
<< Tile8x8
 /PaintProc {0.5 setlinewidth pop 0 -2 M 12 4 L
	-4 0 M 12 8 L -4 4 M 8 10 L stroke}
>> matrix makepattern
/Pat9 exch def
/Pattern1 {PatternBgnd KeepColor Pat1 setpattern} bind def
/Pattern2 {PatternBgnd KeepColor Pat2 setpattern} bind def
/Pattern3 {PatternBgnd KeepColor Pat3 setpattern} bind def
/Pattern4 {PatternBgnd KeepColor Landscape {Pat5} {Pat4} ifelse setpattern} bind def
/Pattern5 {PatternBgnd KeepColor Landscape {Pat4} {Pat5} ifelse setpattern} bind def
/Pattern6 {PatternBgnd KeepColor Landscape {Pat9} {Pat6} ifelse setpattern} bind def
/Pattern7 {PatternBgnd KeepColor Landscape {Pat8} {Pat7} ifelse setpattern} bind def
} def
%
%
%End of PostScript Level 2 code
%
/PatternBgnd {
  TransparentPatterns {} {gsave 1 setgray fill grestore} ifelse
} def
%
% Substitute for Level 2 pattern fill codes with
% grayscale if Level 2 support is not selected.
%
/Level1PatternFill {
/Pattern1 {0.250 Density} bind def
/Pattern2 {0.500 Density} bind def
/Pattern3 {0.750 Density} bind def
/Pattern4 {0.125 Density} bind def
/Pattern5 {0.375 Density} bind def
/Pattern6 {0.625 Density} bind def
/Pattern7 {0.875 Density} bind def
} def
%
% Now test for support of Level 2 code
%
Level1 {Level1PatternFill} {Level2PatternFill} ifelse
%
/Symbol-Oblique /Symbol findfont [1 0 .167 1 0 0] makefont
dup length dict begin {1 index /FID eq {pop pop} {def} ifelse} forall
currentdict end definefont pop
end
%%EndProlog
gnudict begin
gsave
doclip
0 0 translate
0.050 0.050 scale
0 setgray
newpath
1.000 UL
LTb
1220 640 M
63 0 V
11276 0 R
-63 0 V
1220 1610 M
63 0 V
11276 0 R
-63 0 V
1220 2580 M
63 0 V
11276 0 R
-63 0 V
1220 3550 M
63 0 V
11276 0 R
-63 0 V
1220 4520 M
63 0 V
11276 0 R
-63 0 V
1220 5489 M
63 0 V
11276 0 R
-63 0 V
1220 6459 M
63 0 V
11276 0 R
-63 0 V
1220 7429 M
63 0 V
11276 0 R
-63 0 V
1220 8399 M
63 0 V
11276 0 R
-63 0 V
1220 640 M
0 63 V
0 7696 R
0 -63 V
1641 640 M
0 31 V
0 7728 R
0 -31 V
2197 640 M
0 31 V
0 7728 R
0 -31 V
2482 640 M
0 31 V
0 7728 R
0 -31 V
2617 640 M
0 63 V
0 7696 R
0 -63 V
3038 640 M
0 31 V
0 7728 R
0 -31 V
3594 640 M
0 31 V
0 7728 R
0 -31 V
3880 640 M
0 31 V
0 7728 R
0 -31 V
4015 640 M
0 63 V
0 7696 R
0 -63 V
4436 640 M
0 31 V
0 7728 R
0 -31 V
4992 640 M
0 31 V
0 7728 R
0 -31 V
5277 640 M
0 31 V
0 7728 R
0 -31 V
5412 640 M
0 63 V
0 7696 R
0 -63 V
5833 640 M
0 31 V
0 7728 R
0 -31 V
6389 640 M
0 31 V
0 7728 R
0 -31 V
6674 640 M
0 31 V
0 7728 R
0 -31 V
6810 640 M
0 63 V
0 7696 R
0 -63 V
7231 640 M
0 31 V
stroke 7231 671 M
0 7728 R
0 -31 V
7787 640 M
0 31 V
0 7728 R
0 -31 V
8072 640 M
0 31 V
0 7728 R
0 -31 V
8207 640 M
0 63 V
0 7696 R
0 -63 V
8628 640 M
0 31 V
0 7728 R
0 -31 V
9184 640 M
0 31 V
0 7728 R
0 -31 V
9469 640 M
0 31 V
0 7728 R
0 -31 V
9605 640 M
0 63 V
0 7696 R
0 -63 V
10026 640 M
0 31 V
0 7728 R
0 -31 V
10582 640 M
0 31 V
0 7728 R
0 -31 V
10867 640 M
0 31 V
0 7728 R
0 -31 V
11002 640 M
0 63 V
0 7696 R
0 -63 V
11423 640 M
0 31 V
0 7728 R
0 -31 V
11979 640 M
0 31 V
0 7728 R
0 -31 V
12264 640 M
0 31 V
0 7728 R
0 -31 V
12400 640 M
0 63 V
0 7696 R
0 -63 V
stroke
1220 8399 N
0 -7759 V
11339 0 V
0 7759 V
-11339 0 V
Z stroke
LCb setrgbcolor
LTb
LCb setrgbcolor
LTb
LCb setrgbcolor
LTb
LCb setrgbcolor
LTb
1.000 UP
1.000 UL
LTb
1.000 UL
LTb
1340 7656 N
0 680 V
1863 0 V
0 -680 V
-1863 0 V
Z stroke
1340 8336 M
1863 0 V
1.000 UP
stroke
LT0
LCb setrgbcolor
LT0
2540 8116 M
543 0 V
-543 31 R
0 -62 V
543 62 R
0 -62 V
1220 3298 M
556 -2 V
421 1 V
246 -4 V
174 3 V
557 13 V
420 -9 V
246 -5 V
731 5 V
667 -6 V
174 10 V
557 3 V
420 -19 V
246 -8 V
175 12 V
556 0 V
421 4 V
246 -10 V
174 7 V
556 -2 V
421 45 V
246 15 V
175 32 V
977 438 V
420 543 V
977 1911 V
421 1736 V
1220 3289 M
0 19 V
-31 -19 R
62 0 V
-62 19 R
62 0 V
525 -24 R
0 25 V
-31 -25 R
62 0 V
-62 25 R
62 0 V
390 -23 R
0 22 V
-31 -22 R
62 0 V
-62 22 R
62 0 V
215 -25 R
0 20 V
-31 -20 R
62 0 V
-62 20 R
62 0 V
143 -17 R
0 19 V
-31 -19 R
62 0 V
-62 19 R
62 0 V
526 -6 R
0 20 V
-31 -20 R
62 0 V
-62 20 R
62 0 V
389 -29 R
0 21 V
-31 -21 R
62 0 V
-62 21 R
62 0 V
215 -26 R
0 19 V
-31 -19 R
62 0 V
-62 19 R
62 0 V
700 -14 R
0 20 V
-31 -20 R
62 0 V
-62 20 R
62 0 V
636 -26 R
0 19 V
-31 -19 R
62 0 V
-62 19 R
62 0 V
143 -10 R
0 22 V
-31 -22 R
62 0 V
-62 22 R
62 0 V
526 -18 R
0 21 V
-31 -21 R
62 0 V
-62 21 R
62 0 V
stroke 6000 3318 M
389 -38 R
0 16 V
-31 -16 R
62 0 V
-62 16 R
62 0 V
215 -26 R
0 20 V
-31 -20 R
62 0 V
-62 20 R
62 0 V
144 -9 R
0 21 V
-31 -21 R
62 0 V
-62 21 R
62 0 V
525 -18 R
0 16 V
-31 -16 R
62 0 V
-62 16 R
62 0 V
390 -14 R
0 19 V
-31 -19 R
62 0 V
-62 19 R
62 0 V
215 -28 R
0 17 V
-31 -17 R
62 0 V
-62 17 R
62 0 V
143 -9 R
0 15 V
-31 -15 R
62 0 V
-62 15 R
62 0 V
525 -17 R
0 15 V
-31 -15 R
62 0 V
-62 15 R
62 0 V
390 5 R
0 67 V
-31 -67 R
62 0 V
-62 67 R
62 0 V
215 -41 R
0 44 V
-31 -44 R
62 0 V
-62 44 R
62 0 V
144 -25 R
0 70 V
-31 -70 R
62 0 V
-62 70 R
62 0 V
946 369 R
0 68 V
-31 -68 R
62 0 V
-62 68 R
62 0 V
389 472 R
0 75 V
-31 -75 R
62 0 V
-62 75 R
62 0 V
946 1832 R
0 82 V
-31 -82 R
62 0 V
-62 82 R
62 0 V
390 1654 R
0 82 V
-31 -82 R
62 0 V
-62 82 R
62 0 V
1220 3298 Pls
1776 3296 Pls
2197 3297 Pls
2443 3293 Pls
2617 3296 Pls
3174 3309 Pls
3594 3300 Pls
3840 3295 Pls
4571 3300 Pls
5238 3294 Pls
5412 3304 Pls
5969 3307 Pls
6389 3288 Pls
6635 3280 Pls
6810 3292 Pls
7366 3292 Pls
7787 3296 Pls
8033 3286 Pls
8207 3293 Pls
8763 3291 Pls
9184 3336 Pls
9430 3351 Pls
9605 3383 Pls
10582 3821 Pls
11002 4364 Pls
11979 6275 Pls
12400 8011 Pls
2811 8116 Pls
1.000 UP
1.000 UL
LT0
LC1 setrgbcolor
1220 2562 M
556 -7 V
421 1 V
246 -15 V
174 15 V
557 -4 V
420 3 V
246 -12 V
731 -8 V
667 -9 V
174 28 V
557 6 V
420 -9 V
246 19 V
175 -25 V
556 15 V
421 -19 V
246 13 V
174 -12 V
556 9 V
421 8 V
246 5 V
175 31 V
977 186 V
420 207 V
977 1221 V
421 1135 V
1220 2552 M
0 20 V
-31 -20 R
62 0 V
-62 20 R
62 0 V
525 -28 R
0 21 V
-31 -21 R
62 0 V
-62 21 R
62 0 V
390 -18 R
0 17 V
-31 -17 R
62 0 V
-62 17 R
62 0 V
215 -33 R
0 20 V
-31 -20 R
62 0 V
-62 20 R
62 0 V
143 -5 R
0 19 V
-31 -19 R
62 0 V
-62 19 R
62 0 V
526 -24 R
0 21 V
-31 -21 R
62 0 V
-62 21 R
62 0 V
389 -18 R
0 21 V
-31 -21 R
62 0 V
-62 21 R
62 0 V
215 -33 R
0 22 V
-31 -22 R
62 0 V
-62 22 R
62 0 V
700 -30 R
0 22 V
-31 -22 R
62 0 V
-62 22 R
62 0 V
636 -29 R
0 19 V
-31 -19 R
62 0 V
-62 19 R
62 0 V
143 6 R
0 23 V
-31 -23 R
62 0 V
-62 23 R
62 0 V
526 -15 R
0 21 V
-31 -21 R
62 0 V
-62 21 R
62 0 V
389 -30 R
0 19 V
-31 -19 R
62 0 V
-62 19 R
62 0 V
stroke 6420 2560 M
215 0 R
0 21 V
-31 -21 R
62 0 V
-62 21 R
62 0 V
144 -46 R
0 20 V
-31 -20 R
62 0 V
-62 20 R
62 0 V
525 -5 R
0 20 V
-31 -20 R
62 0 V
-62 20 R
62 0 V
390 -39 R
0 20 V
-31 -20 R
62 0 V
-62 20 R
62 0 V
215 -6 R
0 17 V
-31 -17 R
62 0 V
-62 17 R
62 0 V
143 -28 R
0 16 V
-31 -16 R
62 0 V
-62 16 R
62 0 V
525 -8 R
0 18 V
-31 -18 R
62 0 V
-62 18 R
62 0 V
390 -11 R
0 21 V
-31 -21 R
62 0 V
-62 21 R
62 0 V
215 -22 R
0 32 V
-31 -32 R
62 0 V
-62 32 R
62 0 V
144 2 R
0 27 V
-31 -27 R
62 0 V
-62 27 R
62 0 V
946 163 R
0 17 V
-31 -17 R
62 0 V
-62 17 R
62 0 V
389 192 R
0 14 V
-31 -14 R
62 0 V
-62 14 R
62 0 V
946 1203 R
0 22 V
-31 -22 R
62 0 V
-62 22 R
62 0 V
390 1106 R
0 36 V
-31 -36 R
62 0 V
-62 36 R
62 0 V
1220 2562 Crs
1776 2555 Crs
2197 2556 Crs
2443 2541 Crs
2617 2556 Crs
3174 2552 Crs
3594 2555 Crs
3840 2543 Crs
4571 2535 Crs
5238 2526 Crs
5412 2554 Crs
5969 2560 Crs
6389 2551 Crs
6635 2570 Crs
6810 2545 Crs
7366 2560 Crs
7787 2541 Crs
8033 2554 Crs
8207 2542 Crs
8763 2551 Crs
9184 2559 Crs
9430 2564 Crs
9605 2595 Crs
10582 2781 Crs
11002 2988 Crs
11979 4209 Crs
12400 5344 Crs
1.000 UP
1.000 UL
LT0
LC2 setrgbcolor
1220 1340 M
556 -2 V
421 -6 V
246 7 V
174 -7 V
557 7 V
420 -14 V
246 21 V
731 -13 V
667 -3 V
174 18 V
557 -39 V
420 33 V
246 -8 V
175 10 V
556 -1 V
421 -8 V
246 8 V
174 -11 V
556 13 V
421 16 V
246 46 V
175 9 V
977 66 V
420 98 V
977 817 V
421 733 V
1220 1332 M
0 17 V
-31 -17 R
62 0 V
-62 17 R
62 0 V
525 -20 R
0 18 V
-31 -18 R
62 0 V
-62 18 R
62 0 V
390 -24 R
0 19 V
-31 -19 R
62 0 V
-62 19 R
62 0 V
215 -13 R
0 20 V
-31 -20 R
62 0 V
-62 20 R
62 0 V
143 -27 R
0 19 V
-31 -19 R
62 0 V
-62 19 R
62 0 V
526 -12 R
0 20 V
-31 -20 R
62 0 V
-62 20 R
62 0 V
389 -35 R
0 22 V
-31 -22 R
62 0 V
-62 22 R
62 0 V
215 1 R
0 19 V
-31 -19 R
62 0 V
-62 19 R
62 0 V
700 -34 R
0 23 V
-31 -23 R
62 0 V
-62 23 R
62 0 V
636 -24 R
0 17 V
-31 -17 R
62 0 V
-62 17 R
62 0 V
143 0 R
0 20 V
-31 -20 R
62 0 V
-62 20 R
62 0 V
526 -58 R
0 19 V
-31 -19 R
62 0 V
-62 19 R
62 0 V
389 13 R
0 20 V
-31 -20 R
62 0 V
-62 20 R
62 0 V
stroke 6420 1352 M
215 -27 R
0 19 V
-31 -19 R
62 0 V
-62 19 R
62 0 V
144 -8 R
0 16 V
-31 -16 R
62 0 V
-62 16 R
62 0 V
525 -19 R
0 20 V
-31 -20 R
62 0 V
-62 20 R
62 0 V
390 -30 R
0 24 V
-31 -24 R
62 0 V
-62 24 R
62 0 V
215 -12 R
0 16 V
-31 -16 R
62 0 V
-62 16 R
62 0 V
143 -26 R
0 15 V
-31 -15 R
62 0 V
-62 15 R
62 0 V
525 -5 R
0 20 V
-31 -20 R
62 0 V
-62 20 R
62 0 V
390 -11 R
0 33 V
-31 -33 R
62 0 V
-62 33 R
62 0 V
215 11 R
0 38 V
-31 -38 R
62 0 V
-62 38 R
62 0 V
144 -30 R
0 40 V
-31 -40 R
62 0 V
-62 40 R
62 0 V
946 36 R
0 20 V
-31 -20 R
62 0 V
-62 20 R
62 0 V
389 75 R
0 26 V
-31 -26 R
62 0 V
-62 26 R
62 0 V
946 790 R
0 29 V
-31 -29 R
62 0 V
-62 29 R
62 0 V
390 704 R
0 28 V
-31 -28 R
62 0 V
-62 28 R
62 0 V
1220 1340 Star
1776 1338 Star
2197 1332 Star
2443 1339 Star
2617 1332 Star
3174 1339 Star
3594 1325 Star
3840 1346 Star
4571 1333 Star
5238 1330 Star
5412 1348 Star
5969 1309 Star
6389 1342 Star
6635 1334 Star
6810 1344 Star
7366 1343 Star
7787 1335 Star
8033 1343 Star
8207 1332 Star
8763 1345 Star
9184 1361 Star
9430 1407 Star
9605 1416 Star
10582 1482 Star
11002 1580 Star
11979 2397 Star
12400 3130 Star
1.000 UL
LT0
LCb setrgbcolor
1220 3298 M
115 0 V
114 0 V
115 0 V
114 0 V
115 0 V
114 0 V
115 0 V
114 0 V
115 0 V
114 0 V
115 0 V
114 0 V
115 0 V
114 0 V
115 0 V
115 0 V
114 0 V
115 0 V
114 0 V
115 0 V
114 0 V
115 0 V
114 0 V
115 0 V
114 0 V
115 0 V
114 0 V
115 0 V
115 0 V
114 0 V
115 0 V
114 0 V
115 0 V
114 0 V
115 0 V
114 0 V
115 0 V
114 0 V
115 0 V
114 0 V
115 0 V
114 0 V
115 0 V
115 0 V
114 0 V
115 0 V
114 0 V
115 0 V
114 0 V
115 0 V
114 0 V
115 0 V
114 0 V
115 0 V
114 0 V
115 0 V
115 0 V
114 0 V
115 0 V
114 0 V
115 0 V
114 0 V
115 0 V
114 0 V
115 0 V
114 0 V
115 0 V
114 0 V
115 0 V
114 0 V
115 0 V
115 0 V
114 0 V
115 0 V
114 0 V
115 0 V
114 0 V
115 0 V
114 0 V
115 0 V
114 0 V
115 0 V
114 0 V
115 0 V
115 0 V
114 0 V
115 0 V
114 0 V
115 0 V
114 0 V
115 0 V
114 0 V
115 0 V
114 0 V
115 0 V
114 0 V
115 0 V
114 0 V
115 0 V
stroke
LT0
LCb setrgbcolor
1220 2562 M
115 0 V
114 0 V
115 0 V
114 0 V
115 0 V
114 0 V
115 0 V
114 0 V
115 0 V
114 0 V
115 0 V
114 0 V
115 0 V
114 0 V
115 0 V
115 0 V
114 0 V
115 0 V
114 0 V
115 0 V
114 0 V
115 0 V
114 0 V
115 0 V
114 0 V
115 0 V
114 0 V
115 0 V
115 0 V
114 0 V
115 0 V
114 0 V
115 0 V
114 0 V
115 0 V
114 0 V
115 0 V
114 0 V
115 0 V
114 0 V
115 0 V
114 0 V
115 0 V
115 0 V
114 0 V
115 0 V
114 0 V
115 0 V
114 0 V
115 0 V
114 0 V
115 0 V
114 0 V
115 0 V
114 0 V
115 0 V
115 0 V
114 0 V
115 0 V
114 0 V
115 0 V
114 0 V
115 0 V
114 0 V
115 0 V
114 0 V
115 0 V
114 0 V
115 0 V
114 0 V
115 0 V
115 0 V
114 0 V
115 0 V
114 0 V
115 0 V
114 0 V
115 0 V
114 0 V
115 0 V
114 0 V
115 0 V
114 0 V
115 0 V
115 0 V
114 0 V
115 0 V
114 0 V
115 0 V
114 0 V
115 0 V
114 0 V
115 0 V
114 0 V
115 0 V
114 0 V
115 0 V
114 0 V
115 0 V
stroke
LT0
LCb setrgbcolor
1220 1340 M
115 0 V
114 0 V
115 0 V
114 0 V
115 0 V
114 0 V
115 0 V
114 0 V
115 0 V
114 0 V
115 0 V
114 0 V
115 0 V
114 0 V
115 0 V
115 0 V
114 0 V
115 0 V
114 0 V
115 0 V
114 0 V
115 0 V
114 0 V
115 0 V
114 0 V
115 0 V
114 0 V
115 0 V
115 0 V
114 0 V
115 0 V
114 0 V
115 0 V
114 0 V
115 0 V
114 0 V
115 0 V
114 0 V
115 0 V
114 0 V
115 0 V
114 0 V
115 0 V
115 0 V
114 0 V
115 0 V
114 0 V
115 0 V
114 0 V
115 0 V
114 0 V
115 0 V
114 0 V
115 0 V
114 0 V
115 0 V
115 0 V
114 0 V
115 0 V
114 0 V
115 0 V
114 0 V
115 0 V
114 0 V
115 0 V
114 0 V
115 0 V
114 0 V
115 0 V
114 0 V
115 0 V
115 0 V
114 0 V
115 0 V
114 0 V
115 0 V
114 0 V
115 0 V
114 0 V
115 0 V
114 0 V
115 0 V
114 0 V
115 0 V
115 0 V
114 0 V
115 0 V
114 0 V
115 0 V
114 0 V
115 0 V
114 0 V
115 0 V
114 0 V
115 0 V
114 0 V
115 0 V
114 0 V
115 0 V
1.000 UP
stroke
LT1
LC0 setrgbcolor
LCb setrgbcolor
LT1
LC0 setrgbcolor
2540 7876 M
543 0 V
-543 31 R
0 -62 V
543 62 R
0 -62 V
1220 3295 M
556 -8 V
421 8 V
246 7 V
174 -1 V
557 3 V
420 -20 V
246 2 V
731 12 V
667 -3 V
174 -6 V
557 10 V
420 -21 V
246 2 V
175 25 V
556 11 V
421 -22 V
246 2 V
174 -5 V
556 6 V
421 2 V
246 -3 V
175 9 V
977 6 V
420 38 V
556 126 V
421 113 V
246 117 V
175 116 V
1220 3284 M
0 21 V
-31 -21 R
62 0 V
-62 21 R
62 0 V
525 -29 R
0 21 V
-31 -21 R
62 0 V
-62 21 R
62 0 V
390 -14 R
0 23 V
-31 -23 R
62 0 V
-62 23 R
62 0 V
215 -13 R
0 19 V
-31 -19 R
62 0 V
-62 19 R
62 0 V
143 -21 R
0 21 V
-31 -21 R
62 0 V
-62 21 R
62 0 V
526 -18 R
0 21 V
-31 -21 R
62 0 V
-62 21 R
62 0 V
389 -42 R
0 22 V
-31 -22 R
62 0 V
-62 22 R
62 0 V
215 -19 R
0 20 V
-31 -20 R
62 0 V
-62 20 R
62 0 V
700 -9 R
0 23 V
-31 -23 R
62 0 V
-62 23 R
62 0 V
636 -27 R
0 23 V
-31 -23 R
62 0 V
-62 23 R
62 0 V
143 -30 R
0 25 V
-31 -25 R
62 0 V
-62 25 R
62 0 V
526 -11 R
0 19 V
-31 -19 R
62 0 V
stroke 6000 3290 M
-62 19 R
62 0 V
389 -42 R
0 23 V
-31 -23 R
62 0 V
-62 23 R
62 0 V
215 -21 R
0 22 V
-31 -22 R
62 0 V
-62 22 R
62 0 V
144 3 R
0 23 V
-31 -23 R
62 0 V
-62 23 R
62 0 V
525 -10 R
0 18 V
-31 -18 R
62 0 V
-62 18 R
62 0 V
390 -40 R
0 18 V
-31 -18 R
62 0 V
-62 18 R
62 0 V
215 -16 R
0 17 V
-31 -17 R
62 0 V
-62 17 R
62 0 V
143 -22 R
0 17 V
-31 -17 R
62 0 V
-62 17 R
62 0 V
525 -9 R
0 15 V
-31 -15 R
62 0 V
-62 15 R
62 0 V
390 -13 R
0 15 V
-31 -15 R
62 0 V
-62 15 R
62 0 V
215 -19 R
0 15 V
-31 -15 R
62 0 V
-62 15 R
62 0 V
144 -5 R
0 15 V
-31 -15 R
62 0 V
-62 15 R
62 0 V
946 -11 R
0 18 V
-31 -18 R
62 0 V
-62 18 R
62 0 V
389 18 R
0 22 V
-31 -22 R
62 0 V
-62 22 R
62 0 V
525 100 R
0 30 V
-31 -30 R
62 0 V
-62 30 R
62 0 V
390 82 R
0 33 V
-31 -33 R
62 0 V
-62 33 R
62 0 V
215 80 R
0 39 V
-31 -39 R
62 0 V
-62 39 R
62 0 V
144 76 R
0 42 V
-31 -42 R
62 0 V
-62 42 R
62 0 V
stroke 12431 3842 M
1220 3295 Crs
1776 3287 Crs
2197 3295 Crs
2443 3302 Crs
2617 3301 Crs
3174 3304 Crs
3594 3284 Crs
3840 3286 Crs
4571 3298 Crs
5238 3295 Crs
5412 3289 Crs
5969 3299 Crs
6389 3278 Crs
6635 3280 Crs
6810 3305 Crs
7366 3316 Crs
7787 3294 Crs
8033 3296 Crs
8207 3291 Crs
8763 3297 Crs
9184 3299 Crs
9430 3296 Crs
9605 3305 Crs
10582 3311 Crs
11002 3349 Crs
11558 3475 Crs
11979 3588 Crs
12225 3705 Crs
12400 3821 Crs
2811 7876 Crs
1.000 UP
1.000 UL
LT1
LC1 setrgbcolor
1220 2573 M
556 -8 V
421 -16 V
246 7 V
174 -1 V
557 -2 V
420 0 V
246 4 V
731 -19 V
667 18 V
174 12 V
557 4 V
420 -3 V
246 -12 V
175 -16 V
556 -3 V
421 24 V
246 -30 V
174 13 V
556 -12 V
421 13 V
246 3 V
175 -6 V
977 -1 V
420 -19 V
556 59 V
421 46 V
246 53 V
175 53 V
1220 2563 M
0 19 V
-31 -19 R
62 0 V
-62 19 R
62 0 V
525 -28 R
0 22 V
-31 -22 R
62 0 V
-62 22 R
62 0 V
390 -38 R
0 21 V
-31 -21 R
62 0 V
-62 21 R
62 0 V
215 -12 R
0 19 V
-31 -19 R
62 0 V
-62 19 R
62 0 V
143 -21 R
0 19 V
-31 -19 R
62 0 V
-62 19 R
62 0 V
526 -22 R
0 22 V
-31 -22 R
62 0 V
-62 22 R
62 0 V
389 -21 R
0 19 V
-31 -19 R
62 0 V
-62 19 R
62 0 V
215 -15 R
0 21 V
-31 -21 R
62 0 V
-62 21 R
62 0 V
700 -40 R
0 20 V
-31 -20 R
62 0 V
-62 20 R
62 0 V
636 -2 R
0 19 V
-31 -19 R
62 0 V
-62 19 R
62 0 V
143 -7 R
0 20 V
-31 -20 R
62 0 V
-62 20 R
62 0 V
526 -15 R
0 18 V
-31 -18 R
62 0 V
-62 18 R
62 0 V
389 -22 R
0 20 V
-31 -20 R
62 0 V
stroke 6420 2559 M
-62 20 R
62 0 V
215 -31 R
0 19 V
-31 -19 R
62 0 V
-62 19 R
62 0 V
144 -37 R
0 22 V
-31 -22 R
62 0 V
-62 22 R
62 0 V
525 -24 R
0 20 V
-31 -20 R
62 0 V
-62 20 R
62 0 V
390 5 R
0 18 V
-31 -18 R
62 0 V
-62 18 R
62 0 V
215 -48 R
0 18 V
-31 -18 R
62 0 V
-62 18 R
62 0 V
143 -5 R
0 18 V
-31 -18 R
62 0 V
-62 18 R
62 0 V
525 -31 R
0 21 V
-31 -21 R
62 0 V
-62 21 R
62 0 V
390 -6 R
0 16 V
-31 -16 R
62 0 V
-62 16 R
62 0 V
215 -14 R
0 17 V
-31 -17 R
62 0 V
-62 17 R
62 0 V
144 -22 R
0 16 V
-31 -16 R
62 0 V
-62 16 R
62 0 V
946 -21 R
0 24 V
-31 -24 R
62 0 V
-62 24 R
62 0 V
389 -44 R
0 25 V
-31 -25 R
62 0 V
-62 25 R
62 0 V
525 35 R
0 23 V
-31 -23 R
62 0 V
-62 23 R
62 0 V
390 24 R
0 22 V
-31 -22 R
62 0 V
-62 22 R
62 0 V
215 30 R
0 23 V
-31 -23 R
62 0 V
-62 23 R
62 0 V
144 32 R
0 20 V
-31 -20 R
62 0 V
-62 20 R
62 0 V
1220 2573 Crs
1776 2565 Crs
2197 2549 Crs
2443 2556 Crs
2617 2555 Crs
3174 2553 Crs
3594 2553 Crs
3840 2557 Crs
4571 2538 Crs
5238 2556 Crs
5412 2568 Crs
5969 2572 Crs
6389 2569 Crs
6635 2557 Crs
6810 2541 Crs
7366 2538 Crs
7787 2562 Crs
8033 2532 Crs
8207 2545 Crs
8763 2533 Crs
9184 2546 Crs
9430 2549 Crs
9605 2543 Crs
10582 2542 Crs
11002 2523 Crs
11558 2582 Crs
11979 2628 Crs
12225 2681 Crs
12400 2734 Crs
1.000 UP
1.000 UL
LT1
LC2 setrgbcolor
1220 1328 M
556 -12 V
421 25 V
246 5 V
174 -21 V
557 14 V
420 -6 V
246 7 V
731 -3 V
667 -1 V
174 -1 V
557 -3 V
420 2 V
246 -2 V
175 -11 V
556 17 V
421 -6 V
246 17 V
174 -10 V
556 1 V
421 -15 V
246 18 V
175 -9 V
977 -4 V
420 -27 V
556 -21 V
421 -2 V
246 2 V
175 5 V
1220 1319 M
0 17 V
-31 -17 R
62 0 V
-62 17 R
62 0 V
525 -30 R
0 20 V
-31 -20 R
62 0 V
-62 20 R
62 0 V
390 5 R
0 20 V
-31 -20 R
62 0 V
-62 20 R
62 0 V
215 -16 R
0 22 V
-31 -22 R
62 0 V
-62 22 R
62 0 V
143 -42 R
0 19 V
-31 -19 R
62 0 V
-62 19 R
62 0 V
526 -5 R
0 21 V
-31 -21 R
62 0 V
-62 21 R
62 0 V
389 -28 R
0 22 V
-31 -22 R
62 0 V
-62 22 R
62 0 V
215 -15 R
0 23 V
-31 -23 R
62 0 V
-62 23 R
62 0 V
700 -25 R
0 21 V
-31 -21 R
62 0 V
-62 21 R
62 0 V
636 -22 R
0 20 V
-31 -20 R
62 0 V
-62 20 R
62 0 V
143 -21 R
0 21 V
-31 -21 R
62 0 V
-62 21 R
62 0 V
526 -23 R
0 19 V
-31 -19 R
62 0 V
-62 19 R
62 0 V
389 -17 R
0 19 V
-31 -19 R
62 0 V
stroke 6420 1325 M
-62 19 R
62 0 V
215 -22 R
0 21 V
-31 -21 R
62 0 V
-62 21 R
62 0 V
144 -32 R
0 20 V
-31 -20 R
62 0 V
-62 20 R
62 0 V
525 -3 R
0 20 V
-31 -20 R
62 0 V
-62 20 R
62 0 V
390 -26 R
0 19 V
-31 -19 R
62 0 V
-62 19 R
62 0 V
215 0 R
0 16 V
-31 -16 R
62 0 V
-62 16 R
62 0 V
143 -26 R
0 16 V
-31 -16 R
62 0 V
-62 16 R
62 0 V
525 -14 R
0 15 V
-31 -15 R
62 0 V
-62 15 R
62 0 V
390 -30 R
0 14 V
-31 -14 R
62 0 V
-62 14 R
62 0 V
215 4 R
0 14 V
-31 -14 R
62 0 V
-62 14 R
62 0 V
144 -22 R
0 13 V
-31 -13 R
62 0 V
-62 13 R
62 0 V
946 -21 R
0 20 V
-31 -20 R
62 0 V
-62 20 R
62 0 V
389 -49 R
0 25 V
-31 -25 R
62 0 V
-62 25 R
62 0 V
525 -48 R
0 28 V
-31 -28 R
62 0 V
-62 28 R
62 0 V
390 -30 R
0 28 V
-31 -28 R
62 0 V
-62 28 R
62 0 V
215 -27 R
0 30 V
-31 -30 R
62 0 V
-62 30 R
62 0 V
144 -25 R
0 30 V
-31 -30 R
62 0 V
-62 30 R
62 0 V
1220 1328 Crs
1776 1316 Crs
2197 1341 Crs
2443 1346 Crs
2617 1325 Crs
3174 1339 Crs
3594 1333 Crs
3840 1340 Crs
4571 1337 Crs
5238 1336 Crs
5412 1335 Crs
5969 1332 Crs
6389 1334 Crs
6635 1332 Crs
6810 1321 Crs
7366 1338 Crs
7787 1332 Crs
8033 1349 Crs
8207 1339 Crs
8763 1340 Crs
9184 1325 Crs
9430 1343 Crs
9605 1334 Crs
10582 1330 Crs
11002 1303 Crs
11558 1282 Crs
11979 1280 Crs
12225 1282 Crs
12400 1287 Crs
1.000 UL
LTb
1220 8399 N
0 -7759 V
11339 0 V
0 7759 V
-11339 0 V
Z stroke
1.000 UP
1.000 UL
LTb
stroke
grestore
end
showpage
  }}%
  \put(2420,7876){\makebox(0,0)[r]{\strut{}Linear}}%
  \put(2420,8116){\makebox(0,0)[r]{\strut{}Quadratic}}%
  \put(6889,140){\makebox(0,0){\strut{}Relaxation Parameter $\eta$}}%
  \put(280,4519){%
  \special{ps: gsave currentpoint currentpoint translate
630 rotate neg exch neg exch translate}%
  \makebox(0,0){\strut{}Eigenvalue Estimate}%
  \special{ps: currentpoint grestore moveto}%
  }%
  \put(12400,440){\makebox(0,0){\strut{} 1}}%
  \put(11002,440){\makebox(0,0){\strut{} 0.1}}%
  \put(9605,440){\makebox(0,0){\strut{} 0.01}}%
  \put(8207,440){\makebox(0,0){\strut{} 0.001}}%
  \put(6810,440){\makebox(0,0){\strut{} 0.0001}}%
  \put(5412,440){\makebox(0,0){\strut{} 1e-05}}%
  \put(4015,440){\makebox(0,0){\strut{} 1e-06}}%
  \put(2617,440){\makebox(0,0){\strut{} 1e-07}}%
  \put(1220,440){\makebox(0,0){\strut{} 1e-08}}%
  \put(1100,8399){\makebox(0,0)[r]{\strut{} 1.05}}%
  \put(1100,7429){\makebox(0,0)[r]{\strut{} 1.04}}%
  \put(1100,6459){\makebox(0,0)[r]{\strut{} 1.03}}%
  \put(1100,5489){\makebox(0,0)[r]{\strut{} 1.02}}%
  \put(1100,4520){\makebox(0,0)[r]{\strut{} 1.01}}%
  \put(1100,3550){\makebox(0,0)[r]{\strut{} 1}}%
  \put(1100,2580){\makebox(0,0)[r]{\strut{} 0.99}}%
  \put(1100,1610){\makebox(0,0)[r]{\strut{} 0.98}}%
  \put(1100,640){\makebox(0,0)[r]{\strut{} 0.97}}%
\end{picture}%
\endgroup
\endinput

    \caption{Eigenvalue estimates for the fundamental and first two harmonics for varying values of the relaxation parameter $\eta$.  The number of particles tracked in a non-relaxed iteration is 5E5.  The solid lines show a quadratic approach to relaxing and the dashed lines show a linear approach.  The black lines are the reference eigenvalues from \cite{Garis:1991One-s-0} and \cite{Dahl:1979Eigen-0}.}
    \label{fig:MoreandLessValue}
\end{sidewaysfigure}

\Fref{fig:MoreandLessFOM} shows the figure of merit for both strategies.  The dashed lines show the figure of merit for an Arnoldi calculation with no relaxation.  Both strategies demonstrate similar behavior in that a small amount of relaxation is actually worse than no relaxation at all.  The range of values of $\eta$ for which relaxation may benefit the calculation is larger for the linear strategy.  


\begin{sidewaysfigure}\centering
    % GNUPLOT: LaTeX picture with Postscript
\begingroup%
\makeatletter%
\newcommand{\GNUPLOTspecial}{%
  \@sanitize\catcode`\%=14\relax\special}%
\setlength{\unitlength}{0.0500bp}%
\begin{picture}(12960,8640)(0,0)%
  {\GNUPLOTspecial{"
%!PS-Adobe-2.0 EPSF-2.0
%%Title: MoreandLessFOM.tex
%%Creator: gnuplot 4.3 patchlevel 0
%%CreationDate: Sat Jul 18 21:07:55 2009
%%DocumentFonts: 
%%BoundingBox: 0 0 648 432
%%EndComments
%%BeginProlog
/gnudict 256 dict def
gnudict begin
%
% The following true/false flags may be edited by hand if desired.
% The unit line width and grayscale image gamma correction may also be changed.
%
/Color true def
/Blacktext true def
/Solid false def
/Dashlength 1 def
/Landscape false def
/Level1 false def
/Rounded false def
/ClipToBoundingBox false def
/TransparentPatterns false def
/gnulinewidth 5.000 def
/userlinewidth gnulinewidth def
/Gamma 1.0 def
%
/vshift -66 def
/dl1 {
  10.0 Dashlength mul mul
  Rounded { currentlinewidth 0.75 mul sub dup 0 le { pop 0.01 } if } if
} def
/dl2 {
  10.0 Dashlength mul mul
  Rounded { currentlinewidth 0.75 mul add } if
} def
/hpt_ 31.5 def
/vpt_ 31.5 def
/hpt hpt_ def
/vpt vpt_ def
Level1 {} {
/SDict 10 dict def
systemdict /pdfmark known not {
  userdict /pdfmark systemdict /cleartomark get put
} if
SDict begin [
  /Title (MoreandLessFOM.tex)
  /Subject (gnuplot plot)
  /Creator (gnuplot 4.3 patchlevel 0)
  /Author (Jeremy Conlin)
%  /Producer (gnuplot)
%  /Keywords ()
  /CreationDate (Sat Jul 18 21:07:55 2009)
  /DOCINFO pdfmark
end
} ifelse
/doclip {
  ClipToBoundingBox {
    newpath 0 0 moveto 648 0 lineto 648 432 lineto 0 432 lineto closepath
    clip
  } if
} def
%
% Gnuplot Prolog Version 4.2 (November 2007)
%
/M {moveto} bind def
/L {lineto} bind def
/R {rmoveto} bind def
/V {rlineto} bind def
/N {newpath moveto} bind def
/Z {closepath} bind def
/C {setrgbcolor} bind def
/f {rlineto fill} bind def
/Gshow {show} def   % May be redefined later in the file to support UTF-8
/vpt2 vpt 2 mul def
/hpt2 hpt 2 mul def
/Lshow {currentpoint stroke M 0 vshift R 
	Blacktext {gsave 0 setgray show grestore} {show} ifelse} def
/Rshow {currentpoint stroke M dup stringwidth pop neg vshift R
	Blacktext {gsave 0 setgray show grestore} {show} ifelse} def
/Cshow {currentpoint stroke M dup stringwidth pop -2 div vshift R 
	Blacktext {gsave 0 setgray show grestore} {show} ifelse} def
/UP {dup vpt_ mul /vpt exch def hpt_ mul /hpt exch def
  /hpt2 hpt 2 mul def /vpt2 vpt 2 mul def} def
/DL {Color {setrgbcolor Solid {pop []} if 0 setdash}
 {pop pop pop 0 setgray Solid {pop []} if 0 setdash} ifelse} def
/BL {stroke userlinewidth 2 mul setlinewidth
	Rounded {1 setlinejoin 1 setlinecap} if} def
/AL {stroke userlinewidth 2 div setlinewidth
	Rounded {1 setlinejoin 1 setlinecap} if} def
/UL {dup gnulinewidth mul /userlinewidth exch def
	dup 1 lt {pop 1} if 10 mul /udl exch def} def
/PL {stroke userlinewidth setlinewidth
	Rounded {1 setlinejoin 1 setlinecap} if} def
% Default Line colors
/LCw {1 1 1} def
/LCb {0 0 0} def
/LCa {0 0 0} def
/LC0 {1 0 0} def
/LC1 {0 1 0} def
/LC2 {0 0 1} def
/LC3 {1 0 1} def
/LC4 {0 1 1} def
/LC5 {1 1 0} def
/LC6 {0 0 0} def
/LC7 {1 0.3 0} def
/LC8 {0.5 0.5 0.5} def
% Default Line Types
/LTw {PL [] 1 setgray} def
/LTb {BL [] LCb DL} def
/LTa {AL [1 udl mul 2 udl mul] 0 setdash LCa setrgbcolor} def
/LT0 {PL [] LC0 DL} def
/LT1 {PL [4 dl1 2 dl2] LC1 DL} def
/LT2 {PL [2 dl1 3 dl2] LC2 DL} def
/LT3 {PL [1 dl1 1.5 dl2] LC3 DL} def
/LT4 {PL [6 dl1 2 dl2 1 dl1 2 dl2] LC4 DL} def
/LT5 {PL [3 dl1 3 dl2 1 dl1 3 dl2] LC5 DL} def
/LT6 {PL [2 dl1 2 dl2 2 dl1 6 dl2] LC6 DL} def
/LT7 {PL [1 dl1 2 dl2 6 dl1 2 dl2 1 dl1 2 dl2] LC7 DL} def
/LT8 {PL [2 dl1 2 dl2 2 dl1 2 dl2 2 dl1 2 dl2 2 dl1 4 dl2] LC8 DL} def
/Pnt {stroke [] 0 setdash gsave 1 setlinecap M 0 0 V stroke grestore} def
/Dia {stroke [] 0 setdash 2 copy vpt add M
  hpt neg vpt neg V hpt vpt neg V
  hpt vpt V hpt neg vpt V closepath stroke
  Pnt} def
/Pls {stroke [] 0 setdash vpt sub M 0 vpt2 V
  currentpoint stroke M
  hpt neg vpt neg R hpt2 0 V stroke
 } def
/Box {stroke [] 0 setdash 2 copy exch hpt sub exch vpt add M
  0 vpt2 neg V hpt2 0 V 0 vpt2 V
  hpt2 neg 0 V closepath stroke
  Pnt} def
/Crs {stroke [] 0 setdash exch hpt sub exch vpt add M
  hpt2 vpt2 neg V currentpoint stroke M
  hpt2 neg 0 R hpt2 vpt2 V stroke} def
/TriU {stroke [] 0 setdash 2 copy vpt 1.12 mul add M
  hpt neg vpt -1.62 mul V
  hpt 2 mul 0 V
  hpt neg vpt 1.62 mul V closepath stroke
  Pnt} def
/Star {2 copy Pls Crs} def
/BoxF {stroke [] 0 setdash exch hpt sub exch vpt add M
  0 vpt2 neg V hpt2 0 V 0 vpt2 V
  hpt2 neg 0 V closepath fill} def
/TriUF {stroke [] 0 setdash vpt 1.12 mul add M
  hpt neg vpt -1.62 mul V
  hpt 2 mul 0 V
  hpt neg vpt 1.62 mul V closepath fill} def
/TriD {stroke [] 0 setdash 2 copy vpt 1.12 mul sub M
  hpt neg vpt 1.62 mul V
  hpt 2 mul 0 V
  hpt neg vpt -1.62 mul V closepath stroke
  Pnt} def
/TriDF {stroke [] 0 setdash vpt 1.12 mul sub M
  hpt neg vpt 1.62 mul V
  hpt 2 mul 0 V
  hpt neg vpt -1.62 mul V closepath fill} def
/DiaF {stroke [] 0 setdash vpt add M
  hpt neg vpt neg V hpt vpt neg V
  hpt vpt V hpt neg vpt V closepath fill} def
/Pent {stroke [] 0 setdash 2 copy gsave
  translate 0 hpt M 4 {72 rotate 0 hpt L} repeat
  closepath stroke grestore Pnt} def
/PentF {stroke [] 0 setdash gsave
  translate 0 hpt M 4 {72 rotate 0 hpt L} repeat
  closepath fill grestore} def
/Circle {stroke [] 0 setdash 2 copy
  hpt 0 360 arc stroke Pnt} def
/CircleF {stroke [] 0 setdash hpt 0 360 arc fill} def
/C0 {BL [] 0 setdash 2 copy moveto vpt 90 450 arc} bind def
/C1 {BL [] 0 setdash 2 copy moveto
	2 copy vpt 0 90 arc closepath fill
	vpt 0 360 arc closepath} bind def
/C2 {BL [] 0 setdash 2 copy moveto
	2 copy vpt 90 180 arc closepath fill
	vpt 0 360 arc closepath} bind def
/C3 {BL [] 0 setdash 2 copy moveto
	2 copy vpt 0 180 arc closepath fill
	vpt 0 360 arc closepath} bind def
/C4 {BL [] 0 setdash 2 copy moveto
	2 copy vpt 180 270 arc closepath fill
	vpt 0 360 arc closepath} bind def
/C5 {BL [] 0 setdash 2 copy moveto
	2 copy vpt 0 90 arc
	2 copy moveto
	2 copy vpt 180 270 arc closepath fill
	vpt 0 360 arc} bind def
/C6 {BL [] 0 setdash 2 copy moveto
	2 copy vpt 90 270 arc closepath fill
	vpt 0 360 arc closepath} bind def
/C7 {BL [] 0 setdash 2 copy moveto
	2 copy vpt 0 270 arc closepath fill
	vpt 0 360 arc closepath} bind def
/C8 {BL [] 0 setdash 2 copy moveto
	2 copy vpt 270 360 arc closepath fill
	vpt 0 360 arc closepath} bind def
/C9 {BL [] 0 setdash 2 copy moveto
	2 copy vpt 270 450 arc closepath fill
	vpt 0 360 arc closepath} bind def
/C10 {BL [] 0 setdash 2 copy 2 copy moveto vpt 270 360 arc closepath fill
	2 copy moveto
	2 copy vpt 90 180 arc closepath fill
	vpt 0 360 arc closepath} bind def
/C11 {BL [] 0 setdash 2 copy moveto
	2 copy vpt 0 180 arc closepath fill
	2 copy moveto
	2 copy vpt 270 360 arc closepath fill
	vpt 0 360 arc closepath} bind def
/C12 {BL [] 0 setdash 2 copy moveto
	2 copy vpt 180 360 arc closepath fill
	vpt 0 360 arc closepath} bind def
/C13 {BL [] 0 setdash 2 copy moveto
	2 copy vpt 0 90 arc closepath fill
	2 copy moveto
	2 copy vpt 180 360 arc closepath fill
	vpt 0 360 arc closepath} bind def
/C14 {BL [] 0 setdash 2 copy moveto
	2 copy vpt 90 360 arc closepath fill
	vpt 0 360 arc} bind def
/C15 {BL [] 0 setdash 2 copy vpt 0 360 arc closepath fill
	vpt 0 360 arc closepath} bind def
/Rec {newpath 4 2 roll moveto 1 index 0 rlineto 0 exch rlineto
	neg 0 rlineto closepath} bind def
/Square {dup Rec} bind def
/Bsquare {vpt sub exch vpt sub exch vpt2 Square} bind def
/S0 {BL [] 0 setdash 2 copy moveto 0 vpt rlineto BL Bsquare} bind def
/S1 {BL [] 0 setdash 2 copy vpt Square fill Bsquare} bind def
/S2 {BL [] 0 setdash 2 copy exch vpt sub exch vpt Square fill Bsquare} bind def
/S3 {BL [] 0 setdash 2 copy exch vpt sub exch vpt2 vpt Rec fill Bsquare} bind def
/S4 {BL [] 0 setdash 2 copy exch vpt sub exch vpt sub vpt Square fill Bsquare} bind def
/S5 {BL [] 0 setdash 2 copy 2 copy vpt Square fill
	exch vpt sub exch vpt sub vpt Square fill Bsquare} bind def
/S6 {BL [] 0 setdash 2 copy exch vpt sub exch vpt sub vpt vpt2 Rec fill Bsquare} bind def
/S7 {BL [] 0 setdash 2 copy exch vpt sub exch vpt sub vpt vpt2 Rec fill
	2 copy vpt Square fill Bsquare} bind def
/S8 {BL [] 0 setdash 2 copy vpt sub vpt Square fill Bsquare} bind def
/S9 {BL [] 0 setdash 2 copy vpt sub vpt vpt2 Rec fill Bsquare} bind def
/S10 {BL [] 0 setdash 2 copy vpt sub vpt Square fill 2 copy exch vpt sub exch vpt Square fill
	Bsquare} bind def
/S11 {BL [] 0 setdash 2 copy vpt sub vpt Square fill 2 copy exch vpt sub exch vpt2 vpt Rec fill
	Bsquare} bind def
/S12 {BL [] 0 setdash 2 copy exch vpt sub exch vpt sub vpt2 vpt Rec fill Bsquare} bind def
/S13 {BL [] 0 setdash 2 copy exch vpt sub exch vpt sub vpt2 vpt Rec fill
	2 copy vpt Square fill Bsquare} bind def
/S14 {BL [] 0 setdash 2 copy exch vpt sub exch vpt sub vpt2 vpt Rec fill
	2 copy exch vpt sub exch vpt Square fill Bsquare} bind def
/S15 {BL [] 0 setdash 2 copy Bsquare fill Bsquare} bind def
/D0 {gsave translate 45 rotate 0 0 S0 stroke grestore} bind def
/D1 {gsave translate 45 rotate 0 0 S1 stroke grestore} bind def
/D2 {gsave translate 45 rotate 0 0 S2 stroke grestore} bind def
/D3 {gsave translate 45 rotate 0 0 S3 stroke grestore} bind def
/D4 {gsave translate 45 rotate 0 0 S4 stroke grestore} bind def
/D5 {gsave translate 45 rotate 0 0 S5 stroke grestore} bind def
/D6 {gsave translate 45 rotate 0 0 S6 stroke grestore} bind def
/D7 {gsave translate 45 rotate 0 0 S7 stroke grestore} bind def
/D8 {gsave translate 45 rotate 0 0 S8 stroke grestore} bind def
/D9 {gsave translate 45 rotate 0 0 S9 stroke grestore} bind def
/D10 {gsave translate 45 rotate 0 0 S10 stroke grestore} bind def
/D11 {gsave translate 45 rotate 0 0 S11 stroke grestore} bind def
/D12 {gsave translate 45 rotate 0 0 S12 stroke grestore} bind def
/D13 {gsave translate 45 rotate 0 0 S13 stroke grestore} bind def
/D14 {gsave translate 45 rotate 0 0 S14 stroke grestore} bind def
/D15 {gsave translate 45 rotate 0 0 S15 stroke grestore} bind def
/DiaE {stroke [] 0 setdash vpt add M
  hpt neg vpt neg V hpt vpt neg V
  hpt vpt V hpt neg vpt V closepath stroke} def
/BoxE {stroke [] 0 setdash exch hpt sub exch vpt add M
  0 vpt2 neg V hpt2 0 V 0 vpt2 V
  hpt2 neg 0 V closepath stroke} def
/TriUE {stroke [] 0 setdash vpt 1.12 mul add M
  hpt neg vpt -1.62 mul V
  hpt 2 mul 0 V
  hpt neg vpt 1.62 mul V closepath stroke} def
/TriDE {stroke [] 0 setdash vpt 1.12 mul sub M
  hpt neg vpt 1.62 mul V
  hpt 2 mul 0 V
  hpt neg vpt -1.62 mul V closepath stroke} def
/PentE {stroke [] 0 setdash gsave
  translate 0 hpt M 4 {72 rotate 0 hpt L} repeat
  closepath stroke grestore} def
/CircE {stroke [] 0 setdash 
  hpt 0 360 arc stroke} def
/Opaque {gsave closepath 1 setgray fill grestore 0 setgray closepath} def
/DiaW {stroke [] 0 setdash vpt add M
  hpt neg vpt neg V hpt vpt neg V
  hpt vpt V hpt neg vpt V Opaque stroke} def
/BoxW {stroke [] 0 setdash exch hpt sub exch vpt add M
  0 vpt2 neg V hpt2 0 V 0 vpt2 V
  hpt2 neg 0 V Opaque stroke} def
/TriUW {stroke [] 0 setdash vpt 1.12 mul add M
  hpt neg vpt -1.62 mul V
  hpt 2 mul 0 V
  hpt neg vpt 1.62 mul V Opaque stroke} def
/TriDW {stroke [] 0 setdash vpt 1.12 mul sub M
  hpt neg vpt 1.62 mul V
  hpt 2 mul 0 V
  hpt neg vpt -1.62 mul V Opaque stroke} def
/PentW {stroke [] 0 setdash gsave
  translate 0 hpt M 4 {72 rotate 0 hpt L} repeat
  Opaque stroke grestore} def
/CircW {stroke [] 0 setdash 
  hpt 0 360 arc Opaque stroke} def
/BoxFill {gsave Rec 1 setgray fill grestore} def
/Density {
  /Fillden exch def
  currentrgbcolor
  /ColB exch def /ColG exch def /ColR exch def
  /ColR ColR Fillden mul Fillden sub 1 add def
  /ColG ColG Fillden mul Fillden sub 1 add def
  /ColB ColB Fillden mul Fillden sub 1 add def
  ColR ColG ColB setrgbcolor} def
/BoxColFill {gsave Rec PolyFill} def
/PolyFill {gsave Density fill grestore grestore} def
/h {rlineto rlineto rlineto gsave closepath fill grestore} bind def
%
% PostScript Level 1 Pattern Fill routine for rectangles
% Usage: x y w h s a XX PatternFill
%	x,y = lower left corner of box to be filled
%	w,h = width and height of box
%	  a = angle in degrees between lines and x-axis
%	 XX = 0/1 for no/yes cross-hatch
%
/PatternFill {gsave /PFa [ 9 2 roll ] def
  PFa 0 get PFa 2 get 2 div add PFa 1 get PFa 3 get 2 div add translate
  PFa 2 get -2 div PFa 3 get -2 div PFa 2 get PFa 3 get Rec
  gsave 1 setgray fill grestore clip
  currentlinewidth 0.5 mul setlinewidth
  /PFs PFa 2 get dup mul PFa 3 get dup mul add sqrt def
  0 0 M PFa 5 get rotate PFs -2 div dup translate
  0 1 PFs PFa 4 get div 1 add floor cvi
	{PFa 4 get mul 0 M 0 PFs V} for
  0 PFa 6 get ne {
	0 1 PFs PFa 4 get div 1 add floor cvi
	{PFa 4 get mul 0 2 1 roll M PFs 0 V} for
 } if
  stroke grestore} def
%
/languagelevel where
 {pop languagelevel} {1} ifelse
 2 lt
	{/InterpretLevel1 true def}
	{/InterpretLevel1 Level1 def}
 ifelse
%
% PostScript level 2 pattern fill definitions
%
/Level2PatternFill {
/Tile8x8 {/PaintType 2 /PatternType 1 /TilingType 1 /BBox [0 0 8 8] /XStep 8 /YStep 8}
	bind def
/KeepColor {currentrgbcolor [/Pattern /DeviceRGB] setcolorspace} bind def
<< Tile8x8
 /PaintProc {0.5 setlinewidth pop 0 0 M 8 8 L 0 8 M 8 0 L stroke} 
>> matrix makepattern
/Pat1 exch def
<< Tile8x8
 /PaintProc {0.5 setlinewidth pop 0 0 M 8 8 L 0 8 M 8 0 L stroke
	0 4 M 4 8 L 8 4 L 4 0 L 0 4 L stroke}
>> matrix makepattern
/Pat2 exch def
<< Tile8x8
 /PaintProc {0.5 setlinewidth pop 0 0 M 0 8 L
	8 8 L 8 0 L 0 0 L fill}
>> matrix makepattern
/Pat3 exch def
<< Tile8x8
 /PaintProc {0.5 setlinewidth pop -4 8 M 8 -4 L
	0 12 M 12 0 L stroke}
>> matrix makepattern
/Pat4 exch def
<< Tile8x8
 /PaintProc {0.5 setlinewidth pop -4 0 M 8 12 L
	0 -4 M 12 8 L stroke}
>> matrix makepattern
/Pat5 exch def
<< Tile8x8
 /PaintProc {0.5 setlinewidth pop -2 8 M 4 -4 L
	0 12 M 8 -4 L 4 12 M 10 0 L stroke}
>> matrix makepattern
/Pat6 exch def
<< Tile8x8
 /PaintProc {0.5 setlinewidth pop -2 0 M 4 12 L
	0 -4 M 8 12 L 4 -4 M 10 8 L stroke}
>> matrix makepattern
/Pat7 exch def
<< Tile8x8
 /PaintProc {0.5 setlinewidth pop 8 -2 M -4 4 L
	12 0 M -4 8 L 12 4 M 0 10 L stroke}
>> matrix makepattern
/Pat8 exch def
<< Tile8x8
 /PaintProc {0.5 setlinewidth pop 0 -2 M 12 4 L
	-4 0 M 12 8 L -4 4 M 8 10 L stroke}
>> matrix makepattern
/Pat9 exch def
/Pattern1 {PatternBgnd KeepColor Pat1 setpattern} bind def
/Pattern2 {PatternBgnd KeepColor Pat2 setpattern} bind def
/Pattern3 {PatternBgnd KeepColor Pat3 setpattern} bind def
/Pattern4 {PatternBgnd KeepColor Landscape {Pat5} {Pat4} ifelse setpattern} bind def
/Pattern5 {PatternBgnd KeepColor Landscape {Pat4} {Pat5} ifelse setpattern} bind def
/Pattern6 {PatternBgnd KeepColor Landscape {Pat9} {Pat6} ifelse setpattern} bind def
/Pattern7 {PatternBgnd KeepColor Landscape {Pat8} {Pat7} ifelse setpattern} bind def
} def
%
%
%End of PostScript Level 2 code
%
/PatternBgnd {
  TransparentPatterns {} {gsave 1 setgray fill grestore} ifelse
} def
%
% Substitute for Level 2 pattern fill codes with
% grayscale if Level 2 support is not selected.
%
/Level1PatternFill {
/Pattern1 {0.250 Density} bind def
/Pattern2 {0.500 Density} bind def
/Pattern3 {0.750 Density} bind def
/Pattern4 {0.125 Density} bind def
/Pattern5 {0.375 Density} bind def
/Pattern6 {0.625 Density} bind def
/Pattern7 {0.875 Density} bind def
} def
%
% Now test for support of Level 2 code
%
Level1 {Level1PatternFill} {Level2PatternFill} ifelse
%
/Symbol-Oblique /Symbol findfont [1 0 .167 1 0 0] makefont
dup length dict begin {1 index /FID eq {pop pop} {def} ifelse} forall
currentdict end definefont pop
end
%%EndProlog
gnudict begin
gsave
doclip
0 0 translate
0.050 0.050 scale
0 setgray
newpath
1.000 UL
LTb
1340 640 M
63 0 V
-63 1552 R
63 0 V
-63 1552 R
63 0 V
-63 1551 R
63 0 V
-63 1552 R
63 0 V
-63 1552 R
63 0 V
1340 640 M
0 63 V
0 7696 R
0 -63 V
1724 640 M
0 31 V
0 7728 R
0 -31 V
2232 640 M
0 31 V
0 7728 R
0 -31 V
2493 640 M
0 31 V
0 7728 R
0 -31 V
2617 640 M
0 63 V
0 7696 R
0 -63 V
3001 640 M
0 31 V
0 7728 R
0 -31 V
3509 640 M
0 31 V
0 7728 R
0 -31 V
3770 640 M
0 31 V
0 7728 R
0 -31 V
3893 640 M
0 63 V
0 7696 R
0 -63 V
4278 640 M
0 31 V
0 7728 R
0 -31 V
4786 640 M
0 31 V
0 7728 R
0 -31 V
5046 640 M
0 31 V
0 7728 R
0 -31 V
5170 640 M
0 63 V
0 7696 R
0 -63 V
5554 640 M
0 31 V
0 7728 R
0 -31 V
6062 640 M
0 31 V
0 7728 R
0 -31 V
6323 640 M
0 31 V
0 7728 R
0 -31 V
6447 640 M
0 63 V
0 7696 R
0 -63 V
6831 640 M
0 31 V
0 7728 R
0 -31 V
7339 640 M
0 31 V
0 7728 R
0 -31 V
7600 640 M
0 31 V
0 7728 R
0 -31 V
7723 640 M
0 63 V
0 7696 R
0 -63 V
8108 640 M
0 31 V
0 7728 R
0 -31 V
8616 640 M
0 31 V
0 7728 R
0 -31 V
8876 640 M
0 31 V
stroke 8876 671 M
0 7728 R
0 -31 V
9000 640 M
0 63 V
0 7696 R
0 -63 V
9384 640 M
0 31 V
0 7728 R
0 -31 V
9893 640 M
0 31 V
0 7728 R
0 -31 V
10153 640 M
0 31 V
0 7728 R
0 -31 V
10277 640 M
0 63 V
0 7696 R
0 -63 V
10661 640 M
0 31 V
0 7728 R
0 -31 V
11169 640 M
0 31 V
0 7728 R
0 -31 V
11430 640 M
0 31 V
0 7728 R
0 -31 V
11554 640 M
0 63 V
0 7696 R
0 -63 V
11699 640 M
-63 0 V
63 1552 R
-63 0 V
63 1552 R
-63 0 V
63 1551 R
-63 0 V
63 1552 R
-63 0 V
63 1552 R
-63 0 V
stroke
1340 8399 N
0 -7759 V
10359 0 V
0 7759 V
-10359 0 V
Z stroke
LCb setrgbcolor
LTb
LCb setrgbcolor
LTb
LCb setrgbcolor
LTb
LCb setrgbcolor
LTb
1.000 UP
1.000 UL
LTb
1.000 UL
LTb
1460 7656 N
0 680 V
1863 0 V
0 -680 V
-1863 0 V
Z stroke
1460 8336 M
1863 0 V
stroke
LT0
LC0 setrgbcolor
LCb setrgbcolor
LT0
LC0 setrgbcolor
2660 8116 M
543 0 V
1340 4632 M
1848 3238 L
384 528 V
225 553 V
160 123 V
508 -315 V
384 -35 V
225 651 V
667 -252 V
610 533 V
5170 3877 L
508 22 V
384 2446 V
6287 4502 L
160 -361 V
508 2418 V
7339 4499 L
225 1069 V
159 974 V
509 750 V
8616 972 L
225 424 V
9000 939 L
893 25 V
384 -60 V
892 -43 V
385 -1 V
stroke
LT1
LC0 setrgbcolor
1340 4497 M
105 0 V
104 0 V
105 0 V
105 0 V
104 0 V
105 0 V
104 0 V
105 0 V
105 0 V
104 0 V
105 0 V
105 0 V
104 0 V
105 0 V
105 0 V
104 0 V
105 0 V
104 0 V
105 0 V
105 0 V
104 0 V
105 0 V
105 0 V
104 0 V
105 0 V
105 0 V
104 0 V
105 0 V
104 0 V
105 0 V
105 0 V
104 0 V
105 0 V
105 0 V
104 0 V
105 0 V
105 0 V
104 0 V
105 0 V
104 0 V
105 0 V
105 0 V
104 0 V
105 0 V
105 0 V
104 0 V
105 0 V
105 0 V
104 0 V
105 0 V
104 0 V
105 0 V
105 0 V
104 0 V
105 0 V
105 0 V
104 0 V
105 0 V
105 0 V
104 0 V
105 0 V
104 0 V
105 0 V
105 0 V
104 0 V
105 0 V
105 0 V
104 0 V
105 0 V
105 0 V
104 0 V
105 0 V
104 0 V
105 0 V
105 0 V
104 0 V
105 0 V
105 0 V
104 0 V
105 0 V
105 0 V
104 0 V
105 0 V
104 0 V
105 0 V
105 0 V
104 0 V
105 0 V
105 0 V
104 0 V
105 0 V
105 0 V
104 0 V
105 0 V
104 0 V
105 0 V
105 0 V
104 0 V
105 0 V
stroke
LT0
LC2 setrgbcolor
LCb setrgbcolor
LT0
LC2 setrgbcolor
2660 7876 M
543 0 V
1340 4317 M
508 -324 V
384 -375 V
225 1167 V
160 -851 V
508 5 V
384 -7 V
225 533 V
667 -925 V
610 -204 V
159 -204 V
508 1620 V
6062 3458 L
225 426 V
160 -357 V
508 1780 V
384 -13 V
225 218 V
159 -47 V
509 1397 V
384 -14 V
225 878 V
159 -217 V
9893 5166 L
384 -1329 V
508 -1577 V
384 -213 V
225 -431 V
160 -166 V
stroke
LT1
LC2 setrgbcolor
1340 4302 M
105 0 V
104 0 V
105 0 V
105 0 V
104 0 V
105 0 V
104 0 V
105 0 V
105 0 V
104 0 V
105 0 V
105 0 V
104 0 V
105 0 V
105 0 V
104 0 V
105 0 V
104 0 V
105 0 V
105 0 V
104 0 V
105 0 V
105 0 V
104 0 V
105 0 V
105 0 V
104 0 V
105 0 V
104 0 V
105 0 V
105 0 V
104 0 V
105 0 V
105 0 V
104 0 V
105 0 V
105 0 V
104 0 V
105 0 V
104 0 V
105 0 V
105 0 V
104 0 V
105 0 V
105 0 V
104 0 V
105 0 V
105 0 V
104 0 V
105 0 V
104 0 V
105 0 V
105 0 V
104 0 V
105 0 V
105 0 V
104 0 V
105 0 V
105 0 V
104 0 V
105 0 V
104 0 V
105 0 V
105 0 V
104 0 V
105 0 V
105 0 V
104 0 V
105 0 V
105 0 V
104 0 V
105 0 V
104 0 V
105 0 V
105 0 V
104 0 V
105 0 V
105 0 V
104 0 V
105 0 V
105 0 V
104 0 V
105 0 V
104 0 V
105 0 V
105 0 V
104 0 V
105 0 V
105 0 V
104 0 V
105 0 V
105 0 V
104 0 V
105 0 V
104 0 V
105 0 V
105 0 V
104 0 V
105 0 V
stroke
LTb
1340 8399 N
0 -7759 V
10359 0 V
0 7759 V
-10359 0 V
Z stroke
1.000 UP
1.000 UL
LTb
stroke
grestore
end
showpage
  }}%
  \put(2540,7876){\makebox(0,0)[r]{\strut{}Linear}}%
  \put(2540,8116){\makebox(0,0)[r]{\strut{}Quadratic}}%
  \put(6519,140){\makebox(0,0){\strut{}Relaxation Parameter $\eta$}}%
  \put(280,4519){%
  \special{ps: gsave currentpoint currentpoint translate
630 rotate neg exch neg exch translate}%
  \makebox(0,0){\strut{}FOM $\left(1/\sigma^2T\right)$}%
  \special{ps: currentpoint grestore moveto}%
  }%
  \put(11819,8399){\makebox(0,0)[l]{\strut{} 25000}}%
  \put(11819,6847){\makebox(0,0)[l]{\strut{} 20000}}%
  \put(11819,5295){\makebox(0,0)[l]{\strut{} 15000}}%
  \put(11819,3744){\makebox(0,0)[l]{\strut{} 10000}}%
  \put(11819,2192){\makebox(0,0)[l]{\strut{} 5000}}%
  \put(11819,640){\makebox(0,0)[l]{\strut{} 0}}%
  \put(11554,440){\makebox(0,0){\strut{} 1}}%
  \put(10277,440){\makebox(0,0){\strut{} 0.1}}%
  \put(9000,440){\makebox(0,0){\strut{} 0.01}}%
  \put(7723,440){\makebox(0,0){\strut{} 0.001}}%
  \put(6447,440){\makebox(0,0){\strut{} 0.0001}}%
  \put(5170,440){\makebox(0,0){\strut{} 1e-05}}%
  \put(3893,440){\makebox(0,0){\strut{} 1e-06}}%
  \put(2617,440){\makebox(0,0){\strut{} 1e-07}}%
  \put(1340,440){\makebox(0,0){\strut{} 1e-08}}%
  \put(1220,8399){\makebox(0,0)[r]{\strut{} 25000}}%
  \put(1220,6847){\makebox(0,0)[r]{\strut{} 20000}}%
  \put(1220,5295){\makebox(0,0)[r]{\strut{} 15000}}%
  \put(1220,3744){\makebox(0,0)[r]{\strut{} 10000}}%
  \put(1220,2192){\makebox(0,0)[r]{\strut{} 5000}}%
  \put(1220,640){\makebox(0,0)[r]{\strut{} 0}}%
\end{picture}%
\endgroup
\endinput

    \caption{Fundamental eigenvalue estimate and figure of merit for varying values of the relaxation parameter $\eta$.  The red lines show a quadratic approach to relaxing and the blue lines show a linear approach.  The dashed lines is the value of the figure of merit when there is no relaxation.  The number of particles tracked in a non-relaxed iteration is 5E5.}
    \label{fig:MoreandLessFOM}
\end{sidewaysfigure}

% Less aggressive
\begin{table}\centering
    \begin{tabular}{ccccrr}
        \toprule
        \multirow{2}{*}{$\eta$} & Active & Total \# & \multirow{2}{*}{$\lambda$} & \multicolumn{1}{c}{\multirow{2}{*}{FOM}} & \multicolumn{1}{c}{Time} \\
        & Restarts & Particles & & & \multicolumn{1}{c}{(s)} \\
        \midrule
               0 &   100 & 2.500e+09 & 0.9975 $\pm$  1.0\e{-4} &  11799.5 & 7806.2 \\
           1E-08 &   100 & 2.500e+09 & 0.9974 $\pm$  1.0\e{-4} &  11845.8 & 7795.1 \\
         2.5E-08 &   100 & 2.500e+09 & 0.9973 $\pm$  1.1\e{-4} &  10801.6 & 7812.4 \\
           5E-08 &   100 & 2.500e+09 & 0.9974 $\pm$  1.2\e{-4} &   9595.2 & 7817.4 \\
         7.5E-08 &   100 & 2.500e+09 & 0.9974 $\pm$  9.8\e{-5} &  13357.0 & 7790.3 \\
           1E-07 &   100 & 2.500e+09 & 0.9974 $\pm$  1.1\e{-4} &  10614.9 & 7796.8 \\
         2.5E-07 &   100 & 2.500e+09 & 0.9975 $\pm$  1.1\e{-4} &  10629.2 & 7793.8 \\
           5E-07 &   100 & 2.500e+09 & 0.9973 $\pm$  1.1\e{-4} &  10606.7 & 7818.0 \\
         7.5E-07 &   100 & 2.500e+09 & 0.9973 $\pm$  1.0\e{-4} &  12324.6 & 7917.3 \\
         2.5E-06 &   100 & 2.500e+09 & 0.9974 $\pm$  1.2\e{-4} &   9345.4 & 7809.8 \\
         7.5E-06 &   100 & 2.500e+09 & 0.9974 $\pm$  1.2\e{-4} &   8686.7 & 7801.4 \\
           1E-05 &   100 & 2.500e+09 & 0.9973 $\pm$  1.3\e{-4} &   8029.7 & 7803.3 \\
         2.5E-05 &   100 & 2.497e+09 & 0.9974 $\pm$  9.8\e{-5} &  13247.9 & 7800.2 \\
           5E-05 &   100 & 2.484e+09 & 0.9972 $\pm$  1.2\e{-4} &   9080.7 & 7764.0 \\
         7.5E-05 &   100 & 2.471e+09 & 0.9972 $\pm$  1.1\e{-4} &  10451.4 & 7706.1 \\
          0.0001 &   100 & 2.437e+09 & 0.9975 $\pm$  1.2\e{-4} &   9302.1 & 7596.6 \\
         0.00025 &   114 & 2.493e+09 & 0.9976 $\pm$  9.2\e{-5} &  15038.1 & 7785.9 \\
          0.0005 &   124 & 2.481e+09 & 0.9974 $\pm$  9.3\e{-5} &  14997.3 & 7766.4 \\
         0.00075 &   133 & 2.505e+09 & 0.9974 $\pm$  9.0\e{-5} &  15698.0 & 7889.1 \\
           0.001 &   140 & 2.491e+09 & 0.9973 $\pm$  9.1\e{-5} &  15544.8 & 7781.8 \\
          0.0025 &   166 & 2.492e+09 & 0.9974 $\pm$  8.0\e{-5} &  20047.8 & 7783.3 \\
           0.005 &   194 & 2.472e+09 & 0.9974 $\pm$  8.0\e{-5} &  20003.4 & 7728.1 \\
          0.0075 &   224 & 2.475e+09 & 0.9974 $\pm$  7.5\e{-5} &  22830.9 & 7746.0 \\
            0.01 &   262 & 2.501e+09 & 0.9975 $\pm$  7.6\e{-5} &  22131.4 & 7840.6 \\
            0.05 &   817 & 2.520e+09 & 0.9975 $\pm$  9.3\e{-5} &  14582.1 & 7871.5 \\
             0.1 &  1173 & 2.483e+09 & 0.9979 $\pm$  1.1\e{-4} &  10302.1 & 7832.0 \\
            0.25 &  1649 & 2.503e+09 & 0.9992 $\pm$  1.6\e{-4} &   5219.9 & 7918.0 \\
             0.5 &  1917 & 2.506e+09 & 1.0004 $\pm$  1.7\e{-4} &   4533.4 & 7921.6 \\
            0.75 &  2037 & 2.515e+09 & 1.0016 $\pm$  2.0\e{-4} &   3143.8 & 7944.6 \\
               1 &  2090 & 2.496e+09 & 1.0028 $\pm$  2.2\e{-4} &   2609.6 & 7953.4 \\
        \bottomrule
    \end{tabular}
    \caption{Eigenvalue estimates for fundamental eigenvalue, figure of merit, and time for a relaxed Arnoldi simulation of a 20mfp thick slab.  The relaxation for this simulation is linear.  Also shown is the number of  active restarts and the total number of particles tracked in the simulation.  5E5 particles were tracked in each non-relaxed iteration.}
    \label{tab:Linear0}
\end{table}

% Harmonics
\begin{comment}
\begin{table}[ht]\centering
    \begin{tabular}{ccccrr}
        \toprule
        \multirow{2}{*}{$\eta$} & Active & Total \# & \multirow{2}{*}{$\lambda$} & \multicolumn{1}{c}{\multirow{2}{*}{FOM}} & \multicolumn{1}{c}{Time} \\
        & Restarts & Particles & & & \multicolumn{1}{c}{(s)} \\
        \midrule
               0 &   100 & 2.500e+09 & 0.9897 $\pm$  1.1\e{-4} &  11271.5 & 7806.2 \\
           1E-08 &   100 & 2.500e+09 & 0.9899 $\pm$  9.8\e{-5} &  13455.9 & 7795.1 \\
         2.5E-08 &   100 & 2.500e+09 & 0.9899 $\pm$  1.1\e{-4} &   9879.8 & 7812.4 \\
           5E-08 &   100 & 2.500e+09 & 0.9897 $\pm$  1.1\e{-4} &  10855.8 & 7817.4 \\
         7.5E-08 &   100 & 2.500e+09 & 0.9898 $\pm$  9.7\e{-5} &  13556.7 & 7790.3 \\
           1E-07 &   100 & 2.500e+09 & 0.9897 $\pm$  1.0\e{-4} &  12269.7 & 7796.8 \\
         2.5E-07 &   100 & 2.500e+09 & 0.9897 $\pm$  1.1\e{-4} &  10059.1 & 7793.8 \\
           5E-07 &   100 & 2.500e+09 & 0.9897 $\pm$  9.6\e{-5} &  13919.7 & 7818.0 \\
         7.5E-07 &   100 & 2.500e+09 & 0.9898 $\pm$  1.1\e{-4} &  10784.9 & 7917.3 \\
         2.5E-06 &   100 & 2.500e+09 & 0.9896 $\pm$  1.1\e{-4} &  11422.1 & 7809.8 \\
         7.5E-06 &   100 & 2.500e+09 & 0.9897 $\pm$  1.0\e{-4} &  12867.6 & 7801.4 \\
           1E-05 &   100 & 2.500e+09 & 0.9899 $\pm$  1.0\e{-4} &  12645.4 & 7803.3 \\
         2.5E-05 &   100 & 2.497e+09 & 0.9899 $\pm$  9.4\e{-5} &  14554.0 & 7800.2 \\
           5E-05 &   100 & 2.484e+09 & 0.9899 $\pm$  1.1\e{-4} &  11227.4 & 7764.0 \\
         7.5E-05 &   100 & 2.471e+09 & 0.9898 $\pm$  9.8\e{-5} &  13498.3 & 7706.1 \\
          0.0001 &   100 & 2.437e+09 & 0.9896 $\pm$  1.1\e{-4} &  10791.6 & 7596.6 \\
         0.00025 &   114 & 2.493e+09 & 0.9896 $\pm$  1.0\e{-4} &  12260.7 & 7785.9 \\
          0.0005 &   124 & 2.481e+09 & 0.9898 $\pm$  9.2\e{-5} &  15349.2 & 7766.4 \\
         0.00075 &   133 & 2.505e+09 & 0.9895 $\pm$  9.1\e{-5} &  15449.2 & 7889.1 \\
           0.001 &   140 & 2.491e+09 & 0.9896 $\pm$  9.3\e{-5} &  15006.1 & 7781.8 \\
          0.0025 &   166 & 2.492e+09 & 0.9895 $\pm$  1.1\e{-4} &  10929.5 & 7783.3 \\
           0.005 &   194 & 2.472e+09 & 0.9897 $\pm$  8.4\e{-5} &  18344.5 & 7728.1 \\
          0.0075 &   224 & 2.475e+09 & 0.9897 $\pm$  8.8\e{-5} &  16614.6 & 7746.0 \\
            0.01 &   262 & 2.501e+09 & 0.9896 $\pm$  8.2\e{-5} &  18759.5 & 7840.6 \\
            0.05 &   817 & 2.520e+09 & 0.9896 $\pm$  1.2\e{-4} &   8864.6 & 7871.5 \\
             0.1 &  1173 & 2.483e+09 & 0.9894 $\pm$  1.3\e{-4} &   7807.2 & 7832.0 \\
            0.25 &  1649 & 2.503e+09 & 0.9900 $\pm$  1.2\e{-4} &   9157.9 & 7918.0 \\
             0.5 &  1917 & 2.506e+09 & 0.9905 $\pm$  1.1\e{-4} &   9889.5 & 7921.6 \\
            0.75 &  2037 & 2.515e+09 & 0.9910 $\pm$  1.2\e{-4} &   9284.7 & 7944.6 \\
               1 &  2090 & 2.496e+09 & 0.9916 $\pm$  1.1\e{-4} &  10935.1 & 7953.4 \\
        \bottomrule
    \end{tabular}
    \caption{First higher order eigenvalue estimates with relaxation.}
    \label{tab:Linear1}
\end{table}

\begin{table}[ht]\centering
    \begin{tabular}{ccccrr}
        \toprule
        \multirow{2}{*}{$\eta$} & Active & Total \# & \multirow{2}{*}{$\lambda$} & \multicolumn{1}{c}{\multirow{2}{*}{FOM}} & \multicolumn{1}{c}{Time} \\
        & Restarts & Particles & & & \multicolumn{1}{c}{(s)} \\
        \midrule
               0 &   100 & 2.500e+09 & 0.9770 $\pm$  1.0\e{-4} &  12926.0 & 7806.2 \\
           1E-08 &   100 & 2.500e+09 & 0.9771 $\pm$  9.1\e{-5} &  15488.9 & 7795.1 \\
         2.5E-08 &   100 & 2.500e+09 & 0.9770 $\pm$  1.1\e{-4} &  11117.8 & 7812.4 \\
           5E-08 &   100 & 2.500e+09 & 0.9772 $\pm$  1.1\e{-4} &  11558.2 & 7817.4 \\
         7.5E-08 &   100 & 2.500e+09 & 0.9773 $\pm$  1.1\e{-4} &  10344.3 & 7790.3 \\
           1E-07 &   100 & 2.500e+09 & 0.9771 $\pm$  9.9\e{-5} &  13189.6 & 7796.8 \\
         2.5E-07 &   100 & 2.500e+09 & 0.9772 $\pm$  1.1\e{-4} &  10406.1 & 7793.8 \\
           5E-07 &   100 & 2.500e+09 & 0.9771 $\pm$  1.1\e{-4} &   9930.0 & 7818.0 \\
         7.5E-07 &   100 & 2.500e+09 & 0.9772 $\pm$  1.2\e{-4} &   9142.7 & 7917.3 \\
         2.5E-06 &   100 & 2.500e+09 & 0.9772 $\pm$  1.1\e{-4} &  11393.0 & 7809.8 \\
         7.5E-06 &   100 & 2.500e+09 & 0.9772 $\pm$  1.0\e{-4} &  11976.8 & 7801.4 \\
           1E-05 &   100 & 2.500e+09 & 0.9772 $\pm$  1.0\e{-4} &  11704.1 & 7803.3 \\
         2.5E-05 &   100 & 2.497e+09 & 0.9771 $\pm$  1.0\e{-4} &  12906.4 & 7800.2 \\
           5E-05 &   100 & 2.484e+09 & 0.9772 $\pm$  9.9\e{-5} &  13101.2 & 7764.0 \\
         7.5E-05 &   100 & 2.471e+09 & 0.9771 $\pm$  1.1\e{-4} &  10422.3 & 7706.1 \\
          0.0001 &   100 & 2.437e+09 & 0.9770 $\pm$  1.1\e{-4} &  11902.0 & 7596.6 \\
         0.00025 &   114 & 2.493e+09 & 0.9772 $\pm$  1.0\e{-4} &  11887.8 & 7785.9 \\
          0.0005 &   124 & 2.481e+09 & 0.9771 $\pm$  9.8\e{-5} &  13476.7 & 7766.4 \\
         0.00075 &   133 & 2.505e+09 & 0.9773 $\pm$  8.0\e{-5} &  19901.5 & 7889.1 \\
           0.001 &   140 & 2.491e+09 & 0.9772 $\pm$  8.3\e{-5} &  18802.3 & 7781.8 \\
          0.0025 &   166 & 2.492e+09 & 0.9772 $\pm$  7.9\e{-5} &  20801.4 & 7783.3 \\
           0.005 &   194 & 2.472e+09 & 0.9771 $\pm$  7.1\e{-5} &  25648.4 & 7728.1 \\
          0.0075 &   224 & 2.475e+09 & 0.9772 $\pm$  7.5\e{-5} &  23183.8 & 7746.0 \\
            0.01 &   262 & 2.501e+09 & 0.9772 $\pm$  6.8\e{-5} &  27836.6 & 7840.6 \\
            0.05 &   817 & 2.520e+09 & 0.9771 $\pm$  1.0\e{-4} &  11794.3 & 7871.5 \\
             0.1 &  1173 & 2.483e+09 & 0.9768 $\pm$  1.3\e{-4} &   7982.5 & 7832.0 \\
            0.25 &  1649 & 2.503e+09 & 0.9766 $\pm$  1.5\e{-4} &   6002.6 & 7918.0 \\
             0.5 &  1917 & 2.506e+09 & 0.9766 $\pm$  1.5\e{-4} &   5904.3 & 7921.6 \\
            0.75 &  2037 & 2.515e+09 & 0.9766 $\pm$  1.5\e{-4} &   5354.7 & 7944.6 \\
               1 &  2090 & 2.496e+09 & 0.9767 $\pm$  1.5\e{-4} &   5320.2 & 7953.4 \\
        \bottomrule
    \end{tabular}
    \caption{Second higher order eigenvalue estimates with relaxation.}
    \label{tab:Linear2}
\end{table}
\end{comment}

\section{Summary}
Relaxing the precision to which a linear operator is applied to a vector is an interesting topic.  Bouras and Frayss\`{e} have shown---as well as many others citing their work---that relaxing Arnoldi's method can reduce the computational expense of Arnoldi's method while maintaining the convergence properties.  

We have also seen that relaxation can reduce computational expense for Monte Carlo Arnoldi's method.  This allows us to perform more restarts and calculate more eigenvalue estimates for a relaxed Arnoldi's method in the same amount of time required for a non-relaxed Arnoldi's method.  The problem however is that it can be difficult to determine the optimal value for the relaxation parameter $\eta$.  For the problems shown here, the range of values that gives better results than a non-relaxed Arnoldi calculation is small.  If a poor value of $\eta$ is chosen, it can cause the calculation to be slow or even wrong.

We have seen that for a variety of particles tracked in an iteration, the eigenvalue estimate still diverges and the figure of merit plummets when $\eta$ becomes too large, but with more particles tracked $\eta$ can be larger than with fewer particles tracked without any problems in calculating the eigenvalue estimate.  The figure of merit can be larger when tracking more particles.  Even so the range of good values of $\eta$ is relatively small.  
