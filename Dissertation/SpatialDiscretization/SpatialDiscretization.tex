%!TEX root = ../Thesis.tex
\chapter{Spatial Discretization \label{ch:SpatialDiscretization}}
In \Fref{ch:ArnoldiMethod} necessity of spatially discretizing the fission source in order to take the inner product was introduced.  The inner product is necessary for orthogonalizing and normalizing the Arnoldi vectors.  The discretization causes an error in the eigenvalue estimate if the discretization is too coarse.  
 
The effect of discretization was demonstrated in \Fref{sec:ArnoldiResults} and in \Fref{fig:BasicBias} the error in the eigenvalue estimate is shown as a function of the spatial bin width.   We see from this figure the importance of having a sufficient number of spatial bins to eliminate discretization errors.  Using a large number of spatial bins will remove the error in the eigenvalue estimate associated with discretization, but can increase the computational expense of sampling from and scoring in a fission source, as well as taking the inner product of two fission sources if too many bins are used.

When sampling from a discretized fission source, or tallying in a discretized fission source, the bin to sample/tally must be determined.  The time required to find the appropriate bin is proportional to the number of spatial bins used, so as the number of spatial bins increases, the time required for sampling or tallying increases and the figure of merit decreases.

When choosing a discretization strategy, it is important to find the smallest number of spatial bins that will reduce the eigenvalue error due to spatial discretization smaller than the statistical uncertainty.  We must be careful not to use too many spatial bins or the efficiency will suffer.

In \Fref{fig:BasicBias} we see that the slopes of the linear best fit approximations have values ranging from 1.8 to 1.9.  This indicates there is a nearly quadratic or second-order relationship between the spatial discretization and the error in the eigenvalue estimate.  In this chapter I will demonstrate a higher order accurate approximation to the spatial discretization and show how it reduces the error caused by the discretization of the fission source.  This idea is based on work performed by \citet{Griesheimer:2005Funct-0} on Functional Expansion Tallies.

\section{Second-Order Accurate Approximation---Linear in Space  \label{sec:LinearSpace}}
The spatial approximation used in \Fref{ch:ArnoldiMethod} is a first-order accurate approximation, i.e. constant in space.  The fission source was approximated as
\begin{equation}
    \vP(x) = \sum_{b=1}^B a_b \Px,
    \label{eq:ApproximatedSource}
\end{equation}
where $B$ is the number of spatial bins and 
\begin{equation}
    \Pi_b(x) = \begin{cases}
        \left(\frac{1}{x_{b+1}-x_b}\right)^{1/2}, & x_b \leq x < x_{b+1} \\
        0, & \mathrm{otherwise}.
    \end{cases}
    \label{eq:FirstOrderApproximation}
\end{equation}
The fission source, $\vP(x)$ is represented in Arnoldi's method as a vector of the form
\begin{equation}
    \vP = \left[a_1, a_2, \ldots, a_B\right]^T.
    \label{eq:ConstantSpaceArnoldiVector}
\end{equation}

A spatial discretization of order 2 is not very different from the first-order accurate approximation given in \Fref{eq:ApproximatedSource} and \Fref{eq:FirstOrderApproximation}.  We approximate the fission source as a linear combination of functions
\begin{equation}
    \vL(x) = \sum_{b=1}^B \Lx,
    \label{eq:LinearSource}
\end{equation}
where $B$ is the number of spatial bins.  The fission source in each bin is approximated by a linear function, \Lx, over the range of the bin:
\begin{equation}
    \Lx = \begin{cases}
        \alpha_b + \beta_b x, & x_b \leq x < x_{b+1} \\
        0, & \mathrm{otherwise}.
    \end{cases}
    \label{eq:SecondOrderApproximation}
\end{equation}
In this second-order accurate approximation the term $\beta_b x$ is included which preserves some of the spatial information ignored in a first-order accurate approximation.  Similarly to the first-order accurate approximation, the Arnoldi vector representation of \vL{} is a vector of the form
\begin{equation}
    \vL = \left[\alpha_1, \beta_1, \alpha_2, \beta_2, \ldots, \alpha_n, \beta_B\right]^T.
    \label{eq:LinearSpaceArnoldiVector}
\end{equation}
The inner product between two piecewise linear in space fission sources is defined to be
\begin{equation}
    \langle \vP^{(j)},\vP^{(k)}\rangle = \sum_{b=1}^B \left(\alpha_b^{(j)}\alpha_b^{(k)} + \beta_b^{(j)}\beta_b^{(k)} \right).
    \label{eq:InnerProductLinearSources}
\end{equation}

\subsection{Sampling} \label{sec:SecondOrderSample}
The integral 
\begin{equation}
    q_b = \int \left|\Lx\right| \dd x 
    \label{eq:IntegratedSourceBin}
\end{equation}
represents the rate of fission neutrons generated in the range $\left[x_b, x_{b+1}\right)$ and its magnitude---relative to the integrals over every other bin---is the probability of sampling a neutron from that bin.  We can create a normalized, discrete distribution $p(x) = \left\{p_b\right\}_{b=1}^B$ where $p_b = q_b/Q$ and $Q = \sum_{b=1}^B q_b$ is the total source strength.  We can sample a bin from $p(x)$ as we did for a first-order fission source.  Once a bin has been chosen, the position of the neutron is sampled from within the bin with the distribution function defined as
\begin{equation}
    p_b(x) = \frac{1}{q_b} \left|\Lx\right|
    \label{eq:BinPDF}
\end{equation}
where $q_b$ is from \Fref{eq:IntegratedSourceBin}.  Once the position of the neutron has been sampled we give it a weight just as in \Fref{sec:NegativeSource}
\begin{equation}
    \omega = \begin{cases}
        1 & v(x_s) > 0 \\
        -1 & v(x_s) < 1.
    \end{cases}
    \label{eq:SecondOrderInitialWeight}
\end{equation}

\subsection{Determining the Expansion Coefficients $\alpha$ and $\beta$}
With the fission source being approximated by the function in \Fref{eq:SecondOrderApproximation} we turn our attention to determining the expansion coefficients $\alpha$ and $\beta$.  To do this we must evaluate two integrals, something that is is well suited for Monte Carlo methods.  We first define the midpoint of bin $b$
\begin{equation}
    \xmid = \frac{x_{b+1}+x_b}{2}.
    \label{eq:xmid}
\end{equation}
Taking the zeroth and first spatial moments over the bin
\begin{subequations}\label{eq:SpatialMoments}\begin{align}
        \int_{x_b}^{x_{b+1}} \Lx \dd x  &= \frac{1}{2}\left(x_{b+1} - x_b\right)\left[2\alpha_b + \beta_b\left(x_{b+1} + x_b\right)\right] \label{eq:ZerothMoment} \\[2ex]
        \int_{x_b}^{x_{b+1}} \left(x-\xmid\right)\Lx \dd x &= \frac{\beta_b}{12}\left(x_{b+1} - x_b\right)^3 \label{eq:FirstMoment}
    \end{align}
\end{subequations}
gives two equations for $\alpha_b$ and $\beta_b$.  The left-hand side of \Fref{eq:SpatialMoments} can be evaluated via Monte Carlo
\begin{subequations}\label{eq:SpatialMomentsMC}\begin{align}
    \int_{x_b}^{x_{b+1}} \Lx \dd x &= \frac{1}{N}\sum_{i=1}^N \omega_i  \label{eq:ZerothMomentMC} \\[2ex]
    \int_{x_b}^{x_{b+1}} \left(x_i-\xmid\right)\Lx \dd x &= \frac{1}{N}\sum_{i=1}^N \left(x-\xmid\right)\omega_i \label{eq:FirstMomentMC}
    \end{align}
\end{subequations}
where $N$ is the number of source particles and $\omega_i$ and $x_i$ are the weight and position of the particle that induces fission in bin $b$.  Note that $\omega_i$ can be negative.  By equating \Fref{eq:ZerothMoment} with \Fref{eq:ZerothMomentMC} and \Fref{eq:FirstMoment} with \Fref{eq:FirstMomentMC} we can obtain expressions for $\alpha_b$ and $\beta_b$
\begin{subequations}
    \begin{align}
        \alpha_b &= \frac{1}{x_{b+1}-x_b}\frac{1}{N} \sum_{i=1}^N \omega_i - \frac{\beta_b}{2}\left(x_{b+1}+x_b\right) \label{eq:alpha_b} \\
        \beta_b &= \frac{12}{\left(x_{b+1}-x_b\right)^3}\frac{1}{N} \sum_{i=1}^N \left(x_i-\xmid\right)\omega_i \label{eq:beta_b}
    \end{align}
    \label{eq:ExpansionCoefficients}
\end{subequations}
For Monte Carlo particle transport this means every time a fission is caused in bin $b$ the tallies $\omega_i$ and  $\left(x_i - \xmid\right)\omega_i$ are recorded.  At the end of the iteration, when the sampling and tallying are finished, the fission source is normalized similarly to the constant in space approximated source (\Fref{eq:SourceScaling})
\begin{equation}
    \vL^{(j+1)}(x) = \A \vL^{(j)}(x) \int |\vL^{(j)}(x)| \dd x.
    \label{eq:SecondOrderSourceScaling}
\end{equation}
Notice the difference between \Fref{eq:SecondOrderSourceScaling} and \Fref{eq:SourceScaling} is the term $1/N$ is missing here.  The source is still scaled by the number of source particles in \Fref{eq:alpha_b} and \Fref{eq:beta_b}.

\section{Numerical Results}
To demonstrate the difference between a first and second-order accurate approximation to the fission source a series of simulations, similar to those given in \Fref{sec:DiscretizationBias} has been performed.  In \Fref{sec:DiscretizationBias} the effect of the coarseness of the spatial discretization on the bias of the eigenvalue estimate was shown by performing the same calculation but varying the number of spatial bins.  Here, the same set of calculations is performed,  a second-order accurate, linear in space approximation is used to the fission source.  The results will be compared with the results from \Fref{sec:DiscretizationBias}.

The problem is a 20 mfp thick semi-infinite homogeneous slab of multiplying material with cross sections \mbox{$\nu\Sigma_f = 1.0$}, \mbox{$\Sigma_a = 0.2$}, and \mbox{$\Sigma_s = 0.8$}; \mbox{$\Sigma_t = 1.0$}.  In each iteration $10^5$ histories are tracked, 10 iterations per restart with 50 inactive restarts and 500 active restarts.  The number of spatial bins range from 10 to 150.  Both first and second order approximations start with a uniform source across the entire slab.  First-order accurate results are denoted with a subscript $\Pi$ and second-order accurate results are denoted with a subscript $\Lin$.

On the following pages I present tables showing the numerical results of the bias and uncertainty of the estimated eigenvalues.  Tables \ref{tab:Bias0Histogram} and \ref{tab:Bias0Linear} show the results for the fundamental eigenvalue for the first and second-order accurate approximations, respectively.  In each of these tables is the eigenvalue estimate ($\lambda$), the standard deviation ($\sigma$), the error in the eigenvalue estimate ($\mathcal{B}$) and the figure of merit (FOM).  The error is the absolute value of the difference between the estimated eigenvalue and the published reference values from \cite{Garis:1991One-s-0} and \cite{Dahl:1979Eigen-0}.

\begin{table} \centering
    \subfloat[First-order accurate ($\Pi$) spatial discretization]{%
    \begin{tabular}{cccccc}
        \toprule
        \# Bins & Bin Width (mfp) & $\lambda_{\Pi}$ & $\sigma_{\Pi}$ & $\mathcal{B}_{\Pi}$ & FOM ($\Pi$) \\
        \midrule
         10 & 2.00 & 4.8003 & 6.6\e{-4} & 2.7\e{-2} & 831.2 \\
         25 & 0.80 & 4.8224 & 6.8\e{-4} & 5.3\e{-3} & 773.2 \\
         40 & 0.50 & 4.8251 & 6.3\e{-4} & 2.6\e{-3} & 872.0 \\
         50 & 0.40 & 4.8273 & 6.5\e{-4} & 4.2\e{-4} & 829.0 \\
         60 & 0.33 & 4.8258 & 6.9\e{-4} & 2.0\e{-3} & 704.3 \\
         75 & 0.27 & 4.8275 & 6.7\e{-4} & 2.4\e{-4} & 753.0 \\
         90 & 0.22 & 4.8277 & 6.7\e{-4} & 4.2\e{-5} & 746.1 \\
        105 & 0.19 & 4.8277 & 6.9\e{-4} & 5.2\e{-5} & 698.3 \\
        120 & 0.17 & 4.8282 & 6.5\e{-4} & 4.1\e{-4} & 767.8 \\
        135 & 0.15 & 4.8274 & 7.0\e{-4} & 3.1\e{-4} & 656.9 \\
        150 & 0.13 & 4.8285 & 6.4\e{-4} & 7.5\e{-4} & 792.0 \\
        \bottomrule
    \end{tabular}
    \label{tab:Bias0Histogram}}

    \subfloat[Second-order accurate ($\Lin$) spatial discretization]{%
    \begin{tabular}{cccccc}
        \toprule
        \# Bins & Bin Width (mfp) & $\lambda_{\Lin}$ & $\sigma_{\Lin}$ & $\mathcal{B}_{\Lin}$ & FOM ($\Lin$) \\
        \midrule
         10 & 2.00 & 4.8302 & 1.2\e{-3} & 2.5\e{-3} & 259.5 \\
         25 & 0.80 & 4.8280 & 4.4\e{-4} & 2.1\e{-4} & 1765.7 \\
         40 & 0.50 & 4.8283 & 4.5\e{-4} & 5.7\e{-4} & 1675.6 \\
         50 & 0.40 & 4.8277 & 4.5\e{-4} & 6.2\e{-5} & 1470.6 \\
         60 & 0.33 & 4.8283 & 3.9\e{-4} & 5.4\e{-4} & 2171.6 \\
         75 & 0.27 & 4.8276 & 4.0\e{-4} & 1.5\e{-4} & 2069.3 \\
         90 & 0.22 & 4.8275 & 3.8\e{-4} & 2.0\e{-4} & 2251.3 \\
        105 & 0.19 & 4.8259 & 5.9\e{-4} & 1.8\e{-3} & 877.1 \\
        120 & 0.17 & 4.8172 & 1.3\e{-3} & 1.1\e{-2} & 189.4 \\
        135 & 0.15 & 4.8110 & 2.1\e{-3} & 1.7\e{-2} & 66.2 \\
        150 & 0.13 & 4.8102 & 2.0\e{-3} & 1.8\e{-2} & 73.0 \\        
        \bottomrule
    \end{tabular}
    \label{tab:Bias0Linear}}
    \caption{Error ($\mathcal{B}$) in the fundamental eigenvalue estimate ($\lambda$) for first-order accurate \subref{tab:Bias0Histogram} and second-order accurate \subref{tab:Bias0Linear} discretization as a function of the bin width.  Figure of merit is also given for first and second-order accurate spatial discretizations.  1E5 particles were tracked in each iteration.}
\end{table}

\begin{comment}
\begin{table}[p] \centering
    \begin{tabular}{cccccc}
        \toprule
        \# Bins & Bin Width (mfp) & $\lambda_{\Pi}$ & $\sigma_{\Pi}$ & $\mathcal{B}_{\Pi}$ & FOM ($\Pi$) \\
        \midrule
         10 & 2.00 & 4.2839 & 6.4\e{-4} & 2.7\e{-2} & 831.2 \\
         25 & 0.80 & 4.3652 & 6.5\e{-4} & 5.3\e{-3} & 773.2 \\
         40 & 0.50 & 4.3754 & 6.1\e{-4} & 2.6\e{-3} & 872.0 \\
         50 & 0.40 & 4.3794 & 5.8\e{-4} & 4.2\e{-4} & 829.0 \\
         60 & 0.33 & 4.3796 & 6.3\e{-4} & 2.0\e{-3} & 704.3 \\
         75 & 0.27 & 4.3807 & 6.4\e{-4} & 2.4\e{-4} & 753.0 \\
         90 & 0.22 & 4.3819 & 6.3\e{-4} & 4.2\e{-5} & 746.1 \\
        105 & 0.19 & 4.3805 & 6.3\e{-4} & 5.2\e{-5} & 698.3 \\
        120 & 0.17 & 4.3819 & 6.1\e{-4} & 4.1\e{-4} & 767.8 \\
        135 & 0.15 & 4.3810 & 6.3\e{-4} & 3.1\e{-4} & 656.9 \\
        150 & 0.13 & 4.3827 & 6.3\e{-4} & 7.5\e{-4} & 792.0 \\
        \bottomrule
    \end{tabular}
    \caption{First higher order eigenvalue bias for first-order accurate ($\Pi$) spatial discretization.  }
    \label{tab:Bias1Histogram}
\end{table}
\begin{table}[p] \centering
    \begin{tabular}{cccccc}
        \toprule
        \# Bins & Bin Width (mfp) & $\lambda_{\Lin}$ & $\sigma_{\Lin}$ & $\mathcal{B}_{\Lin}$ & FOM ($\Lin$) \\
        \midrule
         10 & 2.00 & 4.3867 & 2.3\e{-3} & 2.5\e{-3} & 259.5 \\
         25 & 0.80 & 4.3832 & 6.6\e{-4} & 2.1\e{-4} & 1765.7 \\
         40 & 0.50 & 4.3825 & 6.6\e{-4} & 5.7\e{-4} & 1675.6 \\
         50 & 0.40 & 4.3839 & 6.0\e{-4} & 6.2\e{-5} & 1470.6 \\
         60 & 0.33 & 4.3835 & 6.2\e{-4} & 5.4\e{-4} & 2171.6 \\
         75 & 0.27 & 4.3814 & 8.7\e{-4} & 1.5\e{-4} & 2069.3 \\
         90 & 0.22 & 4.3744 & 2.1\e{-3} & 2.0\e{-4} & 2251.3 \\
        105 & 0.19 & 4.3544 & 1.7\e{-3} & 1.8\e{-3} & 877.1 \\
        120 & 0.17 & 4.2712 & 5.2\e{-3} & 1.1\e{-2} & 189.4 \\
        135 & 0.15 & 4.1913 & 1.0\e{-2} & 1.7\e{-2} & 66.2 \\
        150 & 0.13 & 3.9961 & 1.2\e{-2} & 1.8\e{-2} & 73.0 \\        
        \bottomrule
    \end{tabular}
    \caption{First higher order eigenvalue bias for second-order accurate ($\Lin$) spatial discretization.  }
    \label{tab:Bias1Linear}
\end{table}

\begin{table}[p] \centering
    \begin{tabular}{cccccc}
        \toprule
        \# Bins & Bin Width (mfp) & $\lambda_{\Pi}$ & $\sigma_{\Pi}$ & $\mathcal{B}_{\Pi}$ & FOM ($\Pi$) \\
        \midrule
         10 & 2.00 & 3.6321 & 5.4\e{-4} & 2.7\e{-2} & 831.2 \\
         25 & 0.80 & 3.7826 & 6.2\e{-4} & 5.3\e{-3} & 773.2 \\
         40 & 0.50 & 3.8032 & 5.9\e{-4} & 2.6\e{-3} & 872.0 \\
         50 & 0.40 & 3.8091 & 5.8\e{-4} & 4.2\e{-4} & 829.0 \\
         60 & 0.33 & 3.8108 & 5.9\e{-4} & 2.0\e{-3} & 704.3 \\
         75 & 0.27 & 3.8127 & 6.1\e{-4} & 2.4\e{-4} & 753.0 \\
         90 & 0.22 & 3.8151 & 5.4\e{-4} & 4.2\e{-5} & 746.1 \\
        105 & 0.19 & 3.8148 & 5.7\e{-4} & 5.2\e{-5} & 698.3 \\
        120 & 0.17 & 3.8171 & 5.8\e{-4} & 4.1\e{-4} & 767.8 \\
        135 & 0.15 & 3.8167 & 5.6\e{-4} & 3.1\e{-4} & 656.9 \\
        150 & 0.13 & 3.8167 & 6.1\e{-4} & 7.5\e{-4} & 792.0 \\        
        \bottomrule
    \end{tabular}
    \caption{Second higher order eigenvalue bias for first-order accurate ($\Pi$) spatial discretization.  }
    \label{tab:Bias2Histogram}
\end{table}
\begin{table}[p] \centering
    \begin{tabular}{cccccc}
        \toprule
        \# Bins & Bin Width (mfp) & $\lambda_{\Lin}$ & $\sigma_{\Lin}$ & $\mathcal{B}_{\Lin}$ & FOM ($\Lin$) \\
        \midrule
         10 & 2.00 & 3.8330 & 4.3\e{-3} & 2.5\e{-3} & 259.5 \\
         25 & 0.80 & 3.8192 & 1.0\e{-3} & 2.1\e{-4} & 1765.7 \\
         40 & 0.50 & 3.8157 & 1.0\e{-3} & 5.7\e{-4} & 1675.6 \\
         50 & 0.40 & 3.8155 & 1.1\e{-3} & 6.2\e{-5} & 1470.6 \\
         60 & 0.33 & 3.8162 & 1.2\e{-3} & 5.4\e{-4} & 2171.6 \\
         75 & 0.27 & 3.8089 & 2.0\e{-3} & 1.5\e{-4} & 2069.3 \\
         90 & 0.22 & 3.2808 & 4.6\e{-3} & 2.0\e{-4} & 2251.3 \\
        105 & 0.19 & 3.1759 & 4.3\e{-3} & 1.8\e{-3} & 877.1 \\
        120 & 0.17 & 2.8474 & 1.7\e{-2} & 1.1\e{-2} & 189.4 \\
        135 & 0.15 & 1.7544 & 2.0\e{-2} & 1.7\e{-2} & 66.2 \\
        150 & 0.13 & 1.4275 & 3.6\e{-3} & 1.8\e{-2} & 73.0 \\        
        \bottomrule
    \end{tabular}
    \caption{Second higher order eigenvalue bias for second-order accurate ($\Lin$) spatial discretization.  }
    \label{tab:Bias2Linear}
\end{table}
\end{comment}

The purpose of going to a second-order accurate accurate approximation to the fission source is to reduce the error in the eigenvalue estimate associated with discretizing the fission source.  We can see from tables \ref{tab:Bias0Histogram} and \ref{tab:Bias0Linear} that for bin widths between 0.4 and 2.0 mfp thick the error in the eigenvalue estimate from the second-order accurate approximation is an order of magnitude smaller than the error from the first-order accurate approximation.  Thus, moving to a second-order accurate approximation to the fission source can greatly reduce the error.  

The second-order accurate approximation has a smaller statistical uncertainty than the first-order accurate approximation for bin widths between 0.8 and 0.22 mfp and is fairly independent of the bin width in this range.  The uncertainty for the first-order accurate approximation appears to be independent of the size of the bin width over the whole range of bin widths.  The largest and three smallest bin widths from the second-order accurate approximation however have large errors in the eigenvalue estimate and large statistical uncertainties.  For thin bins, the number of neutrons that are born in those bins is small, causing the Monte Carlo noise to dominate the results.  The behavior of the linear-in-space approximation in the 2 mfp wide bins is a mystery.

The figure of merit is 2-3 times larger for the second-order accurate approximation than the first-order for bin widths between 0.8 and 0.22 mfp.  For very small bin widths or for the largest bin width, the figure of merit is smaller for the second-order accurate approximation than the first-order.  The figure of merit as a function of bin width is shown graphically in \Fref{fig:BiasFOM}.
\begin{sidewaysfigure} \centering
    % GNUPLOT: LaTeX picture with Postscript
\begingroup%
\makeatletter%
\newcommand{\GNUPLOTspecial}{%
  \@sanitize\catcode`\%=14\relax\special}%
\setlength{\unitlength}{0.0500bp}%
\begin{picture}(12960,8640)(0,0)%
  {\GNUPLOTspecial{"
%!PS-Adobe-2.0 EPSF-2.0
%%Title: BiasFOM.tex
%%Creator: gnuplot 4.3 patchlevel 0
%%CreationDate: Wed Aug 12 16:28:22 2009
%%DocumentFonts: 
%%BoundingBox: 0 0 648 432
%%EndComments
%%BeginProlog
/gnudict 256 dict def
gnudict begin
%
% The following true/false flags may be edited by hand if desired.
% The unit line width and grayscale image gamma correction may also be changed.
%
/Color true def
/Blacktext true def
/Solid true def
/Dashlength 1 def
/Landscape false def
/Level1 false def
/Rounded false def
/ClipToBoundingBox false def
/TransparentPatterns false def
/gnulinewidth 5.000 def
/userlinewidth gnulinewidth def
/Gamma 1.0 def
%
/vshift -66 def
/dl1 {
  10.0 Dashlength mul mul
  Rounded { currentlinewidth 0.75 mul sub dup 0 le { pop 0.01 } if } if
} def
/dl2 {
  10.0 Dashlength mul mul
  Rounded { currentlinewidth 0.75 mul add } if
} def
/hpt_ 31.5 def
/vpt_ 31.5 def
/hpt hpt_ def
/vpt vpt_ def
Level1 {} {
/SDict 10 dict def
systemdict /pdfmark known not {
  userdict /pdfmark systemdict /cleartomark get put
} if
SDict begin [
  /Title (BiasFOM.tex)
  /Subject (gnuplot plot)
  /Creator (gnuplot 4.3 patchlevel 0)
  /Author (Jeremy Conlin)
%  /Producer (gnuplot)
%  /Keywords ()
  /CreationDate (Wed Aug 12 16:28:22 2009)
  /DOCINFO pdfmark
end
} ifelse
/doclip {
  ClipToBoundingBox {
    newpath 0 0 moveto 648 0 lineto 648 432 lineto 0 432 lineto closepath
    clip
  } if
} def
%
% Gnuplot Prolog Version 4.2 (November 2007)
%
/M {moveto} bind def
/L {lineto} bind def
/R {rmoveto} bind def
/V {rlineto} bind def
/N {newpath moveto} bind def
/Z {closepath} bind def
/C {setrgbcolor} bind def
/f {rlineto fill} bind def
/Gshow {show} def   % May be redefined later in the file to support UTF-8
/vpt2 vpt 2 mul def
/hpt2 hpt 2 mul def
/Lshow {currentpoint stroke M 0 vshift R 
	Blacktext {gsave 0 setgray show grestore} {show} ifelse} def
/Rshow {currentpoint stroke M dup stringwidth pop neg vshift R
	Blacktext {gsave 0 setgray show grestore} {show} ifelse} def
/Cshow {currentpoint stroke M dup stringwidth pop -2 div vshift R 
	Blacktext {gsave 0 setgray show grestore} {show} ifelse} def
/UP {dup vpt_ mul /vpt exch def hpt_ mul /hpt exch def
  /hpt2 hpt 2 mul def /vpt2 vpt 2 mul def} def
/DL {Color {setrgbcolor Solid {pop []} if 0 setdash}
 {pop pop pop 0 setgray Solid {pop []} if 0 setdash} ifelse} def
/BL {stroke userlinewidth 2 mul setlinewidth
	Rounded {1 setlinejoin 1 setlinecap} if} def
/AL {stroke userlinewidth 2 div setlinewidth
	Rounded {1 setlinejoin 1 setlinecap} if} def
/UL {dup gnulinewidth mul /userlinewidth exch def
	dup 1 lt {pop 1} if 10 mul /udl exch def} def
/PL {stroke userlinewidth setlinewidth
	Rounded {1 setlinejoin 1 setlinecap} if} def
% Default Line colors
/LCw {1 1 1} def
/LCb {0 0 0} def
/LCa {0 0 0} def
/LC0 {1 0 0} def
/LC1 {0 1 0} def
/LC2 {0 0 1} def
/LC3 {1 0 1} def
/LC4 {0 1 1} def
/LC5 {1 1 0} def
/LC6 {0 0 0} def
/LC7 {1 0.3 0} def
/LC8 {0.5 0.5 0.5} def
% Default Line Types
/LTw {PL [] 1 setgray} def
/LTb {BL [] LCb DL} def
/LTa {AL [1 udl mul 2 udl mul] 0 setdash LCa setrgbcolor} def
/LT0 {PL [] LC0 DL} def
/LT1 {PL [4 dl1 2 dl2] LC1 DL} def
/LT2 {PL [2 dl1 3 dl2] LC2 DL} def
/LT3 {PL [1 dl1 1.5 dl2] LC3 DL} def
/LT4 {PL [6 dl1 2 dl2 1 dl1 2 dl2] LC4 DL} def
/LT5 {PL [3 dl1 3 dl2 1 dl1 3 dl2] LC5 DL} def
/LT6 {PL [2 dl1 2 dl2 2 dl1 6 dl2] LC6 DL} def
/LT7 {PL [1 dl1 2 dl2 6 dl1 2 dl2 1 dl1 2 dl2] LC7 DL} def
/LT8 {PL [2 dl1 2 dl2 2 dl1 2 dl2 2 dl1 2 dl2 2 dl1 4 dl2] LC8 DL} def
/Pnt {stroke [] 0 setdash gsave 1 setlinecap M 0 0 V stroke grestore} def
/Dia {stroke [] 0 setdash 2 copy vpt add M
  hpt neg vpt neg V hpt vpt neg V
  hpt vpt V hpt neg vpt V closepath stroke
  Pnt} def
/Pls {stroke [] 0 setdash vpt sub M 0 vpt2 V
  currentpoint stroke M
  hpt neg vpt neg R hpt2 0 V stroke
 } def
/Box {stroke [] 0 setdash 2 copy exch hpt sub exch vpt add M
  0 vpt2 neg V hpt2 0 V 0 vpt2 V
  hpt2 neg 0 V closepath stroke
  Pnt} def
/Crs {stroke [] 0 setdash exch hpt sub exch vpt add M
  hpt2 vpt2 neg V currentpoint stroke M
  hpt2 neg 0 R hpt2 vpt2 V stroke} def
/TriU {stroke [] 0 setdash 2 copy vpt 1.12 mul add M
  hpt neg vpt -1.62 mul V
  hpt 2 mul 0 V
  hpt neg vpt 1.62 mul V closepath stroke
  Pnt} def
/Star {2 copy Pls Crs} def
/BoxF {stroke [] 0 setdash exch hpt sub exch vpt add M
  0 vpt2 neg V hpt2 0 V 0 vpt2 V
  hpt2 neg 0 V closepath fill} def
/TriUF {stroke [] 0 setdash vpt 1.12 mul add M
  hpt neg vpt -1.62 mul V
  hpt 2 mul 0 V
  hpt neg vpt 1.62 mul V closepath fill} def
/TriD {stroke [] 0 setdash 2 copy vpt 1.12 mul sub M
  hpt neg vpt 1.62 mul V
  hpt 2 mul 0 V
  hpt neg vpt -1.62 mul V closepath stroke
  Pnt} def
/TriDF {stroke [] 0 setdash vpt 1.12 mul sub M
  hpt neg vpt 1.62 mul V
  hpt 2 mul 0 V
  hpt neg vpt -1.62 mul V closepath fill} def
/DiaF {stroke [] 0 setdash vpt add M
  hpt neg vpt neg V hpt vpt neg V
  hpt vpt V hpt neg vpt V closepath fill} def
/Pent {stroke [] 0 setdash 2 copy gsave
  translate 0 hpt M 4 {72 rotate 0 hpt L} repeat
  closepath stroke grestore Pnt} def
/PentF {stroke [] 0 setdash gsave
  translate 0 hpt M 4 {72 rotate 0 hpt L} repeat
  closepath fill grestore} def
/Circle {stroke [] 0 setdash 2 copy
  hpt 0 360 arc stroke Pnt} def
/CircleF {stroke [] 0 setdash hpt 0 360 arc fill} def
/C0 {BL [] 0 setdash 2 copy moveto vpt 90 450 arc} bind def
/C1 {BL [] 0 setdash 2 copy moveto
	2 copy vpt 0 90 arc closepath fill
	vpt 0 360 arc closepath} bind def
/C2 {BL [] 0 setdash 2 copy moveto
	2 copy vpt 90 180 arc closepath fill
	vpt 0 360 arc closepath} bind def
/C3 {BL [] 0 setdash 2 copy moveto
	2 copy vpt 0 180 arc closepath fill
	vpt 0 360 arc closepath} bind def
/C4 {BL [] 0 setdash 2 copy moveto
	2 copy vpt 180 270 arc closepath fill
	vpt 0 360 arc closepath} bind def
/C5 {BL [] 0 setdash 2 copy moveto
	2 copy vpt 0 90 arc
	2 copy moveto
	2 copy vpt 180 270 arc closepath fill
	vpt 0 360 arc} bind def
/C6 {BL [] 0 setdash 2 copy moveto
	2 copy vpt 90 270 arc closepath fill
	vpt 0 360 arc closepath} bind def
/C7 {BL [] 0 setdash 2 copy moveto
	2 copy vpt 0 270 arc closepath fill
	vpt 0 360 arc closepath} bind def
/C8 {BL [] 0 setdash 2 copy moveto
	2 copy vpt 270 360 arc closepath fill
	vpt 0 360 arc closepath} bind def
/C9 {BL [] 0 setdash 2 copy moveto
	2 copy vpt 270 450 arc closepath fill
	vpt 0 360 arc closepath} bind def
/C10 {BL [] 0 setdash 2 copy 2 copy moveto vpt 270 360 arc closepath fill
	2 copy moveto
	2 copy vpt 90 180 arc closepath fill
	vpt 0 360 arc closepath} bind def
/C11 {BL [] 0 setdash 2 copy moveto
	2 copy vpt 0 180 arc closepath fill
	2 copy moveto
	2 copy vpt 270 360 arc closepath fill
	vpt 0 360 arc closepath} bind def
/C12 {BL [] 0 setdash 2 copy moveto
	2 copy vpt 180 360 arc closepath fill
	vpt 0 360 arc closepath} bind def
/C13 {BL [] 0 setdash 2 copy moveto
	2 copy vpt 0 90 arc closepath fill
	2 copy moveto
	2 copy vpt 180 360 arc closepath fill
	vpt 0 360 arc closepath} bind def
/C14 {BL [] 0 setdash 2 copy moveto
	2 copy vpt 90 360 arc closepath fill
	vpt 0 360 arc} bind def
/C15 {BL [] 0 setdash 2 copy vpt 0 360 arc closepath fill
	vpt 0 360 arc closepath} bind def
/Rec {newpath 4 2 roll moveto 1 index 0 rlineto 0 exch rlineto
	neg 0 rlineto closepath} bind def
/Square {dup Rec} bind def
/Bsquare {vpt sub exch vpt sub exch vpt2 Square} bind def
/S0 {BL [] 0 setdash 2 copy moveto 0 vpt rlineto BL Bsquare} bind def
/S1 {BL [] 0 setdash 2 copy vpt Square fill Bsquare} bind def
/S2 {BL [] 0 setdash 2 copy exch vpt sub exch vpt Square fill Bsquare} bind def
/S3 {BL [] 0 setdash 2 copy exch vpt sub exch vpt2 vpt Rec fill Bsquare} bind def
/S4 {BL [] 0 setdash 2 copy exch vpt sub exch vpt sub vpt Square fill Bsquare} bind def
/S5 {BL [] 0 setdash 2 copy 2 copy vpt Square fill
	exch vpt sub exch vpt sub vpt Square fill Bsquare} bind def
/S6 {BL [] 0 setdash 2 copy exch vpt sub exch vpt sub vpt vpt2 Rec fill Bsquare} bind def
/S7 {BL [] 0 setdash 2 copy exch vpt sub exch vpt sub vpt vpt2 Rec fill
	2 copy vpt Square fill Bsquare} bind def
/S8 {BL [] 0 setdash 2 copy vpt sub vpt Square fill Bsquare} bind def
/S9 {BL [] 0 setdash 2 copy vpt sub vpt vpt2 Rec fill Bsquare} bind def
/S10 {BL [] 0 setdash 2 copy vpt sub vpt Square fill 2 copy exch vpt sub exch vpt Square fill
	Bsquare} bind def
/S11 {BL [] 0 setdash 2 copy vpt sub vpt Square fill 2 copy exch vpt sub exch vpt2 vpt Rec fill
	Bsquare} bind def
/S12 {BL [] 0 setdash 2 copy exch vpt sub exch vpt sub vpt2 vpt Rec fill Bsquare} bind def
/S13 {BL [] 0 setdash 2 copy exch vpt sub exch vpt sub vpt2 vpt Rec fill
	2 copy vpt Square fill Bsquare} bind def
/S14 {BL [] 0 setdash 2 copy exch vpt sub exch vpt sub vpt2 vpt Rec fill
	2 copy exch vpt sub exch vpt Square fill Bsquare} bind def
/S15 {BL [] 0 setdash 2 copy Bsquare fill Bsquare} bind def
/D0 {gsave translate 45 rotate 0 0 S0 stroke grestore} bind def
/D1 {gsave translate 45 rotate 0 0 S1 stroke grestore} bind def
/D2 {gsave translate 45 rotate 0 0 S2 stroke grestore} bind def
/D3 {gsave translate 45 rotate 0 0 S3 stroke grestore} bind def
/D4 {gsave translate 45 rotate 0 0 S4 stroke grestore} bind def
/D5 {gsave translate 45 rotate 0 0 S5 stroke grestore} bind def
/D6 {gsave translate 45 rotate 0 0 S6 stroke grestore} bind def
/D7 {gsave translate 45 rotate 0 0 S7 stroke grestore} bind def
/D8 {gsave translate 45 rotate 0 0 S8 stroke grestore} bind def
/D9 {gsave translate 45 rotate 0 0 S9 stroke grestore} bind def
/D10 {gsave translate 45 rotate 0 0 S10 stroke grestore} bind def
/D11 {gsave translate 45 rotate 0 0 S11 stroke grestore} bind def
/D12 {gsave translate 45 rotate 0 0 S12 stroke grestore} bind def
/D13 {gsave translate 45 rotate 0 0 S13 stroke grestore} bind def
/D14 {gsave translate 45 rotate 0 0 S14 stroke grestore} bind def
/D15 {gsave translate 45 rotate 0 0 S15 stroke grestore} bind def
/DiaE {stroke [] 0 setdash vpt add M
  hpt neg vpt neg V hpt vpt neg V
  hpt vpt V hpt neg vpt V closepath stroke} def
/BoxE {stroke [] 0 setdash exch hpt sub exch vpt add M
  0 vpt2 neg V hpt2 0 V 0 vpt2 V
  hpt2 neg 0 V closepath stroke} def
/TriUE {stroke [] 0 setdash vpt 1.12 mul add M
  hpt neg vpt -1.62 mul V
  hpt 2 mul 0 V
  hpt neg vpt 1.62 mul V closepath stroke} def
/TriDE {stroke [] 0 setdash vpt 1.12 mul sub M
  hpt neg vpt 1.62 mul V
  hpt 2 mul 0 V
  hpt neg vpt -1.62 mul V closepath stroke} def
/PentE {stroke [] 0 setdash gsave
  translate 0 hpt M 4 {72 rotate 0 hpt L} repeat
  closepath stroke grestore} def
/CircE {stroke [] 0 setdash 
  hpt 0 360 arc stroke} def
/Opaque {gsave closepath 1 setgray fill grestore 0 setgray closepath} def
/DiaW {stroke [] 0 setdash vpt add M
  hpt neg vpt neg V hpt vpt neg V
  hpt vpt V hpt neg vpt V Opaque stroke} def
/BoxW {stroke [] 0 setdash exch hpt sub exch vpt add M
  0 vpt2 neg V hpt2 0 V 0 vpt2 V
  hpt2 neg 0 V Opaque stroke} def
/TriUW {stroke [] 0 setdash vpt 1.12 mul add M
  hpt neg vpt -1.62 mul V
  hpt 2 mul 0 V
  hpt neg vpt 1.62 mul V Opaque stroke} def
/TriDW {stroke [] 0 setdash vpt 1.12 mul sub M
  hpt neg vpt 1.62 mul V
  hpt 2 mul 0 V
  hpt neg vpt -1.62 mul V Opaque stroke} def
/PentW {stroke [] 0 setdash gsave
  translate 0 hpt M 4 {72 rotate 0 hpt L} repeat
  Opaque stroke grestore} def
/CircW {stroke [] 0 setdash 
  hpt 0 360 arc Opaque stroke} def
/BoxFill {gsave Rec 1 setgray fill grestore} def
/Density {
  /Fillden exch def
  currentrgbcolor
  /ColB exch def /ColG exch def /ColR exch def
  /ColR ColR Fillden mul Fillden sub 1 add def
  /ColG ColG Fillden mul Fillden sub 1 add def
  /ColB ColB Fillden mul Fillden sub 1 add def
  ColR ColG ColB setrgbcolor} def
/BoxColFill {gsave Rec PolyFill} def
/PolyFill {gsave Density fill grestore grestore} def
/h {rlineto rlineto rlineto gsave closepath fill grestore} bind def
%
% PostScript Level 1 Pattern Fill routine for rectangles
% Usage: x y w h s a XX PatternFill
%	x,y = lower left corner of box to be filled
%	w,h = width and height of box
%	  a = angle in degrees between lines and x-axis
%	 XX = 0/1 for no/yes cross-hatch
%
/PatternFill {gsave /PFa [ 9 2 roll ] def
  PFa 0 get PFa 2 get 2 div add PFa 1 get PFa 3 get 2 div add translate
  PFa 2 get -2 div PFa 3 get -2 div PFa 2 get PFa 3 get Rec
  gsave 1 setgray fill grestore clip
  currentlinewidth 0.5 mul setlinewidth
  /PFs PFa 2 get dup mul PFa 3 get dup mul add sqrt def
  0 0 M PFa 5 get rotate PFs -2 div dup translate
  0 1 PFs PFa 4 get div 1 add floor cvi
	{PFa 4 get mul 0 M 0 PFs V} for
  0 PFa 6 get ne {
	0 1 PFs PFa 4 get div 1 add floor cvi
	{PFa 4 get mul 0 2 1 roll M PFs 0 V} for
 } if
  stroke grestore} def
%
/languagelevel where
 {pop languagelevel} {1} ifelse
 2 lt
	{/InterpretLevel1 true def}
	{/InterpretLevel1 Level1 def}
 ifelse
%
% PostScript level 2 pattern fill definitions
%
/Level2PatternFill {
/Tile8x8 {/PaintType 2 /PatternType 1 /TilingType 1 /BBox [0 0 8 8] /XStep 8 /YStep 8}
	bind def
/KeepColor {currentrgbcolor [/Pattern /DeviceRGB] setcolorspace} bind def
<< Tile8x8
 /PaintProc {0.5 setlinewidth pop 0 0 M 8 8 L 0 8 M 8 0 L stroke} 
>> matrix makepattern
/Pat1 exch def
<< Tile8x8
 /PaintProc {0.5 setlinewidth pop 0 0 M 8 8 L 0 8 M 8 0 L stroke
	0 4 M 4 8 L 8 4 L 4 0 L 0 4 L stroke}
>> matrix makepattern
/Pat2 exch def
<< Tile8x8
 /PaintProc {0.5 setlinewidth pop 0 0 M 0 8 L
	8 8 L 8 0 L 0 0 L fill}
>> matrix makepattern
/Pat3 exch def
<< Tile8x8
 /PaintProc {0.5 setlinewidth pop -4 8 M 8 -4 L
	0 12 M 12 0 L stroke}
>> matrix makepattern
/Pat4 exch def
<< Tile8x8
 /PaintProc {0.5 setlinewidth pop -4 0 M 8 12 L
	0 -4 M 12 8 L stroke}
>> matrix makepattern
/Pat5 exch def
<< Tile8x8
 /PaintProc {0.5 setlinewidth pop -2 8 M 4 -4 L
	0 12 M 8 -4 L 4 12 M 10 0 L stroke}
>> matrix makepattern
/Pat6 exch def
<< Tile8x8
 /PaintProc {0.5 setlinewidth pop -2 0 M 4 12 L
	0 -4 M 8 12 L 4 -4 M 10 8 L stroke}
>> matrix makepattern
/Pat7 exch def
<< Tile8x8
 /PaintProc {0.5 setlinewidth pop 8 -2 M -4 4 L
	12 0 M -4 8 L 12 4 M 0 10 L stroke}
>> matrix makepattern
/Pat8 exch def
<< Tile8x8
 /PaintProc {0.5 setlinewidth pop 0 -2 M 12 4 L
	-4 0 M 12 8 L -4 4 M 8 10 L stroke}
>> matrix makepattern
/Pat9 exch def
/Pattern1 {PatternBgnd KeepColor Pat1 setpattern} bind def
/Pattern2 {PatternBgnd KeepColor Pat2 setpattern} bind def
/Pattern3 {PatternBgnd KeepColor Pat3 setpattern} bind def
/Pattern4 {PatternBgnd KeepColor Landscape {Pat5} {Pat4} ifelse setpattern} bind def
/Pattern5 {PatternBgnd KeepColor Landscape {Pat4} {Pat5} ifelse setpattern} bind def
/Pattern6 {PatternBgnd KeepColor Landscape {Pat9} {Pat6} ifelse setpattern} bind def
/Pattern7 {PatternBgnd KeepColor Landscape {Pat8} {Pat7} ifelse setpattern} bind def
} def
%
%
%End of PostScript Level 2 code
%
/PatternBgnd {
  TransparentPatterns {} {gsave 1 setgray fill grestore} ifelse
} def
%
% Substitute for Level 2 pattern fill codes with
% grayscale if Level 2 support is not selected.
%
/Level1PatternFill {
/Pattern1 {0.250 Density} bind def
/Pattern2 {0.500 Density} bind def
/Pattern3 {0.750 Density} bind def
/Pattern4 {0.125 Density} bind def
/Pattern5 {0.375 Density} bind def
/Pattern6 {0.625 Density} bind def
/Pattern7 {0.875 Density} bind def
} def
%
% Now test for support of Level 2 code
%
Level1 {Level1PatternFill} {Level2PatternFill} ifelse
%
/Symbol-Oblique /Symbol findfont [1 0 .167 1 0 0] makefont
dup length dict begin {1 index /FID eq {pop pop} {def} ifelse} forall
currentdict end definefont pop
end
%%EndProlog
gnudict begin
gsave
doclip
0 0 translate
0.050 0.050 scale
0 setgray
newpath
1.000 UL
LTb
1220 640 M
63 0 V
11276 0 R
-63 0 V
1220 2192 M
63 0 V
11276 0 R
-63 0 V
1220 3744 M
63 0 V
11276 0 R
-63 0 V
1220 5295 M
63 0 V
11276 0 R
-63 0 V
1220 6847 M
63 0 V
11276 0 R
-63 0 V
1220 8399 M
63 0 V
11276 0 R
-63 0 V
1220 640 M
0 63 V
0 7696 R
0 -63 V
3797 640 M
0 63 V
0 7696 R
0 -63 V
6374 640 M
0 63 V
0 7696 R
0 -63 V
8951 640 M
0 63 V
0 7696 R
0 -63 V
11528 640 M
0 63 V
0 7696 R
0 -63 V
stroke
1220 8399 N
0 -7759 V
11339 0 V
0 7759 V
-11339 0 V
Z stroke
LCb setrgbcolor
LTb
LCb setrgbcolor
LTb
LCb setrgbcolor
LTb
LCb setrgbcolor
LTb
1.500 UP
1.000 UL
LTb
1.000 UL
LTb
11265 7736 N
0 600 V
1174 0 V
0 -600 V
-1174 0 V
Z stroke
11265 8336 M
1174 0 V
1.500 UP
stroke
LT0
LCb setrgbcolor
LT0
11745 8136 M
574 0 V
11528 3220 M
5343 3040 L
3797 3346 L
3282 3213 L
2938 2826 L
-344 151 V
-229 -21 V
2202 2807 L
-123 216 V
-95 -344 V
-77 419 V
11528 3220 Pls
5343 3040 Pls
3797 3346 Pls
3282 3213 Pls
2938 2826 Pls
2594 2977 Pls
2365 2956 Pls
2202 2807 Pls
2079 3023 Pls
1984 2679 Pls
1907 3098 Pls
12032 8136 Pls
1.500 UP
1.000 UL
LT2
LCb setrgbcolor
LT2
11745 7936 M
574 0 V
11528 1445 M
5343 6120 L
3797 5840 L
3282 5204 L
2938 7380 L
2594 7062 L
-229 565 V
2202 3362 L
2079 1228 L
1984 846 L
-77 20 V
11528 1445 Star
5343 6120 Star
3797 5840 Star
3282 5204 Star
2938 7380 Star
2594 7062 Star
2365 7627 Star
2202 3362 Star
2079 1228 Star
1984 846 Star
1907 866 Star
12032 7936 Star
1.000 UL
LTb
1220 8399 N
0 -7759 V
11339 0 V
0 7759 V
-11339 0 V
Z stroke
1.500 UP
1.000 UL
LTb
stroke
grestore
end
showpage
  }}%
  \put(11625,7936){\makebox(0,0)[r]{\strut{}$\Lin$}}%
  \put(11625,8136){\makebox(0,0)[r]{\strut{}$\Pi$}}%
  \put(6889,140){\makebox(0,0){\strut{}Bin Width (mfp)}}%
  \put(280,4519){%
  \special{ps: gsave currentpoint currentpoint translate
630 rotate neg exch neg exch translate}%
  \makebox(0,0){\strut{}FOM $1/\sigma^2T$}%
  \special{ps: currentpoint grestore moveto}%
  }%
  \put(11528,440){\makebox(0,0){\strut{} 2}}%
  \put(8951,440){\makebox(0,0){\strut{} 1.5}}%
  \put(6374,440){\makebox(0,0){\strut{} 1}}%
  \put(3797,440){\makebox(0,0){\strut{} 0.5}}%
  \put(1220,440){\makebox(0,0){\strut{} 0}}%
  \put(1100,8399){\makebox(0,0)[r]{\strut{} 2500}}%
  \put(1100,6847){\makebox(0,0)[r]{\strut{} 2000}}%
  \put(1100,5295){\makebox(0,0)[r]{\strut{} 1500}}%
  \put(1100,3744){\makebox(0,0)[r]{\strut{} 1000}}%
  \put(1100,2192){\makebox(0,0)[r]{\strut{} 500}}%
  \put(1100,640){\makebox(0,0)[r]{\strut{} 0}}%
\end{picture}%
\endgroup
\endinput

    \caption{Figure of merit as a function of bin width for a slab of width 20 mfp and tracking 1E5 particles per iteration.  Included are results from a first-order ($\Pi$) and second-order ($\Lin$) approximation to the fission source.}
    \label{fig:BiasFOM}
\end{sidewaysfigure}

In the first-order accurate approximation, the shape of the eigenvectors were limited by the flat approximation in each bin.  Since the second-order accurate approximation is a linear approximation in each bin, we expect the second-order accurate approximation to have an improved eigenvector than the first-order accurate approximation.  In \Fref{fig:LinearVectors} the eigenvectors from the second-order accurate approximation with 60 spatial bins are shown.  In \Fref{fig:SpatialDiscretizationVectors}, the same eigenvectors are plotted along with the eigenvectors from the first-order accurate approximation to show they have the same shape.  From these figures we see that the second-order accurate approximation is a smooth approximation to the fission source and appears continuous across the bin boundaries.
\begin{sidewaysfigure} \centering
    % GNUPLOT: LaTeX picture with Postscript
\begingroup%
\makeatletter%
\newcommand{\GNUPLOTspecial}{%
  \@sanitize\catcode`\%=14\relax\special}%
\setlength{\unitlength}{0.0500bp}%
\begin{picture}(12960,8640)(0,0)%
  {\GNUPLOTspecial{"
%!PS-Adobe-2.0 EPSF-2.0
%%Title: LinearVectors.tex
%%Creator: gnuplot 4.3 patchlevel 0
%%CreationDate: Wed Aug 12 16:28:22 2009
%%DocumentFonts: 
%%BoundingBox: 0 0 648 432
%%EndComments
%%BeginProlog
/gnudict 256 dict def
gnudict begin
%
% The following true/false flags may be edited by hand if desired.
% The unit line width and grayscale image gamma correction may also be changed.
%
/Color true def
/Blacktext true def
/Solid true def
/Dashlength 1 def
/Landscape false def
/Level1 false def
/Rounded false def
/ClipToBoundingBox false def
/TransparentPatterns false def
/gnulinewidth 5.000 def
/userlinewidth gnulinewidth def
/Gamma 1.0 def
%
/vshift -66 def
/dl1 {
  10.0 Dashlength mul mul
  Rounded { currentlinewidth 0.75 mul sub dup 0 le { pop 0.01 } if } if
} def
/dl2 {
  10.0 Dashlength mul mul
  Rounded { currentlinewidth 0.75 mul add } if
} def
/hpt_ 31.5 def
/vpt_ 31.5 def
/hpt hpt_ def
/vpt vpt_ def
Level1 {} {
/SDict 10 dict def
systemdict /pdfmark known not {
  userdict /pdfmark systemdict /cleartomark get put
} if
SDict begin [
  /Title (LinearVectors.tex)
  /Subject (gnuplot plot)
  /Creator (gnuplot 4.3 patchlevel 0)
  /Author (Jeremy Conlin)
%  /Producer (gnuplot)
%  /Keywords ()
  /CreationDate (Wed Aug 12 16:28:22 2009)
  /DOCINFO pdfmark
end
} ifelse
/doclip {
  ClipToBoundingBox {
    newpath 0 0 moveto 648 0 lineto 648 432 lineto 0 432 lineto closepath
    clip
  } if
} def
%
% Gnuplot Prolog Version 4.2 (November 2007)
%
/M {moveto} bind def
/L {lineto} bind def
/R {rmoveto} bind def
/V {rlineto} bind def
/N {newpath moveto} bind def
/Z {closepath} bind def
/C {setrgbcolor} bind def
/f {rlineto fill} bind def
/Gshow {show} def   % May be redefined later in the file to support UTF-8
/vpt2 vpt 2 mul def
/hpt2 hpt 2 mul def
/Lshow {currentpoint stroke M 0 vshift R 
	Blacktext {gsave 0 setgray show grestore} {show} ifelse} def
/Rshow {currentpoint stroke M dup stringwidth pop neg vshift R
	Blacktext {gsave 0 setgray show grestore} {show} ifelse} def
/Cshow {currentpoint stroke M dup stringwidth pop -2 div vshift R 
	Blacktext {gsave 0 setgray show grestore} {show} ifelse} def
/UP {dup vpt_ mul /vpt exch def hpt_ mul /hpt exch def
  /hpt2 hpt 2 mul def /vpt2 vpt 2 mul def} def
/DL {Color {setrgbcolor Solid {pop []} if 0 setdash}
 {pop pop pop 0 setgray Solid {pop []} if 0 setdash} ifelse} def
/BL {stroke userlinewidth 2 mul setlinewidth
	Rounded {1 setlinejoin 1 setlinecap} if} def
/AL {stroke userlinewidth 2 div setlinewidth
	Rounded {1 setlinejoin 1 setlinecap} if} def
/UL {dup gnulinewidth mul /userlinewidth exch def
	dup 1 lt {pop 1} if 10 mul /udl exch def} def
/PL {stroke userlinewidth setlinewidth
	Rounded {1 setlinejoin 1 setlinecap} if} def
% Default Line colors
/LCw {1 1 1} def
/LCb {0 0 0} def
/LCa {0 0 0} def
/LC0 {1 0 0} def
/LC1 {0 1 0} def
/LC2 {0 0 1} def
/LC3 {1 0 1} def
/LC4 {0 1 1} def
/LC5 {1 1 0} def
/LC6 {0 0 0} def
/LC7 {1 0.3 0} def
/LC8 {0.5 0.5 0.5} def
% Default Line Types
/LTw {PL [] 1 setgray} def
/LTb {BL [] LCb DL} def
/LTa {AL [1 udl mul 2 udl mul] 0 setdash LCa setrgbcolor} def
/LT0 {PL [] LC0 DL} def
/LT1 {PL [4 dl1 2 dl2] LC1 DL} def
/LT2 {PL [2 dl1 3 dl2] LC2 DL} def
/LT3 {PL [1 dl1 1.5 dl2] LC3 DL} def
/LT4 {PL [6 dl1 2 dl2 1 dl1 2 dl2] LC4 DL} def
/LT5 {PL [3 dl1 3 dl2 1 dl1 3 dl2] LC5 DL} def
/LT6 {PL [2 dl1 2 dl2 2 dl1 6 dl2] LC6 DL} def
/LT7 {PL [1 dl1 2 dl2 6 dl1 2 dl2 1 dl1 2 dl2] LC7 DL} def
/LT8 {PL [2 dl1 2 dl2 2 dl1 2 dl2 2 dl1 2 dl2 2 dl1 4 dl2] LC8 DL} def
/Pnt {stroke [] 0 setdash gsave 1 setlinecap M 0 0 V stroke grestore} def
/Dia {stroke [] 0 setdash 2 copy vpt add M
  hpt neg vpt neg V hpt vpt neg V
  hpt vpt V hpt neg vpt V closepath stroke
  Pnt} def
/Pls {stroke [] 0 setdash vpt sub M 0 vpt2 V
  currentpoint stroke M
  hpt neg vpt neg R hpt2 0 V stroke
 } def
/Box {stroke [] 0 setdash 2 copy exch hpt sub exch vpt add M
  0 vpt2 neg V hpt2 0 V 0 vpt2 V
  hpt2 neg 0 V closepath stroke
  Pnt} def
/Crs {stroke [] 0 setdash exch hpt sub exch vpt add M
  hpt2 vpt2 neg V currentpoint stroke M
  hpt2 neg 0 R hpt2 vpt2 V stroke} def
/TriU {stroke [] 0 setdash 2 copy vpt 1.12 mul add M
  hpt neg vpt -1.62 mul V
  hpt 2 mul 0 V
  hpt neg vpt 1.62 mul V closepath stroke
  Pnt} def
/Star {2 copy Pls Crs} def
/BoxF {stroke [] 0 setdash exch hpt sub exch vpt add M
  0 vpt2 neg V hpt2 0 V 0 vpt2 V
  hpt2 neg 0 V closepath fill} def
/TriUF {stroke [] 0 setdash vpt 1.12 mul add M
  hpt neg vpt -1.62 mul V
  hpt 2 mul 0 V
  hpt neg vpt 1.62 mul V closepath fill} def
/TriD {stroke [] 0 setdash 2 copy vpt 1.12 mul sub M
  hpt neg vpt 1.62 mul V
  hpt 2 mul 0 V
  hpt neg vpt -1.62 mul V closepath stroke
  Pnt} def
/TriDF {stroke [] 0 setdash vpt 1.12 mul sub M
  hpt neg vpt 1.62 mul V
  hpt 2 mul 0 V
  hpt neg vpt -1.62 mul V closepath fill} def
/DiaF {stroke [] 0 setdash vpt add M
  hpt neg vpt neg V hpt vpt neg V
  hpt vpt V hpt neg vpt V closepath fill} def
/Pent {stroke [] 0 setdash 2 copy gsave
  translate 0 hpt M 4 {72 rotate 0 hpt L} repeat
  closepath stroke grestore Pnt} def
/PentF {stroke [] 0 setdash gsave
  translate 0 hpt M 4 {72 rotate 0 hpt L} repeat
  closepath fill grestore} def
/Circle {stroke [] 0 setdash 2 copy
  hpt 0 360 arc stroke Pnt} def
/CircleF {stroke [] 0 setdash hpt 0 360 arc fill} def
/C0 {BL [] 0 setdash 2 copy moveto vpt 90 450 arc} bind def
/C1 {BL [] 0 setdash 2 copy moveto
	2 copy vpt 0 90 arc closepath fill
	vpt 0 360 arc closepath} bind def
/C2 {BL [] 0 setdash 2 copy moveto
	2 copy vpt 90 180 arc closepath fill
	vpt 0 360 arc closepath} bind def
/C3 {BL [] 0 setdash 2 copy moveto
	2 copy vpt 0 180 arc closepath fill
	vpt 0 360 arc closepath} bind def
/C4 {BL [] 0 setdash 2 copy moveto
	2 copy vpt 180 270 arc closepath fill
	vpt 0 360 arc closepath} bind def
/C5 {BL [] 0 setdash 2 copy moveto
	2 copy vpt 0 90 arc
	2 copy moveto
	2 copy vpt 180 270 arc closepath fill
	vpt 0 360 arc} bind def
/C6 {BL [] 0 setdash 2 copy moveto
	2 copy vpt 90 270 arc closepath fill
	vpt 0 360 arc closepath} bind def
/C7 {BL [] 0 setdash 2 copy moveto
	2 copy vpt 0 270 arc closepath fill
	vpt 0 360 arc closepath} bind def
/C8 {BL [] 0 setdash 2 copy moveto
	2 copy vpt 270 360 arc closepath fill
	vpt 0 360 arc closepath} bind def
/C9 {BL [] 0 setdash 2 copy moveto
	2 copy vpt 270 450 arc closepath fill
	vpt 0 360 arc closepath} bind def
/C10 {BL [] 0 setdash 2 copy 2 copy moveto vpt 270 360 arc closepath fill
	2 copy moveto
	2 copy vpt 90 180 arc closepath fill
	vpt 0 360 arc closepath} bind def
/C11 {BL [] 0 setdash 2 copy moveto
	2 copy vpt 0 180 arc closepath fill
	2 copy moveto
	2 copy vpt 270 360 arc closepath fill
	vpt 0 360 arc closepath} bind def
/C12 {BL [] 0 setdash 2 copy moveto
	2 copy vpt 180 360 arc closepath fill
	vpt 0 360 arc closepath} bind def
/C13 {BL [] 0 setdash 2 copy moveto
	2 copy vpt 0 90 arc closepath fill
	2 copy moveto
	2 copy vpt 180 360 arc closepath fill
	vpt 0 360 arc closepath} bind def
/C14 {BL [] 0 setdash 2 copy moveto
	2 copy vpt 90 360 arc closepath fill
	vpt 0 360 arc} bind def
/C15 {BL [] 0 setdash 2 copy vpt 0 360 arc closepath fill
	vpt 0 360 arc closepath} bind def
/Rec {newpath 4 2 roll moveto 1 index 0 rlineto 0 exch rlineto
	neg 0 rlineto closepath} bind def
/Square {dup Rec} bind def
/Bsquare {vpt sub exch vpt sub exch vpt2 Square} bind def
/S0 {BL [] 0 setdash 2 copy moveto 0 vpt rlineto BL Bsquare} bind def
/S1 {BL [] 0 setdash 2 copy vpt Square fill Bsquare} bind def
/S2 {BL [] 0 setdash 2 copy exch vpt sub exch vpt Square fill Bsquare} bind def
/S3 {BL [] 0 setdash 2 copy exch vpt sub exch vpt2 vpt Rec fill Bsquare} bind def
/S4 {BL [] 0 setdash 2 copy exch vpt sub exch vpt sub vpt Square fill Bsquare} bind def
/S5 {BL [] 0 setdash 2 copy 2 copy vpt Square fill
	exch vpt sub exch vpt sub vpt Square fill Bsquare} bind def
/S6 {BL [] 0 setdash 2 copy exch vpt sub exch vpt sub vpt vpt2 Rec fill Bsquare} bind def
/S7 {BL [] 0 setdash 2 copy exch vpt sub exch vpt sub vpt vpt2 Rec fill
	2 copy vpt Square fill Bsquare} bind def
/S8 {BL [] 0 setdash 2 copy vpt sub vpt Square fill Bsquare} bind def
/S9 {BL [] 0 setdash 2 copy vpt sub vpt vpt2 Rec fill Bsquare} bind def
/S10 {BL [] 0 setdash 2 copy vpt sub vpt Square fill 2 copy exch vpt sub exch vpt Square fill
	Bsquare} bind def
/S11 {BL [] 0 setdash 2 copy vpt sub vpt Square fill 2 copy exch vpt sub exch vpt2 vpt Rec fill
	Bsquare} bind def
/S12 {BL [] 0 setdash 2 copy exch vpt sub exch vpt sub vpt2 vpt Rec fill Bsquare} bind def
/S13 {BL [] 0 setdash 2 copy exch vpt sub exch vpt sub vpt2 vpt Rec fill
	2 copy vpt Square fill Bsquare} bind def
/S14 {BL [] 0 setdash 2 copy exch vpt sub exch vpt sub vpt2 vpt Rec fill
	2 copy exch vpt sub exch vpt Square fill Bsquare} bind def
/S15 {BL [] 0 setdash 2 copy Bsquare fill Bsquare} bind def
/D0 {gsave translate 45 rotate 0 0 S0 stroke grestore} bind def
/D1 {gsave translate 45 rotate 0 0 S1 stroke grestore} bind def
/D2 {gsave translate 45 rotate 0 0 S2 stroke grestore} bind def
/D3 {gsave translate 45 rotate 0 0 S3 stroke grestore} bind def
/D4 {gsave translate 45 rotate 0 0 S4 stroke grestore} bind def
/D5 {gsave translate 45 rotate 0 0 S5 stroke grestore} bind def
/D6 {gsave translate 45 rotate 0 0 S6 stroke grestore} bind def
/D7 {gsave translate 45 rotate 0 0 S7 stroke grestore} bind def
/D8 {gsave translate 45 rotate 0 0 S8 stroke grestore} bind def
/D9 {gsave translate 45 rotate 0 0 S9 stroke grestore} bind def
/D10 {gsave translate 45 rotate 0 0 S10 stroke grestore} bind def
/D11 {gsave translate 45 rotate 0 0 S11 stroke grestore} bind def
/D12 {gsave translate 45 rotate 0 0 S12 stroke grestore} bind def
/D13 {gsave translate 45 rotate 0 0 S13 stroke grestore} bind def
/D14 {gsave translate 45 rotate 0 0 S14 stroke grestore} bind def
/D15 {gsave translate 45 rotate 0 0 S15 stroke grestore} bind def
/DiaE {stroke [] 0 setdash vpt add M
  hpt neg vpt neg V hpt vpt neg V
  hpt vpt V hpt neg vpt V closepath stroke} def
/BoxE {stroke [] 0 setdash exch hpt sub exch vpt add M
  0 vpt2 neg V hpt2 0 V 0 vpt2 V
  hpt2 neg 0 V closepath stroke} def
/TriUE {stroke [] 0 setdash vpt 1.12 mul add M
  hpt neg vpt -1.62 mul V
  hpt 2 mul 0 V
  hpt neg vpt 1.62 mul V closepath stroke} def
/TriDE {stroke [] 0 setdash vpt 1.12 mul sub M
  hpt neg vpt 1.62 mul V
  hpt 2 mul 0 V
  hpt neg vpt -1.62 mul V closepath stroke} def
/PentE {stroke [] 0 setdash gsave
  translate 0 hpt M 4 {72 rotate 0 hpt L} repeat
  closepath stroke grestore} def
/CircE {stroke [] 0 setdash 
  hpt 0 360 arc stroke} def
/Opaque {gsave closepath 1 setgray fill grestore 0 setgray closepath} def
/DiaW {stroke [] 0 setdash vpt add M
  hpt neg vpt neg V hpt vpt neg V
  hpt vpt V hpt neg vpt V Opaque stroke} def
/BoxW {stroke [] 0 setdash exch hpt sub exch vpt add M
  0 vpt2 neg V hpt2 0 V 0 vpt2 V
  hpt2 neg 0 V Opaque stroke} def
/TriUW {stroke [] 0 setdash vpt 1.12 mul add M
  hpt neg vpt -1.62 mul V
  hpt 2 mul 0 V
  hpt neg vpt 1.62 mul V Opaque stroke} def
/TriDW {stroke [] 0 setdash vpt 1.12 mul sub M
  hpt neg vpt 1.62 mul V
  hpt 2 mul 0 V
  hpt neg vpt -1.62 mul V Opaque stroke} def
/PentW {stroke [] 0 setdash gsave
  translate 0 hpt M 4 {72 rotate 0 hpt L} repeat
  Opaque stroke grestore} def
/CircW {stroke [] 0 setdash 
  hpt 0 360 arc Opaque stroke} def
/BoxFill {gsave Rec 1 setgray fill grestore} def
/Density {
  /Fillden exch def
  currentrgbcolor
  /ColB exch def /ColG exch def /ColR exch def
  /ColR ColR Fillden mul Fillden sub 1 add def
  /ColG ColG Fillden mul Fillden sub 1 add def
  /ColB ColB Fillden mul Fillden sub 1 add def
  ColR ColG ColB setrgbcolor} def
/BoxColFill {gsave Rec PolyFill} def
/PolyFill {gsave Density fill grestore grestore} def
/h {rlineto rlineto rlineto gsave closepath fill grestore} bind def
%
% PostScript Level 1 Pattern Fill routine for rectangles
% Usage: x y w h s a XX PatternFill
%	x,y = lower left corner of box to be filled
%	w,h = width and height of box
%	  a = angle in degrees between lines and x-axis
%	 XX = 0/1 for no/yes cross-hatch
%
/PatternFill {gsave /PFa [ 9 2 roll ] def
  PFa 0 get PFa 2 get 2 div add PFa 1 get PFa 3 get 2 div add translate
  PFa 2 get -2 div PFa 3 get -2 div PFa 2 get PFa 3 get Rec
  gsave 1 setgray fill grestore clip
  currentlinewidth 0.5 mul setlinewidth
  /PFs PFa 2 get dup mul PFa 3 get dup mul add sqrt def
  0 0 M PFa 5 get rotate PFs -2 div dup translate
  0 1 PFs PFa 4 get div 1 add floor cvi
	{PFa 4 get mul 0 M 0 PFs V} for
  0 PFa 6 get ne {
	0 1 PFs PFa 4 get div 1 add floor cvi
	{PFa 4 get mul 0 2 1 roll M PFs 0 V} for
 } if
  stroke grestore} def
%
/languagelevel where
 {pop languagelevel} {1} ifelse
 2 lt
	{/InterpretLevel1 true def}
	{/InterpretLevel1 Level1 def}
 ifelse
%
% PostScript level 2 pattern fill definitions
%
/Level2PatternFill {
/Tile8x8 {/PaintType 2 /PatternType 1 /TilingType 1 /BBox [0 0 8 8] /XStep 8 /YStep 8}
	bind def
/KeepColor {currentrgbcolor [/Pattern /DeviceRGB] setcolorspace} bind def
<< Tile8x8
 /PaintProc {0.5 setlinewidth pop 0 0 M 8 8 L 0 8 M 8 0 L stroke} 
>> matrix makepattern
/Pat1 exch def
<< Tile8x8
 /PaintProc {0.5 setlinewidth pop 0 0 M 8 8 L 0 8 M 8 0 L stroke
	0 4 M 4 8 L 8 4 L 4 0 L 0 4 L stroke}
>> matrix makepattern
/Pat2 exch def
<< Tile8x8
 /PaintProc {0.5 setlinewidth pop 0 0 M 0 8 L
	8 8 L 8 0 L 0 0 L fill}
>> matrix makepattern
/Pat3 exch def
<< Tile8x8
 /PaintProc {0.5 setlinewidth pop -4 8 M 8 -4 L
	0 12 M 12 0 L stroke}
>> matrix makepattern
/Pat4 exch def
<< Tile8x8
 /PaintProc {0.5 setlinewidth pop -4 0 M 8 12 L
	0 -4 M 12 8 L stroke}
>> matrix makepattern
/Pat5 exch def
<< Tile8x8
 /PaintProc {0.5 setlinewidth pop -2 8 M 4 -4 L
	0 12 M 8 -4 L 4 12 M 10 0 L stroke}
>> matrix makepattern
/Pat6 exch def
<< Tile8x8
 /PaintProc {0.5 setlinewidth pop -2 0 M 4 12 L
	0 -4 M 8 12 L 4 -4 M 10 8 L stroke}
>> matrix makepattern
/Pat7 exch def
<< Tile8x8
 /PaintProc {0.5 setlinewidth pop 8 -2 M -4 4 L
	12 0 M -4 8 L 12 4 M 0 10 L stroke}
>> matrix makepattern
/Pat8 exch def
<< Tile8x8
 /PaintProc {0.5 setlinewidth pop 0 -2 M 12 4 L
	-4 0 M 12 8 L -4 4 M 8 10 L stroke}
>> matrix makepattern
/Pat9 exch def
/Pattern1 {PatternBgnd KeepColor Pat1 setpattern} bind def
/Pattern2 {PatternBgnd KeepColor Pat2 setpattern} bind def
/Pattern3 {PatternBgnd KeepColor Pat3 setpattern} bind def
/Pattern4 {PatternBgnd KeepColor Landscape {Pat5} {Pat4} ifelse setpattern} bind def
/Pattern5 {PatternBgnd KeepColor Landscape {Pat4} {Pat5} ifelse setpattern} bind def
/Pattern6 {PatternBgnd KeepColor Landscape {Pat9} {Pat6} ifelse setpattern} bind def
/Pattern7 {PatternBgnd KeepColor Landscape {Pat8} {Pat7} ifelse setpattern} bind def
} def
%
%
%End of PostScript Level 2 code
%
/PatternBgnd {
  TransparentPatterns {} {gsave 1 setgray fill grestore} ifelse
} def
%
% Substitute for Level 2 pattern fill codes with
% grayscale if Level 2 support is not selected.
%
/Level1PatternFill {
/Pattern1 {0.250 Density} bind def
/Pattern2 {0.500 Density} bind def
/Pattern3 {0.750 Density} bind def
/Pattern4 {0.125 Density} bind def
/Pattern5 {0.375 Density} bind def
/Pattern6 {0.625 Density} bind def
/Pattern7 {0.875 Density} bind def
} def
%
% Now test for support of Level 2 code
%
Level1 {Level1PatternFill} {Level2PatternFill} ifelse
%
/Symbol-Oblique /Symbol findfont [1 0 .167 1 0 0] makefont
dup length dict begin {1 index /FID eq {pop pop} {def} ifelse} forall
currentdict end definefont pop
end
%%EndProlog
gnudict begin
gsave
doclip
0 0 translate
0.050 0.050 scale
0 setgray
newpath
1.000 UL
LTb
1100 640 M
63 0 V
11396 0 R
-63 0 V
1100 1553 M
63 0 V
11396 0 R
-63 0 V
1100 2466 M
63 0 V
11396 0 R
-63 0 V
1100 3378 M
63 0 V
11396 0 R
-63 0 V
1100 4291 M
63 0 V
11396 0 R
-63 0 V
1100 5204 M
63 0 V
11396 0 R
-63 0 V
1100 6117 M
63 0 V
11396 0 R
-63 0 V
1100 7030 M
63 0 V
11396 0 R
-63 0 V
1100 7943 M
63 0 V
11396 0 R
-63 0 V
1100 640 M
0 63 V
0 7696 R
0 -63 V
3965 640 M
0 63 V
0 7696 R
0 -63 V
6830 640 M
0 63 V
0 7696 R
0 -63 V
9694 640 M
0 63 V
0 7696 R
0 -63 V
12559 640 M
0 63 V
0 7696 R
0 -63 V
stroke
1100 8399 N
0 -7759 V
11459 0 V
0 7759 V
-11459 0 V
Z stroke
LCb setrgbcolor
LTb
LCb setrgbcolor
LTb
LCb setrgbcolor
LTb
LCb setrgbcolor
LTb
1.000 UP
1.000 UL
LTb
1.000 UL
LTb
11296 703 N
0 800 V
1143 0 V
0 -800 V
-1143 0 V
Z stroke
11296 1503 M
1143 0 V
stroke
LT0
LCb setrgbcolor
LT0
11776 1303 M
543 0 V
1100 3779 M
191 269 V
0 -5 V
191 236 V
0 -4 V
191 228 V
0 -1 V
191 217 V
0 1 V
191 214 V
0 -2 V
191 217 V
0 -5 V
191 205 V
0 1 V
191 198 V
0 1 V
191 195 V
0 -4 V
191 191 V
191 178 V
0 4 V
191 171 V
0 -2 V
191 175 V
0 4 V
191 156 V
0 -1 V
191 154 V
0 1 V
191 141 V
0 4 V
191 132 V
0 2 V
191 131 V
0 1 V
191 115 V
0 4 V
191 103 V
0 5 V
191 94 V
0 3 V
191 88 V
0 3 V
191 77 V
191 72 V
0 -2 V
191 59 V
0 2 V
191 49 V
0 -1 V
191 37 V
191 31 V
0 5 V
191 10 V
0 2 V
191 5 V
0 -4 V
190 -1 V
0 -2 V
192 -14 V
0 4 V
190 -30 V
191 -38 V
192 -47 V
0 -3 V
190 -55 V
0 -1 V
191 -71 V
0 -2 V
192 -83 V
0 5 V
190 -92 V
0 -1 V
191 -100 V
192 -106 V
0 -1 V
190 -122 V
0 -1 V
191 -125 V
0 -1 V
191 -134 V
191 -145 V
0 1 V
191 -156 V
0 2 V
191 -160 V
0 -3 V
191 -166 V
0 1 V
191 -177 V
191 -183 V
191 -187 V
0 -1 V
191 -195 V
191 -198 V
0 -1 V
191 -205 V
0 1 V
191 -209 V
191 -215 V
0 -2 V
191 -213 V
stroke 11986 4507 M
0 -3 V
191 -226 V
0 2 V
191 -235 V
0 4 V
191 -272 V
stroke
LC1 setrgbcolor
LCb setrgbcolor
LT0
LC1 setrgbcolor
11776 1103 M
543 0 V
1100 2990 M
191 -267 V
0 4 V
191 -223 V
191 -209 V
0 2 V
191 -195 V
0 2 V
191 -183 V
0 -1 V
191 -165 V
0 2 V
191 -151 V
0 -1 V
191 -132 V
0 -1 V
191 -110 V
0 -5 V
191 -91 V
0 -4 V
191 -73 V
0 -1 V
191 -52 V
0 -2 V
191 -33 V
0 2 V
191 -11 V
191 13 V
191 34 V
191 59 V
0 -2 V
191 81 V
0 -3 V
191 93 V
0 2 V
191 111 V
0 5 V
191 128 V
0 6 V
191 147 V
0 2 V
191 165 V
0 -1 V
191 183 V
0 -2 V
191 196 V
0 -2 V
191 201 V
0 1 V
191 208 V
0 4 V
191 214 V
0 7 V
191 212 V
191 222 V
0 3 V
190 219 V
0 1 V
192 224 V
0 -2 V
190 218 V
191 211 V
0 -1 V
192 202 V
0 3 V
190 185 V
0 5 V
191 180 V
0 -1 V
192 166 V
0 -1 V
190 151 V
0 -1 V
191 133 V
0 1 V
192 116 V
0 -1 V
190 93 V
191 78 V
191 53 V
0 1 V
191 34 V
0 2 V
191 8 V
0 5 V
191 -11 V
0 -3 V
191 -30 V
0 3 V
191 -57 V
0 2 V
191 -76 V
0 -1 V
191 -92 V
0 -2 V
191 -112 V
0 -3 V
191 -129 V
0 -1 V
191 -148 V
191 -165 V
0 1 V
191 -181 V
stroke 11795 4656 M
0 2 V
191 -196 V
0 -3 V
191 -205 V
191 -224 V
0 5 V
191 -267 V
stroke
LT0
LC2 setrgbcolor
LCb setrgbcolor
LT0
LC2 setrgbcolor
11776 903 M
543 0 V
1100 3935 M
191 381 V
0 -7 V
191 309 V
0 -2 V
191 270 V
0 2 V
191 228 V
0 5 V
191 190 V
0 2 V
191 153 V
0 1 V
191 108 V
0 -5 V
191 60 V
0 -2 V
191 17 V
0 -10 V
191 -38 V
0 -2 V
191 -90 V
0 3 V
191 -135 V
0 2 V
191 -174 V
0 -3 V
191 -219 V
0 3 V
191 -251 V
0 2 V
191 -280 V
0 -10 V
191 -289 V
0 1 V
191 -325 V
0 -1 V
191 -330 V
0 -4 V
191 -334 V
0 1 V
191 -329 V
0 6 V
191 -318 V
0 -1 V
191 -298 V
0 3 V
191 -274 V
0 -3 V
191 -233 V
0 1 V
191 -218 V
191 -159 V
0 -5 V
191 -121 V
0 8 V
191 -86 V
0 3 V
191 -33 V
0 4 V
190 18 V
0 1 V
192 78 V
0 -11 V
190 126 V
0 1 V
191 166 V
0 -8 V
192 211 V
0 5 V
190 227 V
0 2 V
191 281 V
0 -3 V
192 298 V
0 1 V
190 316 V
0 3 V
191 324 V
0 8 V
192 329 V
0 3 V
190 332 V
191 321 V
0 3 V
191 308 V
0 -2 V
191 285 V
0 4 V
191 251 V
0 4 V
191 214 V
0 4 V
191 177 V
0 7 V
191 128 V
0 5 V
191 90 V
0 1 V
191 44 V
0 -2 V
191 -2 V
0 -2 V
stroke 11031 5655 M
191 -53 V
0 -4 V
191 -102 V
0 -4 V
191 -148 V
0 2 V
191 -198 V
0 -1 V
191 -234 V
0 -4 V
191 -274 V
0 2 V
191 -314 V
0 7 V
191 -386 V
stroke
LTb
1100 8399 N
0 -7759 V
11459 0 V
0 7759 V
-11459 0 V
Z stroke
1.000 UP
1.000 UL
LTb
stroke
grestore
end
showpage
  }}%
  \put(11656,903){\makebox(0,0)[r]{\strut{}$\lambda_2$}}%
  \put(11656,1103){\makebox(0,0)[r]{\strut{}$\lambda_1$}}%
  \put(11656,1303){\makebox(0,0)[r]{\strut{}$\lambda_0$}}%
  \put(6829,140){\makebox(0,0){\strut{}Slab width (mfp)}}%
  \put(280,4519){%
  \special{ps: gsave currentpoint currentpoint translate
630 rotate neg exch neg exch translate}%
  \makebox(0,0){\strut{}Eigenvector}%
  \special{ps: currentpoint grestore moveto}%
  }%
  \put(12559,440){\makebox(0,0){\strut{} 20}}%
  \put(9694,440){\makebox(0,0){\strut{} 15}}%
  \put(6830,440){\makebox(0,0){\strut{} 10}}%
  \put(3965,440){\makebox(0,0){\strut{} 5}}%
  \put(1100,440){\makebox(0,0){\strut{} 0}}%
  \put(980,7943){\makebox(0,0)[r]{\strut{} 1}}%
  \put(980,7030){\makebox(0,0)[r]{\strut{} 0.8}}%
  \put(980,6117){\makebox(0,0)[r]{\strut{} 0.6}}%
  \put(980,5204){\makebox(0,0)[r]{\strut{} 0.4}}%
  \put(980,4291){\makebox(0,0)[r]{\strut{} 0.2}}%
  \put(980,3378){\makebox(0,0)[r]{\strut{} 0}}%
  \put(980,2466){\makebox(0,0)[r]{\strut{}-0.2}}%
  \put(980,1553){\makebox(0,0)[r]{\strut{}-0.4}}%
  \put(980,640){\makebox(0,0)[r]{\strut{}-0.6}}%
\end{picture}%
\endgroup
\endinput

    \caption{Estimates of the fundamental and first two harmonic eigenvectors, tracking 1E5 particles per iteration and, using 60 spatial bins with a second-order accurate approximation to the fission source.}
    \label{fig:LinearVectors}
\end{sidewaysfigure}

\begin{sidewaysfigure} \centering
    % GNUPLOT: LaTeX picture with Postscript
\begingroup%
\makeatletter%
\newcommand{\GNUPLOTspecial}{%
  \@sanitize\catcode`\%=14\relax\special}%
\setlength{\unitlength}{0.0500bp}%
\begin{picture}(12960,8640)(0,0)%
  {\GNUPLOTspecial{"
%!PS-Adobe-2.0 EPSF-2.0
%%Title: Vectors.tex
%%Creator: gnuplot 4.3 patchlevel 0
%%CreationDate: Wed Aug 12 16:28:22 2009
%%DocumentFonts: 
%%BoundingBox: 0 0 648 432
%%EndComments
%%BeginProlog
/gnudict 256 dict def
gnudict begin
%
% The following true/false flags may be edited by hand if desired.
% The unit line width and grayscale image gamma correction may also be changed.
%
/Color true def
/Blacktext true def
/Solid true def
/Dashlength 1 def
/Landscape false def
/Level1 false def
/Rounded false def
/ClipToBoundingBox false def
/TransparentPatterns false def
/gnulinewidth 5.000 def
/userlinewidth gnulinewidth def
/Gamma 1.0 def
%
/vshift -66 def
/dl1 {
  10.0 Dashlength mul mul
  Rounded { currentlinewidth 0.75 mul sub dup 0 le { pop 0.01 } if } if
} def
/dl2 {
  10.0 Dashlength mul mul
  Rounded { currentlinewidth 0.75 mul add } if
} def
/hpt_ 31.5 def
/vpt_ 31.5 def
/hpt hpt_ def
/vpt vpt_ def
Level1 {} {
/SDict 10 dict def
systemdict /pdfmark known not {
  userdict /pdfmark systemdict /cleartomark get put
} if
SDict begin [
  /Title (Vectors.tex)
  /Subject (gnuplot plot)
  /Creator (gnuplot 4.3 patchlevel 0)
  /Author (Jeremy Conlin)
%  /Producer (gnuplot)
%  /Keywords ()
  /CreationDate (Wed Aug 12 16:28:22 2009)
  /DOCINFO pdfmark
end
} ifelse
/doclip {
  ClipToBoundingBox {
    newpath 0 0 moveto 648 0 lineto 648 432 lineto 0 432 lineto closepath
    clip
  } if
} def
%
% Gnuplot Prolog Version 4.2 (November 2007)
%
/M {moveto} bind def
/L {lineto} bind def
/R {rmoveto} bind def
/V {rlineto} bind def
/N {newpath moveto} bind def
/Z {closepath} bind def
/C {setrgbcolor} bind def
/f {rlineto fill} bind def
/Gshow {show} def   % May be redefined later in the file to support UTF-8
/vpt2 vpt 2 mul def
/hpt2 hpt 2 mul def
/Lshow {currentpoint stroke M 0 vshift R 
	Blacktext {gsave 0 setgray show grestore} {show} ifelse} def
/Rshow {currentpoint stroke M dup stringwidth pop neg vshift R
	Blacktext {gsave 0 setgray show grestore} {show} ifelse} def
/Cshow {currentpoint stroke M dup stringwidth pop -2 div vshift R 
	Blacktext {gsave 0 setgray show grestore} {show} ifelse} def
/UP {dup vpt_ mul /vpt exch def hpt_ mul /hpt exch def
  /hpt2 hpt 2 mul def /vpt2 vpt 2 mul def} def
/DL {Color {setrgbcolor Solid {pop []} if 0 setdash}
 {pop pop pop 0 setgray Solid {pop []} if 0 setdash} ifelse} def
/BL {stroke userlinewidth 2 mul setlinewidth
	Rounded {1 setlinejoin 1 setlinecap} if} def
/AL {stroke userlinewidth 2 div setlinewidth
	Rounded {1 setlinejoin 1 setlinecap} if} def
/UL {dup gnulinewidth mul /userlinewidth exch def
	dup 1 lt {pop 1} if 10 mul /udl exch def} def
/PL {stroke userlinewidth setlinewidth
	Rounded {1 setlinejoin 1 setlinecap} if} def
% Default Line colors
/LCw {1 1 1} def
/LCb {0 0 0} def
/LCa {0 0 0} def
/LC0 {1 0 0} def
/LC1 {0 1 0} def
/LC2 {0 0 1} def
/LC3 {1 0 1} def
/LC4 {0 1 1} def
/LC5 {1 1 0} def
/LC6 {0 0 0} def
/LC7 {1 0.3 0} def
/LC8 {0.5 0.5 0.5} def
% Default Line Types
/LTw {PL [] 1 setgray} def
/LTb {BL [] LCb DL} def
/LTa {AL [1 udl mul 2 udl mul] 0 setdash LCa setrgbcolor} def
/LT0 {PL [] LC0 DL} def
/LT1 {PL [4 dl1 2 dl2] LC1 DL} def
/LT2 {PL [2 dl1 3 dl2] LC2 DL} def
/LT3 {PL [1 dl1 1.5 dl2] LC3 DL} def
/LT4 {PL [6 dl1 2 dl2 1 dl1 2 dl2] LC4 DL} def
/LT5 {PL [3 dl1 3 dl2 1 dl1 3 dl2] LC5 DL} def
/LT6 {PL [2 dl1 2 dl2 2 dl1 6 dl2] LC6 DL} def
/LT7 {PL [1 dl1 2 dl2 6 dl1 2 dl2 1 dl1 2 dl2] LC7 DL} def
/LT8 {PL [2 dl1 2 dl2 2 dl1 2 dl2 2 dl1 2 dl2 2 dl1 4 dl2] LC8 DL} def
/Pnt {stroke [] 0 setdash gsave 1 setlinecap M 0 0 V stroke grestore} def
/Dia {stroke [] 0 setdash 2 copy vpt add M
  hpt neg vpt neg V hpt vpt neg V
  hpt vpt V hpt neg vpt V closepath stroke
  Pnt} def
/Pls {stroke [] 0 setdash vpt sub M 0 vpt2 V
  currentpoint stroke M
  hpt neg vpt neg R hpt2 0 V stroke
 } def
/Box {stroke [] 0 setdash 2 copy exch hpt sub exch vpt add M
  0 vpt2 neg V hpt2 0 V 0 vpt2 V
  hpt2 neg 0 V closepath stroke
  Pnt} def
/Crs {stroke [] 0 setdash exch hpt sub exch vpt add M
  hpt2 vpt2 neg V currentpoint stroke M
  hpt2 neg 0 R hpt2 vpt2 V stroke} def
/TriU {stroke [] 0 setdash 2 copy vpt 1.12 mul add M
  hpt neg vpt -1.62 mul V
  hpt 2 mul 0 V
  hpt neg vpt 1.62 mul V closepath stroke
  Pnt} def
/Star {2 copy Pls Crs} def
/BoxF {stroke [] 0 setdash exch hpt sub exch vpt add M
  0 vpt2 neg V hpt2 0 V 0 vpt2 V
  hpt2 neg 0 V closepath fill} def
/TriUF {stroke [] 0 setdash vpt 1.12 mul add M
  hpt neg vpt -1.62 mul V
  hpt 2 mul 0 V
  hpt neg vpt 1.62 mul V closepath fill} def
/TriD {stroke [] 0 setdash 2 copy vpt 1.12 mul sub M
  hpt neg vpt 1.62 mul V
  hpt 2 mul 0 V
  hpt neg vpt -1.62 mul V closepath stroke
  Pnt} def
/TriDF {stroke [] 0 setdash vpt 1.12 mul sub M
  hpt neg vpt 1.62 mul V
  hpt 2 mul 0 V
  hpt neg vpt -1.62 mul V closepath fill} def
/DiaF {stroke [] 0 setdash vpt add M
  hpt neg vpt neg V hpt vpt neg V
  hpt vpt V hpt neg vpt V closepath fill} def
/Pent {stroke [] 0 setdash 2 copy gsave
  translate 0 hpt M 4 {72 rotate 0 hpt L} repeat
  closepath stroke grestore Pnt} def
/PentF {stroke [] 0 setdash gsave
  translate 0 hpt M 4 {72 rotate 0 hpt L} repeat
  closepath fill grestore} def
/Circle {stroke [] 0 setdash 2 copy
  hpt 0 360 arc stroke Pnt} def
/CircleF {stroke [] 0 setdash hpt 0 360 arc fill} def
/C0 {BL [] 0 setdash 2 copy moveto vpt 90 450 arc} bind def
/C1 {BL [] 0 setdash 2 copy moveto
	2 copy vpt 0 90 arc closepath fill
	vpt 0 360 arc closepath} bind def
/C2 {BL [] 0 setdash 2 copy moveto
	2 copy vpt 90 180 arc closepath fill
	vpt 0 360 arc closepath} bind def
/C3 {BL [] 0 setdash 2 copy moveto
	2 copy vpt 0 180 arc closepath fill
	vpt 0 360 arc closepath} bind def
/C4 {BL [] 0 setdash 2 copy moveto
	2 copy vpt 180 270 arc closepath fill
	vpt 0 360 arc closepath} bind def
/C5 {BL [] 0 setdash 2 copy moveto
	2 copy vpt 0 90 arc
	2 copy moveto
	2 copy vpt 180 270 arc closepath fill
	vpt 0 360 arc} bind def
/C6 {BL [] 0 setdash 2 copy moveto
	2 copy vpt 90 270 arc closepath fill
	vpt 0 360 arc closepath} bind def
/C7 {BL [] 0 setdash 2 copy moveto
	2 copy vpt 0 270 arc closepath fill
	vpt 0 360 arc closepath} bind def
/C8 {BL [] 0 setdash 2 copy moveto
	2 copy vpt 270 360 arc closepath fill
	vpt 0 360 arc closepath} bind def
/C9 {BL [] 0 setdash 2 copy moveto
	2 copy vpt 270 450 arc closepath fill
	vpt 0 360 arc closepath} bind def
/C10 {BL [] 0 setdash 2 copy 2 copy moveto vpt 270 360 arc closepath fill
	2 copy moveto
	2 copy vpt 90 180 arc closepath fill
	vpt 0 360 arc closepath} bind def
/C11 {BL [] 0 setdash 2 copy moveto
	2 copy vpt 0 180 arc closepath fill
	2 copy moveto
	2 copy vpt 270 360 arc closepath fill
	vpt 0 360 arc closepath} bind def
/C12 {BL [] 0 setdash 2 copy moveto
	2 copy vpt 180 360 arc closepath fill
	vpt 0 360 arc closepath} bind def
/C13 {BL [] 0 setdash 2 copy moveto
	2 copy vpt 0 90 arc closepath fill
	2 copy moveto
	2 copy vpt 180 360 arc closepath fill
	vpt 0 360 arc closepath} bind def
/C14 {BL [] 0 setdash 2 copy moveto
	2 copy vpt 90 360 arc closepath fill
	vpt 0 360 arc} bind def
/C15 {BL [] 0 setdash 2 copy vpt 0 360 arc closepath fill
	vpt 0 360 arc closepath} bind def
/Rec {newpath 4 2 roll moveto 1 index 0 rlineto 0 exch rlineto
	neg 0 rlineto closepath} bind def
/Square {dup Rec} bind def
/Bsquare {vpt sub exch vpt sub exch vpt2 Square} bind def
/S0 {BL [] 0 setdash 2 copy moveto 0 vpt rlineto BL Bsquare} bind def
/S1 {BL [] 0 setdash 2 copy vpt Square fill Bsquare} bind def
/S2 {BL [] 0 setdash 2 copy exch vpt sub exch vpt Square fill Bsquare} bind def
/S3 {BL [] 0 setdash 2 copy exch vpt sub exch vpt2 vpt Rec fill Bsquare} bind def
/S4 {BL [] 0 setdash 2 copy exch vpt sub exch vpt sub vpt Square fill Bsquare} bind def
/S5 {BL [] 0 setdash 2 copy 2 copy vpt Square fill
	exch vpt sub exch vpt sub vpt Square fill Bsquare} bind def
/S6 {BL [] 0 setdash 2 copy exch vpt sub exch vpt sub vpt vpt2 Rec fill Bsquare} bind def
/S7 {BL [] 0 setdash 2 copy exch vpt sub exch vpt sub vpt vpt2 Rec fill
	2 copy vpt Square fill Bsquare} bind def
/S8 {BL [] 0 setdash 2 copy vpt sub vpt Square fill Bsquare} bind def
/S9 {BL [] 0 setdash 2 copy vpt sub vpt vpt2 Rec fill Bsquare} bind def
/S10 {BL [] 0 setdash 2 copy vpt sub vpt Square fill 2 copy exch vpt sub exch vpt Square fill
	Bsquare} bind def
/S11 {BL [] 0 setdash 2 copy vpt sub vpt Square fill 2 copy exch vpt sub exch vpt2 vpt Rec fill
	Bsquare} bind def
/S12 {BL [] 0 setdash 2 copy exch vpt sub exch vpt sub vpt2 vpt Rec fill Bsquare} bind def
/S13 {BL [] 0 setdash 2 copy exch vpt sub exch vpt sub vpt2 vpt Rec fill
	2 copy vpt Square fill Bsquare} bind def
/S14 {BL [] 0 setdash 2 copy exch vpt sub exch vpt sub vpt2 vpt Rec fill
	2 copy exch vpt sub exch vpt Square fill Bsquare} bind def
/S15 {BL [] 0 setdash 2 copy Bsquare fill Bsquare} bind def
/D0 {gsave translate 45 rotate 0 0 S0 stroke grestore} bind def
/D1 {gsave translate 45 rotate 0 0 S1 stroke grestore} bind def
/D2 {gsave translate 45 rotate 0 0 S2 stroke grestore} bind def
/D3 {gsave translate 45 rotate 0 0 S3 stroke grestore} bind def
/D4 {gsave translate 45 rotate 0 0 S4 stroke grestore} bind def
/D5 {gsave translate 45 rotate 0 0 S5 stroke grestore} bind def
/D6 {gsave translate 45 rotate 0 0 S6 stroke grestore} bind def
/D7 {gsave translate 45 rotate 0 0 S7 stroke grestore} bind def
/D8 {gsave translate 45 rotate 0 0 S8 stroke grestore} bind def
/D9 {gsave translate 45 rotate 0 0 S9 stroke grestore} bind def
/D10 {gsave translate 45 rotate 0 0 S10 stroke grestore} bind def
/D11 {gsave translate 45 rotate 0 0 S11 stroke grestore} bind def
/D12 {gsave translate 45 rotate 0 0 S12 stroke grestore} bind def
/D13 {gsave translate 45 rotate 0 0 S13 stroke grestore} bind def
/D14 {gsave translate 45 rotate 0 0 S14 stroke grestore} bind def
/D15 {gsave translate 45 rotate 0 0 S15 stroke grestore} bind def
/DiaE {stroke [] 0 setdash vpt add M
  hpt neg vpt neg V hpt vpt neg V
  hpt vpt V hpt neg vpt V closepath stroke} def
/BoxE {stroke [] 0 setdash exch hpt sub exch vpt add M
  0 vpt2 neg V hpt2 0 V 0 vpt2 V
  hpt2 neg 0 V closepath stroke} def
/TriUE {stroke [] 0 setdash vpt 1.12 mul add M
  hpt neg vpt -1.62 mul V
  hpt 2 mul 0 V
  hpt neg vpt 1.62 mul V closepath stroke} def
/TriDE {stroke [] 0 setdash vpt 1.12 mul sub M
  hpt neg vpt 1.62 mul V
  hpt 2 mul 0 V
  hpt neg vpt -1.62 mul V closepath stroke} def
/PentE {stroke [] 0 setdash gsave
  translate 0 hpt M 4 {72 rotate 0 hpt L} repeat
  closepath stroke grestore} def
/CircE {stroke [] 0 setdash 
  hpt 0 360 arc stroke} def
/Opaque {gsave closepath 1 setgray fill grestore 0 setgray closepath} def
/DiaW {stroke [] 0 setdash vpt add M
  hpt neg vpt neg V hpt vpt neg V
  hpt vpt V hpt neg vpt V Opaque stroke} def
/BoxW {stroke [] 0 setdash exch hpt sub exch vpt add M
  0 vpt2 neg V hpt2 0 V 0 vpt2 V
  hpt2 neg 0 V Opaque stroke} def
/TriUW {stroke [] 0 setdash vpt 1.12 mul add M
  hpt neg vpt -1.62 mul V
  hpt 2 mul 0 V
  hpt neg vpt 1.62 mul V Opaque stroke} def
/TriDW {stroke [] 0 setdash vpt 1.12 mul sub M
  hpt neg vpt 1.62 mul V
  hpt 2 mul 0 V
  hpt neg vpt -1.62 mul V Opaque stroke} def
/PentW {stroke [] 0 setdash gsave
  translate 0 hpt M 4 {72 rotate 0 hpt L} repeat
  Opaque stroke grestore} def
/CircW {stroke [] 0 setdash 
  hpt 0 360 arc Opaque stroke} def
/BoxFill {gsave Rec 1 setgray fill grestore} def
/Density {
  /Fillden exch def
  currentrgbcolor
  /ColB exch def /ColG exch def /ColR exch def
  /ColR ColR Fillden mul Fillden sub 1 add def
  /ColG ColG Fillden mul Fillden sub 1 add def
  /ColB ColB Fillden mul Fillden sub 1 add def
  ColR ColG ColB setrgbcolor} def
/BoxColFill {gsave Rec PolyFill} def
/PolyFill {gsave Density fill grestore grestore} def
/h {rlineto rlineto rlineto gsave closepath fill grestore} bind def
%
% PostScript Level 1 Pattern Fill routine for rectangles
% Usage: x y w h s a XX PatternFill
%	x,y = lower left corner of box to be filled
%	w,h = width and height of box
%	  a = angle in degrees between lines and x-axis
%	 XX = 0/1 for no/yes cross-hatch
%
/PatternFill {gsave /PFa [ 9 2 roll ] def
  PFa 0 get PFa 2 get 2 div add PFa 1 get PFa 3 get 2 div add translate
  PFa 2 get -2 div PFa 3 get -2 div PFa 2 get PFa 3 get Rec
  gsave 1 setgray fill grestore clip
  currentlinewidth 0.5 mul setlinewidth
  /PFs PFa 2 get dup mul PFa 3 get dup mul add sqrt def
  0 0 M PFa 5 get rotate PFs -2 div dup translate
  0 1 PFs PFa 4 get div 1 add floor cvi
	{PFa 4 get mul 0 M 0 PFs V} for
  0 PFa 6 get ne {
	0 1 PFs PFa 4 get div 1 add floor cvi
	{PFa 4 get mul 0 2 1 roll M PFs 0 V} for
 } if
  stroke grestore} def
%
/languagelevel where
 {pop languagelevel} {1} ifelse
 2 lt
	{/InterpretLevel1 true def}
	{/InterpretLevel1 Level1 def}
 ifelse
%
% PostScript level 2 pattern fill definitions
%
/Level2PatternFill {
/Tile8x8 {/PaintType 2 /PatternType 1 /TilingType 1 /BBox [0 0 8 8] /XStep 8 /YStep 8}
	bind def
/KeepColor {currentrgbcolor [/Pattern /DeviceRGB] setcolorspace} bind def
<< Tile8x8
 /PaintProc {0.5 setlinewidth pop 0 0 M 8 8 L 0 8 M 8 0 L stroke} 
>> matrix makepattern
/Pat1 exch def
<< Tile8x8
 /PaintProc {0.5 setlinewidth pop 0 0 M 8 8 L 0 8 M 8 0 L stroke
	0 4 M 4 8 L 8 4 L 4 0 L 0 4 L stroke}
>> matrix makepattern
/Pat2 exch def
<< Tile8x8
 /PaintProc {0.5 setlinewidth pop 0 0 M 0 8 L
	8 8 L 8 0 L 0 0 L fill}
>> matrix makepattern
/Pat3 exch def
<< Tile8x8
 /PaintProc {0.5 setlinewidth pop -4 8 M 8 -4 L
	0 12 M 12 0 L stroke}
>> matrix makepattern
/Pat4 exch def
<< Tile8x8
 /PaintProc {0.5 setlinewidth pop -4 0 M 8 12 L
	0 -4 M 12 8 L stroke}
>> matrix makepattern
/Pat5 exch def
<< Tile8x8
 /PaintProc {0.5 setlinewidth pop -2 8 M 4 -4 L
	0 12 M 8 -4 L 4 12 M 10 0 L stroke}
>> matrix makepattern
/Pat6 exch def
<< Tile8x8
 /PaintProc {0.5 setlinewidth pop -2 0 M 4 12 L
	0 -4 M 8 12 L 4 -4 M 10 8 L stroke}
>> matrix makepattern
/Pat7 exch def
<< Tile8x8
 /PaintProc {0.5 setlinewidth pop 8 -2 M -4 4 L
	12 0 M -4 8 L 12 4 M 0 10 L stroke}
>> matrix makepattern
/Pat8 exch def
<< Tile8x8
 /PaintProc {0.5 setlinewidth pop 0 -2 M 12 4 L
	-4 0 M 12 8 L -4 4 M 8 10 L stroke}
>> matrix makepattern
/Pat9 exch def
/Pattern1 {PatternBgnd KeepColor Pat1 setpattern} bind def
/Pattern2 {PatternBgnd KeepColor Pat2 setpattern} bind def
/Pattern3 {PatternBgnd KeepColor Pat3 setpattern} bind def
/Pattern4 {PatternBgnd KeepColor Landscape {Pat5} {Pat4} ifelse setpattern} bind def
/Pattern5 {PatternBgnd KeepColor Landscape {Pat4} {Pat5} ifelse setpattern} bind def
/Pattern6 {PatternBgnd KeepColor Landscape {Pat9} {Pat6} ifelse setpattern} bind def
/Pattern7 {PatternBgnd KeepColor Landscape {Pat8} {Pat7} ifelse setpattern} bind def
} def
%
%
%End of PostScript Level 2 code
%
/PatternBgnd {
  TransparentPatterns {} {gsave 1 setgray fill grestore} ifelse
} def
%
% Substitute for Level 2 pattern fill codes with
% grayscale if Level 2 support is not selected.
%
/Level1PatternFill {
/Pattern1 {0.250 Density} bind def
/Pattern2 {0.500 Density} bind def
/Pattern3 {0.750 Density} bind def
/Pattern4 {0.125 Density} bind def
/Pattern5 {0.375 Density} bind def
/Pattern6 {0.625 Density} bind def
/Pattern7 {0.875 Density} bind def
} def
%
% Now test for support of Level 2 code
%
Level1 {Level1PatternFill} {Level2PatternFill} ifelse
%
/Symbol-Oblique /Symbol findfont [1 0 .167 1 0 0] makefont
dup length dict begin {1 index /FID eq {pop pop} {def} ifelse} forall
currentdict end definefont pop
end
%%EndProlog
gnudict begin
gsave
doclip
0 0 translate
0.050 0.050 scale
0 setgray
newpath
1.000 UL
LTb
1100 640 M
63 0 V
11396 0 R
-63 0 V
1100 1553 M
63 0 V
11396 0 R
-63 0 V
1100 2466 M
63 0 V
11396 0 R
-63 0 V
1100 3378 M
63 0 V
11396 0 R
-63 0 V
1100 4291 M
63 0 V
11396 0 R
-63 0 V
1100 5204 M
63 0 V
11396 0 R
-63 0 V
1100 6117 M
63 0 V
11396 0 R
-63 0 V
1100 7030 M
63 0 V
11396 0 R
-63 0 V
1100 7943 M
63 0 V
11396 0 R
-63 0 V
1100 640 M
0 63 V
0 7696 R
0 -63 V
3965 640 M
0 63 V
0 7696 R
0 -63 V
6830 640 M
0 63 V
0 7696 R
0 -63 V
9694 640 M
0 63 V
0 7696 R
0 -63 V
12559 640 M
0 63 V
0 7696 R
0 -63 V
stroke
1100 8399 N
0 -7759 V
11459 0 V
0 7759 V
-11459 0 V
Z stroke
LCb setrgbcolor
LTb
LCb setrgbcolor
LTb
LCb setrgbcolor
LTb
LCb setrgbcolor
LTb
1.000 UP
1.000 UL
LTb
1.000 UL
LTb
11296 703 N
0 800 V
1143 0 V
0 -800 V
-1143 0 V
Z stroke
11296 1503 M
1143 0 V
stroke
LT0
LCb setrgbcolor
LT0
11776 1303 M
543 0 V
1100 3779 M
191 269 V
0 -5 V
191 236 V
0 -4 V
191 228 V
0 -1 V
191 217 V
0 1 V
191 214 V
0 -2 V
191 217 V
0 -5 V
191 205 V
0 1 V
191 198 V
0 1 V
191 195 V
0 -4 V
191 191 V
191 178 V
0 4 V
191 171 V
0 -2 V
191 175 V
0 4 V
191 156 V
0 -1 V
191 154 V
0 1 V
191 141 V
0 4 V
191 132 V
0 2 V
191 131 V
0 1 V
191 115 V
0 4 V
191 103 V
0 5 V
191 94 V
0 3 V
191 88 V
0 3 V
191 77 V
191 72 V
0 -2 V
191 59 V
0 2 V
191 49 V
0 -1 V
191 37 V
191 31 V
0 5 V
191 10 V
0 2 V
191 5 V
0 -4 V
190 -1 V
0 -2 V
192 -14 V
0 4 V
190 -30 V
191 -38 V
192 -47 V
0 -3 V
190 -55 V
0 -1 V
191 -71 V
0 -2 V
192 -83 V
0 5 V
190 -92 V
0 -1 V
191 -100 V
192 -106 V
0 -1 V
190 -122 V
0 -1 V
191 -125 V
0 -1 V
191 -134 V
191 -145 V
0 1 V
191 -156 V
0 2 V
191 -160 V
0 -3 V
191 -166 V
0 1 V
191 -177 V
191 -183 V
191 -187 V
0 -1 V
191 -195 V
191 -198 V
0 -1 V
191 -205 V
0 1 V
191 -209 V
191 -215 V
0 -2 V
191 -213 V
stroke 11986 4507 M
0 -3 V
191 -226 V
0 2 V
191 -235 V
0 4 V
191 -272 V
stroke
LC1 setrgbcolor
LCb setrgbcolor
LT0
LC1 setrgbcolor
11776 1103 M
543 0 V
1100 2990 M
191 -267 V
0 4 V
191 -223 V
191 -209 V
0 2 V
191 -195 V
0 2 V
191 -183 V
0 -1 V
191 -165 V
0 2 V
191 -151 V
0 -1 V
191 -132 V
0 -1 V
191 -110 V
0 -5 V
191 -91 V
0 -4 V
191 -73 V
0 -1 V
191 -52 V
0 -2 V
191 -33 V
0 2 V
191 -11 V
191 13 V
191 34 V
191 59 V
0 -2 V
191 81 V
0 -3 V
191 93 V
0 2 V
191 111 V
0 5 V
191 128 V
0 6 V
191 147 V
0 2 V
191 165 V
0 -1 V
191 183 V
0 -2 V
191 196 V
0 -2 V
191 201 V
0 1 V
191 208 V
0 4 V
191 214 V
0 7 V
191 212 V
191 222 V
0 3 V
190 219 V
0 1 V
192 224 V
0 -2 V
190 218 V
191 211 V
0 -1 V
192 202 V
0 3 V
190 185 V
0 5 V
191 180 V
0 -1 V
192 166 V
0 -1 V
190 151 V
0 -1 V
191 133 V
0 1 V
192 116 V
0 -1 V
190 93 V
191 78 V
191 53 V
0 1 V
191 34 V
0 2 V
191 8 V
0 5 V
191 -11 V
0 -3 V
191 -30 V
0 3 V
191 -57 V
0 2 V
191 -76 V
0 -1 V
191 -92 V
0 -2 V
191 -112 V
0 -3 V
191 -129 V
0 -1 V
191 -148 V
191 -165 V
0 1 V
191 -181 V
stroke 11795 4656 M
0 2 V
191 -196 V
0 -3 V
191 -205 V
191 -224 V
0 5 V
191 -267 V
stroke
LT0
LC2 setrgbcolor
LCb setrgbcolor
LT0
LC2 setrgbcolor
11776 903 M
543 0 V
1100 3935 M
191 381 V
0 -7 V
191 309 V
0 -2 V
191 270 V
0 2 V
191 228 V
0 5 V
191 190 V
0 2 V
191 153 V
0 1 V
191 108 V
0 -5 V
191 60 V
0 -2 V
191 17 V
0 -10 V
191 -38 V
0 -2 V
191 -90 V
0 3 V
191 -135 V
0 2 V
191 -174 V
0 -3 V
191 -219 V
0 3 V
191 -251 V
0 2 V
191 -280 V
0 -10 V
191 -289 V
0 1 V
191 -325 V
0 -1 V
191 -330 V
0 -4 V
191 -334 V
0 1 V
191 -329 V
0 6 V
191 -318 V
0 -1 V
191 -298 V
0 3 V
191 -274 V
0 -3 V
191 -233 V
0 1 V
191 -218 V
191 -159 V
0 -5 V
191 -121 V
0 8 V
191 -86 V
0 3 V
191 -33 V
0 4 V
190 18 V
0 1 V
192 78 V
0 -11 V
190 126 V
0 1 V
191 166 V
0 -8 V
192 211 V
0 5 V
190 227 V
0 2 V
191 281 V
0 -3 V
192 298 V
0 1 V
190 316 V
0 3 V
191 324 V
0 8 V
192 329 V
0 3 V
190 332 V
191 321 V
0 3 V
191 308 V
0 -2 V
191 285 V
0 4 V
191 251 V
0 4 V
191 214 V
0 4 V
191 177 V
0 7 V
191 128 V
0 5 V
191 90 V
0 1 V
191 44 V
0 -2 V
191 -2 V
0 -2 V
stroke 11031 5655 M
191 -53 V
0 -4 V
191 -102 V
0 -4 V
191 -148 V
0 2 V
191 -198 V
0 -1 V
191 -234 V
0 -4 V
191 -274 V
0 2 V
191 -314 V
0 7 V
191 -386 V
stroke
LT0
1100 3378 M
0 539 V
191 0 V
0 249 V
191 0 V
0 228 V
191 0 V
0 223 V
191 0 V
0 215 V
191 0 V
0 207 V
191 0 V
0 210 V
191 0 V
0 198 V
191 0 V
0 199 V
191 0 V
0 189 V
191 0 V
0 187 V
191 0 V
0 179 V
191 0 V
0 171 V
191 0 V
0 167 V
191 0 V
0 158 V
191 0 V
0 149 V
191 0 V
0 143 V
191 0 V
0 134 V
191 0 V
0 122 V
191 0 V
0 113 V
191 0 V
0 102 V
191 0 V
0 91 V
191 0 V
0 90 V
191 0 V
0 74 V
191 0 V
0 63 V
191 0 V
0 55 V
191 0 V
0 45 V
191 0 V
0 35 V
191 0 V
0 18 V
191 0 V
0 10 V
191 0 V
0 2 V
191 0 V
0 -10 V
190 0 V
0 -25 V
191 0 V
0 -28 V
192 0 V
0 -42 V
190 0 V
0 -53 V
191 0 V
0 -68 V
191 0 V
0 -73 V
191 0 V
0 -86 V
191 0 V
0 -94 V
191 0 V
0 -108 V
191 0 V
0 -115 V
191 0 V
0 -123 V
191 0 V
0 -134 V
191 0 V
0 -143 V
191 0 V
0 -151 V
191 0 V
0 -159 V
191 0 V
0 -166 V
191 0 V
0 -169 V
191 0 V
0 -181 V
191 0 V
0 -190 V
191 0 V
0 -185 V
191 0 V
stroke 11031 5640 M
0 -195 V
191 0 V
0 -202 V
191 0 V
0 -208 V
191 0 V
0 -209 V
191 0 V
0 -217 V
191 0 V
0 -219 V
191 0 V
0 -228 V
191 0 V
0 -247 V
191 0 V
0 -537 V
stroke
LC1 setrgbcolor
1100 3378 M
0 -523 V
191 0 V
0 -239 V
191 0 V
0 -216 V
191 0 V
0 -202 V
191 0 V
0 -186 V
191 0 V
0 -170 V
191 0 V
0 -160 V
191 0 V
0 -138 V
191 0 V
0 -122 V
191 0 V
0 -105 V
191 0 V
0 -84 V
191 0 V
0 -66 V
191 0 V
0 -44 V
191 0 V
0 -20 V
191 0 V
191 0 V
0 24 V
191 0 V
0 43 V
191 0 V
0 65 V
191 0 V
0 85 V
191 0 V
0 105 V
191 0 V
0 123 V
191 0 V
0 143 V
191 0 V
0 156 V
191 0 V
0 171 V
191 0 V
0 188 V
191 0 V
0 197 V
191 0 V
0 207 V
191 0 V
0 214 V
191 0 V
0 219 V
191 0 V
0 223 V
191 0 V
0 224 V
191 0 V
0 224 V
190 0 V
0 218 V
191 0 V
0 212 V
192 0 V
0 207 V
190 0 V
0 197 V
191 0 V
0 186 V
191 0 V
0 173 V
191 0 V
0 156 V
191 0 V
0 145 V
191 0 V
0 126 V
191 0 V
0 107 V
191 0 V
0 85 V
191 0 V
0 65 V
191 0 V
0 44 V
191 0 V
0 25 V
191 0 V
0 1 V
191 0 V
0 -22 V
191 0 V
0 -41 V
191 0 V
0 -64 V
191 0 V
0 -84 V
191 0 V
0 -105 V
191 0 V
0 -123 V
stroke 11031 5222 M
191 0 V
0 -141 V
191 0 V
0 -160 V
191 0 V
0 -172 V
191 0 V
0 -190 V
191 0 V
0 -198 V
191 0 V
0 -216 V
191 0 V
0 -240 V
191 0 V
0 -527 V
stroke
LT0
LC2 setrgbcolor
1100 3378 M
0 757 V
191 0 V
0 342 V
191 0 V
0 290 V
191 0 V
0 255 V
191 0 V
0 215 V
191 0 V
0 171 V
191 0 V
0 132 V
191 0 V
0 82 V
191 0 V
0 35 V
191 0 V
0 -20 V
191 0 V
0 -64 V
191 0 V
0 -113 V
191 0 V
0 -156 V
191 0 V
0 -197 V
191 0 V
0 -239 V
191 0 V
0 -269 V
191 0 V
0 -295 V
191 0 V
0 -314 V
191 0 V
0 -328 V
191 0 V
0 -337 V
191 0 V
0 -334 V
191 0 V
0 -326 V
191 0 V
0 -308 V
191 0 V
0 -287 V
191 0 V
0 -258 V
191 0 V
0 -227 V
191 0 V
0 -188 V
191 0 V
0 -146 V
191 0 V
0 -96 V
191 0 V
0 -50 V
191 0 V
0 -2 V
191 0 V
0 49 V
190 0 V
0 100 V
191 0 V
0 143 V
192 0 V
0 186 V
190 0 V
0 224 V
191 0 V
0 261 V
191 0 V
0 288 V
191 0 V
0 309 V
191 0 V
0 325 V
191 0 V
0 334 V
191 0 V
0 334 V
191 0 V
0 330 V
191 0 V
0 316 V
191 0 V
0 298 V
191 0 V
0 266 V
191 0 V
0 238 V
191 0 V
0 200 V
191 0 V
0 161 V
191 0 V
0 111 V
191 0 V
0 64 V
191 0 V
0 21 V
191 0 V
stroke 11031 5661 M
0 -33 V
191 0 V
0 -80 V
191 0 V
0 -130 V
191 0 V
0 -171 V
191 0 V
0 -218 V
191 0 V
0 -254 V
191 0 V
0 -294 V
191 0 V
0 -342 V
191 0 V
0 -761 V
stroke
LTb
1100 8399 N
0 -7759 V
11459 0 V
0 7759 V
-11459 0 V
Z stroke
1.000 UP
1.000 UL
LTb
stroke
grestore
end
showpage
  }}%
  \put(11656,903){\makebox(0,0)[r]{\strut{}$\lambda_2$}}%
  \put(11656,1103){\makebox(0,0)[r]{\strut{}$\lambda_1$}}%
  \put(11656,1303){\makebox(0,0)[r]{\strut{}$\lambda_0$}}%
  \put(6829,140){\makebox(0,0){\strut{}Slab width (mfp)}}%
  \put(280,4519){%
  \special{ps: gsave currentpoint currentpoint translate
630 rotate neg exch neg exch translate}%
  \makebox(0,0){\strut{}Eigenvector}%
  \special{ps: currentpoint grestore moveto}%
  }%
  \put(12559,440){\makebox(0,0){\strut{} 20}}%
  \put(9694,440){\makebox(0,0){\strut{} 15}}%
  \put(6830,440){\makebox(0,0){\strut{} 10}}%
  \put(3965,440){\makebox(0,0){\strut{} 5}}%
  \put(1100,440){\makebox(0,0){\strut{} 0}}%
  \put(980,7943){\makebox(0,0)[r]{\strut{} 1}}%
  \put(980,7030){\makebox(0,0)[r]{\strut{} 0.8}}%
  \put(980,6117){\makebox(0,0)[r]{\strut{} 0.6}}%
  \put(980,5204){\makebox(0,0)[r]{\strut{} 0.4}}%
  \put(980,4291){\makebox(0,0)[r]{\strut{} 0.2}}%
  \put(980,3378){\makebox(0,0)[r]{\strut{} 0}}%
  \put(980,2466){\makebox(0,0)[r]{\strut{}-0.2}}%
  \put(980,1553){\makebox(0,0)[r]{\strut{}-0.4}}%
  \put(980,640){\makebox(0,0)[r]{\strut{}-0.6}}%
\end{picture}%
\endgroup
\endinput

    \caption{Estimates of the fundamental and first two harmonic eigenvectors tracking 1E5 particles per iteration and using 60 spatial bins.  The sawtooth shaped curves use a  first-order accurate approximation while the smooth curves use a second-order accurate approximation to the fission source.}
    \label{fig:SpatialDiscretizationVectors}
\end{sidewaysfigure}

For small bin widths the second-order accurate approximation seems to have difficulties due to too few particles scoring in a bin.  If more particles were tracked in an iteration we expect more particles to score in each bin and the statistical uncertainty to improve.  The above simulations have been repeated, but have tracked 1E6 particles in each iteration---ten times more particles---to improve the statistics in each simulation.  The results for the first-order accurate approximation are given in \Fref{tab:Bias0Histogram1E6} and the results for the second-order accurate approximation  are given in \Fref{tab:Bias0Linear1E6}.

Tracking more particles in an iteration seems to have made a big improvement for the second-order accurate approximation with very small bin widths.  Now all the results, except the large bin width of 2 mfp, estimates the fundamental eigenvalue within statistical uncertainty.  Tracking ten times more particles per iteration has improved the poor results from the simulation with small bins and has reduced the standard deviation by approximately one-third, as we expect for the second-order accurate approximation.  The standard deviation was also reduced for the first-order accurate approximation simulations.  The error in the eigenvalue estimate is also significantly smaller for the second-order accurate approximation but remains essentially unchanged for the first-order accurate approximation.

In \Fref{fig:BiasFOM1E6} the figure of merit is shown for the first and second-order spatial approximations, where 1 million particles are tracked for each iteration.  We see that when tracking more particles in an iteration, the figure of merit improves for second-order accurate approximations with small bin widths.  This is because now we have an accurate estimate of the eigenvalue as well as a reduced standard deviation.  With the exception of simulations with very large bin widths, the figure of merit for second-order accurate approximations are two times larger than that for first-order accurate approximations.  It should be noted that the linear-in-space approximation is not only more efficient than the power method, but it is in fact computing 3 eigenpairs compared to the power methods single eigenvector. 

The anomaly in these results is the second-order accurate approximation with a bin width of 2.0 mfp.  The error in the eigenvalue estimate is smaller than the error for the same problem but with a first-order accurate approximation, but the eigenvalue estimate is not within statistical uncertainty regardless of the number of particles tracked.  In addition, the error in the eigenvalue estimate is larger than for any other bin width.

\begin{comment}
\begin{sidewaysfigure} \centering
    % GNUPLOT: LaTeX picture with Postscript
\begingroup%
\makeatletter%
\newcommand{\GNUPLOTspecial}{%
  \@sanitize\catcode`\%=14\relax\special}%
\setlength{\unitlength}{0.0500bp}%
\begin{picture}(12960,8640)(0,0)%
  {\GNUPLOTspecial{"
%!PS-Adobe-2.0 EPSF-2.0
%%Title: BiasLinear1E6.tex
%%Creator: gnuplot 4.3 patchlevel 0
%%CreationDate: Wed Aug 12 16:28:22 2009
%%DocumentFonts: 
%%BoundingBox: 0 0 648 432
%%EndComments
%%BeginProlog
/gnudict 256 dict def
gnudict begin
%
% The following true/false flags may be edited by hand if desired.
% The unit line width and grayscale image gamma correction may also be changed.
%
/Color true def
/Blacktext true def
/Solid true def
/Dashlength 1 def
/Landscape false def
/Level1 false def
/Rounded false def
/ClipToBoundingBox false def
/TransparentPatterns false def
/gnulinewidth 5.000 def
/userlinewidth gnulinewidth def
/Gamma 1.0 def
%
/vshift -66 def
/dl1 {
  10.0 Dashlength mul mul
  Rounded { currentlinewidth 0.75 mul sub dup 0 le { pop 0.01 } if } if
} def
/dl2 {
  10.0 Dashlength mul mul
  Rounded { currentlinewidth 0.75 mul add } if
} def
/hpt_ 31.5 def
/vpt_ 31.5 def
/hpt hpt_ def
/vpt vpt_ def
Level1 {} {
/SDict 10 dict def
systemdict /pdfmark known not {
  userdict /pdfmark systemdict /cleartomark get put
} if
SDict begin [
  /Title (BiasLinear1E6.tex)
  /Subject (gnuplot plot)
  /Creator (gnuplot 4.3 patchlevel 0)
  /Author (Jeremy Conlin)
%  /Producer (gnuplot)
%  /Keywords ()
  /CreationDate (Wed Aug 12 16:28:22 2009)
  /DOCINFO pdfmark
end
} ifelse
/doclip {
  ClipToBoundingBox {
    newpath 0 0 moveto 648 0 lineto 648 432 lineto 0 432 lineto closepath
    clip
  } if
} def
%
% Gnuplot Prolog Version 4.2 (November 2007)
%
/M {moveto} bind def
/L {lineto} bind def
/R {rmoveto} bind def
/V {rlineto} bind def
/N {newpath moveto} bind def
/Z {closepath} bind def
/C {setrgbcolor} bind def
/f {rlineto fill} bind def
/Gshow {show} def   % May be redefined later in the file to support UTF-8
/vpt2 vpt 2 mul def
/hpt2 hpt 2 mul def
/Lshow {currentpoint stroke M 0 vshift R 
	Blacktext {gsave 0 setgray show grestore} {show} ifelse} def
/Rshow {currentpoint stroke M dup stringwidth pop neg vshift R
	Blacktext {gsave 0 setgray show grestore} {show} ifelse} def
/Cshow {currentpoint stroke M dup stringwidth pop -2 div vshift R 
	Blacktext {gsave 0 setgray show grestore} {show} ifelse} def
/UP {dup vpt_ mul /vpt exch def hpt_ mul /hpt exch def
  /hpt2 hpt 2 mul def /vpt2 vpt 2 mul def} def
/DL {Color {setrgbcolor Solid {pop []} if 0 setdash}
 {pop pop pop 0 setgray Solid {pop []} if 0 setdash} ifelse} def
/BL {stroke userlinewidth 2 mul setlinewidth
	Rounded {1 setlinejoin 1 setlinecap} if} def
/AL {stroke userlinewidth 2 div setlinewidth
	Rounded {1 setlinejoin 1 setlinecap} if} def
/UL {dup gnulinewidth mul /userlinewidth exch def
	dup 1 lt {pop 1} if 10 mul /udl exch def} def
/PL {stroke userlinewidth setlinewidth
	Rounded {1 setlinejoin 1 setlinecap} if} def
% Default Line colors
/LCw {1 1 1} def
/LCb {0 0 0} def
/LCa {0 0 0} def
/LC0 {1 0 0} def
/LC1 {0 1 0} def
/LC2 {0 0 1} def
/LC3 {1 0 1} def
/LC4 {0 1 1} def
/LC5 {1 1 0} def
/LC6 {0 0 0} def
/LC7 {1 0.3 0} def
/LC8 {0.5 0.5 0.5} def
% Default Line Types
/LTw {PL [] 1 setgray} def
/LTb {BL [] LCb DL} def
/LTa {AL [1 udl mul 2 udl mul] 0 setdash LCa setrgbcolor} def
/LT0 {PL [] LC0 DL} def
/LT1 {PL [4 dl1 2 dl2] LC1 DL} def
/LT2 {PL [2 dl1 3 dl2] LC2 DL} def
/LT3 {PL [1 dl1 1.5 dl2] LC3 DL} def
/LT4 {PL [6 dl1 2 dl2 1 dl1 2 dl2] LC4 DL} def
/LT5 {PL [3 dl1 3 dl2 1 dl1 3 dl2] LC5 DL} def
/LT6 {PL [2 dl1 2 dl2 2 dl1 6 dl2] LC6 DL} def
/LT7 {PL [1 dl1 2 dl2 6 dl1 2 dl2 1 dl1 2 dl2] LC7 DL} def
/LT8 {PL [2 dl1 2 dl2 2 dl1 2 dl2 2 dl1 2 dl2 2 dl1 4 dl2] LC8 DL} def
/Pnt {stroke [] 0 setdash gsave 1 setlinecap M 0 0 V stroke grestore} def
/Dia {stroke [] 0 setdash 2 copy vpt add M
  hpt neg vpt neg V hpt vpt neg V
  hpt vpt V hpt neg vpt V closepath stroke
  Pnt} def
/Pls {stroke [] 0 setdash vpt sub M 0 vpt2 V
  currentpoint stroke M
  hpt neg vpt neg R hpt2 0 V stroke
 } def
/Box {stroke [] 0 setdash 2 copy exch hpt sub exch vpt add M
  0 vpt2 neg V hpt2 0 V 0 vpt2 V
  hpt2 neg 0 V closepath stroke
  Pnt} def
/Crs {stroke [] 0 setdash exch hpt sub exch vpt add M
  hpt2 vpt2 neg V currentpoint stroke M
  hpt2 neg 0 R hpt2 vpt2 V stroke} def
/TriU {stroke [] 0 setdash 2 copy vpt 1.12 mul add M
  hpt neg vpt -1.62 mul V
  hpt 2 mul 0 V
  hpt neg vpt 1.62 mul V closepath stroke
  Pnt} def
/Star {2 copy Pls Crs} def
/BoxF {stroke [] 0 setdash exch hpt sub exch vpt add M
  0 vpt2 neg V hpt2 0 V 0 vpt2 V
  hpt2 neg 0 V closepath fill} def
/TriUF {stroke [] 0 setdash vpt 1.12 mul add M
  hpt neg vpt -1.62 mul V
  hpt 2 mul 0 V
  hpt neg vpt 1.62 mul V closepath fill} def
/TriD {stroke [] 0 setdash 2 copy vpt 1.12 mul sub M
  hpt neg vpt 1.62 mul V
  hpt 2 mul 0 V
  hpt neg vpt -1.62 mul V closepath stroke
  Pnt} def
/TriDF {stroke [] 0 setdash vpt 1.12 mul sub M
  hpt neg vpt 1.62 mul V
  hpt 2 mul 0 V
  hpt neg vpt -1.62 mul V closepath fill} def
/DiaF {stroke [] 0 setdash vpt add M
  hpt neg vpt neg V hpt vpt neg V
  hpt vpt V hpt neg vpt V closepath fill} def
/Pent {stroke [] 0 setdash 2 copy gsave
  translate 0 hpt M 4 {72 rotate 0 hpt L} repeat
  closepath stroke grestore Pnt} def
/PentF {stroke [] 0 setdash gsave
  translate 0 hpt M 4 {72 rotate 0 hpt L} repeat
  closepath fill grestore} def
/Circle {stroke [] 0 setdash 2 copy
  hpt 0 360 arc stroke Pnt} def
/CircleF {stroke [] 0 setdash hpt 0 360 arc fill} def
/C0 {BL [] 0 setdash 2 copy moveto vpt 90 450 arc} bind def
/C1 {BL [] 0 setdash 2 copy moveto
	2 copy vpt 0 90 arc closepath fill
	vpt 0 360 arc closepath} bind def
/C2 {BL [] 0 setdash 2 copy moveto
	2 copy vpt 90 180 arc closepath fill
	vpt 0 360 arc closepath} bind def
/C3 {BL [] 0 setdash 2 copy moveto
	2 copy vpt 0 180 arc closepath fill
	vpt 0 360 arc closepath} bind def
/C4 {BL [] 0 setdash 2 copy moveto
	2 copy vpt 180 270 arc closepath fill
	vpt 0 360 arc closepath} bind def
/C5 {BL [] 0 setdash 2 copy moveto
	2 copy vpt 0 90 arc
	2 copy moveto
	2 copy vpt 180 270 arc closepath fill
	vpt 0 360 arc} bind def
/C6 {BL [] 0 setdash 2 copy moveto
	2 copy vpt 90 270 arc closepath fill
	vpt 0 360 arc closepath} bind def
/C7 {BL [] 0 setdash 2 copy moveto
	2 copy vpt 0 270 arc closepath fill
	vpt 0 360 arc closepath} bind def
/C8 {BL [] 0 setdash 2 copy moveto
	2 copy vpt 270 360 arc closepath fill
	vpt 0 360 arc closepath} bind def
/C9 {BL [] 0 setdash 2 copy moveto
	2 copy vpt 270 450 arc closepath fill
	vpt 0 360 arc closepath} bind def
/C10 {BL [] 0 setdash 2 copy 2 copy moveto vpt 270 360 arc closepath fill
	2 copy moveto
	2 copy vpt 90 180 arc closepath fill
	vpt 0 360 arc closepath} bind def
/C11 {BL [] 0 setdash 2 copy moveto
	2 copy vpt 0 180 arc closepath fill
	2 copy moveto
	2 copy vpt 270 360 arc closepath fill
	vpt 0 360 arc closepath} bind def
/C12 {BL [] 0 setdash 2 copy moveto
	2 copy vpt 180 360 arc closepath fill
	vpt 0 360 arc closepath} bind def
/C13 {BL [] 0 setdash 2 copy moveto
	2 copy vpt 0 90 arc closepath fill
	2 copy moveto
	2 copy vpt 180 360 arc closepath fill
	vpt 0 360 arc closepath} bind def
/C14 {BL [] 0 setdash 2 copy moveto
	2 copy vpt 90 360 arc closepath fill
	vpt 0 360 arc} bind def
/C15 {BL [] 0 setdash 2 copy vpt 0 360 arc closepath fill
	vpt 0 360 arc closepath} bind def
/Rec {newpath 4 2 roll moveto 1 index 0 rlineto 0 exch rlineto
	neg 0 rlineto closepath} bind def
/Square {dup Rec} bind def
/Bsquare {vpt sub exch vpt sub exch vpt2 Square} bind def
/S0 {BL [] 0 setdash 2 copy moveto 0 vpt rlineto BL Bsquare} bind def
/S1 {BL [] 0 setdash 2 copy vpt Square fill Bsquare} bind def
/S2 {BL [] 0 setdash 2 copy exch vpt sub exch vpt Square fill Bsquare} bind def
/S3 {BL [] 0 setdash 2 copy exch vpt sub exch vpt2 vpt Rec fill Bsquare} bind def
/S4 {BL [] 0 setdash 2 copy exch vpt sub exch vpt sub vpt Square fill Bsquare} bind def
/S5 {BL [] 0 setdash 2 copy 2 copy vpt Square fill
	exch vpt sub exch vpt sub vpt Square fill Bsquare} bind def
/S6 {BL [] 0 setdash 2 copy exch vpt sub exch vpt sub vpt vpt2 Rec fill Bsquare} bind def
/S7 {BL [] 0 setdash 2 copy exch vpt sub exch vpt sub vpt vpt2 Rec fill
	2 copy vpt Square fill Bsquare} bind def
/S8 {BL [] 0 setdash 2 copy vpt sub vpt Square fill Bsquare} bind def
/S9 {BL [] 0 setdash 2 copy vpt sub vpt vpt2 Rec fill Bsquare} bind def
/S10 {BL [] 0 setdash 2 copy vpt sub vpt Square fill 2 copy exch vpt sub exch vpt Square fill
	Bsquare} bind def
/S11 {BL [] 0 setdash 2 copy vpt sub vpt Square fill 2 copy exch vpt sub exch vpt2 vpt Rec fill
	Bsquare} bind def
/S12 {BL [] 0 setdash 2 copy exch vpt sub exch vpt sub vpt2 vpt Rec fill Bsquare} bind def
/S13 {BL [] 0 setdash 2 copy exch vpt sub exch vpt sub vpt2 vpt Rec fill
	2 copy vpt Square fill Bsquare} bind def
/S14 {BL [] 0 setdash 2 copy exch vpt sub exch vpt sub vpt2 vpt Rec fill
	2 copy exch vpt sub exch vpt Square fill Bsquare} bind def
/S15 {BL [] 0 setdash 2 copy Bsquare fill Bsquare} bind def
/D0 {gsave translate 45 rotate 0 0 S0 stroke grestore} bind def
/D1 {gsave translate 45 rotate 0 0 S1 stroke grestore} bind def
/D2 {gsave translate 45 rotate 0 0 S2 stroke grestore} bind def
/D3 {gsave translate 45 rotate 0 0 S3 stroke grestore} bind def
/D4 {gsave translate 45 rotate 0 0 S4 stroke grestore} bind def
/D5 {gsave translate 45 rotate 0 0 S5 stroke grestore} bind def
/D6 {gsave translate 45 rotate 0 0 S6 stroke grestore} bind def
/D7 {gsave translate 45 rotate 0 0 S7 stroke grestore} bind def
/D8 {gsave translate 45 rotate 0 0 S8 stroke grestore} bind def
/D9 {gsave translate 45 rotate 0 0 S9 stroke grestore} bind def
/D10 {gsave translate 45 rotate 0 0 S10 stroke grestore} bind def
/D11 {gsave translate 45 rotate 0 0 S11 stroke grestore} bind def
/D12 {gsave translate 45 rotate 0 0 S12 stroke grestore} bind def
/D13 {gsave translate 45 rotate 0 0 S13 stroke grestore} bind def
/D14 {gsave translate 45 rotate 0 0 S14 stroke grestore} bind def
/D15 {gsave translate 45 rotate 0 0 S15 stroke grestore} bind def
/DiaE {stroke [] 0 setdash vpt add M
  hpt neg vpt neg V hpt vpt neg V
  hpt vpt V hpt neg vpt V closepath stroke} def
/BoxE {stroke [] 0 setdash exch hpt sub exch vpt add M
  0 vpt2 neg V hpt2 0 V 0 vpt2 V
  hpt2 neg 0 V closepath stroke} def
/TriUE {stroke [] 0 setdash vpt 1.12 mul add M
  hpt neg vpt -1.62 mul V
  hpt 2 mul 0 V
  hpt neg vpt 1.62 mul V closepath stroke} def
/TriDE {stroke [] 0 setdash vpt 1.12 mul sub M
  hpt neg vpt 1.62 mul V
  hpt 2 mul 0 V
  hpt neg vpt -1.62 mul V closepath stroke} def
/PentE {stroke [] 0 setdash gsave
  translate 0 hpt M 4 {72 rotate 0 hpt L} repeat
  closepath stroke grestore} def
/CircE {stroke [] 0 setdash 
  hpt 0 360 arc stroke} def
/Opaque {gsave closepath 1 setgray fill grestore 0 setgray closepath} def
/DiaW {stroke [] 0 setdash vpt add M
  hpt neg vpt neg V hpt vpt neg V
  hpt vpt V hpt neg vpt V Opaque stroke} def
/BoxW {stroke [] 0 setdash exch hpt sub exch vpt add M
  0 vpt2 neg V hpt2 0 V 0 vpt2 V
  hpt2 neg 0 V Opaque stroke} def
/TriUW {stroke [] 0 setdash vpt 1.12 mul add M
  hpt neg vpt -1.62 mul V
  hpt 2 mul 0 V
  hpt neg vpt 1.62 mul V Opaque stroke} def
/TriDW {stroke [] 0 setdash vpt 1.12 mul sub M
  hpt neg vpt 1.62 mul V
  hpt 2 mul 0 V
  hpt neg vpt -1.62 mul V Opaque stroke} def
/PentW {stroke [] 0 setdash gsave
  translate 0 hpt M 4 {72 rotate 0 hpt L} repeat
  Opaque stroke grestore} def
/CircW {stroke [] 0 setdash 
  hpt 0 360 arc Opaque stroke} def
/BoxFill {gsave Rec 1 setgray fill grestore} def
/Density {
  /Fillden exch def
  currentrgbcolor
  /ColB exch def /ColG exch def /ColR exch def
  /ColR ColR Fillden mul Fillden sub 1 add def
  /ColG ColG Fillden mul Fillden sub 1 add def
  /ColB ColB Fillden mul Fillden sub 1 add def
  ColR ColG ColB setrgbcolor} def
/BoxColFill {gsave Rec PolyFill} def
/PolyFill {gsave Density fill grestore grestore} def
/h {rlineto rlineto rlineto gsave closepath fill grestore} bind def
%
% PostScript Level 1 Pattern Fill routine for rectangles
% Usage: x y w h s a XX PatternFill
%	x,y = lower left corner of box to be filled
%	w,h = width and height of box
%	  a = angle in degrees between lines and x-axis
%	 XX = 0/1 for no/yes cross-hatch
%
/PatternFill {gsave /PFa [ 9 2 roll ] def
  PFa 0 get PFa 2 get 2 div add PFa 1 get PFa 3 get 2 div add translate
  PFa 2 get -2 div PFa 3 get -2 div PFa 2 get PFa 3 get Rec
  gsave 1 setgray fill grestore clip
  currentlinewidth 0.5 mul setlinewidth
  /PFs PFa 2 get dup mul PFa 3 get dup mul add sqrt def
  0 0 M PFa 5 get rotate PFs -2 div dup translate
  0 1 PFs PFa 4 get div 1 add floor cvi
	{PFa 4 get mul 0 M 0 PFs V} for
  0 PFa 6 get ne {
	0 1 PFs PFa 4 get div 1 add floor cvi
	{PFa 4 get mul 0 2 1 roll M PFs 0 V} for
 } if
  stroke grestore} def
%
/languagelevel where
 {pop languagelevel} {1} ifelse
 2 lt
	{/InterpretLevel1 true def}
	{/InterpretLevel1 Level1 def}
 ifelse
%
% PostScript level 2 pattern fill definitions
%
/Level2PatternFill {
/Tile8x8 {/PaintType 2 /PatternType 1 /TilingType 1 /BBox [0 0 8 8] /XStep 8 /YStep 8}
	bind def
/KeepColor {currentrgbcolor [/Pattern /DeviceRGB] setcolorspace} bind def
<< Tile8x8
 /PaintProc {0.5 setlinewidth pop 0 0 M 8 8 L 0 8 M 8 0 L stroke} 
>> matrix makepattern
/Pat1 exch def
<< Tile8x8
 /PaintProc {0.5 setlinewidth pop 0 0 M 8 8 L 0 8 M 8 0 L stroke
	0 4 M 4 8 L 8 4 L 4 0 L 0 4 L stroke}
>> matrix makepattern
/Pat2 exch def
<< Tile8x8
 /PaintProc {0.5 setlinewidth pop 0 0 M 0 8 L
	8 8 L 8 0 L 0 0 L fill}
>> matrix makepattern
/Pat3 exch def
<< Tile8x8
 /PaintProc {0.5 setlinewidth pop -4 8 M 8 -4 L
	0 12 M 12 0 L stroke}
>> matrix makepattern
/Pat4 exch def
<< Tile8x8
 /PaintProc {0.5 setlinewidth pop -4 0 M 8 12 L
	0 -4 M 12 8 L stroke}
>> matrix makepattern
/Pat5 exch def
<< Tile8x8
 /PaintProc {0.5 setlinewidth pop -2 8 M 4 -4 L
	0 12 M 8 -4 L 4 12 M 10 0 L stroke}
>> matrix makepattern
/Pat6 exch def
<< Tile8x8
 /PaintProc {0.5 setlinewidth pop -2 0 M 4 12 L
	0 -4 M 8 12 L 4 -4 M 10 8 L stroke}
>> matrix makepattern
/Pat7 exch def
<< Tile8x8
 /PaintProc {0.5 setlinewidth pop 8 -2 M -4 4 L
	12 0 M -4 8 L 12 4 M 0 10 L stroke}
>> matrix makepattern
/Pat8 exch def
<< Tile8x8
 /PaintProc {0.5 setlinewidth pop 0 -2 M 12 4 L
	-4 0 M 12 8 L -4 4 M 8 10 L stroke}
>> matrix makepattern
/Pat9 exch def
/Pattern1 {PatternBgnd KeepColor Pat1 setpattern} bind def
/Pattern2 {PatternBgnd KeepColor Pat2 setpattern} bind def
/Pattern3 {PatternBgnd KeepColor Pat3 setpattern} bind def
/Pattern4 {PatternBgnd KeepColor Landscape {Pat5} {Pat4} ifelse setpattern} bind def
/Pattern5 {PatternBgnd KeepColor Landscape {Pat4} {Pat5} ifelse setpattern} bind def
/Pattern6 {PatternBgnd KeepColor Landscape {Pat9} {Pat6} ifelse setpattern} bind def
/Pattern7 {PatternBgnd KeepColor Landscape {Pat8} {Pat7} ifelse setpattern} bind def
} def
%
%
%End of PostScript Level 2 code
%
/PatternBgnd {
  TransparentPatterns {} {gsave 1 setgray fill grestore} ifelse
} def
%
% Substitute for Level 2 pattern fill codes with
% grayscale if Level 2 support is not selected.
%
/Level1PatternFill {
/Pattern1 {0.250 Density} bind def
/Pattern2 {0.500 Density} bind def
/Pattern3 {0.750 Density} bind def
/Pattern4 {0.125 Density} bind def
/Pattern5 {0.375 Density} bind def
/Pattern6 {0.625 Density} bind def
/Pattern7 {0.875 Density} bind def
} def
%
% Now test for support of Level 2 code
%
Level1 {Level1PatternFill} {Level2PatternFill} ifelse
%
/Symbol-Oblique /Symbol findfont [1 0 .167 1 0 0] makefont
dup length dict begin {1 index /FID eq {pop pop} {def} ifelse} forall
currentdict end definefont pop
end
%%EndProlog
gnudict begin
gsave
doclip
0 0 translate
0.050 0.050 scale
0 setgray
newpath
1.000 UL
LTb
1460 640 M
63 0 V
11036 0 R
-63 0 V
1460 1419 M
31 0 V
11068 0 R
-31 0 V
1460 1874 M
31 0 V
11068 0 R
-31 0 V
1460 2197 M
31 0 V
11068 0 R
-31 0 V
1460 2448 M
31 0 V
11068 0 R
-31 0 V
1460 2653 M
31 0 V
11068 0 R
-31 0 V
1460 2826 M
31 0 V
11068 0 R
-31 0 V
1460 2976 M
31 0 V
11068 0 R
-31 0 V
1460 3108 M
31 0 V
11068 0 R
-31 0 V
1460 3226 M
63 0 V
11036 0 R
-63 0 V
1460 4005 M
31 0 V
11068 0 R
-31 0 V
1460 4460 M
31 0 V
11068 0 R
-31 0 V
1460 4783 M
31 0 V
11068 0 R
-31 0 V
1460 5034 M
31 0 V
11068 0 R
-31 0 V
1460 5239 M
31 0 V
11068 0 R
-31 0 V
1460 5412 M
31 0 V
11068 0 R
-31 0 V
1460 5562 M
31 0 V
11068 0 R
-31 0 V
1460 5694 M
31 0 V
11068 0 R
-31 0 V
1460 5813 M
63 0 V
11036 0 R
-63 0 V
1460 6591 M
31 0 V
11068 0 R
-31 0 V
1460 7047 M
31 0 V
11068 0 R
-31 0 V
1460 7370 M
31 0 V
11068 0 R
-31 0 V
1460 7620 M
31 0 V
11068 0 R
-31 0 V
1460 7825 M
31 0 V
11068 0 R
-31 0 V
1460 7998 M
31 0 V
11068 0 R
-31 0 V
1460 8148 M
31 0 V
11068 0 R
-31 0 V
1460 8281 M
31 0 V
stroke 1491 8281 M
11068 0 R
-31 0 V
1460 8399 M
63 0 V
11036 0 R
-63 0 V
1460 640 M
0 63 V
0 7696 R
0 -63 V
3850 640 M
0 31 V
0 7728 R
0 -31 V
5248 640 M
0 31 V
0 7728 R
0 -31 V
6240 640 M
0 31 V
0 7728 R
0 -31 V
7010 640 M
0 31 V
0 7728 R
0 -31 V
7638 640 M
0 31 V
0 7728 R
0 -31 V
8170 640 M
0 31 V
0 7728 R
0 -31 V
8630 640 M
0 31 V
0 7728 R
0 -31 V
9036 640 M
0 31 V
0 7728 R
0 -31 V
9400 640 M
0 63 V
0 7696 R
0 -63 V
11790 640 M
0 31 V
0 7728 R
0 -31 V
stroke
1460 8399 N
0 -7759 V
11099 0 V
0 7759 V
-11099 0 V
Z stroke
LCb setrgbcolor
LTb
LCb setrgbcolor
LTb
LCb setrgbcolor
LTb
LCb setrgbcolor
LTb
0.500 UP
1.000 UL
LTb
1.000 UL
LTb
11328 703 N
0 800 V
1111 0 V
0 -800 V
-1111 0 V
Z stroke
11328 1503 M
1111 0 V
0.500 UP
stroke
LT0
LCb setrgbcolor
LT0
11808 1303 M
511 0 V
-511 31 R
0 -62 V
511 62 R
0 -62 V
-529 4523 R
0 2514 V
-31 -2514 R
62 0 V
-62 2514 R
62 0 V
8630 3794 M
0 1142 V
8599 3794 M
62 0 V
-62 1142 R
62 0 V
7010 640 M
0 3404 V
6979 640 M
62 0 V
-62 3404 R
62 0 V
6240 1590 M
0 2945 V
6209 1590 M
62 0 V
-62 2945 R
62 0 V
5611 640 M
0 3599 V
5580 640 M
62 0 V
-62 3599 R
62 0 V
4842 640 M
0 3310 V
4811 640 M
62 0 V
-62 3310 R
62 0 V
4213 640 M
0 3443 V
4182 640 M
62 0 V
-62 3443 R
62 0 V
3682 640 M
0 3594 V
3651 640 M
62 0 V
-62 3594 R
62 0 V
3221 640 M
0 3211 V
3190 640 M
62 0 V
-62 3211 R
62 0 V
2815 2561 M
0 1974 V
2784 2561 M
62 0 V
-62 1974 R
62 0 V
2452 640 M
0 3280 V
2421 640 M
62 0 V
-62 3280 R
62 0 V
11790 7644 Pls
8630 4504 Pls
7010 2793 Pls
6240 3835 Pls
5611 3354 Pls
4842 2492 Pls
4213 2947 Pls
3682 3313 Pls
3221 2402 Pls
2815 3935 Pls
2452 2689 Pls
12063 1303 Pls
0.500 UP
1.000 UL
LT1
LCb setrgbcolor
LT1
11808 1103 M
511 0 V
-511 31 R
0 -62 V
511 62 R
0 -62 V
-529 4333 R
0 1931 V
-31 -1931 R
62 0 V
-62 1931 R
62 0 V
8630 640 M
0 4016 V
8599 640 M
62 0 V
-62 4016 R
62 0 V
7010 3214 M
0 1862 V
6979 3214 M
62 0 V
-62 1862 R
62 0 V
6240 2459 M
0 2460 V
6209 2459 M
62 0 V
-62 2460 R
62 0 V
5611 640 M
0 3800 V
5580 640 M
62 0 V
-62 3800 R
62 0 V
4842 640 M
0 3675 V
4811 640 M
62 0 V
-62 3675 R
62 0 V
4213 1588 M
0 3289 V
4182 1588 M
62 0 V
-62 3289 R
62 0 V
3682 640 M
0 3758 V
3651 640 M
62 0 V
-62 3758 R
62 0 V
3221 3342 M
0 1717 V
3190 3342 M
62 0 V
-62 1717 R
62 0 V
2815 640 M
0 3919 V
2784 640 M
62 0 V
-62 3919 R
62 0 V
2452 640 M
0 4092 V
2421 640 M
62 0 V
-62 4092 R
62 0 V
11790 6743 Crs
8630 3689 Crs
7010 4494 Crs
6240 4260 Crs
5611 2990 Crs
4842 2225 Crs
4213 4157 Crs
3682 2842 Crs
3221 4501 Crs
2815 3422 Crs
2452 3512 Crs
12063 1103 Crs
0.500 UP
1.000 UL
LT2
LCb setrgbcolor
LT2
11808 903 M
511 0 V
-511 31 R
0 -62 V
511 62 R
0 -62 V
-529 6071 R
0 1341 V
-31 -1341 R
62 0 V
-62 1341 R
62 0 V
8630 2702 M
0 2702 V
8599 2702 M
62 0 V
-62 2702 R
62 0 V
7010 640 M
0 4567 V
6979 640 M
62 0 V
-62 4567 R
62 0 V
6240 640 M
0 4198 V
6209 640 M
62 0 V
-62 4198 R
62 0 V
5611 4368 M
0 1353 V
5580 4368 M
62 0 V
-62 1353 R
62 0 V
4842 4483 M
0 1274 V
4811 4483 M
62 0 V
-62 1274 R
62 0 V
4213 640 M
0 4241 V
4182 640 M
62 0 V
-62 4241 R
62 0 V
3682 3569 M
0 2002 V
3651 3569 M
62 0 V
-62 2002 R
62 0 V
3221 640 M
0 4723 V
3190 640 M
62 0 V
-62 4723 R
62 0 V
2815 640 M
0 4810 V
2784 640 M
62 0 V
-62 4810 R
62 0 V
2452 4599 M
0 1668 V
2421 4599 M
62 0 V
-62 1668 R
62 0 V
11790 7803 Star
8630 4723 Star
7010 4346 Star
6240 3323 Star
5611 5237 Star
4842 5292 Star
4213 3215 Star
3682 4967 Star
3221 4457 Star
2815 4464 Star
2452 5718 Star
12063 903 Star
1.000 UL
LTb
1460 8399 N
0 -7759 V
11099 0 V
0 7759 V
-11099 0 V
Z stroke
0.500 UP
1.000 UL
LTb
stroke
grestore
end
showpage
  }}%
  \put(11688,903){\makebox(0,0)[r]{\strut{}$\mathcal{B}_2$}}%
  \put(11688,1103){\makebox(0,0)[r]{\strut{}$\mathcal{B}_1$}}%
  \put(11688,1303){\makebox(0,0)[r]{\strut{}$\mathcal{B}_0$}}%
  \put(7009,140){\makebox(0,0){\strut{}Bin Width (mfp)}}%
  \put(280,4519){%
  \special{ps: gsave currentpoint currentpoint translate
270 rotate neg exch neg exch translate}%
  \makebox(0,0){\strut{}Eigenvalue Bias}%
  \special{ps: currentpoint grestore moveto}%
  }%
  \put(9400,440){\makebox(0,0){\strut{} 1}}%
  \put(1460,440){\makebox(0,0){\strut{} 0.1}}%
  \put(1340,8399){\makebox(0,0)[r]{\strut{} 0.01}}%
  \put(1340,5813){\makebox(0,0)[r]{\strut{} 0.001}}%
  \put(1340,3226){\makebox(0,0)[r]{\strut{} 0.0001}}%
  \put(1340,640){\makebox(0,0)[r]{\strut{} 1e-05}}%
\end{picture}%
\endgroup
\endinput

    \caption{Eigenvalue bias for second-order spatial discretization with 1 million particles tracked.}
    \label{fig:BiasSecondOrder1E6}
\end{sidewaysfigure}
\end{comment}

\begin{table}[p] \centering
    \subfloat[First-order ($\Pi$) spatial discretization]{%
    \begin{tabular}{cccccc}
        \toprule
        \# Bins & Bin Width (mfp) & $\lambda_{\Pi}$ & $\sigma_{\Pi}$ & $\mathcal{B}_{\Pi}$ & FOM ($\Pi$) \\
        \midrule
         10 & 2.00 & 4.8003 & 2.1\e{-4} & 2.7\e{-2} & 838.5 \\
         25 & 0.80 & 4.8232 & 2.2\e{-4} & 4.5\e{-3} & 751.1 \\
         40 & 0.50 & 4.8258 & 2.1\e{-4} & 1.9\e{-3} & 832.9 \\
         50 & 0.40 & 4.8267 & 2.1\e{-4} & 1.1\e{-3} & 796.1 \\
         60 & 0.33 & 4.8270 & 2.1\e{-4} & 7.3\e{-4} & 751.0 \\
         75 & 0.27 & 4.8269 & 2.1\e{-4} & 8.3\e{-4} & 787.5 \\
         90 & 0.22 & 4.8275 & 2.2\e{-4} & 2.7\e{-4} & 694.3 \\
        105 & 0.19 & 4.8274 & 2.1\e{-4} & 3.4\e{-4} & 768.8 \\
        120 & 0.17 & 4.8276 & 2.0\e{-4} & 1.6\e{-4} & 846.8 \\
        135 & 0.15 & 4.8275 & 2.0\e{-4} & 2.6\e{-4} & 793.9 \\
        150 & 0.13 & 4.8280 & 2.1\e{-4} & 2.7\e{-4} & 746.8 \\        
        \bottomrule
    \end{tabular}
    \label{tab:Bias0Histogram1E6} }

    \subfloat[Second-order ($\Lin$) spatial discretization]{%
    \begin{tabular}{cccccc}
        \toprule
        \# Bins & Bin Width (mfp) & $\lambda_{\Lin}$ & $\sigma_{\Lin}$ & $\mathcal{B}_{\Lin}$ & FOM ($\Lin$) \\
        \midrule
         10 & 2.00 & 4.8329 & 4.1\e{-3} & 5.1\e{-3} & 2.0 \\
         25 & 0.80 & 4.8274 & 1.5\e{-4} & 3.1\e{-4} & 1607.8 \\
         40 & 0.50 & 4.8278 & 1.4\e{-4} & 6.8\e{-5} & 1742.0 \\
         50 & 0.40 & 4.8276 & 1.5\e{-4} & 1.7\e{-4} & 1512.8 \\
         60 & 0.33 & 4.8276 & 1.3\e{-4} & 1.1\e{-4} & 1832.8 \\
         75 & 0.27 & 4.8277 & 1.4\e{-4} & 5.2\e{-5} & 1692.6 \\
         90 & 0.22 & 4.8278 & 1.4\e{-4} & 7.8\e{-5} & 1730.6 \\
        105 & 0.19 & 4.8279 & 1.4\e{-4} & 1.1\e{-4} & 1701.9 \\
        120 & 0.17 & 4.8278 & 1.3\e{-4} & 4.8\e{-5} & 1985.2 \\
        135 & 0.15 & 4.8279 & 1.3\e{-4} & 1.9\e{-4} & 1757.3 \\
        150 & 0.13 & 4.8277 & 1.2\e{-4} & 6.2\e{-5} & 2015.0 \\        
        \bottomrule
    \end{tabular}
    \label{tab:Bias0Linear1E6} }
    \caption{Error ($\mathcal{B}$) in the fundamental eigenvalue estimate ($\lambda$) for first-order \subref{tab:Bias0Histogram} and second-order \subref{tab:Bias0Linear} discretization as a function of the bin width.  Figure of merit is also given for first and second-order spatial discretizations.  1E6 particles were tracked in each iteration.}
\end{table}

\begin{comment}
\begin{table}[ht] \centering
    \begin{tabular}{cccccc}
        \toprule
        \# Bins & Bin Width (mfp) & $\lambda_{\Pi}$ & $\sigma_{\Pi}$ & $\mathcal{B}_{\Pi}$ & FOM ($\Pi$) \\
        \midrule
         10 & 2.00 & 4.2853 & 2.0\e{-4} & 2.7\e{-2} & 838.5 \\
         25 & 0.80 & 4.3658 & 1.9\e{-4} & 4.5\e{-3} & 751.1 \\
         40 & 0.50 & 4.3758 & 1.9\e{-4} & 1.9\e{-3} & 832.9 \\
         50 & 0.40 & 4.3784 & 1.9\e{-4} & 1.1\e{-3} & 796.1 \\
         60 & 0.33 & 4.3802 & 2.0\e{-4} & 7.3\e{-4} & 751.0 \\
         75 & 0.27 & 4.3811 & 2.1\e{-4} & 8.3\e{-4} & 787.5 \\
         90 & 0.22 & 4.3818 & 1.9\e{-4} & 2.7\e{-4} & 694.3 \\
        105 & 0.19 & 4.3820 & 2.1\e{-4} & 3.4\e{-4} & 768.8 \\
        120 & 0.17 & 4.3821 & 2.0\e{-4} & 1.6\e{-4} & 846.8 \\
        135 & 0.15 & 4.3823 & 2.0\e{-4} & 2.6\e{-4} & 793.9 \\
        150 & 0.13 & 4.3824 & 2.0\e{-4} & 2.7\e{-4} & 746.8 \\        
        \bottomrule
    \end{tabular}
    \caption{First higher order eigenvalue bias for first-order ($\Pi$) spatial discretization with 1 million particles tracked.  }
    \label{tab:Bias1Histogram1E6}
\end{table}
\begin{table}[hb] \centering
    \begin{tabular}{cccccc}
        \toprule
        \# Bins & Bin Width (mfp) & $\lambda_{\Lin}$ & $\sigma_{\Lin}$ & $\mathcal{B}_{\Lin}$ & FOM ($\Lin$) \\
        \midrule
         10 & 2.00 & 4.3854 & 1.6\e{-3} & 5.1\e{-3} & 2.0 \\
         25 & 0.80 & 4.3830 & 2.1\e{-4} & 3.1\e{-4} & 1607.8 \\
         40 & 0.50 & 4.3834 & 2.1\e{-4} & 6.8\e{-5} & 1742.0 \\
         50 & 0.40 & 4.3829 & 2.0\e{-4} & 1.7\e{-4} & 1512.8 \\
         60 & 0.33 & 4.3830 & 2.1\e{-4} & 1.1\e{-4} & 1832.8 \\
         75 & 0.27 & 4.3831 & 2.2\e{-4} & 5.2\e{-5} & 1692.6 \\
         90 & 0.22 & 4.3833 & 2.1\e{-4} & 7.8\e{-5} & 1730.6 \\
        105 & 0.19 & 4.3830 & 2.1\e{-4} & 1.1\e{-4} & 1701.9 \\
        120 & 0.17 & 4.3828 & 2.0\e{-4} & 4.8\e{-5} & 1985.2 \\
        135 & 0.15 & 4.3832 & 2.1\e{-4} & 1.9\e{-4} & 1757.3 \\
        150 & 0.13 & 4.3832 & 2.5\e{-4} & 6.2\e{-5} & 2015.0 \\        
        \bottomrule
    \end{tabular}
    \caption{First higher order eigenvalue bias for second-order ($\Lin$) spatial discretization with 1 million particles tracked.  }
    \label{tab:Bias1Linear1E6}
\end{table}

\begin{table}[ht] \centering
    \begin{tabular}{cccccc}
        \toprule
        \# Bins & Bin Width (mfp) & $\lambda_{\Pi}$ & $\sigma_{\Pi}$ & $\mathcal{B}_{\Pi}$ & FOM ($\Pi$) \\
        \midrule
         10 & 2.00 & 3.6321 & 1.7\e{-4} & 2.7\e{-2} & 838.5 \\
         25 & 0.80 & 3.7830 & 1.8\e{-4} & 4.5\e{-3} & 751.1 \\
         40 & 0.50 & 3.8037 & 1.8\e{-4} & 1.9\e{-3} & 832.9 \\
         50 & 0.40 & 3.8081 & 1.8\e{-4} & 1.1\e{-3} & 796.1 \\
         60 & 0.33 & 3.8111 & 1.9\e{-4} & 7.3\e{-4} & 751.0 \\
         75 & 0.27 & 3.8133 & 1.9\e{-4} & 8.3\e{-4} & 787.5 \\
         90 & 0.22 & 3.8148 & 1.9\e{-4} & 2.7\e{-4} & 694.3 \\
        105 & 0.19 & 3.8151 & 1.9\e{-4} & 3.4\e{-4} & 768.8 \\
        120 & 0.17 & 3.8159 & 1.8\e{-4} & 1.6\e{-4} & 846.8 \\
        135 & 0.15 & 3.8164 & 1.8\e{-4} & 2.6\e{-4} & 793.9 \\
        150 & 0.13 & 3.8166 & 1.9\e{-4} & 2.7\e{-4} & 746.8 \\        
        \bottomrule
    \end{tabular}
    \caption{Second higher order eigenvalue bias for first-order ($\Pi$) spatial discretization with 1 million particles tracked.  }
    \label{tab:Bias2Histogram1E6}
\end{table}
\begin{table}[hb] \centering
    \begin{tabular}{cccccc}
        \toprule
        \# Bins & Bin Width (mfp) & $\lambda_{\Lin}$ & $\sigma_{\Lin}$ & $\mathcal{B}_{\Lin}$ & FOM ($\Lin$) \\
        \midrule
         10 & 2.00 & 3.8234 & 3.1\e{-3} & 5.1\e{-3} & 2.0 \\
         25 & 0.80 & 3.8171 & 3.2\e{-4} & 3.1\e{-4} & 1607.8 \\
         40 & 0.50 & 3.8177 & 3.1\e{-4} & 6.8\e{-5} & 1742.0 \\
         50 & 0.40 & 3.8174 & 3.1\e{-4} & 1.7\e{-4} & 1512.8 \\
         60 & 0.33 & 3.8169 & 3.2\e{-4} & 1.1\e{-4} & 1832.8 \\
         75 & 0.27 & 3.8168 & 3.2\e{-4} & 5.2\e{-5} & 1692.6 \\
         90 & 0.22 & 3.8174 & 3.4\e{-4} & 7.8\e{-5} & 1730.6 \\
        105 & 0.19 & 3.8179 & 3.4\e{-4} & 1.1\e{-4} & 1701.9 \\
        120 & 0.17 & 3.8172 & 3.7\e{-4} & 4.8\e{-5} & 1985.2 \\
        135 & 0.15 & 3.8178 & 4.2\e{-4} & 1.9\e{-4} & 1757.3 \\
        150 & 0.13 & 3.8165 & 5.8\e{-4} & 6.2\e{-5} & 2015.0 \\        
        \bottomrule
    \end{tabular}
    \caption{Second higher order eigenvalue bias for second-order ($\Lin$) spatial discretization with 1 million particles tracked.  }
    \label{tab:Bias2Linear1E6}
\end{table}
\end{comment}

\begin{sidewaysfigure} \centering
    % GNUPLOT: LaTeX picture with Postscript
\begingroup%
\makeatletter%
\newcommand{\GNUPLOTspecial}{%
  \@sanitize\catcode`\%=14\relax\special}%
\setlength{\unitlength}{0.0500bp}%
\begin{picture}(12960,8640)(0,0)%
  {\GNUPLOTspecial{"
%!PS-Adobe-2.0 EPSF-2.0
%%Title: BiasFOM1E6.tex
%%Creator: gnuplot 4.3 patchlevel 0
%%CreationDate: Wed Aug 12 16:28:22 2009
%%DocumentFonts: 
%%BoundingBox: 0 0 648 432
%%EndComments
%%BeginProlog
/gnudict 256 dict def
gnudict begin
%
% The following true/false flags may be edited by hand if desired.
% The unit line width and grayscale image gamma correction may also be changed.
%
/Color true def
/Blacktext true def
/Solid true def
/Dashlength 1 def
/Landscape false def
/Level1 false def
/Rounded false def
/ClipToBoundingBox false def
/TransparentPatterns false def
/gnulinewidth 5.000 def
/userlinewidth gnulinewidth def
/Gamma 1.0 def
%
/vshift -66 def
/dl1 {
  10.0 Dashlength mul mul
  Rounded { currentlinewidth 0.75 mul sub dup 0 le { pop 0.01 } if } if
} def
/dl2 {
  10.0 Dashlength mul mul
  Rounded { currentlinewidth 0.75 mul add } if
} def
/hpt_ 31.5 def
/vpt_ 31.5 def
/hpt hpt_ def
/vpt vpt_ def
Level1 {} {
/SDict 10 dict def
systemdict /pdfmark known not {
  userdict /pdfmark systemdict /cleartomark get put
} if
SDict begin [
  /Title (BiasFOM1E6.tex)
  /Subject (gnuplot plot)
  /Creator (gnuplot 4.3 patchlevel 0)
  /Author (Jeremy Conlin)
%  /Producer (gnuplot)
%  /Keywords ()
  /CreationDate (Wed Aug 12 16:28:22 2009)
  /DOCINFO pdfmark
end
} ifelse
/doclip {
  ClipToBoundingBox {
    newpath 0 0 moveto 648 0 lineto 648 432 lineto 0 432 lineto closepath
    clip
  } if
} def
%
% Gnuplot Prolog Version 4.2 (November 2007)
%
/M {moveto} bind def
/L {lineto} bind def
/R {rmoveto} bind def
/V {rlineto} bind def
/N {newpath moveto} bind def
/Z {closepath} bind def
/C {setrgbcolor} bind def
/f {rlineto fill} bind def
/Gshow {show} def   % May be redefined later in the file to support UTF-8
/vpt2 vpt 2 mul def
/hpt2 hpt 2 mul def
/Lshow {currentpoint stroke M 0 vshift R 
	Blacktext {gsave 0 setgray show grestore} {show} ifelse} def
/Rshow {currentpoint stroke M dup stringwidth pop neg vshift R
	Blacktext {gsave 0 setgray show grestore} {show} ifelse} def
/Cshow {currentpoint stroke M dup stringwidth pop -2 div vshift R 
	Blacktext {gsave 0 setgray show grestore} {show} ifelse} def
/UP {dup vpt_ mul /vpt exch def hpt_ mul /hpt exch def
  /hpt2 hpt 2 mul def /vpt2 vpt 2 mul def} def
/DL {Color {setrgbcolor Solid {pop []} if 0 setdash}
 {pop pop pop 0 setgray Solid {pop []} if 0 setdash} ifelse} def
/BL {stroke userlinewidth 2 mul setlinewidth
	Rounded {1 setlinejoin 1 setlinecap} if} def
/AL {stroke userlinewidth 2 div setlinewidth
	Rounded {1 setlinejoin 1 setlinecap} if} def
/UL {dup gnulinewidth mul /userlinewidth exch def
	dup 1 lt {pop 1} if 10 mul /udl exch def} def
/PL {stroke userlinewidth setlinewidth
	Rounded {1 setlinejoin 1 setlinecap} if} def
% Default Line colors
/LCw {1 1 1} def
/LCb {0 0 0} def
/LCa {0 0 0} def
/LC0 {1 0 0} def
/LC1 {0 1 0} def
/LC2 {0 0 1} def
/LC3 {1 0 1} def
/LC4 {0 1 1} def
/LC5 {1 1 0} def
/LC6 {0 0 0} def
/LC7 {1 0.3 0} def
/LC8 {0.5 0.5 0.5} def
% Default Line Types
/LTw {PL [] 1 setgray} def
/LTb {BL [] LCb DL} def
/LTa {AL [1 udl mul 2 udl mul] 0 setdash LCa setrgbcolor} def
/LT0 {PL [] LC0 DL} def
/LT1 {PL [4 dl1 2 dl2] LC1 DL} def
/LT2 {PL [2 dl1 3 dl2] LC2 DL} def
/LT3 {PL [1 dl1 1.5 dl2] LC3 DL} def
/LT4 {PL [6 dl1 2 dl2 1 dl1 2 dl2] LC4 DL} def
/LT5 {PL [3 dl1 3 dl2 1 dl1 3 dl2] LC5 DL} def
/LT6 {PL [2 dl1 2 dl2 2 dl1 6 dl2] LC6 DL} def
/LT7 {PL [1 dl1 2 dl2 6 dl1 2 dl2 1 dl1 2 dl2] LC7 DL} def
/LT8 {PL [2 dl1 2 dl2 2 dl1 2 dl2 2 dl1 2 dl2 2 dl1 4 dl2] LC8 DL} def
/Pnt {stroke [] 0 setdash gsave 1 setlinecap M 0 0 V stroke grestore} def
/Dia {stroke [] 0 setdash 2 copy vpt add M
  hpt neg vpt neg V hpt vpt neg V
  hpt vpt V hpt neg vpt V closepath stroke
  Pnt} def
/Pls {stroke [] 0 setdash vpt sub M 0 vpt2 V
  currentpoint stroke M
  hpt neg vpt neg R hpt2 0 V stroke
 } def
/Box {stroke [] 0 setdash 2 copy exch hpt sub exch vpt add M
  0 vpt2 neg V hpt2 0 V 0 vpt2 V
  hpt2 neg 0 V closepath stroke
  Pnt} def
/Crs {stroke [] 0 setdash exch hpt sub exch vpt add M
  hpt2 vpt2 neg V currentpoint stroke M
  hpt2 neg 0 R hpt2 vpt2 V stroke} def
/TriU {stroke [] 0 setdash 2 copy vpt 1.12 mul add M
  hpt neg vpt -1.62 mul V
  hpt 2 mul 0 V
  hpt neg vpt 1.62 mul V closepath stroke
  Pnt} def
/Star {2 copy Pls Crs} def
/BoxF {stroke [] 0 setdash exch hpt sub exch vpt add M
  0 vpt2 neg V hpt2 0 V 0 vpt2 V
  hpt2 neg 0 V closepath fill} def
/TriUF {stroke [] 0 setdash vpt 1.12 mul add M
  hpt neg vpt -1.62 mul V
  hpt 2 mul 0 V
  hpt neg vpt 1.62 mul V closepath fill} def
/TriD {stroke [] 0 setdash 2 copy vpt 1.12 mul sub M
  hpt neg vpt 1.62 mul V
  hpt 2 mul 0 V
  hpt neg vpt -1.62 mul V closepath stroke
  Pnt} def
/TriDF {stroke [] 0 setdash vpt 1.12 mul sub M
  hpt neg vpt 1.62 mul V
  hpt 2 mul 0 V
  hpt neg vpt -1.62 mul V closepath fill} def
/DiaF {stroke [] 0 setdash vpt add M
  hpt neg vpt neg V hpt vpt neg V
  hpt vpt V hpt neg vpt V closepath fill} def
/Pent {stroke [] 0 setdash 2 copy gsave
  translate 0 hpt M 4 {72 rotate 0 hpt L} repeat
  closepath stroke grestore Pnt} def
/PentF {stroke [] 0 setdash gsave
  translate 0 hpt M 4 {72 rotate 0 hpt L} repeat
  closepath fill grestore} def
/Circle {stroke [] 0 setdash 2 copy
  hpt 0 360 arc stroke Pnt} def
/CircleF {stroke [] 0 setdash hpt 0 360 arc fill} def
/C0 {BL [] 0 setdash 2 copy moveto vpt 90 450 arc} bind def
/C1 {BL [] 0 setdash 2 copy moveto
	2 copy vpt 0 90 arc closepath fill
	vpt 0 360 arc closepath} bind def
/C2 {BL [] 0 setdash 2 copy moveto
	2 copy vpt 90 180 arc closepath fill
	vpt 0 360 arc closepath} bind def
/C3 {BL [] 0 setdash 2 copy moveto
	2 copy vpt 0 180 arc closepath fill
	vpt 0 360 arc closepath} bind def
/C4 {BL [] 0 setdash 2 copy moveto
	2 copy vpt 180 270 arc closepath fill
	vpt 0 360 arc closepath} bind def
/C5 {BL [] 0 setdash 2 copy moveto
	2 copy vpt 0 90 arc
	2 copy moveto
	2 copy vpt 180 270 arc closepath fill
	vpt 0 360 arc} bind def
/C6 {BL [] 0 setdash 2 copy moveto
	2 copy vpt 90 270 arc closepath fill
	vpt 0 360 arc closepath} bind def
/C7 {BL [] 0 setdash 2 copy moveto
	2 copy vpt 0 270 arc closepath fill
	vpt 0 360 arc closepath} bind def
/C8 {BL [] 0 setdash 2 copy moveto
	2 copy vpt 270 360 arc closepath fill
	vpt 0 360 arc closepath} bind def
/C9 {BL [] 0 setdash 2 copy moveto
	2 copy vpt 270 450 arc closepath fill
	vpt 0 360 arc closepath} bind def
/C10 {BL [] 0 setdash 2 copy 2 copy moveto vpt 270 360 arc closepath fill
	2 copy moveto
	2 copy vpt 90 180 arc closepath fill
	vpt 0 360 arc closepath} bind def
/C11 {BL [] 0 setdash 2 copy moveto
	2 copy vpt 0 180 arc closepath fill
	2 copy moveto
	2 copy vpt 270 360 arc closepath fill
	vpt 0 360 arc closepath} bind def
/C12 {BL [] 0 setdash 2 copy moveto
	2 copy vpt 180 360 arc closepath fill
	vpt 0 360 arc closepath} bind def
/C13 {BL [] 0 setdash 2 copy moveto
	2 copy vpt 0 90 arc closepath fill
	2 copy moveto
	2 copy vpt 180 360 arc closepath fill
	vpt 0 360 arc closepath} bind def
/C14 {BL [] 0 setdash 2 copy moveto
	2 copy vpt 90 360 arc closepath fill
	vpt 0 360 arc} bind def
/C15 {BL [] 0 setdash 2 copy vpt 0 360 arc closepath fill
	vpt 0 360 arc closepath} bind def
/Rec {newpath 4 2 roll moveto 1 index 0 rlineto 0 exch rlineto
	neg 0 rlineto closepath} bind def
/Square {dup Rec} bind def
/Bsquare {vpt sub exch vpt sub exch vpt2 Square} bind def
/S0 {BL [] 0 setdash 2 copy moveto 0 vpt rlineto BL Bsquare} bind def
/S1 {BL [] 0 setdash 2 copy vpt Square fill Bsquare} bind def
/S2 {BL [] 0 setdash 2 copy exch vpt sub exch vpt Square fill Bsquare} bind def
/S3 {BL [] 0 setdash 2 copy exch vpt sub exch vpt2 vpt Rec fill Bsquare} bind def
/S4 {BL [] 0 setdash 2 copy exch vpt sub exch vpt sub vpt Square fill Bsquare} bind def
/S5 {BL [] 0 setdash 2 copy 2 copy vpt Square fill
	exch vpt sub exch vpt sub vpt Square fill Bsquare} bind def
/S6 {BL [] 0 setdash 2 copy exch vpt sub exch vpt sub vpt vpt2 Rec fill Bsquare} bind def
/S7 {BL [] 0 setdash 2 copy exch vpt sub exch vpt sub vpt vpt2 Rec fill
	2 copy vpt Square fill Bsquare} bind def
/S8 {BL [] 0 setdash 2 copy vpt sub vpt Square fill Bsquare} bind def
/S9 {BL [] 0 setdash 2 copy vpt sub vpt vpt2 Rec fill Bsquare} bind def
/S10 {BL [] 0 setdash 2 copy vpt sub vpt Square fill 2 copy exch vpt sub exch vpt Square fill
	Bsquare} bind def
/S11 {BL [] 0 setdash 2 copy vpt sub vpt Square fill 2 copy exch vpt sub exch vpt2 vpt Rec fill
	Bsquare} bind def
/S12 {BL [] 0 setdash 2 copy exch vpt sub exch vpt sub vpt2 vpt Rec fill Bsquare} bind def
/S13 {BL [] 0 setdash 2 copy exch vpt sub exch vpt sub vpt2 vpt Rec fill
	2 copy vpt Square fill Bsquare} bind def
/S14 {BL [] 0 setdash 2 copy exch vpt sub exch vpt sub vpt2 vpt Rec fill
	2 copy exch vpt sub exch vpt Square fill Bsquare} bind def
/S15 {BL [] 0 setdash 2 copy Bsquare fill Bsquare} bind def
/D0 {gsave translate 45 rotate 0 0 S0 stroke grestore} bind def
/D1 {gsave translate 45 rotate 0 0 S1 stroke grestore} bind def
/D2 {gsave translate 45 rotate 0 0 S2 stroke grestore} bind def
/D3 {gsave translate 45 rotate 0 0 S3 stroke grestore} bind def
/D4 {gsave translate 45 rotate 0 0 S4 stroke grestore} bind def
/D5 {gsave translate 45 rotate 0 0 S5 stroke grestore} bind def
/D6 {gsave translate 45 rotate 0 0 S6 stroke grestore} bind def
/D7 {gsave translate 45 rotate 0 0 S7 stroke grestore} bind def
/D8 {gsave translate 45 rotate 0 0 S8 stroke grestore} bind def
/D9 {gsave translate 45 rotate 0 0 S9 stroke grestore} bind def
/D10 {gsave translate 45 rotate 0 0 S10 stroke grestore} bind def
/D11 {gsave translate 45 rotate 0 0 S11 stroke grestore} bind def
/D12 {gsave translate 45 rotate 0 0 S12 stroke grestore} bind def
/D13 {gsave translate 45 rotate 0 0 S13 stroke grestore} bind def
/D14 {gsave translate 45 rotate 0 0 S14 stroke grestore} bind def
/D15 {gsave translate 45 rotate 0 0 S15 stroke grestore} bind def
/DiaE {stroke [] 0 setdash vpt add M
  hpt neg vpt neg V hpt vpt neg V
  hpt vpt V hpt neg vpt V closepath stroke} def
/BoxE {stroke [] 0 setdash exch hpt sub exch vpt add M
  0 vpt2 neg V hpt2 0 V 0 vpt2 V
  hpt2 neg 0 V closepath stroke} def
/TriUE {stroke [] 0 setdash vpt 1.12 mul add M
  hpt neg vpt -1.62 mul V
  hpt 2 mul 0 V
  hpt neg vpt 1.62 mul V closepath stroke} def
/TriDE {stroke [] 0 setdash vpt 1.12 mul sub M
  hpt neg vpt 1.62 mul V
  hpt 2 mul 0 V
  hpt neg vpt -1.62 mul V closepath stroke} def
/PentE {stroke [] 0 setdash gsave
  translate 0 hpt M 4 {72 rotate 0 hpt L} repeat
  closepath stroke grestore} def
/CircE {stroke [] 0 setdash 
  hpt 0 360 arc stroke} def
/Opaque {gsave closepath 1 setgray fill grestore 0 setgray closepath} def
/DiaW {stroke [] 0 setdash vpt add M
  hpt neg vpt neg V hpt vpt neg V
  hpt vpt V hpt neg vpt V Opaque stroke} def
/BoxW {stroke [] 0 setdash exch hpt sub exch vpt add M
  0 vpt2 neg V hpt2 0 V 0 vpt2 V
  hpt2 neg 0 V Opaque stroke} def
/TriUW {stroke [] 0 setdash vpt 1.12 mul add M
  hpt neg vpt -1.62 mul V
  hpt 2 mul 0 V
  hpt neg vpt 1.62 mul V Opaque stroke} def
/TriDW {stroke [] 0 setdash vpt 1.12 mul sub M
  hpt neg vpt 1.62 mul V
  hpt 2 mul 0 V
  hpt neg vpt -1.62 mul V Opaque stroke} def
/PentW {stroke [] 0 setdash gsave
  translate 0 hpt M 4 {72 rotate 0 hpt L} repeat
  Opaque stroke grestore} def
/CircW {stroke [] 0 setdash 
  hpt 0 360 arc Opaque stroke} def
/BoxFill {gsave Rec 1 setgray fill grestore} def
/Density {
  /Fillden exch def
  currentrgbcolor
  /ColB exch def /ColG exch def /ColR exch def
  /ColR ColR Fillden mul Fillden sub 1 add def
  /ColG ColG Fillden mul Fillden sub 1 add def
  /ColB ColB Fillden mul Fillden sub 1 add def
  ColR ColG ColB setrgbcolor} def
/BoxColFill {gsave Rec PolyFill} def
/PolyFill {gsave Density fill grestore grestore} def
/h {rlineto rlineto rlineto gsave closepath fill grestore} bind def
%
% PostScript Level 1 Pattern Fill routine for rectangles
% Usage: x y w h s a XX PatternFill
%	x,y = lower left corner of box to be filled
%	w,h = width and height of box
%	  a = angle in degrees between lines and x-axis
%	 XX = 0/1 for no/yes cross-hatch
%
/PatternFill {gsave /PFa [ 9 2 roll ] def
  PFa 0 get PFa 2 get 2 div add PFa 1 get PFa 3 get 2 div add translate
  PFa 2 get -2 div PFa 3 get -2 div PFa 2 get PFa 3 get Rec
  gsave 1 setgray fill grestore clip
  currentlinewidth 0.5 mul setlinewidth
  /PFs PFa 2 get dup mul PFa 3 get dup mul add sqrt def
  0 0 M PFa 5 get rotate PFs -2 div dup translate
  0 1 PFs PFa 4 get div 1 add floor cvi
	{PFa 4 get mul 0 M 0 PFs V} for
  0 PFa 6 get ne {
	0 1 PFs PFa 4 get div 1 add floor cvi
	{PFa 4 get mul 0 2 1 roll M PFs 0 V} for
 } if
  stroke grestore} def
%
/languagelevel where
 {pop languagelevel} {1} ifelse
 2 lt
	{/InterpretLevel1 true def}
	{/InterpretLevel1 Level1 def}
 ifelse
%
% PostScript level 2 pattern fill definitions
%
/Level2PatternFill {
/Tile8x8 {/PaintType 2 /PatternType 1 /TilingType 1 /BBox [0 0 8 8] /XStep 8 /YStep 8}
	bind def
/KeepColor {currentrgbcolor [/Pattern /DeviceRGB] setcolorspace} bind def
<< Tile8x8
 /PaintProc {0.5 setlinewidth pop 0 0 M 8 8 L 0 8 M 8 0 L stroke} 
>> matrix makepattern
/Pat1 exch def
<< Tile8x8
 /PaintProc {0.5 setlinewidth pop 0 0 M 8 8 L 0 8 M 8 0 L stroke
	0 4 M 4 8 L 8 4 L 4 0 L 0 4 L stroke}
>> matrix makepattern
/Pat2 exch def
<< Tile8x8
 /PaintProc {0.5 setlinewidth pop 0 0 M 0 8 L
	8 8 L 8 0 L 0 0 L fill}
>> matrix makepattern
/Pat3 exch def
<< Tile8x8
 /PaintProc {0.5 setlinewidth pop -4 8 M 8 -4 L
	0 12 M 12 0 L stroke}
>> matrix makepattern
/Pat4 exch def
<< Tile8x8
 /PaintProc {0.5 setlinewidth pop -4 0 M 8 12 L
	0 -4 M 12 8 L stroke}
>> matrix makepattern
/Pat5 exch def
<< Tile8x8
 /PaintProc {0.5 setlinewidth pop -2 8 M 4 -4 L
	0 12 M 8 -4 L 4 12 M 10 0 L stroke}
>> matrix makepattern
/Pat6 exch def
<< Tile8x8
 /PaintProc {0.5 setlinewidth pop -2 0 M 4 12 L
	0 -4 M 8 12 L 4 -4 M 10 8 L stroke}
>> matrix makepattern
/Pat7 exch def
<< Tile8x8
 /PaintProc {0.5 setlinewidth pop 8 -2 M -4 4 L
	12 0 M -4 8 L 12 4 M 0 10 L stroke}
>> matrix makepattern
/Pat8 exch def
<< Tile8x8
 /PaintProc {0.5 setlinewidth pop 0 -2 M 12 4 L
	-4 0 M 12 8 L -4 4 M 8 10 L stroke}
>> matrix makepattern
/Pat9 exch def
/Pattern1 {PatternBgnd KeepColor Pat1 setpattern} bind def
/Pattern2 {PatternBgnd KeepColor Pat2 setpattern} bind def
/Pattern3 {PatternBgnd KeepColor Pat3 setpattern} bind def
/Pattern4 {PatternBgnd KeepColor Landscape {Pat5} {Pat4} ifelse setpattern} bind def
/Pattern5 {PatternBgnd KeepColor Landscape {Pat4} {Pat5} ifelse setpattern} bind def
/Pattern6 {PatternBgnd KeepColor Landscape {Pat9} {Pat6} ifelse setpattern} bind def
/Pattern7 {PatternBgnd KeepColor Landscape {Pat8} {Pat7} ifelse setpattern} bind def
} def
%
%
%End of PostScript Level 2 code
%
/PatternBgnd {
  TransparentPatterns {} {gsave 1 setgray fill grestore} ifelse
} def
%
% Substitute for Level 2 pattern fill codes with
% grayscale if Level 2 support is not selected.
%
/Level1PatternFill {
/Pattern1 {0.250 Density} bind def
/Pattern2 {0.500 Density} bind def
/Pattern3 {0.750 Density} bind def
/Pattern4 {0.125 Density} bind def
/Pattern5 {0.375 Density} bind def
/Pattern6 {0.625 Density} bind def
/Pattern7 {0.875 Density} bind def
} def
%
% Now test for support of Level 2 code
%
Level1 {Level1PatternFill} {Level2PatternFill} ifelse
%
/Symbol-Oblique /Symbol findfont [1 0 .167 1 0 0] makefont
dup length dict begin {1 index /FID eq {pop pop} {def} ifelse} forall
currentdict end definefont pop
end
%%EndProlog
gnudict begin
gsave
doclip
0 0 translate
0.050 0.050 scale
0 setgray
newpath
1.000 UL
LTb
1220 640 M
63 0 V
11276 0 R
-63 0 V
1220 2192 M
63 0 V
11276 0 R
-63 0 V
1220 3744 M
63 0 V
11276 0 R
-63 0 V
1220 5295 M
63 0 V
11276 0 R
-63 0 V
1220 6847 M
63 0 V
11276 0 R
-63 0 V
1220 8399 M
63 0 V
11276 0 R
-63 0 V
1220 640 M
0 63 V
0 7696 R
0 -63 V
3797 640 M
0 63 V
0 7696 R
0 -63 V
6374 640 M
0 63 V
0 7696 R
0 -63 V
8951 640 M
0 63 V
0 7696 R
0 -63 V
11528 640 M
0 63 V
0 7696 R
0 -63 V
stroke
1220 8399 N
0 -7759 V
11339 0 V
0 7759 V
-11339 0 V
Z stroke
LCb setrgbcolor
LTb
LCb setrgbcolor
LTb
LCb setrgbcolor
LTb
LCb setrgbcolor
LTb
1.500 UP
1.000 UL
LTb
1.000 UL
LTb
11265 7736 N
0 600 V
1174 0 V
0 -600 V
-1174 0 V
Z stroke
11265 8336 M
1174 0 V
1.500 UP
stroke
LT0
LCb setrgbcolor
LT0
11745 8136 M
574 0 V
11528 3242 M
5343 2971 L
3797 3225 L
3282 3111 L
2938 2971 L
-344 113 V
2365 2795 L
-163 231 V
-123 242 V
-95 -164 V
-77 -146 V
11528 3242 Pls
5343 2971 Pls
3797 3225 Pls
3282 3111 Pls
2938 2971 Pls
2594 3084 Pls
2365 2795 Pls
2202 3026 Pls
2079 3268 Pls
1984 3104 Pls
1907 2958 Pls
12032 8136 Pls
1.500 UP
1.000 UL
LT2
LCb setrgbcolor
LT2
11745 7936 M
574 0 V
11528 646 M
5343 5630 L
3797 6046 L
3282 5335 L
-344 993 V
2594 5893 L
-229 118 V
-163 -89 V
-123 879 V
-95 -707 V
-77 800 V
11528 646 Star
5343 5630 Star
3797 6046 Star
3282 5335 Star
2938 6328 Star
2594 5893 Star
2365 6011 Star
2202 5922 Star
2079 6801 Star
1984 6094 Star
1907 6894 Star
12032 7936 Star
1.000 UL
LTb
1220 8399 N
0 -7759 V
11339 0 V
0 7759 V
-11339 0 V
Z stroke
1.500 UP
1.000 UL
LTb
stroke
grestore
end
showpage
  }}%
  \put(11625,7936){\makebox(0,0)[r]{\strut{}$\Lin$}}%
  \put(11625,8136){\makebox(0,0)[r]{\strut{}$\Pi$}}%
  \put(6889,140){\makebox(0,0){\strut{}Bin Width (mfp)}}%
  \put(280,4519){%
  \special{ps: gsave currentpoint currentpoint translate
630 rotate neg exch neg exch translate}%
  \makebox(0,0){\strut{}FOM $1/\sigma^2T$}%
  \special{ps: currentpoint grestore moveto}%
  }%
  \put(11528,440){\makebox(0,0){\strut{} 2}}%
  \put(8951,440){\makebox(0,0){\strut{} 1.5}}%
  \put(6374,440){\makebox(0,0){\strut{} 1}}%
  \put(3797,440){\makebox(0,0){\strut{} 0.5}}%
  \put(1220,440){\makebox(0,0){\strut{} 0}}%
  \put(1100,8399){\makebox(0,0)[r]{\strut{} 2500}}%
  \put(1100,6847){\makebox(0,0)[r]{\strut{} 2000}}%
  \put(1100,5295){\makebox(0,0)[r]{\strut{} 1500}}%
  \put(1100,3744){\makebox(0,0)[r]{\strut{} 1000}}%
  \put(1100,2192){\makebox(0,0)[r]{\strut{} 500}}%
  \put(1100,640){\makebox(0,0)[r]{\strut{} 0}}%
\end{picture}%
\endgroup
\endinput

    \caption{Figure of merit as a function of bin width for a slab of width 20 mfp and tracking 1E6 particles per iteration.  Included are results from a first-order ($\Pi$) and second-order ($\Lin$) approximation to the fission source.}
    \label{fig:BiasFOM1E6}
\end{sidewaysfigure}

\section{Summary} \label{sec:SpatialDiscretizationSummary}
In this chapter first and second-order accurate approximations to the fission source have been investigated.  These were demonstrated on a semi-infinite, homogeneous slab of multiplying material of width 20 mfp.  Both techniques have an error associated with the discretization.  The power method does not suffer from this kind of error.  It has been shown that for a sufficient number of spatial bins the error in the eigenvalue estimate is smaller than the standard deviation of the eigenvalue estimates.  

Using a second-order accurate approximation for the fission source is a significant improvement over the first-order accurate approximation; the standard deviation is smaller and the error in the eigenvalue estimate is an order of magnitude smaller when a sufficient number of particles is used.  The standard deviation becomes smaller for both first and second-order accurate approximations as the number of particles tracked in an iteration increases.  The error in the eigenvalue estimate is not significantly improved for the first-order accurate approximation as the number of particles tracked increases, while the error decreases for the second-order accurate approximation.

The second-order accurate approximation has much better approximation to the eigenvectors.  Because the approximation is linear in space, the eigenvector approximation is smoother in the second-order accurate approximation than in the first-order accurate approximation.  The results shown in this chapter show that the eigenvectors are nearly continuous across bin boundaries for the second-order accurate approximation.

Both the first and second-order accurate approximations have errors in the eigenvalue estimate that are less than the statistical uncertainty for bin widths in the 0.5--0.2 mfp range.  If the spatial discretization is too coarse, too much information is lost; if the discretization is too fine, the Monte Carlo transport is too noisy within each bin to give good results.  In both extreme cases an error eigenvalue estimates arises.  In the second-order accurate approximation, the figure of merit significantly decreases in these extreme case because the variance of the mean of the eigenvalue estimates is larger.  

Using a moderate number of spatial bins can give excellent results.  When tracking more particles per iteration, a finer discretization can be used.  Further work needs to be performed to identify---if possible---a general rule for the granularity of the discretization.  

