\documentclass[10pt,letterpaper]{article}

\usepackage{amsmath, amsfonts}
\usepackage{amsthm}
\usepackage[usenames,dvipsnames]{color}
\usepackage{paralist}   % For better control of itemizations
\usepackage[margin=1in]{geometry}
\usepackage{parallel}

\newtheorem{thm}{Theorem}

% New commands
\newcommand{\QR}{\ensuremath{QR} }

\title{\QR Algorithm}
\author{Jeremy Lloyd Conlin}

\begin{document}
\maketitle
\begin{abstract}
    This document is used to help formulate some of the important proofs resulting from the QR algorithm.  Many of them are exercises from David S. Watkins' \emph{Fundamentals of Matrix Computations}.
\end{abstract}

\section{General Infomation}
Arnoldi factorization
\begin{equation}
    A = \QR
    \label{eq:QRFactorization}
\end{equation}
Recombining:
\begin{equation}
    \hat{A} = RQ.
\end{equation}

\subsection{QR Iteration}
\begin{subequations}\begin{align}
    A_{m-1} &= Q_mR_m \\
    A_{m} &= R_mQ_m,
\end{align}
    \label{eq:QRIteration}
\end{subequations}
Alternatively
\begin{subequations}\begin{align}
    A_{m} &= Q_m^*A_{m-1}Q_m \\
    A_{m} &= R_mA_{m-1}R_m^{-1}.
\end{align}
    \label{eq:QRIterationAlternate}
\end{subequations}
Also
\begin{equation}
    Q_mA_m = A_{m-1}Q_m
\end{equation}

\subsection{Shifted QR Iteration}
\begin{subequations}\begin{gather}
    A_{m-1} - \nu_{m-1} I = Q_mR_m \label{eq:ShiftedQRFactorization} \\
    A_m = R_mQ_m = \nu_{m-1}I
\end{gather}
    \label{eq:ShiftedQRIteration}
\end{subequations}
Alternatively
\begin{subequations}\begin{align}
    A_m &= Q_m^*A_{m-1}Q_m \\
    A_m &= R_mA_{m-1}R_{m-1}^{-1}.
\end{align}
    \label{eq:ShifteDQRIterationAlternative}
\end{subequations}


\section{$p(A) = \hat{Q}_j \hat{R}_j$ (Exercise 6.2.36)} \label{sec:QRPolynomial}
This exercise shows several things.  First it shows that the shifted \QR algorithm creates a polynomials of $A$ with roots at the chosen shifts.
Let $A_m$ be generated by \QR algorithm:
\begin{subequations}\begin{align}
    A_{m-1} - \nu_{m-1}I = Q_mR_m \\
    A_m = R_mQ_m + \nu_{m-1}
\end{align}\end{subequations}
where $A_0 = A$.  Let $\hat{Q}_m = Q_1\cdots Q_m$ and $\hat{R}_m = R_m \cdots R_1$.

\begin{enumerate}[(a)]
    \item Show that 
        \begin{equation}
            A_m = \hat{Q}_m^*A\hat{Q}_m
        \end{equation}
        for $m=1,2,3,\ldots$.

        \begin{enumerate}[$m=1$: ]
            \item $\hat{Q}_1 = Q_1$, 
                \begin{equation}
                    \boxed{A_1 = Q_1^*AQ_1} \text{ by definition, see Eq. \ref{eq:ShifteDQRIterationAlternative} }.
                \end{equation}

            \item $\hat{Q}_2 = Q_1Q_2, \hspace{2ex} \hat{Q}_2^* = Q_2^*Q_1^*$
                \begin{subequations}\begin{align}
                    \text{we know: } A_2 &= Q_2^*A_1Q_2 \\
                     &= Q_2^*\left(Q_1^*AQ_1\right)Q_2 \\
                    A_2 &= \hat{Q}_2^*A\hat{Q}_2
                \end{align}\end{subequations}

            \item $\hat{Q}_3 = Q_1Q_2Q_3, \hspace{2ex} \hat{Q}_3^* = Q_3^*Q_2^*Q_1^*$
                \begin{subequations}\begin{align}
                    \text{we know: } A_3 &= Q_3^*A_2Q_3 \\
                     &= Q_3^*\left[Q_2^*A_1Q_2\right]Q_3 \\
                     &= Q_3^*\left[Q_2^*\left(Q_1^*AQ_1\right)Q_2\right]Q_3 \\
                    A_3 &= \hat{Q}_3^*A\hat{Q}_3
                \end{align}\end{subequations}
                Clearly this can continue for all $m$.
        \end{enumerate}

    \item Deduce that 
        \begin{equation}
            \left(A-\nu_mI\right)\hat{Q}_m = \hat{Q}_m\left(A_m - \nu_mI\right)
        \end{equation}
        for $m=1,2,3,\ldots$.

        \begin{enumerate}[$m=1$: ]
            \item $\hat{Q}_1 = Q_1$
                \begin{equation}
                    \left(A-\nu I\right)Q_1 = AQ_1 - \nu_1Q_1
                    \label{eq:m1Results}
                \end{equation}

                First we must make note:
                \begin{subequations}\begin{align}
                    A_1 &= Q_1^*AQ_1 \\[2mm]
                    Q_1A_1 &= Q_1Q_1^*AQ_1 \\
                    Q_1A_1 &= AQ_1
                \end{align}\end{subequations}
                Plug this result into Eq. (\ref{eq:m1Results}) to get our answer
                \begin{subequations}\begin{align}
                    \left(A-\nu I\right)Q_1 &= Q_1A_1 - \nu_1Q_1 \\
                    \left(A-\nu I\right)\hat{Q}_1 &= \hat{Q}_1\left( A_1 - \nu_1 I \right)
                \end{align}\end{subequations}

            \item $\hat{Q}_2 = Q_1Q_2, \hspace{2ex} \hat{Q}_2^* = Q_2^*Q_1^*$
                \begin{equation}
                    \left(A-\nu_2 I\right)\hat{Q}_2 = \left(A-\nu_2 I\right)Q_1Q_2
                    \label{eq:m1Results}
                \end{equation}

                Again:
                \begin{subequations}\begin{align}
                    A_2 &= Q_2^*A_1Q_2 \\[2mm]
                    Q_2A_2 &= Q_2Q_2^*A_1Q_2 \\
                    Q_2A_2 &= A_1Q_2
                \end{align}\end{subequations}
                We will use this result shortly
                \begin{subequations}\begin{align}
                    \left(A-\nu_2 I\right)\hat{Q}_2 &= \left(A-\nu_2 I\right)Q_1Q_2 \\
                    &= \left(AQ_1-\nu_2 Q_1\right)Q_2 \\
                    &= \left(Q_1A_1-\nu_2 Q_1\right)Q_2 \\
                    &= Q_1\left(A_1-\nu_2 I\right)Q_2 \\
                    &= Q_1\left(A_1Q_2-\nu_2 Q_2\right) \\
                    &= Q_1\left(Q_2A_2-\nu_2 Q_2\right) \\[2mm]
                    \left(A-\nu_2 I\right)\hat{Q}_2 &= Q_1Q_2\left(A_2-\nu_2 I\right)
                \end{align}\end{subequations}
        \end{enumerate}
        I think it's clear that the rest follows by induction.

    \item Now prove by induction on $m$ that
        \begin{equation}
            \left(A - \nu_{m-1}I\right)\cdots\left(A-\nu_0I\right) = \hat{Q}_m\hat{R}_m, \hspace{5ex}
        \end{equation}
        for $m=1,2,3,\ldots$.

        \begin{enumerate}[$m=1$: ]
            \item $\left(A-\nu_0I\right) = \hat{Q}_1\hat{R}_1$
                \begin{equation}
                    \left(A-\nu_0I\right) = Q_1R_1
                \end{equation}
                This is true by construction of shifted \QR algorithm.

            \item $\left(A-\nu_1I\right)\left(A-\nu_0I\right) = \hat{Q}_2\hat{R}_2$
                \begin{subequations}\begin{align}
                    \left(A-\nu_1I\right)\left(A-\nu_0I\right) &= \left(A-\nu_1I\right)\left(Q_1R_1\right) \\
                     &= \left[ \left( A - \nu_1 \right)Q_1 \right]R_1 \\
                     &= \left[ \hat{Q}_1\left(A_1 - \nu_1I\right) \right]R_1 \\
                     &= \hat{Q}_1\left(Q_2R_2\right)R_1 \\
                    \left(A-\nu_1I\right)\left(A-\nu_0I\right) &= \hat{Q}_2\hat{R}_2
                \end{align}\end{subequations}

            \item $\left(A-\nu_2I\right)\left(A-\nu_1I\right)\left(A-\nu_0I\right) = \hat{Q}_3\hat{R}_3$
                \begin{subequations}\begin{align}
                    \left(A-\nu_2I\right)\left(A-\nu_1I\right)\left(A-\nu_0I\right) &= \left(A-\nu_2I\right)\hat{Q}_2\hat{R}_2 \\
                    &= \left(A-\nu_2I\right)Q_1Q_2\hat{R}_2 \\
                    &= \left(AQ_1-\nu_2Q_1\right)Q_2\hat{R}_2 \\
                    &= \left(Q_1A_1-\nu_2Q_1\right)Q_2\hat{R}_2 \\
                    &= Q_1\left(A_1-\nu_2I\right)Q_2\hat{R}_2 \\
                    &= Q_1\left(A_1Q_2-\nu_2Q_2\right)\hat{R}_2 \\
                    &= Q_1\left(Q_2A_2-\nu_2Q_2\right)\hat{R}_2 \\
                    &= Q_1Q_2\left(A_2-\nu_2I\right)\hat{R}_2 \\
                    &= Q_1Q_2\left(Q_3R_3\right)R_2R_1 \\
                    \left(A-\nu_2I\right)\left(A-\nu_1I\right)\left(A-\nu_0I\right) &= \hat{Q}_3\hat{R}_3
                \end{align}\end{subequations}
        \end{enumerate}
        The rest follows by induction.

\end{enumerate}


\section{Proving Theorem 6.4.11, (Exercise 6.4.22)}
\begin{thm}
    Suppose
    \begin{equation*}
        AV_m = V_mH_m + v_{m+1}h_{m+1,m}e_m^T
    \end{equation*}
    and let $p$ be a polynomial of degree $j < m$.  Then
    \begin{equation}
        p(A)V_m = V_mp(H_m) + E_j,
        \label{eq:ArnoldiPolynomial}
    \end{equation}
    where $E_j \in \mathbb{C}^{n \times m}$ is identically zero, except in the last $j$ columns.
    \label{thm:First}
\end{thm}

We will prove Theorem \ref{thm:First} by induction
\begin{enumerate}[(a)]
    \item Show that the theorem holds when $j=1$.  In this case $p(z) = \alpha_1\left(z-\nu_1\right)$, where $\alpha_1$ is some nonzero constant.
        \begin{subequations}
            \begin{align}
                \left(A-\nu_1I\right)V_m &= V_m\left(H_m-\nu_1I\right) + v_{m+1}h_{m+1,m}e_m^T \\
                p(A)V_m &= V_mp(H_m) + E_1 \\
            \end{align}
            \label{eq:j=1}
        \end{subequations}

        This one is trivial.
    \item Show that the theorem holds when $j=2$.  In this case $p(z) = k\alpha_2\left(z-\nu_1\right)\left(z-\nu_2\right)$.  This step is just for practice; it is not crucial to the proof of the Theorem.

        We begin with the Arnoldi factorization after applying one shift
        \begin{equation}
            \left(A-\nu_1I\right)V_m = V_m\left(H_m-\nu_1I\right) + E_1.
            \label{eq:OneShift}
        \end{equation}
        Now we can apply a second shift (this isn't applying a shift rather just multiplying by $\left(A-\nu_2I\right)$):
        \begin{equation}
            \left(A-\nu_2I\right)\left(A-\nu_1I\right)V_m = \left(A-\nu_2I\right)V_m\left(H_m-\nu_1I\right) + \left(A-\nu_2I\right)E_1.
            \label{eq:SecondShift}
        \end{equation}
        We can substitute $\nu_2$ for $\nu_1$ in Equation \ref{eq:OneShift} to obtain the identity
        \begin{equation}
            \left(A-\nu_2I\right)V_m = V_m\left(H_m-\nu_2I\right) + E_1.
        \end{equation}
        and insert this into Equation \ref{eq:SecondShift}
        \begin{subequations}
            \begin{align}
                \begin{split}
                    \left(A-\nu_2I\right)\left(A-\nu_1I\right)V_m &= \left(A-\nu_2I\right)V_m\left(H_m-\nu_1I\right) \\
                     &+ \left(A-\nu_2I\right)E_1.
                \end{split} \\
                \begin{split}
                    &= \left[V_m\left(H_m-\nu_2I\right) + E_1\right]\left(H_m-\nu_1I\right) \\
                    &+ \left(A-\nu_2I\right)E_1.
                \end{split} \\
                \begin{split}
                    &= V_m\left(H_m-\nu_2I\right)\left(H_m-\nu_1I\right) + E_1\left(H_m-\nu_1I\right) \\
                    &+ \left(A-\nu_2I\right)E_1.
                \end{split} \\
                    \left(A-\nu_2I\right)\left(A-\nu_1I\right)V_m &= V_m\left(H_m-\nu_2I\right)\left(H_m-\nu_1I\right) + E_2
            \end{align}
        \end{subequations}
        where $E_2 = E_1\left(H_m-\nu_1I\right) + \left(A-\nu_2I\right)E_1$.which is identically zero, except in the last 2 columns.

    \item Show that if the theorem holds for polynomials of degree $j-1$, then it holds for polynomials of degree $j$.

        We know:
        \begin{equation}
            \left(A-\nu_{j-1}I\right)\cdots\left(A-\nu_1I\right)V_m = V_m\left(H_m-\nu_{j-1}I\right)\cdots\left(H_m-\nu_1I\right) + E_{j-1}
        \end{equation}
        Multiply this by $\left(A-\nu_jI\right)$,
        \begin{equation}\begin{split}
            \left(A-\nu_jI\right)\left(A-\nu_{j-1}I\right)\cdots\left(A-\nu_1I\right)V_m &= \\
            & \left(A-\nu_jI\right)V_m\left(H_m-\nu_{j-1}I\right)\cdots\left(H_m-\nu_1I\right) \\
            &+ \left(A-\nu_jI\right)E_{j-1}
        \end{split}\end{equation}
        Of course:
        \begin{equation}
            \left(A-\nu_jI\right)V_m = V_m\left(H_m-\nu_jI\right) + E_1.
        \end{equation}
        which can be inserted into the previous equation  as before
        \begin{subequations}\begin{align}
            \begin{split}
                \left(A-\nu_jI\right)\cdots\left(A-\nu_1I\right)V_m &= \\
                &\left[ \left(A-\nu_jI\right)V_m \right]\left(H_m-\nu_{j-1}I\right)\cdots\left(H_m-\nu_1I\right) \\
                &+ \left(A-\nu_jI\right)E_{j-1}
            \end{split} \\
            \begin{split}
                &= \left[ V_m\left(H_m-\nu_jI\right) + E_1 \right]\left(H_m-\nu_{j-1}I\right)\cdots\left(H_m-\nu_1I\right) \\
                &+ \left(A-\nu_jI\right)E_{j-1}
            \end{split} \\
            \begin{split}
                &= V_m\left(H_m-\nu_jI\right)\left(H_m-\nu_{j-1}I\right)\cdots\left(H_m-\nu_1I\right) \\
                &+ E_1\left(H_m-\nu_{j-1}I\right)\cdots\left(H_m-\nu_1I\right) + \left(A-\nu_jI\right)E_{j-1}
            \end{split} \\
               \left(A-\nu_jI\right)\cdots\left(A-\nu_1I\right) &= V_m\left(H_m-\nu_jI\right)\left(H_m-\nu_{j-1}I\right)\cdots\left(H_m-\nu_1I\right) + E_j \\
       \intertext{Finally}
       p(A)V_m &= V_mp(H_m) + E_j.
        \end{align}\end{subequations}
\end{enumerate}

This isn't how IRAM actually proceeds.  However we can create an equation as in eq. \ref{eq:ArnoldiPolynomial} by simply operating on the left by some polynomial of $A$; in this case we have chosen a polynomial with root $\nu_j$.  It just so happens (not coincidentally) that using the same $\nu_j$'s as shifts for the shifted \QR algorithm on $H_m$.  Now we have proven in section \ref{sec:QRPolynomial} that 
\begin{equation}
    p\left(H_m\right) = \hat{Q}_m\hat{R}_m
\end{equation}
where $\hat{Q}_m$ and $\hat{R}_m$ come from the shifted \QR algorithm.  Continue on from equation 6.4.12 in Watkins.

\section{Exercise (6.4.21) pg. 460}
Let $j$ be a non-negative integer.  A matrix $B$ is called $j$-Hessenberg if $b_{ij} = 0$ whenever $i-k>j$.  A $j$-Hessenberg matrix is said to be \emph{properly} $j$-Hessenberg if $b_{ij} \neq 0$ whenever $i-k=j$.

\begin{enumerate}[(a)]
    \item \emph{What are the common names for 0-Hessenberg and 1-Hessenberg matrices?}

    A 0-Hessenberg matrix is \emph{upper triangular}.  A 1-Hessenberg is commonly called \emph{upper Hessenberg}.

    \item Show that the product of a properly $j$-Hessenberg and a properly $k$-Hessenberg matrix is properly $\left(j+k\right)$-Hessenberg.\label{itm:j+k}

    The $r$th row of a properly $j$-Hessenberg matrix has
    \begin{equation*}
        \max\left(r-j-1,0\right)
    \end{equation*}
    leading zeros.  The $c$th column of a properly $k$-Hessenberg matrix has
    \begin{equation*}
        \max\left[s-\left(c+k)\right),0\right]
    \end{equation*}
    trailing zeros where $s$ is the size of the matrix.

    Suppose $\mathcal{J}$ is properly $j$-Hessenberg and $\mathcal{K}$ is properly $k$-Hessenberg and $\mathcal{J},\mathcal{K} \in \mathbb{R}^{s\times s}$.  The resulting product $\mathcal{A} = \mathcal{JK}$.  The elements of $\mathcal{A}$, $a_{mn}$, are the inner product of the $m$-th row of $\mathcal{J}$ and the $n$-th column of $\mathcal{K}$.  This element will be zero if the sum of the number of leading zeros in the $m$-th row of $\mathcal{J}$ and the number of trailing zeros in the $n$-th column of $\mathcal{K}$ are greater than the size of the matrix, $s$:
    \begin{equation}
        a_{mn} = \begin{cases}
            0 & \left( \max\left[(r-j-1),0\right] + \max\left[s-\left(c+k\right),0\right] \right) \geq s \\
            \mathcal{J}_m\cdot\mathcal{K}_n & \mathrm{otherwise}
        \end{cases}
    \end{equation}
    where $\mathcal{J}_m$ is the $m$-th row of $\mathcal{J}$ and $\mathcal{K}_n$ is the $n$-th column of $\mathcal{K}$.  For $\mathcal{A}$ to be properly $\left(j+k\right)$-Hessenberg $a_{mn} = 0$ whenever $m-n>\left(j+k\right)$ and $a_{mn}\neq 0$ whenever $m-n = \left(j+k\right)$.

    \item  Show that if $B \in \mathbb{C}^{m\times m}$ is properly$j$-Hessenberg ($j<m$>, then the first $m-j$ columns of $B$ are linearly independent.
    
    If $B$ is properly $j$-Hessenberg then column $k$ will have one more non-zero element than column \mbox{$k-1$} for \mbox{$k =2,\ldots j$}.  Thus they must be linearly independent.

    \item Show that if $H_m$ is properly upper Hessenberg and $p$ is a polynomial of degree $j$, say $p(z) = \left(z-\nu_1\right)\cdots\left(z-\nu_j\right)$, then $p(H_m)$ is properly $j$-Hessenberg.

    In $p(H_m)$ there will be a term $H_m^j$, which by item \ref{itm:j+k} will be $j$-Hessenberg.  Adding in the other lower order terms will not affect this.


\end{enumerate}

\section{Manual Application of Shifted Algorithm}
I write this section to demonstrate manually, and laboriously what happens when you perform shifted \QR iterations.  The traditional way to show the iterations is given in the left column.  The right column is just some algebra to get the shift in the form
\begin{equation}
    \left( A_0-\nu_nI\right)\cdots\left( A_0 - \nu_1I \right) = \hat{Q}_n\hat{R}_n.
\end{equation}
This is necessary for some proofs I am doing.

\clearpage
\begin{Parallel}[v]{0.49\textwidth}{0.49\textwidth}
\textbf{First Shift}
\ParallelLText{
    \begin{equation}
        \left(A_0 - \nu_1I\right) = Q_1R_1
    \end{equation}

    \begin{subequations}
        \begin{align}
            A_1 &= R_1Q_1  + \nu_1I \\
            A_1 &= Q_1^*A_0Q_1 \\
            A_1 &= \hat{Q}_1^*A_0\hat{Q}_1
        \end{align}
    \end{subequations}
}
\ParallelRText{
    \begin{align}
        \left(A_0 - \nu_1I\right) &= Q_1R_1 \\[2mm]
        Q_1A_1 &= A_0Q_1
    \end{align}
}
\ParallelPar

\vspace{\baselineskip}
\textbf{Second Shift}
\ParallelLText{
    \begin{equation}
        \left(A_1 - \nu_2I\right) = Q_2R_2
    \end{equation}

    \begin{subequations}
        \begin{align}
            A_2 &= R_2Q_2  + \nu_2I \\
            A_2 &= Q_2^*A_1Q_2 \\
            A_2 &= Q_2^*\left(Q_1^*A_0Q_1\right)Q_2 \\
            A_2 &= \hat{Q}_2^*A_0\hat{Q}_2
        \end{align}
    \end{subequations}
}
\ParallelRText{
    \begin{align}
        \left(A_1 - \nu_2I\right) &= Q_2R_2 \\[2mm]
        Q_2A_2 &= A_0Q_2    \label{eq:2Identity}
    \end{align}

    \begin{subequations}
        \begin{align}
            \hat{Q}_1\left(A_1-\nu_2I\right)\hat{R}_1 &= \hat{Q}_1Q_2R_2\hat{R}_1 \\
            \hat{Q}_1A_1\hat{R}_1-\nu_2\hat{Q}_1\hat{R}_1 &= \hat{Q}_2\hat{R}_2 \\
            A_0\hat{Q}_1\hat{R}_1-\nu_2\hat{Q}_1\hat{R}_1 &= \hat{Q}_2\hat{R}_2 \\
            \left( A_0-\nu_2I\right) \hat{Q}_1\hat{R}_1 &= \hat{Q}_2\hat{R}_2 \\
            \left( A_0-\nu_2I\right)\left( A_0 - \nu_1I \right) &= \hat{Q}_2\hat{R}_2
        \end{align}
    \end{subequations}

}
\ParallelPar

\vspace{\baselineskip}
\textbf{$n$'th Shift}
\ParallelLText{
    \begin{equation}
        \left(A_{n-1} - \nu_n\right) = Q_nR_n
    \end{equation}

    \begin{subequations}
        \begin{align}
            A_n &= R_nQ_n  + \nu_nI \\
            A_n &= Q_n^*A_{n-1}Q_n \\
            A_n &= Q_n^*\left(Q_{n-1}^*A_{n-2}Q_{n-1}\right)Q_n \\
            \vdots \\
            A_n &= \hat{Q}_n^*A_0\hat{Q}_n
        \end{align}
    \end{subequations}
}
\ParallelRText{
    \begin{align}
        \left(A_{n-1} - \nu_nI\right) &= Q_nR_n \\[2mm]
        \hat{Q}_{n-1}A_{n-1} &= A_0\hat{Q}_{n-1}    
    \end{align}

    \begin{subequations}
        \begin{align}
            \hat{Q}_{n-1}\left(A_{n-1}-\nu_nI\right)\hat{R}_{n-1} &= \hat{Q}_{n-1}Q_nR_n\hat{R}_{n-1} \\
            \hat{Q}_{n-1}A_{n-1}\hat{R}_{n-1}-\nu_n\hat{Q}_{n-1}\hat{R}_{n-1} &= \hat{Q}_n\hat{R}_n \\
            A_0\hat{Q}_{n-1}\hat{R}_{n-1}-\nu_n\hat{Q}_{n-1}\hat{R}_{n-1} &= \hat{Q}_n\hat{R}_n \\
            \left( A_0-\nu_nI\right)\left( \hat{Q}_{n-1}\hat{R}_{n-1} \right) &= \hat{Q}_n\hat{R}_n \\
            \left( A_0-\nu_nI\right)\cdots\left( A_0 - \nu_1I \right) &= \hat{Q}_n\hat{R}_n
        \end{align}
    \end{subequations}

}
\end{Parallel}


\end{document}
