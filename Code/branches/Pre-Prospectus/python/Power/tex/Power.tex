% How does my Power Method proceed
%
\documentclass[11pt]{article}
\usepackage[marginparwidth=1.0in]{geometry}
\usepackage[usenames]{color}
\usepackage{amsmath}

\title{How Does My Power Method Proceed?\\\large{(or: Why Doesn't it Work?)}}
\author{Jeremy L. Conlin}
\begin{document}
\maketitle


\section{Setup}

\section{Power Method}

\subsection{Sample fissionBank}
When sampling from a fissionBank, to create a new fissionBank I select a particle at random, remove it from the bank.  I set its weight to 1 and give it a random direction.\marginpar{If I don't set a random direction, $k_{eff}$ is much higer.  0.43 vs. 0.36.}

If I need more neutrons than are in the bank I sample from, I pick neutrons randomly from the bank, without removing them until the number of particles I need is the length of the current bank.  Then I proceed as described in proceeding paragraph.

\subsection{Transport}
This is (I suppose) the meat of the Power Method, also where most things could possibly go wrong.

For ecah neutron in the fissionBank, I do the following

\subsubsection{Move}\label{sec:Move} I move my particle using the following equations:
\begin{subequations}
    \begin{eqnarray}
        d = -\frac{1}{\Sigma_t}\ln(\xi) \\
        x = x + d*u \\
        y = y + d*v \\
        z = z + d*w \\
    \end{eqnarray}
\end{subequations}
where $d$ is the distance traveled between collisions and $(u,v,w)$ is the direction-vector of travel and, of course, $(x,y,z)$ is the position of the particle before moving.

\subsubsection{Check if still inside}  If the particle is still inside my geometry (slab) I collide it as follows.
\begin{itemize}
    \item Add neutrons to next fissionBank.

    Th number of neutrons I add to the next fissionBank is:
    \begin{equation}
        N = \left\lfloor \omega'\left( \frac{1}{k}*\frac{\nu\Sigma_f}{\Sigma_t}\right) +  \xi \right\rfloor,
    \end{equation}
    where $\omega^{\prime}$ is the weight of the neutron before the collision.

    \item Calculate new weight.

    I play implicit capture so I need to change the weight of my particle at each collision.  The weight, after a collision is:
    \begin{equation}
        \omega = \omega'\left(1 - \frac{\Sigma_a}{\Sigma_t}\right) = \omega^{\prime}\left(\frac{\Sigma_s}{\Sigma_t}\right).
    \end{equation}

    \item Change Direction.

    I sample uniformly over $4\pi$ my new travel direction;
    \begin{subequations}
        \begin{gather}
            \phi = 2\pi\xi \\
            u = 2\xi - 1 \\
            v = \sqrt{1-u^2}\cos{\phi} \\
            w = \sqrt{1-u^2}\sin{\phi}
        \end{gather}
    \end{subequations}
\end{itemize}

\subsubsection{Weight Cutoff Game(s)}  If my particle weight happens to become zero or less, I kill it.  This won't happen unless $\Sigma_s$ is zero.  If the particle weight is less than some cutoff weight, $\omega < \omega_c$, I play Russian Roulette.

In Russian Roulette, I pick a random number, $\xi$.  If $\xi < p_{kill}$ I kill the particle; otherwise I increase the weight of the particle like:
\begin{equation}
    \omega = \frac{\omega^{''}}{1 - p_{kill}}
\end{equation}
and return to Section \ref{sec:Move}.

\subsection{Calculate $k_{eff}$}
After one iteration is complete, I calculate a new estimate of $k_{eff}$ as:
\begin{equation}
    k^{n+1} = k^n\frac{Q^{n+1}}{N},
\end{equation}
where $Q^{n+1}$ is the number of fission neutrons in the fissionBank for the next cycle and $N$ is the number of histories followed in the current cycle.

\end{document}
