%$Author: jlconlin $
%$Date: 2007-11-20 11:23:01 -0700 (Tue, 20 Nov 2007) $
%$Revision: 192 $
%$Id: CodeDoc.tex 192 2007-11-20 18:23:01Z jlconlin $
\documentclass[12pt]{article}
\usepackage[margin=1in]{geometry}
\usepackage[algo2e, ruled, linesnumbered]{algorithm2e}

\SetKwComment{Comment}{$\triangleright$ }{}
\dontprintsemicolon

% Define some new commands so document is uniform throughout
\newcommand{\type}[1]{\texttt{#1}}
\author{Jeremy Conlin}
\title{Code Documentation}

\begin{document}
\maketitle
\section{Class \type{Particle}}
\subsection{Methods}
\subsubsection{\texttt{move()}}
\texttt{move} is perhaps the workhorse of the \type{Particle} class.  This method is responsible for moving the particle in the \type{Field}.
\begin{algorithm2e}
    \SetVline
    \caption{\texttt{move} method for \type{Particle} class.}
    Calculate distance to collision, 
    $d = \frac{1}{\Sigma_T}$\;
    \While{Still in geometry}{
        \eIf{\type{Particle} has left \type{Zone}}{
            \eIf{\type{Particle} has left geometry}{
                Throw NoZoneConnected Error \Comment*[f]{Must be caught by caller function!}
            }
            {
                Increment pointer to \type{Zone} so \type{Particle} knows which \type{Zone} it is in.\;
                Move \type{Particle} to \type{Zone} edge.\;
            }
            }
            {
            Update \type{Particle} position.
}
}
\end{algorithm2e}
Note: it is the responsibility of the caller function to \texttt{move} to make sure \texttt{collision} and other methods are also called.

\end{document}
